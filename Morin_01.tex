\chapter{The Principle of Relativity}
\label{ch:Morin_01}
Einstein based the theory of relativity on two postulates. 
The first is known as the principle of relativity:\\
\tit{No phenomena have properties corresponding to the concept of absolute rest.}

The principle was first enunciated by Galileo, three centuries earlier, and was built into the classical mechanics of Newton. In this chapter we shall elaborate Einstein's concise statement of the principle of relativity, and then explore how the principle can be used to discover some elementary but not entirely obvious facts about how things behave. 

The principle of relativity fits the same pattern of \tit{invariance} principles. It would naturally appear as the last one in the following list of four, where each principle states that the content\footnote{By the \tit{content} of a physical law we mean the set of all assertions that are explicitly made (or are logically implied) by that law.} of some physical laws remains the same under a specified change in the \tit{description} of the physical world.  

\begin{itemize}[noitemsep,topsep=0pt]
\item \tbi{Translational invariance in space}\\
The change is a rigid \tbi{translation} of the origin of the spatial axes.
\item \tbi{Rotational invariance in space}\\
The change is a rigid \tbi{rotation} of the the three spatial axes.
\item \tbi{Translational invariance in time}\\
The change is a \tbi{translation} of the origin of time.
\item \tbi{Principle of Relativity}\\
The change is the \tit{uniform, rigid motion} of the spatial axes along a fixed direction.
\end{itemize}

In working with the principle of relativity, one uses the term \tit{frame of reference}. This is the system in terms of which you have chosen to describe things. For example, it is natural for flight attendants and passengers of an airplane to choose a reference frame that is fixed with respect to the airplane, but any motion occurring inside the airplane can be described equally well with a reference frame that is fixed with respect to the ground. 

Another important term is \tit{inertial frame of reference}.  In an inertial frame, objects on which no forces act remain stationary (if they were at rest) or keep moving with constant speed along a fixed direction (if they were in motion). Equivalently: if the reference frame is an inertial one, objects subject to no forces do not accelerate
\footnote{Acceleration is the \tbi{vector} $\dot{\vb{v}}$ representing the change of velocity $\vb{v}$ in the unit of time. At a given time $t$, $\dot{\vb{v}}$ can be decomposed as the sum of a vector parallel to $\vb{v}$ and a vector $\vb{\omega}$ orthogonal to $\vb{v}$. If $\vb{\omega} = \vb{0}$ there is no change in the \tit{direction} of $\vb{v}$.}.  

Any frame whose axes move with constant velocity with respect to an inertial frame is itself an inertial frame. Each frame thus defines a whole equivalence class of reference frames in uniform relative motion with respect to each other. 

People sometimes take the principle of relativity to mean, loosely speaking, that the behavior of a uniformly moving object should not depend on how fast it is moving, or that motion with uniform velocity cannot affect any properties of an object. This is simply wrong. The principle of relativity only requires that if an  object has certain properties in a frame of reference in which the object is stationary, then if the same object moves uniformly, it will have the same properties \tit{in a frame of reference that moves uniformly with it}. 

Take for example the \tit{Doppler effect}. If a yellow light moves away from you at an enormous speed, the color you see changes from yellow to red; it it moves toward you at an enormous speed, the color changes from yellow to blue. So the color of an object in a fixed frame of reference can depend on whether it is moving or at rest, and in what direction it is moving. What the relativity principle guarantees is that if a light is seen to be yellow when it is stationary, then when it moves with uniform velocity it will still be seen as yellow \tit{by someone who moves with that same velocity}.

We will be applying the principle of relativity to learn some quite extraordinary things by examining the same sets of events in different frames of reference. Some of the things we shall learn in this way are so surprising that they are hard to believe at first. The general procedure for doing this is always the same: \tit{Take a situation which you don't fully understand. Find a new frame of reference in which you do understand it. Examine it in that new frame of reference. Then translate your understanding in the new frame back into the language of the old one}.
\\\tbf{Example 1.}\\
Newton first law of motion states that in the absence of an external force a uniformly moving body continues to move uniformly. This law follows from the principle of relativity and a very much simpler law. The simpler law merely states that in the absence of an external force, a stationary body continues to remain stationary. 

Suppose we only know the simpler law. The principle of relativity tells us that it must be be true in all inertial frames of reference. If we want to learn about the subsequent behavior of a ball initially moving at 50 f/sec in the absence of an external force, all we have to do is find an inertial frame of reference in which we can apply the simpler law. The frame we need is clearly the one that moves at 50 f/sec in the same direction as the ball, since in that frame the ball is stationary. In that frame we can apply the law that in absence of an external force a stationary body remains stationary. Assuming that if an object is undisturbed in one inertial frame of reference, then it is undisturbed in any other inertial frame of reference (that the condition of no force acting on an object is an \tit{invariant} condition independent of the frame of reference in which the object is described), we conclude that in the original frame the ball must continue to move at 50 f/sec in the absence of an external force. 
\\\tbf{Example 2.}\\
Suppose we have two identical perfectly elastic balls. Identical elastic balls have the property that if you shoot them directly at each other with the same speed, then after they collide each bounces back in the direction it came from with the same speed that it had before the collision. Question: What happens if one of the balls is at rest and you shoot the other one directly at it?

There is a long tradition of answering such questions by invoking the conservation of energy and momentum.At this stage it is both entertaining and instructive to understand how this and many related questions can be answered using nothing but the principle of relativity.

To figure out what happens, using only the principle of relativity, first draw a picture illustrating the rule you know: when the balls move at each other with equal speeds, they simply rebound with the same speeds. 

Then draw a picture of the new situation. The white ball moves to the right along the tracks, in a railroad station, at 10 f/sec toward the stationary black ball. 

Now think about how this would look if we sere describing it from the frame of reference of a train moving through the station to the right at 5 f/sec. Since the white ball covers 10 feet of track per second, and the train covers 5 feet of track per second, every second the white ball gains 5 feet on the train. So in the frame of reference of a train moving to the right at 5 f/sec, the white ball moves at 5 f/sec. Since the black ball is stationary with respect to the tracks, in the train frame it moves to the \tit{left} at 5 f/sec, just as the tracks do. 

Therefore, in the frame of this particular train the unknown situation before the collision becomes an instance of the known situation, in which the balls approach each other with the same speed. By the principle of relativity, the equality of conditions \tit{before} the collision implies the equality of conditions \tit{after} the collision, namely that the balls will bounce back, away from each other, with equal speeds. 

Finding what happens in the frame of reference of the station is now a matter of translating the experiment outcome as described in the frame of the train to the frame of the station. After the collision the white ball moves to the left at 5 f/sec in the train frame, so it must be stationary in the station frame. After the collision, the black ball moves to the right at 5 f/sec in the train frame, so it must be moving to the right at 10 f/sec in the station frame. 

So we have used the principle of relativity to learn something new about identical elastic balls: if one is at rest and the other bumps it head-on, then the moving one comes to a complete stop and the stationary one moves off with the velocity the formerly moving one (the white ball) originally had. This is a fact familiar to all plyers of billiards, but not many of them realize that it is simply a consequence of the much more obvious fact (less frequently encountered in billiards) that when two balls collide head-on with equal and opposite speeds, each bounces back the way it came with its original speed. 
\\\tbf{Example 3.}\\
Two identical sticky balls have the property that if they are fired directly at each other with equal speeds, then they stick together upon collision and the resulting compound ball is stationary. If a sticky ball is fired at 10 f/sec directly at another identical sticky ball that is stationary and the two stick together, with what speed and in what direction will the compound ball move after the collision?

We can again answer the question using only the principle of relativity, by viewing the initial moving white ball and initially stationary black ball from the frame of reference of a train in which both are moving with the same speed but in opposite directions. Such a train moves along the direction of motion of the white ball but only at 5 f/sec. In the train frame the situation before the collision is the one we know about: the balls move at each other at the same speeds. Therefore we know that in the train frame the compound ball is stationary after the collision. But since the train moves down the tracks at 5 f/sec and the compount is stationary in the train frame, in the track frame it will move down the tracks at 5 f/sec -- the same speed as the train moves in the track frame. This solves the problem: when the moving ball strickes the stationary ball, the compound ball moves at half the original speed of the moving ball.    
\\\tbf{Example 4.}\\
This one has the virtue that it will not be obvious how to solve it by exploiting the conservation of momentum, but it is easily solved using the principle of relativity. 

Suppose we have two elastic balls, but one of them is very big and the other is very small. If the big ball is stationary and the small ball is fired directly at it, the small ball simply bounces back in the direction it came from with the same speed, and the big ball stays at rest.(Think of throwing a table-tennis ball directly at a bowling ball.) With what speed will each ball move after the collision, if the small ball is stationary and the big ball is fired directly at it with a speed of 10 f/sec?

We wish to examine the initial situation in a frame of reference in which the big ball is stationary, so we must now view the collision in the frame of a train moving, with the big ball, at 10 f/sec to the left. In that frame the small ball will move at 10 f/sec to the right, and the situation before the collision is the one we understand. So in the train frame we know that after the collision the big ball will remain stationary and the small ball will move at 10 f/sec to the left. Returning to the description in the station frame, we note that after the collision the big ball moves with the train, at 10 f/sec to the left. The little ball, however, moves at 20 f/sec to the left, since in each second it gains 10 feet on the train, which has itself moved 10 feet to the left. So if the little ball is initially stationary, then after a collision with the big ball it moves off at \tit{twice} the speed of the big one. 
\\\tbf{Example 5.}\\
What happens if the big and little ball of the previous example approach each other with the \tit{same} speed--say 5f/sec. In that case the train providing the frame of reference in which we know the answer moves, with the big ball, at 5 f/sec to the left, so the little ball moves to the right at 10 f/sec in the train frame. After the collision the big ball remains stationary in the train frame, while the little ball moves to the left at 10 f/sec. So back in the station frame the little ball moves to the left at 15 f/sec, with \tit{triple} its original speed.

You can see a spectacular demonstration of this by placing the little ball, for example a tennis ball, at the very top of the big ball, for example a basketball, and then dropping them on a hard surface very carefully, so that the little ball does not roll off the top of the big one. When the big ball hits the floor it reverses its direction of motion without a change in speed, so for a very brief moment the big ball is moving up and the little ball is moving down, both going at the same speed. Immediately after that, the little ball flies up at nearly three times its original speed. As it happens, the height reached by a ball moving up is proportional to the \tit{square} of its initial speed, so if losses due to various kinds of friction are unimportant, the little ball can shoot up to almost nine times the height from which it was originally dropped!     

I hope these examples will give you a feeling for how the principle of relativity is actually used, and for the power it can have to predict behavior under apparently unfamiliar conditions. Before starting to apply it under genuinely unfamiliar conditions, we must look a little more closely at some of the reasoning we used in these simpler examples.

