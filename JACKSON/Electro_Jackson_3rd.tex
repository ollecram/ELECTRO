%%%%%%%%%%%%%%%%%%%% book.tex %%%%%%%%%%%%%%%%%%%%%%%%%%%%%

\documentclass[english, 11pt, a4paper]{book}

% Some specific typographic conventions used in Griffiths I2QM   START 
\usepackage{mathtools}			% equation tag with [..] instead of (..)
\newtagform{brackets}{[}{]}		% equation tag with [..] instead of (..)
\usetagform{brackets}			% equation tag with [..] instead of (..)
% Some specific typographic conventions used in Griffiths I2QM   END 

%\usepackage[nomath]{lmodern}
\usepackage[T1]{fontenc}
%\usepackage[italian]{babel}
% The following changes the Chapter heading from 'Chapter' to 'Lecture'
%$\addto\captionsenglish{\renewcommand{\chaptername}{Lecture}}
%$%\usepackage{fancyhdr}
%$\newcommand\chap[1]{%
%$ \chapter*{#1}%
%$  \addcontentsline{toc}{chapter}{#1}}
%$\newcommand\sect[1]{%
%$  \section*{#1}%
%$  \addcontentsline{toc}{section}{#1}}

% The following for using the Euro symbol
\usepackage[utf8x]{inputenc}
\usepackage{lmodern, textcomp}
  
% choose options for [] as required from the list
% in the Reference Guide, Sect. 2.2

\usepackage{makeidx}         % allows index generation
\usepackage{graphicx}        % standard LaTeX graphics tool
\usepackage{subcaption}      % for subfigures environments 
                             % when including figure files
\usepackage{multicol}        % used for the two-column index
\usepackage[bottom, symbol]{footmisc} % See https://tex.stackexchange.com/questions/826/symbols-instead-of-numbers-as-footnote-markers
% etc.
% see the list of further useful packages
% in the Reference Guide, Sects. 2.3, 3.1-3.3
\usepackage[normalem]{ulem}

\usepackage[shortlabels]{enumitem}	% to be able to resume enumerated lists

\usepackage{amsmath}	% To be able to slash
\usepackage{bm}	        % To use bold greek letters in math mode with \bm{}
\usepackage{amsfonts}	% To be able to use \mathbb ... 
\usepackage{amssymb}	% To be able to use \nmid ... 
\usepackage{amsthm}		% \qed, \qedhere
\usepackage{slashed}	% any character (dirac)
\usepackage[title,toc,page]{appendix}

% See https://tex.stackexchange.com/questions/36524/how-to-put-a-framed-box-around-text-math-environment/36528
\usepackage{collectbox}	% To make box around formulas

% *** AFTER THIS LINE *** 
%     put \usepackage{} for shared packages kept under ~\Links\repos\git\LaTeX_Styles

% Physics package 
% https://tex.stackexchange.com/questions/38978/how-can-i-manually-install-a-latex-package-debian-ubuntu-linux
\usepackage[italicdiff]{/home/marcello/Links/repos/git/LaTeX_Styles/physics}	
% To put accents below letters
\usepackage{/home/marcello/Links/repos/git/LaTeX_Styles//accents}

% To attach a symbols (instead of a number) to footnotes 
%\usepackage[symbol]{footmisc}

% To control vertical white space above and below equations
% see https://tex.stackexchange.com/questions/69662/how-to-globally-change-the-spacing-around-equations
\expandafter\def\expandafter\normalsize\expandafter{%
    \normalsize
    \setlength\abovedisplayskip{16pt}
    \setlength\belowdisplayskip{16pt}
    \setlength\abovedisplayshortskip{16pt}
    \setlength\belowdisplayshortskip{16pt}
}

% FROM BOXED_TEXT_ETC.tex

% To write two equations side by side
\usepackage{multicol}

% To use PGF/TikZ https://tex.stackexchange.com/questions/3622/best-way-to-generate-nice-function-plots-in-latex
\usepackage{tikz}
\usetikzlibrary{datavisualization}
\usetikzlibrary{datavisualization.formats.functions}

% To create a placeholder paragraph with Latin text
\usepackage{lipsum}

% To create framed text boxes with custom defined styles 
%\usepackage[linewidth=1pt]{mdframed}
\usepackage[framemethod=TikZ]{mdframed}
\mdfdefinestyle{MyFrame}{%
    linecolor=brown,			% blue, orange, brown, ...
    outerlinewidth=1pt,
    roundcorner=10pt,
    innertopmargin=\baselineskip,
    innerbottommargin=\baselineskip,
    innerrightmargin=15pt,
    innerleftmargin=15pt,
    backgroundcolor=gray!5!white}

		%Use for creating boxed/framed parts of text with nice borders

% To use extra symbols like dagger and double dagger in numbering footnotes 
\usepackage{footmisc}

% Allows aligning numbers at decimal point within `tabular environment
%\usepackage{siunitx}
%\sisetup{
%  round-mode          = places, % Rounds numbers
%  round-precision     = 2, % to 2 places
%}

% Force chapter numbering to restart within each part
\makeatletter
%\@addtoreset{chapter}{part}
\makeatletter


\makeindex             % used for the subject index
                       % please use the style svind.ist with
                       % your makeindex program


%%%%%%%%%%%%%%%%%%%%%%%%%%%%%%%%%%%%%%%%%%%%%%%%%%%%%%%%%%%%%%%%%%%%%

\begin{document}

% Useful within \begin{equation*}...\end{equation*} to have ONE equation with number & label
\newcommand\numberthis{\addtocounter{equation}{1}\tag{\theequation}}

\newcommand{\umlaut}[1]{\"#1}
\newcommand{\quotes}[1]{``#1''}
\newcommand{\ovr}[1]{\overline{#1}}
\newcommand{\sfT}{$\mathsf{T}$}
\newcommand{\udT}{\rotatebox[origin=c]{180}{$\mathsf{T}$}}
\newcommand{\avg}[1]{\langle{#1}\rangle}

%Bold calligraphic letters 
\newcommand{\N}{\mathbb{N}}	% integers
\newcommand{\Z}{\mathbb{Z}}	% relative
\newcommand{\Q}{\mathbb{Q}}	% rationals
\newcommand{\R}{\mathbb{R}}	% reals
\newcommand{\C}{\mathbb{C}}	% complex
\newcommand{\F}{\mathbb{F}}	% generic field 1
\newcommand{\K}{\mathbb{K}}	% generic field 2
\newcommand{\V}{\mathbb{V}}	% Shankar's for vector space V

%Plain calligraphic letters 
\newcommand{\cC}{\mathcal{C}}    % space 1
\newcommand{\cF}{\mathcal{F}}    % space 2
\newcommand{\cH}{\mathcal{H}}    % Calligraphic H for Hilbert space
\newcommand{\cS}{\mathcal{S}}    % space 3, Flow of energy (e.g in electromagnetism)
\newcommand{\cT}{\mathcal{T}}    % space 4 

\newcommand{\cI}{\mathcal{I}}    % Moment of Inertia
\newcommand{\cU}{\mathcal{U}}    % sets 1
\newcommand{\cV}{\mathcal{V}}    % sets 2
\newcommand{\cW}{\mathcal{W}}    % sets 3
\newcommand{\cP}{\mathcal{P}}    % sets 4, Momentum density (e.g in electromagnetism) 
\newcommand{\cQ}{\mathcal{Q}}    % sets 5
\newcommand{\cR}{\mathcal{R}}    % sets 6

\newcommand{\cL}{\mathcal{L}}    % Lagrangian density
\newcommand{\cE}{\mathcal{E}}    % Energy density (e.g in electromagnetism)
\newcommand{\cY}{\mathcal{Y}}    % Y

% Quaternions
\newcommand{\Qt}{\mathbb{H}}	% Hamilton's quaternions ('\H' APPARENTLY defined elsewhere by LaTeX}
\newcommand{\qu}{\mathbf{1}}     % 1
\newcommand{\qi}{\mathbf{i}}     % i
\newcommand{\qj}{\mathbf{j}}     % j
\newcommand{\qk}{\mathbf{k}}     % k

% Fraktur (Gothic) font (e.g for algebras)
\newcommand{\frk}[1]{\mathfrak{#1}}  

% To show argument of the exponential function vertically, i.e., as a superscript 
\newcommand{\vexp}[1]{\,e^{#1}}

% To type an angle as a number of degrees like 45^\circ
\newcommand{\degree}[1]{{#1}^\circ}

% To create not-bold vectors with a hat or check accent 
\newcommand{\hatv}[1]{\hat{#1}}
\newcommand{\chkv}[1]{\check{#1}}

% To create boldface vectors with a hat or check accent 
\newcommand{\hatvb}[1]{\vb{\hat{#1}}}
\newcommand{\chkvb}[1]{\vb{\check{#1}}}

% To create boldface greek letters (e.g. for denoting vectors) 
\newcommand{\bmath}[1]{\bm{#1}}  				% SAME AS \bm{#1} - NOT WORTH USING 
\newcommand{\chkbm}[1]{\boldmath{\check{#1}}}	% bold-check
\newcommand{\hatbm}[1]{\boldmath{\hat{#1}}}		% bold-hat

% To create <x|, |x> and <x|y> with unit vectors inside
\newcommand{\ubra}[1]{\bra*{\vu{#1}}}
\newcommand{\uket}[1]{\ket*{\vu{#1}}}
\newcommand{\uip}[2]{\ip*{\vu{#1}}{\vu{#2}}}

% To put accents below letters. 
\newcommand{\ut}[1]{\underaccent{\tilde}{#1}}
\newcommand{\uh}[1]{\underaccent{\hat}{#1}}
\newcommand{\form}[1]{\uh{#1}}

% To create italic, bold, bolditalic text
\newcommand{\tit}[1]{\textit{#1}}
\newcommand{\tbf}[1]{\textbf{#1}}
\newcommand{\tbi}[1]{\textit{\textbf{#1}}}

% Latin Modern sans serif |OR| Helvetica (SELECT)
\newcommand{\textlmss}{\fontfamily{lmss}\selectfont}
\newcommand{\texthv}{\fontfamily{phv}\selectfont}

% Latin Modern sans serif |OR| Helvetica (USE, within OR outside MATH !)
\newcommand{\tlmss}[1]{\text{\textlmss{#1}}}
\newcommand{\thv}[1]{\text{\texthv{#1}}}

% To use \tlmss{T} symbol to denote transpose 
\newcommand{\transp}[1]{{#1}^{\tlmss{T}}}

% To use \dagger symbol to denote operator Adjoint
\newcommand{\Adj}[1]{{#1}^\dagger}

% To denote the Hermitian conjugate with a '+' superscript
\newcommand{\Hconj}[1]{{#1}^{+}}

% To use \tlmss{Ker}, \tlmss{Coker} and \tlmss{Img} to denote Kernel, Co-Kernel & Image 
\newcommand{\Ker}{\tlmss{Ker}\,}
\newcommand{\Coker}{\tlmss{Coker}\,}
\newcommand{\Img}{\tlmss{Im}\,}

% To use \tlmss{Alt} and \tlmss{alt} to denote alternation 
\newcommand{\Alt}{\tlmss{Alt}\,}
\newcommand{\alt}{\tlmss{alt}\,}

% To use \tlmss{Ann} to denote annulets 
\newcommand{\Ann}{\tlmss{Ann}\,}

% Misc abbreviations
\newcommand{\ora}[1]{\overrightarrow{#1}}

\DeclareRobustCommand{\rchi}{{\mathpalette\irchi\relax}}
\newcommand{\irchi}[2]{\raisebox{\depth}{$#1\chi$}} % inner command, used by \rchi

% See https://tex.stackexchange.com/questions/36524/how-to-put-a-framed-box-around-text-math-environment/36528
\makeatletter
\newcommand{\mybox}{%
    \collectbox{%
        \setlength{\fboxsep}{1pt}%
        \fbox{\BOXCONTENT}%
    }%
}
\makeatother

\author{Marcello Vitaletti}
\title{Classical Electrodynamics\\
\small{Selected material from J.D. Jackson 3rd Edition}}
\maketitle

\frontmatter%%%%%%%%%%%%%%%%%%%%%%%%%%%%%%%%%%%%%%%%%%%%%%%%%%%%%%

%\include{dedic}

%\chapter*{Plan}
\label{plan} 

In this book I am keeping notes about the theory of classical electromagnetism, 
as exposed in various books. In particular, I intend to cover the following materials:

\begin{itemize}

\item B. Felsager -- Geometry Particles and Fields
\begin{enumerate}
\setcounter{enumi}{0}
\item Electromagnetism (1.1 to 1.4)
\end{enumerate}

\item C. Cattaneo -- Teoria Einsteniana della Gravitazione
\begin{enumerate}
\setcounter{enumi}{0}
\item Elementi di Algebra e Analisi Lineare
\end{enumerate}

\item D.J. Griffiths -- Introduction to Electrodynamics
\begin{enumerate}
\setcounter{enumi}{0}
\item Vector Analysis
\item Electrostatics
\item Potentials
\item Electric Fields in Matter
\item Magnetostatics
\item Magnetic Fields in Matter
\item Electrodynamics
\item Conservation Laws
\item Electromagnetic Waves
\item Radiation
\item Electrodynamics and Relativity
\item Potentials and Fields
\item Helmoltz Theorem
\end{enumerate}

\item J.D. Jackson -- Classical Electrodynamics, 2nd Edition
\begin{enumerate}
\setcounter{enumi}{0}
\item Introduction to Electrostatics
\item Boundary Value Problems in Electrostatics - I
\item Boundary Value Problems in Electrostatics - II
\item Multipoles, Electrostatics of Macroscopic Media, Dielectrics
%\item Magnetostatics
%\item Time Varying Fields, Maxwell Equations, Conservation Laws
%\item Plane Electromagnetic Waves and Wave Propagation
%\item Wave Guides and Resonant Cavities
%\item Simple Radiating Systems, Scattering and Diffraction
%\item Magnetohydrodynamics and Plasma Physics
\end{enumerate}

\item J.D. Jackson -- Classical Electrodynamics, 3rd Edition
\begin{enumerate}
\setcounter{enumi}{4}
\item Magnetostatics, Faraday's Law, Quasi-Static Fields
\item Maxwell Equations, Macroscopic Electromagnetism, Conservation Laws
\item Plane Electromagnetic Waves and Wave Propagation
\item Wave Guides, Resonant Cavities and Optical Fibers
\item Radiating Systems, Multipole Fields and Radiation
\item Scattering and Diffraction
\item Special Theory of Relativity
\item Dynamics of Relativistic Particles and Electromagnetic Fields
\end{enumerate}

\item B. Felsager -- Geometry Particles and Fields
% Contacts with quantum theory of particles dynamics in EM fields
\begin{enumerate}
\setcounter{enumi}{1}
\item Interaction of Fields and Particles
\end{enumerate}

\item J. Franklin -- Advanced Mechanics and General Relativity
\begin{enumerate}
\setcounter{enumi}{1}
\item Relativistic Mechanics
\item Tensors
\item Curved Space
\item Scalar Field Theory
\item Tensor Field Theory (6.1 to 6.5)
\end{enumerate}

\item J.D. Jackson -- Classical Electrodynamics, 3rd Edition
\begin{enumerate}
\setcounter{enumi}{12}
\item Collisions, Energy Loss and Scattering of Charged Particles, Cherenkov and Transition Radiation
\item Radiation by Moving Charges
\item Bremsstrahlung, Method of Virtual Quanta, Radiative Beta Processes
\item Radiation Damping, Classical Models of Charged Particles
\end{enumerate}

\item B. Felsager -- Geometry Particles and Fields
% Contacts with quantum theory of fields dynamics + differential geometry math
\begin{enumerate}
\setcounter{enumi}{2}
\item Dynamics of Classical Fields
\end{enumerate}

\begin{enumerate}
\setcounter{enumi}{5}
\item Differentiable Manifolds, Tensor analysis
\item Differential Forms, Exterior Calculus
\item Integral Calculus on Manifolds
\end{enumerate}

\item C.W. Misner, K.S. Thorne, J.A. Wheeler -- Gravitation
\begin{enumerate}
\setcounter{enumi}{1}
\item Foundations of Special Relativity
\item The Electromagnetic Field
\item Electromagnetism and Differential Forms
\end{enumerate}

\item L.D. Landau, E.M. Lifshitz -- Teoria dei Campi
\begin{enumerate}
\setcounter{enumi}{0}
\item Principio di Relatività
\item Meccanica Relativistica
\item Carica in un Campo Elettromagnetico
\item Equazioni del Campo Elettromagnetico
\item Campo Elettromagnetico Costante
\item Onde Elettromagnetiche
\item Propagazione della Luce
\item Campo di Cariche in Moto
\item Radiazione Elettromagnetica
\end{enumerate}

\item L.D. Landau, E.M. Lifshitz -- Elettrodinamica dei Mezzi Continui
\begin{enumerate}
\setcounter{enumi}{0}
\item Elettrostatica dei Conduttori
\item Elettrostatica nei Dielettrici
\item Corrente Continua
\item Campo Magnetico Costante
\item Ferromgnetismo e Antiferromagnetismo
\item Superconduttività
\item Campo Magnetico Quasi Stazionario
\item Idrodinamica Magnetica
\item Equazioni delle Onde Elettromagnetiche
\item Propagazione delle Onde Elettromagnetiche
\item Onde Elettromagnetiche in Mezzi Anisotropi
\item Dispersione Spaziale
\item Ottica non Lineare
\item Passaggio delle Particelle Veloci attraverso la Materia
\item Diffusione delle Onde Elettromagnetiche
\item Diffrazione dei Raggi X nei Cristalli
\end{enumerate}

\end{itemize}
	
\setcounter{tocdepth}{1}	% Must appear BEFORE \tableofcontents!
\tableofcontents
%\addappheadtotoc

\mainmatter%%%%%%%%%%%%%%%%%%%%%%%%%%%%%%%%%%%%%%%%%%%%%%%%%%%%%%%
%\setcounter{chapter}{-1}	% To start with Chapter 0 !!  
%\input{../FANCYBOX}		% Example of boxed/framed parts of text with nice borders

\begin{flushright}
\tit{Quassù tutto vi appare regolato\\
dal sorgere e calare di una stella;\\
Si scambiano il pensiero con dei suoni,\\
movimenti del viso e delle mani;\\
Cercano l'armonia, ma spesso in guerra;\\
Eppur vorrei restarci sulla Terra!} 
\end{flushright} 

\counterwithin{equation}{chapter}	% Equation numbering will be A.1 A.2   A.3   ... 


\renewcommand{\thechapter}{I}
\chapter{Introduction and Survey}
\setcounter{equation}{0}	        % To start with Equation I.1
\counterwithin{equation}{chapter}	% Equation numbering will be I.1 I.2   I.3   ... 

\section{Maxwell Equations in Vacuum, Fields, and Sources}

\section{Inverse Square Law, or the Mass of the Photon}

\section{Linear Superposition}

\section{Maxwell Equations in Macroscopic Media}

\section{Boundary Conditions at Interfaces Between Different Media}

\section{Some Remarks on Idealizations in Electromagnetism}

\section*{References and Suggested Reading}


   % Introduction and Survey  
\renewcommand{\thechapter}{I}
\chapter{Introduction and Survey}
\setcounter{equation}{0}	        % To start with Equation I.1
\counterwithin{equation}{chapter}	% Equation numbering will be I.1 I.2   I.3   ... 

\section{Maxwell Equations in Vacuum, Fields, and Sources}

\section{Inverse Square Law, or the Mass of the Photon}

\section{Linear Superposition}

\section{Maxwell Equations in Macroscopic Media}

\section{Boundary Conditions at Interfaces Between Different Media}

\section{Some Remarks on Idealizations in Electromagnetism}

\section*{References and Suggested Reading}


   % Introduction to Electrostatics  
\setcounter{chapter}{0}
\renewcommand{\thechapter}{2}
\chapter{Boundary-Value Problems in Electrostatics: I}
\setcounter{equation}{0}	        % To start with Equation 1
\counterwithin{equation}{chapter}	% Equation numbering will be 2.1 2.2 2.3   ... 

\section{Method of Images}

\section{Point Charge in the Presence of a Grounded Conducting Sphere}

\section{Point Charge in the Presence of a Charged, Insulated, Conducting Sphere}

\section{Point Charge Near a Conducting Sphere at Fixed Potential}

\section{Conducting Sphere in a Uniform Electric Field by Method of Images}

\section{Green Function for the Sphere; General Solution for the Potential}

\section{Conducting Sphere with Hemispheres at Different Potentials}

\section{Orthogonal Functions and Expansions}

\section{Separation of Variables; Laplace Equation in Rectangular Coordinates}

\section{A Two-Dimensional Potential Problem; Summation of Fourier Series}

\section{Field and Charge Densities in Two-Dimensional Corners and Along Edges}

\section{Introduction to Finite Element Analysis for Electrostatics}

%===================================================================================

\section*{References and Suggested Reading}


\section*{Problems}

   % Boundary-Value Problems in Electrostatics: I  
\setcounter{chapter}{0}
\renewcommand{\thechapter}{3}
\chapter{Boundary-Value Problems in Electrostatics: II}
\setcounter{equation}{0}	        % To start with Equation 1
\counterwithin{equation}{chapter}	% Equation numbering will be 2.1 2.2 2.3   ... 

\section{Laplace Equation in Spherical Coordinates}

\section{Legendre Equation and Legendre Polynomials}

\section{Boundary-Value Problems with Azimuthal Symmetry}

\section{Behavior of Fields in a  Conical Hole or Near a Sharp Point}

\section{Associated Legendre Functions and the Spherical Harmonics $Y_{l m}(\theta, \phi)$}

\section{Addition Theorem for Spherical Harmonics}

\section{Laplace Equation in Cylindrical Coordinates; Bessel Functions}

\section{Boundary-Value Problems in Cylindrical Coordinates}

\section{Expansion of Green Functions in Cylindrical Coordinates}

\section{Eigenfunction Expansions for Green Functions}

\section{Mixed Boundary Conditions, Conducting Plane with a Circular Hole}

%===================================================================================

\section*{References and Suggested Reading}


\section*{Problems}

   % Boundary-Value Problems in Electrostatics: II
%\setcounter{chapter}{0}
\renewcommand{\thechapter}{4}
\chapter{Multipoles, Electrostatics of Macroscopic Media, Dielectrics}
\setcounter{equation}{0}	        % To start with Equation 1
\counterwithin{equation}{chapter}	% Equation numbering will be 2.1 2.2 2.3   ... 

\section{Multipole Expansion}

\section{Multipole Expansion of the Energy of a Charge Distribution in an External Field}

\section{Elementary Treatment of Electrostatics with Ponderable Media}

\section{Boundary-Value Problems with Dielectrics}

\section{Molecular Polarizability and Electric Susceptibility}

\section{Models of Electric Polarizability}

\section{Electrostatic Energy in Dielectric Media}


%===================================================================================

\section*{References and Suggested Reading}


\section*{Problems}

   % Multipoles, Electrostatics in Dielectric Media: II


% 1 -- Vector Analysis
%\renewcommand{\thechapter}{I}
\chapter{Introduction and Survey}
\setcounter{equation}{0}	        % To start with Equation I.1
\counterwithin{equation}{chapter}	% Equation numbering will be I.1 I.2   I.3   ... 

\section{Maxwell Equations in Vacuum, Fields, and Sources}

\section{Inverse Square Law, or the Mass of the Photon}

\section{Linear Superposition}

\section{Maxwell Equations in Macroscopic Media}

\section{Boundary Conditions at Interfaces Between Different Media}

\section{Some Remarks on Idealizations in Electromagnetism}

\section*{References and Suggested Reading}


  % Material from J.D. Jackson Classical Electrodynamics, 3rd Edition - 
% 2 -- Electrostatics
%\setcounter{chapter}{0}
\renewcommand{\thechapter}{2}
\chapter{Boundary-Value Problems in Electrostatics: I}
\setcounter{equation}{0}	        % To start with Equation 1
\counterwithin{equation}{chapter}	% Equation numbering will be 2.1 2.2 2.3   ... 

\section{Method of Images}

\section{Point Charge in the Presence of a Grounded Conducting Sphere}

\section{Point Charge in the Presence of a Charged, Insulated, Conducting Sphere}

\section{Point Charge Near a Conducting Sphere at Fixed Potential}

\section{Conducting Sphere in a Uniform Electric Field by Method of Images}

\section{Green Function for the Sphere; General Solution for the Potential}

\section{Conducting Sphere with Hemispheres at Different Potentials}

\section{Orthogonal Functions and Expansions}

\section{Separation of Variables; Laplace Equation in Rectangular Coordinates}

\section{A Two-Dimensional Potential Problem; Summation of Fourier Series}

\section{Field and Charge Densities in Two-Dimensional Corners and Along Edges}

\section{Introduction to Finite Element Analysis for Electrostatics}

%===================================================================================

\section*{References and Suggested Reading}


\section*{Problems}

  % Material from J.D. Jackson Classical Electrodynamics, 3rd Edition - 


\appendixpage
% Appendix on Units and Dimensions
\appendix
\chapter*{Appendix on Units and Dimensions}
\renewcommand{\thechapter}{A}
\setcounter{equation}{0}	        % To start with Equation A.1
\counterwithin{equation}{chapter}	% Equation numbering will be A.1 A.2   A.3   ... 

The question of units and dimensions in electricity and magnetism has exercised a great number of physicists and engineers over the years. This situation is in marked contrast with the almost universal agreement on the basic units of length (centimeter or meter), mass (gram or kilogram), and time (mean solar second). The reason perhaps is that mechanical units were defined when the idea of \quotes{absolute} standards was a novel concept (just before 1800), and they were urged on the professional and commercial world by a group of scientific giants (Borda, Laplace, and others). By the time the problem of electromagnetic units arose there were (and still are) many experts. The purpose of this appendix is to add as little and as much light as possible without belaboring the issue.

\section{Units and Dimensions; Basic Units and\\ Derived Units}\label{sec:A.1}

The \tit{arbitrariness}, in the \tit{number} of fundamental units and in the \tit{dimensions} of any physical quantity in terms of those units has been emphasized by Abraham, Planck, Bridgman,\footnote[1]{P.W. Bridgman, \tit{Dimensional Analysis}, Yale University Press,  (1931).} Birge,\footnote[2]{R.T. Birge, \tit{Am. Phys. Teacher} (now \tit{Am. J. Phys.}), $\vb{2}$, 41 (1934); $\vb{3}$, 102 (1935).} and others. The reader interested in units as such will do well to become familiar with the excellent series of articles by Birge.

The desirable features of a system of units in any field are convenience and clarity. For example, theoretical physicists active in relativistic quantum field theory and the theory of elementary particles find it convenient to \tit{choose}
the universal constants such as Planck's quantum of action and the velocity of light in vacuum to be \tit{dimensionless}
and of \tit{unit magnitude}. The resulting system of units (called \quotes{natural} units) has only \tit{one} basic unit, customarily chosen to be length. All quantities, whether length or time or force or energy, etc., are expressed in terms of this one unit and have dimensions which are powers of its dimension. There is nothing contrived or less fundamental about such a system than one involving the meter, the kilogram, and the second as basic units. It is merely a matter of convenience.\footnote[3]{In quantum field theory, powers of the coupling constant play the role of other basic units in doing dimensional analysis.} 

One word needs to be said about basic units or standards, considered as independent quantities, and derived units or standards, which are defined in both magnitude and dimension through theory and experiment in terms of the basic units. 

Tradition requires that mass ($m$), length ($l$), and time ($t$) be treated as basic. But for electrical quantities there has been no compelling tradition. Consider, for example, the unit of current. The \quotes{international} ampere (for a long period the accepted practical unit of current) is defined in terms of the mass of silver deposited per unit time by electrolysis in a standard silver voltameter. Such a unit of current is properly considered a basic unit, independent of the mass, length, and time units, since the amount of current serving as the unit is found from a supposedly reproducible experiment in electrolysis. 

On the other hand, the presently accepted standard of current, the \quotes{absolute} ampere \quotes{is that constant current which, if maintained in two straight parallel conductors of infinite length, of negligible circular cross section, and placed one meter apart in vacuum, would produce between these conductors a force equal to $2\,\cdot\,10^{-7}$ newton per metre of length.}
This means that the \quotes{absolute} ampere is a derived unit, since its definition is in terms of the mechanical force between two wires through equation (\ref{eq:A.4}) below.\footnote[1]{The proportionality constant $k_2$ in (\ref{eq:A.4}) is thereby given the magnitude $k_2 = 10^{-7}$ in the SI system.} The \quotes{absolute} ampere is, by this definition, exactly one-tenth of the em unit of current, the abampere.

Since 1948 the international accepted system of electromagnetic standards has been based on the meter, the kilogram, the second, and the above definition of the absolute ampere plus other derived units for resistance, voltage, etc. This seems a desirable state of affairs. It avoids such difficulties as arose when in 1894, by act of Congress (based on recommendations of an international commission of engineers and scientists), independent basic units of current, voltage, and resistance were defined in terms of three independent experiments (silver voltameter, Clark standard cell, specified column of mercury).\footnote[2]{See, for example, F. A. Laws, \tit{Electrical Measurements}, McGraw-Hill, (1917), pp. 705-706.} Soon afterwards, because of systematic errors in the experiments outside the claimed accuracy, Ohm's law was no longer valid, by act of Congress!
 
The Système International d'Unités (SI) has the unit of mass defined since 1889 by a platinum-iridium \tit{kilogram} prototype kept in Sevres, France. In 1967 the SI \tit{second} was defined to be \quotes{the duration of 9 192 631 770 periods of the radiation corresponding to the transition between the two hyperfine levels of the ground state of the cesium-133 atom.} The General Conference on Weights and Measures in 1983 adopted a definition of the \tit{meter} based on the speed of light, namely the \tit{meter} is \quotes{the length of the distance traveled in vacuum by light during a time 1/299 792 458 of a second.} The speed of light is therefore no longer an experimental number; it is, by definition of the meter, exactly $c = 299 792 458$ m/s. For electricity and magnetism, the Système International adds the absolute ampere as an additional unit, as already noted. In practice, metrology laboratories around the world define the ampere through the units of electromotive force, the volt, and resistance, the ohm, as determined experimentally from the Josephson effect ($2e/h$) and the quantum Hall effect ($h/e^2$), respectively.\footnote[3]{For a general discussion of SI units in electricity and magnetism and the use of quantum phenomena to define standards, see B. W. Petley, in \tit{Metrology at the Frontiers of Physics and Technology}, eds. L. Corvini and T.J. Quinn, Proceedings of the International School of Physics \quotes{Enrico Fermi,} Course CX, 27 June--7 July 1989, North-Holland, (1992), pp. 33-61.}

\section{Electromagnetic Units and Equations}\label{sec:A.2}

In discussing the units and dimensions of electromagnetism we will take as our starting point the traditional choice of length ($l$), mass ($m$) and time ($t$) as independent, basic dimensions. Furthermore, we will make the commonly accepted definition of current as the time rate of change of charge ($I = \dd{q}/\dd{t}$). This means that the dimension of the ratio of charge and current is that of time.\footnote[1]{From the point of view of special relativity it would be more natural to give current the dimension of charge divided by length. Then current density $J$ and charge density $\rho$ would have the same dimensions and would form a \quotes{natural} $4$-vector. This is the choice made in a modified Gaussian system.} The continuity equation for charge and current densities then takes the form:
\begin{equation}\label{eq:A.1}
\div \vb{J} + \pdv{\rho}{t} = 0
\end{equation}

To simplify matters we initially consider only electromagnetic phenomena in free space, apart from the presence of charges and currents.

The basic physical law governing electrostatics is Coulomb's law on the force between two point charges $q$ and $q'$, separated by a distance $r$. In symbols this law is
\begin{equation}\label{eq:A.2}
F_1 = k_1 \frac{q\,q'}{r^2}
\end{equation}

The constant $k_1$ is a proportionality constant whose magnitude and dimensions \tit{either} are determined by the equation (if the magnitude and dimensions of the unit of charge have been specified independently) \tit{or} are chosen arbitrarily in order to define the unit of charge. Within our present framework all that is determined at the moment is that the product ($k_1 q q'$) has the dimensions\footnote{[$F$] = $m\,l\,t^{-2}$ =  [$k_1 q q' r^{-2}$], hence [$k_1 q q'$] = $m\,l^3\,t^{-2}$. } ($m l^3 t^{-2}$). 
The electric field $\vb{E}$ is a derived quantity, customarily defined to be the force per unit charge. A more general definition would be that the electric field be numerically proportional to the force per unit charge, with a proportionality constant perhaps having dimensions such that the electric field id dimensionally different from force per unit charge. There is, however, nothing to be gained by this extra freedom in the definition of $\vb{E}$, since $\vb{E}$ is the first derived field quantity to be defined. Only when we define other field quantities may it be convenient to insert dimensional proportionality constants in the definitions in order to adjust the dimensions and magnitude of these fields relative to the electric field. Consequently, with no significant loss of generality the electric field of a point charge $q$ may be defined from (\ref{eq:A.2}) as the force per unit charge, 
\begin{equation}\label{eq:A.3}
F_1 = k_1 \frac{q}{r^2}
\end{equation}
All systems of units known to the author use this definition of electric field.

For steady-state magnetic phenomena Ampère's observations form a basis for specifying the interaction and defining the magnetic induction. According to Ampère, the force per unit length between two infinitely long, parallel wires separated by a distance $d$ and carrying currents $I$ and $I'$ is
\begin{equation}\label{eq:A.4}
\frac{\dd{F_2}}{\dd{l}} = 2 k_2 \frac{I\,I'}{d}
\end{equation}
The constant $k_2$ is a proportionality constant akin to $k_1$ in (\ref{eq:A.2}). The dimensionless number $2$ is inserted in (\ref{eq:A.4}) for later convenience in specifying $k_2$. Because of our choice of the dimensions of current and charge embodied in (\ref{eq:A.1}), the dimensions of $k_2$ relative to $k_1$ are determined. From (\ref{eq:A.2}) and (\ref{eq:A.4}) it is easily found that the ratio $k_1/k_2$ 
has the dimension of a velocity squared ($l^2\,t^{-2}$), Furthermore, by comparison of the magnitude of the two mechanical forces (\ref{eq:A.2}) and (\ref{eq:A.4}) for known charges and currents, the magnitude of the ratio $k_1/k_2$ in free space can be found. The numerical value is closely given by the square of the velocity of light in vacuum. Therefore in symbols we can write
\begin{equation}\label{eq:A.5}
\frac{k_1}{k_2} = c^2
\end{equation}
where $c$ stands for the velocity of light in magnitude and dimensions.

The magnetic induction $\vb{B}$ is derived from the force laws of Ampère as being numerically proportional to the force per unit current with a proportionality constant $\alpha$ that may have certain dimensions chosen for convenience. Thus for a long straight wire carrying a current $I$, the magnetic induction $\vb{B}$ at a distance $d$ has the magnitude (and dimensions)
\begin{equation}\label{eq:A.6}
B = 2 k_2 \alpha \frac{I}{d}
\end{equation}

The dimensions of the ratio of electric field to magnetic induction can be found from (\ref{eq:A.1}) (\ref{eq:A.3}) (\ref{eq:A.5}), and (\ref{eq:A.6}). The result is that ($E/B$) has the dimension ($l/t\alpha$). 

The third and final relation in the specification of electromagnetic units and dimensions is Faraday's law of induction, which connects electric and magnetic phenomena. The observed law that the electromotive force induced around a circuit is proportional to the rate of change of magnetic flux through it takes on the differential form
\begin{equation}\label{eq:A.7}
\curl \vb{E} + k_3 \pdv{\vb{B}}{t} = 0
\end{equation}
where $k_3$ is a constant of proportionality. Since the dimensions of $\vb{E}$ relative to $\vb{B}$ are established, the dimensions of $k_3$ can be expressed in terms of previously defined quantities merely by demanding that both terms in (\ref{eq:A.7}) have the same dimensions. Then it is found that $k_3$ has dimensions of $\alpha^{-1}$. Actually, $k_3$ is \tit{equal} to $\alpha^{-1}$. This is established on the basis of Galilean invariance\footnote{For this, see section 6.1 in Jackson's 2nd edition, or section 5.15 in Jackson's 3rd edition.}. But the easiest way to prove the equality is to write all the Maxwell equations in terms of the fields defined here:
\begin{equation}\label{eq:A.8}
\begin{aligned}
\div  \vb{E} &= 4\pi\, k_1\, \rho\\
\curl \vb{B} &= 4\pi\, k_2\, \alpha\, \, \vb{J}\:+\:\frac{k_2 \alpha}{k_1} \pdv{\vb{E}}{t}\\
\curl \vb{E} + k_3\, \pdv{\vb{B}}{t} &= 0\\
\div  \vb{B} &= 0
\end{aligned}
\end{equation}

Then for source-free regions the two curl equations can be combined into the wave equation,
\begin{equation}\label{eq:A.9}
\laplacian \vb{B} - k_3 \:\frac{k_2 \alpha}{k_1}\, \pdv[2]{\vb{B}}{t} = 0
\end{equation}

The velocity of propagation of the waves described by (\ref{eq:A.9}) is related to the combination of constants appearing there. Since this velocity is known to be that of light, we may write
\begin{equation}\label{eq:A.10}
\frac{k_1}{\alpha}{k_3\,k_2\,\alpha} = c^2
\end{equation}

Combining (\ref{eq:A.5}) with (\ref{eq:A.10}), we find
\begin{equation}\label{eq:A.11}
k_3 = \frac{1}{\alpha}
\end{equation}
an equality holding for both magnitude and dimensions.








 





  % Appendix A from J.D. Jackson Classical Electrodynamics, 3rd Edition




\backmatter%%%%%%%%%%%%%%%%%%%%%%%%%%%%%%%%%%%%%%%%%%%%%%%%%%%%%%%
%%%%%%%%%%%%%%%%%%%%%%%%% referenc.tex %%%%%%%%%%%%%%%%%%%%%%%%%%%%%%
% sample references
% 
% Use this file as a template for your own input.
%
%%%%%%%%%%%%%%%%%%%%%%%% Springer-Verlag %%%%%%%%%%%%%%%%%%%%%%%%%%

%
% BibTeX users please use
% \bibliographystyle{}
% \bibliography{}
%
% Non-BibTeX users please use
\begin{thebibliography}{99.}
%
% and use \bibitem to create references.
%
% Use the following syntax and markup for your references
%
% Monograph
\bibitem{Griffiths_4th} D.J. Griffiths (2017)
Introduction to Electrodynamics. Cambridge University Press, Cambridge

% Monograph
\bibitem{Felsager_1981} B. Felsager (1981)
Geometry, Particles and Fields. Odense University Press

% Monograph
\bibitem{BudakFomin_1973} B.M. Budak, S.V. Fomin (1973)
Multiple Integrals, Field Theory and Series. Mir Publishers, Moscow

% Monograph
\bibitem{Postnikov_II_1982} Mikhail Postnikov (1982)
Lectures in Geometry, Semester II. Linear Algebra and Differential Geometry. Mir Publishers, Moscow

\end{thebibliography}

\printindex

%%%%%%%%%%%%%%%%%%%%%%%%%%%%%%%%%%%%%%%%%%%%%%%%%%%%%%%%%%%%%%%%%%%%%%

\end{document}





