\chapter{Felsager -- Electromagnetism}
\label{ch:Felsager_01} 

\section{The Electromagnetic Field}

The two fundamental quantities of the electromagnetic field are the field strengths

\begin{equation*}
\vb{E}(\vb{r}, t); \:\:\: \vb{B}(\vb{r}, t)
\end{equation*}

$\vb{E}$ is an ordinary (i.e. \textit{polar}) vector field, and as such it can be defined before any choice of a reference system. Instead, $\vb{B}$ is an \textit{axial} vector field and thus it can only be specified with respect to a given reference system. Reference systems can be separated in two \textit{equivalence} classes (a.k.a \textit{orientations}), according to the sign (positive or negative) of the determinant associated to the transformation between two such systems. 

Both polar and axial vectors transform in the same way under spatial \textit{rotations} (a rotation connects two systems with the same orientation). Different behaviors arise when dealing with a \textit{reflection} of the three coordinate axes. This one connects two systems with opposite orientations. Polar vectors are not affected at all by this transformation while any axial vector turns into its opposite.

A large set of experimental data can be interpreted as the manifestation of the \textit{Lorentz} force, defining the effects of the fields $\vb{E}$ and $\vb{B}$ on a particle possessing a \textit{charge} $q$

\begin{equation}
\vb{F}(\vb{r}, t) = q \left(\vb{E}(\vb{r}, t) + \frac{1}{c} \vb{v} \cross \vb{B}(\vb{r}, t) \right)
\label{eq:Lorentzforce1}
\end{equation}

It should be noted that $\vb{v} \cross \vb{B}$ is an ordinary (\textit{polar}) vector because $\vb{B}$ is an axial vector and $\vb{v}$ is an ordinary vector. 

According to special relativity the particle's motion obeys the following equations 

\begin{equation}
\dv{\vb{p}}{t} = q \left(\vb{E} + \frac{1}{c} \vb{v} \cross \vb{B} \right) ; \:\: \vb{p} = \gamma m \vb{v} ; \:\: \gamma = \frac{1}{\sqrt{1 - \frac{v^2}{c^2}}}
\label{eq:Lorentzforce2}
\end{equation}

This relation is of extreme importance and it actually serves to define the electromagnetic field. The argument is that the charge being small allows one to neglect the influence of the particle on the electromagnetic field\footnote{The field generated by the charge is certainly very large in its proximity, but let assume for now that a charge is not affected by its own field.}, so that by analyzing the motion of a swarm of particles through the electromagnetic field one can determine the field strengths $\vb{E}$ and $\vb{B}$.

In 1860 Maxwell derived the equations governing the motion of the electromagnetic field:

\begin{subequations}
\label{eq:Maxwell}
\begin{align}
\div{\vb{B}} &= 0 \label{eq:maxwell_divB}\\
\curl{\vb{E}} + \frac{1}{c} \pdv{\vb{B}}{t} &= 0 \label{eq:maxwell_rotE}\\
\div{\vb{E}} &= 4 \pi \rho \label{eq:maxwell_divE}\\
\curl{\vb{B}} &= \frac{4 \pi}{c} \vb{J} + \frac{1}{c} \pdv{\vb{E}}{t} \label{eq:maxwell_rotB}
\end{align}
\end{subequations}

where $\rho = \rho(\vb{r}, t)$ is the \textit{charge}  density, $\vb{J} = \rho \vb{v}$  the \textit{current}  and $c$ a constant equal to the speed of light in empty space. 

\subsection*{Vector calculus}
%Some formulas of vector calculus are reproduced here to help in the following sections
 
\begin{subequations}
\label{eq:vcalc}
\begin{align}
\vb{A} \cross \vb{B} &= - \vb{B} \cross \vb{A} \label{eq:vcalc_a}\\
\vb{A} \vdot (\vb{B} \cross \vb{C}) &= (\vb{A} \cross \vb{B}) \vdot \vb{C}) \label{eq:vcalc_b}\\
\vb{A} \cross (\vb{B} \cross \vb{C}) &= \vb{B} (\vb{A} \vdot \vb{C}) - (\vb{A} \vdot \vb{B})\vb{C} \label{eq:vcalc_c}\\
(\vb{A} \cross \vb{B}) \cross \vb{C} &= (\vb{A} \vdot \vb{C}) \vb{B} - \vb{A} (\vb{B} \vdot \vb{C}) \label{eq:vcalc_d}\\
\div{(\grad{\phi})} &= \laplacian{\phi} \label{eq:vcalc_e}\\
\curl{(\grad{\phi})} &= \vb{0} \label{eq:vcalc_f}\\
\div{(\curl{\vb{B}})} &= 0 \label{eq:vcalc_g}\\
\curl{(\curl{\vb{B}})} &= \grad(\div{\vb{B}}) - \laplacian{\vb{B}}  \label{eq:vcalc_h}
\end{align}
\end{subequations}

\textbf{Gauss' theorem}

\begin{equation}
\int\limits_{\Omega}\div{\vb{E}} \; \mathrm{d}V = \int\limits_{S}\vb{E} \vdot \vu{n} \; \mathrm{d}A \label{eq:Gauss}
\end{equation}

\textbf{Stokes' theorem}

\begin{equation}
\int\limits_{S}(\curl{\vb{B}}) \vdot \vu{n} \; \mathrm{d}A = \oint\limits_{\Gamma} \vb{B} \vdot \mathrm{d}\vb{r}  \label{eq:Stokes}
\end{equation}

\textbf{Theorem of line-integrals}\\
The integral of the gradient of a scalar function $\phi$ between points $P$ and $Q$ along a curve $\Gamma$ is
  
\begin{equation*}
\int\limits_{\Gamma} \grad{\phi} \vdot \mathrm{d}\vb{r} = \phi(Q) - \phi(P)  \label{eq:LineInt}
\end{equation*}


\subsection*{Charge conservation}
By taking the divergence of \ref{eq:maxwell_rotB} and considering calculus rule \ref{eq:vcalc_g}

\begin{align*}
0 &= \div{(\curl{\vb{B}})} = \frac{4 \pi}{c} \div{\vb{J}} + \frac{1}{c} \pdv{\div{\vb{E}}}{t}
\end{align*}

In force of equation \ref{eq:maxwell_divE} the divergence appearing in the last term can be replaced by $4\pi\rho$ yelding the so called \textit{equation of continuity}  

\begin{equation}
\pdv{\rho}{t} + \div{\vb{J}}= 0 \label{eq:continuity}
\end{equation}

The physical significance of the above equation becomes more evident by applying Gauss' theorem to a control volume $\Omega$ to determine the rate of change of the total charge $Q$ contained inside that volume. Indeed, by definition the total charge located within the control volume at a given point in time is

\begin{equation*}
Q(t) = \int\limits_{\Omega}\rho(t) \; \mathrm{d}V 
\end{equation*}

while the istantaneous \textit{flux} of charge (charge entering or leaving the control volume in a unit of time) is given by the surface integral

\begin{equation*}
\int\limits_{S}\rho \,\vb{v} \vdot \vu{n} \; \mathrm{d}A = \int\limits_{S} \vb{J} \vdot \vu{n} \; \mathrm{d}A 
\end{equation*}

The continuity equation \ref{eq:continuity} just links, via the Gauss' theorem, the total flux of charge to the rate of change of $Q$  

\begin{equation*}
\frac{\mathrm{d} \, Q(t)}{\mathrm{d} \, t}  = \int\limits_{\Omega} \pdv{\rho}{t} \; \mathrm{d}V = - \int\limits_{S}\vb{J} \vdot \vu{n} \; \mathrm{d}A 
\end{equation*}

This result is equivalent to the assertion that a decrease[increase] of the total charge in the control volume $V$ by an amount $\mathrm{d}Q$ does always occur with an equal amount of charge flowing across the boundary toward the exterior[interior] of the control volume during the same interval of time. 

In short, electric charge is not created nor destroyed by electromagnetic processes. 

\section{Gauge Potentials}

Identities \ref{eq:vcalc_f} and \ref{eq:vcalc_g} from vector calculus can be exploited to express the fields $\vb{E}$ and $\vb{B}$ in terms of a \textit{scalar} field $\phi$ and a \textit{vector} field $\vb{A}$, thus causing the \textit{homogeneous} Maxwell equations \ref{eq:maxwell_divB} and \ref{eq:maxwell_rotE} to be automatically satisfied. 

Indeed, by equating $\vb{B}$ to the curl of a \textit{vector} field $\vb{A}$ the divergence of $\vb{B}$ does necessarily vanish, as required by \ref{eq:maxwell_divB}:

\begin{equation}
\vb{B} = \curl{\vb{A}} \:\: \longrightarrow \:\: \div{\vb{B}} = \div{(\curl{\vb{A}})} = 0 \label{eq:Bdef}
\end{equation}
   
Once $\vb{B}$ satisfies \ref{eq:Bdef}, the companion Maxwell equation \ref{eq:maxwell_rotE} reads:

\begin{equation*}
\curl{\vb{E}} + \frac{1}{c} \pdv{\vb{B}}{t} = \curl{\left( \vb{E} + \frac{1}{c} \pdv{\vb{A}}{t} \right)} = 0 
\end{equation*}
 
Therefore, by equating the argument of the curl operator in the above equation to the \textit{gradient} of a \textit{scalar} field $\phi$ the Maxwell equation  \ref{eq:maxwell_rotE} is automatically satisfied:

\begin{equation}
\left( \vb{E} + \frac{1}{c} \pdv{\vb{A}}{t} \right) = - \grad{\phi} \:\: \longrightarrow \:\: \curl{\vb{E}} + \frac{1}{c} \pdv{\vb{B}}{t} = - \curl{(\grad{\phi})} = \vb{0} \label{eq:Edef}
\end{equation}

What we have demonstrated so far is that the fields can be conveniently expressed in terms of the potentials $\phi$ and $\vb{A}$ as follows

\begin{subequations}
\label{eq:potentials_1}
\begin{align}
\vb{B} &= \curl{\vb{A}} \label{eq:potentials_11} \\
\vb{E} &= - \grad{\phi} - \frac{1}{c} \pdv{\vb{A}}{t} \label{eq:potentials_12} 
\end{align}
\end{subequations}

It is important to note that the equations \ref{eq:potentials_1} do not restrict the range of admissible solutions to the Maxwell equations \ref{eq:Maxwell}\footnote{Helmoltz theorem states that any vector field can be decomposed into a field with zero divergence and a field with zero curl, where the latter is the gradient of a scalar field and the former is the curl of a vector field.}. 

By substituting the equations \ref{eq:potentials_1} in the two \textit{inhomogeneous} Maxwell equations \ref{eq:maxwell_divE} and \ref{eq:maxwell_rotB} we get the potential dependency from \textit{sources}: 

\begin{align*}
- \div{\vb{E}} &= - 4 \pi \rho \\
			&= \laplacian{\phi} + \frac{1}{c} \pdv{\div{\vb{A}}}{t} \\
- \curl{\vb{B}} &= - \curl{(\curl{\vb{A}})} \\
			&= \laplacian{\vb{A}} - \grad{(\div{\vb{A})}} \\
			&= - \frac{4 \pi}{c} \vb{J} - \frac{1}{c} \pdv{\vb{E}}{t} \\
			&= - \frac{4 \pi}{c} \vb{J} + \frac{1}{c} \pdv{}{t} \left( \grad{\phi} + \frac{1}{c} \pdv{\vb{A}}{t}   \right)	
\end{align*}

Terms in the above equations can be rearranged to give 
\begin{align*}
\Box \, \phi &= \laplacian{\phi} - \frac{1}{c^2} \pdv[2]{\phi}{t} \\
			&= - 4 \pi \rho - \frac{1}{c} \pdv{\div{\vb{A}}}{t} - \frac{1}{c^2} \pdv[2]{\phi}{t} \\
			&= - 4 \pi \rho - \frac{1}{c} \pdv{}{t} \left( \div{\vb{A}} + \frac{1}{c} \pdv{\phi}{t}  \right) \\
\Box \, \vb{A} &= \laplacian{\vb{A}} - \frac{1}{c^2} \pdv[2]{\vb{A}}{t} \\
			&= - \frac{4 \pi}{c} \vb{J} + \frac{1}{c} \pdv{}{t} \left( \grad{\phi} + \frac{1}{c} \pdv{\vb{A}}{t}   \right) + \grad{(\div{\vb{A})}} - \frac{1}{c^2} \pdv[2]{\vb{A}}{t} \\
			&= - \frac{4 \pi}{c} \vb{J} + \frac{1}{c} \pdv{}{t} \left( \grad{\phi} \right) + \grad{(\div{\vb{A})}} \\
			&= - \frac{4 \pi}{c} \vb{J} + \grad{\left( \div{\vb{A}} + \frac{1}{c} \pdv{\phi}{t}  \right)}
\end{align*}

The above tour de force is summarized in the following two equations:
 
\begin{subequations}
\label{eq:potentials_2}
\begin{align} 
\Box \, \phi &= - 4 \pi \rho - \frac{1}{c} \pdv{}{t} \left( \div{\vb{A}} + \frac{1}{c} \pdv{\phi}{t}  \right)
\label{eq:potentials_21} \\
\Box \, \vb{A} &= - \frac{4 \pi}{c} \vb{J} + \:\:\grad{\left( \div{\vb{A}} + \frac{1}{c} \pdv{\phi}{t}  \right)}
\label{eq:potentials_22} 
\end{align}
\end{subequations}

The term appearing between parentheses causes an unwanted coupling of the two equations, yet there is some freedom left in the choice of potentials to get rid of this term. Indeed, let $\phi$ and $\vb{A}$ be solutions of \ref{eq:potentials_2} and let $\vb{B}$ and $\vb{E}$ be the field strengths obtained via \ref{eq:potentials_1}

\begin{align*}
\vb{B} &= \curl{\vb{A}} \\
\vb{E} &= - \grad{\phi} - \frac{1}{c} \pdv{\vb{A}}{t} 
\end{align*}

The following transformation, where $\rchi(\vb{r}, t)$ is an arbitrary scalar field

\begin{equation}
\begin{aligned}
\phi & \longrightarrow \phi' = \phi - \frac{1}{c} \pdv{\rchi}{t} \\
\vb{A} & \longrightarrow \vb{A}' = \vb{A} + \grad{\rchi}
\end{aligned}
\label{eq:f01_gaugexform}
\end{equation}

has the property of leaving the \textit{field strengths}  $\vb{B}$ and $\vb{E}$ unchanged:

\begin{align*}
\vb{B}' &= \curl{\vb{A}'} = \curl{(\vb{A} + \grad{\rchi})} &= \vb{B} \\    
\vb{E}' &= - \grad{\phi'} - \frac{1}{c} \pdv{\vb{A}'}{t} =  \ldots  &= \vb{E}
\end{align*}

As a further step, we show that a suitable choice of the scalar field $\rchi$ exists for which the coupling term 

\begin{equation*}
K(\phi, \vb{A}) \equiv \left( \div{\vb{A}} + \frac{1}{c} \pdv{\phi}{t}  \right)
\end{equation*}

in \ref{eq:potentials_2} is identically zero. Indeed, the transformation \ref{eq:f01_gaugexform} causes the coupling term to change by minus the D'Alembertian of $\rchi$: 

\begin{align*}
K(\phi', \vb{A}') = K(\phi, \vb{A}) + \Box \rchi
\end{align*}

Therefore, requiring $K(\phi', \vb{A}')$ to vanish, amounts to finding a scalar field $\rchi$ which obeys the following PDE

\begin{equation}
\Box \rchi = - K(\phi, \vb{A}) = - \left( \div{\vb{A}} + \frac{1}{c} \pdv{\phi}{t}  \right)
\end{equation}

One never needs to solve the above PDE: the whole point in the preceding discussion was to demonstrate that for any pair of potentials $(\phi, \vb{A})$ satisfying the equations \ref{eq:potentials_2} there is a corresponding pair $(\phi', \vb{A}')$ that produces identical field strengths $\vb{B}$ and $\vb{E}$ and that can be obtained by solving the decoupled system

\begin{equation}
\begin{aligned}
\Box \;\phi' &= - 4\pi\rho \\
\Box \vb{A}' &= - \frac{4 \pi}{c} \vb{J}
\end{aligned}
\label{eq:f01_decoupled}
\end{equation}

which consists of two \textit{wave equations} that are solvable by standard techniques.

The system \ref{eq:f01_decoupled} still admits an \textit{infinity of solutions} because applying to any solution $(\phi, \vb{A})$ the \textit{gauge transformation} \ref{eq:f01_gaugexform} with $\rchi$ a non-vanishing solution of the homogeneous wave equation $\Box \rchi =0$ yelds a distinct $(\phi', \vb{A}')$ pair, that is still a valid solution as it produces the exact same field strengths.

The gauge condition required to arrive at the decoupled system is called the \textit{Lorenz gauge}\footnote{L.V. Lorenz (1829-91) is a Danish physicist who developed a theory of electromagnetic waves independently of Maxwell.} 
\begin{equation}
\left( \div{\vb{A}} + \frac{1}{c} \pdv{\phi}{t}  \right) = 0
\label{eq:f01_lorenzgauge}
\end{equation}

and we observe that it has the same form as the equation of continuity \ref{eq:continuity}. It is indeed common practice, in gauge theories, to let the gauge condition resemble the characteristic equation of the source. 

\section{Field of a Point Charge} 
We want to deduce the field generated by a point-like charged particle at rest. 

For lack of any other privileged direction the field $\vb{E}$ at any point $P$ due to a charge $q$ located at the origin $O$ must be directed along the direction of the \textit{radius vector} $\vb{r} = \overline{OP}$. Maxwell equation \ref{eq:maxwell_divE} is then sufficient to derive the result with use of Gauss theorem \ref{eq:Gauss} which relates the integral of $\div{\vb{E}}$ over a volume $V$ enclosing the particle to the flux of $\vb{E}$ over the volume's bounding surface $S$. \\

Indeed, let $\Omega$ be a spherical volume centered on the origin where also lies the center of the spherically symmetric, sharply peaked charge density distribution of a particle so that 

\begin{equation}
q = \int\limits_{\Omega} \rho(r) \; \mathrm{d}V 
\end{equation}

is the total charge possessed by the particle. This integral can be transformed by the Gauss theorem into the surface integral 

\begin{equation}
4 \pi q = \int\limits_{\Omega}\div{\vb{E}} \; \mathrm{d}V = \int\limits_{S}\vb{E} \vdot \vu{n} \; \mathrm{d}A = 4 \pi r^2 E(r) \label{eq:pointcharge_flux}
\end{equation}

where $r$ is the radius of the sphere and $E(r)$ is the strength of the electric field at the bounding surface. By this result we conclude that the electric field associated with a point charge $q$ is directed radially and its strength is
$E(r) = q/r^2$.



\section{Radial fields} 
Let establish some results about \textit{radial} fields, i.e those taking the form 
%\footnote{This section was inserted to establish some useful results before tackling the successive section of Felsager's book.} 

\begin{equation*}
\vb{v}(\vb{r}) = f(r) \hat{\vb{r}} 
\end{equation*}

where $\hat{\vb{r}}$  is a unit vector having the same direction as the vector $\vb{r}$, the latter denoting the position associated to the field value $\vb{v}(\vb{r})$.

\begin{subequations}
\label{eq:radialfields_A}
\begin{align} 
r \equiv \abs{\vb{r}} &= \sqrt{x^2 + y^2 + z^2}
\label{eq:radialfields_1} \\
\grad{r} &= \frac{\vb{r}}{r} = \hat{\vb{r}}
\label{eq:radialfields_2} \\
\grad{r^2} &= 2 r \grad{r} = 2r \frac{\vb{r}}{r} = 2 \vb{r}
\label{eq:radialfields_3} \\
\grad{\left( \frac{1}{r} \right)} &= - \frac{1}{r^2} \grad{r} = - \frac{\hat{\vb{r}}}{r^2} = - \frac{\vb{r}}{r^3}  
\label{eq:radialfields_4} \\
\grad{\left( \frac{1}{r^n} \right)} &= - \frac{n}{r^{n+1}} \grad{r} = - n \frac{\hat{\vb{r}}}{r^{n+1}} = - n \frac{\vb{r}}{r^{n+2}}
\label{eq:radialfields_5} \\
\div{\vb{r}}  = \div{\left( x \hat{\vb{i}}, y \hat{\vb{j}}, z \hat{\vb{k}} \right)} &= 1 + 1 + 1 = 3
\label{eq:radialfields_6} \\
\div{\hat{\vb{r}}}  = \div{\left(\frac{\hat{\vb{i}}}{\sqrt{3}}, \frac{\hat{\vb{j}}}{\sqrt{3}}, \frac{\hat{\vb{k}}}{\sqrt{3}} \right)}  &= 0 + 0 + 0 = 0
\label{eq:radialfields_7}
\end{align}
\end{subequations}

\begin{equation}
\begin{aligned} 
\div{\vb{A}} \equiv \div{\left[ f(r) \, \hat{\vb{r}} \right]} &= \div{\left[ h(r) \, \vb{r} \right]}; \; h(r) \equiv \frac{f(r)}{r}\\
&= \sum\limits_i{ \pdv{{\left[ h(r) \, \vb{r}_i \right]}}{x_i} } 
 = \sum\limits_i{ \left[ h \, \pdv{\vb{r}_i}{x_i} + \vb{r}_i \, \dv{h}{r} \pdv{r}{x_i} \right]}
\label{eq:radialfields_divergence_A}
\end{aligned}
\end{equation}

Equation \ref{eq:radialfields_divergence_A} can be simplified by use of the following identities 

\begin{equation*}
\vb{r}_i = x_i ; \:\: \pdv{\vb{r}_i}{x_i} = 1 ;  \:\: \pdv{r}{x_i} = \frac{x_i}{r};  \:\: \dv{h}{r} = \frac{f'}{r}
- \frac{f}{r^2}
\end{equation*}

whence 
\begin{equation}
\begin{aligned} 
\div{\vb{A}} \equiv \div{\left[ f(r) \, \hat{\vb{r}} \right]} &= 
\sum\limits_i{ \left[ h + \dv{h}{r} \frac{x_i^2}{r} \right]} = 3h + r \, \dv{h}{r}  \\
&= \frac{2 \, f}{r} + f' = \frac{1}{r^2} \pdv{\left( r^2 f \right) }{r}
\label{eq:radialfields_divergence_B}
\end{aligned}
\end{equation}

where the last expression coincides with the known formula for the contribution of the radial term to the divergence when the field components are given in \textit{spherical coordinates}. 

For a radial field whose intensity is proportional to $1/r^2$ this expression is identically zero everywhere except at the origin $r=0$ where the expression becomes undefined. Therefore, the flux of an \textit{inverse-square law} electric field $\vb{E}$ over any closed surface with no charge inside, is identically zero in force of Gauss theorem. This is easy to visualize when considering the volume between any two concentrical, spherical shells where the field is generated by a charge located at their center, because the inverse square-law causes the flux contributions at the two spherical shells to be equal in magnitude and opposite in sign. It is also easy to convince ourselves that this result holds for volumes of any shape. 

On the other hand, equation \ref{eq:pointcharge_flux} states that the field $\vb{E}$ of a point charge at rest does indeed follow an inverse-square law, and the field total flux of $\vb{E}$ over a surface enclosing a charge $q$ must be $4\pi$ times the charge.

These findings can be put together by adopting the following prescription for the divergence of inverse-square law fields generated by a point source, namely 

\begin{equation}
\div{\left( \frac{\hat{\vb{r}}}{r^2}\right)} = 4 \pi \delta^{3}(\vb{r}) 
\label{eq:pointcharge_div}
\end{equation}

where $\delta^{3}(\vb{r})$ is the three-dimensional \textit{Dirac delta} function.

\section{Magnetic Flux}

Together, Maxwell equation \ref{eq:maxwell_divB} and Gauss theorem require the flux of $\vb{B}$ over a closed surface to vanish.  This is actually true as well of the flux of $\vb{E}$, over the same surface, if there is zero net electric charge $q$ within the bounded volume. In general, in \textit{empty zones} within a \textit{region} of space possibly populated by sparse point-like charges, Maxwell equations reduce to 

\begin{equation}
\begin{aligned}
\div{\vb{B}} &= 0 \\
\div{\vb{E}} &= 0 \\
\curl{\vb{B}} &= + \frac{1}{c} \pdv{\vb{E}}{t} \\
\curl{\vb{E}} &= - \frac{1}{c} \pdv{\vb{B}}{t}
\end{aligned}
\label{eq:emptyMaxwell}
\end{equation}

and we see that under a \textit{steady state} condition (resting charges, steady currents) $\vb{B}$ and $\vb{E}$ obey \textit{identical} partial differential equations, each vector field being required to have \textit{zero curl} and \textit{zero divergence}


\begin{equation}
\begin{aligned}
\div{\vb{B}} &= 0 \:;\:\:\:\: \curl{\vb{B}} = 0 \\
\div{\vb{E}} &= 0 \:;\:\:\:\: \curl{\vb{E}} = 0 
\end{aligned}
\label{eq:emptySteadyMaxwell}
\end{equation}

That both fields are governed by identical equations in zones without sources is a fact that contrasts with the sharp distinction between the scalar potential $\phi$ and the vector potential $\vb{A}$ and the way each field is related to them. In particular, having required the magnetic field $\vb{B}$ to be the curl of $\vb{A}$ implies the flux of $\vb{B}$ over a closed surface to vanish under all conditions, regardless of any sources or field singularity located within the bounded volume. 

Indeed, assuming the bounding surface to be smooth, and the field to be smooth close to the surface, one can use Stokes theorem to demonstrate that the total flux of $\vb{B}$ must be zero\footnote{A closed non-empty curve $\Gamma$ divides the surface in two parts and the flux of $\vb{B}$ over each part can be computed as the line integral of the vector potential $\vb{A}$ over $\Gamma$ but moving in opposite directions for each part.}. 
A similar result holds for the flux of $\vb{E}$ -- due to Gauss theorem -- only when there is no net charge inside.

The whole point of this analysis is that having tied the field $\vb{B}$ to the vector potential $\vb{A}$ it appears not possible to accomodate \textit{magnetic monopoles} (point sources of the magnetic field) into the electromagnetic theory. 

\section{Vector Potential of a Solenoid}

It is a well known result of magnetostatic that a steady uniform current running along the surface of a long cylinder\footnote{A cylinder whose length $L$ is much greater than its diameter $D$.} along a direction orthogonal to its axis, generates a constant magnetic field $\vb{B}_o$ within the cylinder, oriented along the axis. The greater the cylinder aspect ratio ($L/D$) the smaller are disuniformities within the cylinder and the strength of the field outside of it. Here we want to find a vector potential $\vb{A}$ generating the field $\vb{B}_o$, assuming the cylinder axis to run along $\hat{\vb{k}}$, so that 

\begin{equation*}
\vb{B}  = B_o \hat{\vb{k}} = (0, 0, B_o)
\end{equation*}

The following expression, represents a field \textit{circulating} around the cylinder axis
\begin{equation*}
\begin{aligned}
\vb{B} \cross \vb{r} &= B_o \; \hat{\vb{k}} \cross \vb{r} \\
					 &= B_o \; \hat{\vb{k}} \cross \left( x \hat{\vb{i}} + y \hat{\vb{j}} + z \hat{\vb{k}} \right) \\
                     &= B_o \left( x \hat{\vb{k}} \cross \hat{\vb{i}} + y \hat{\vb{k}} \cross \hat{\vb{j}} \right) \\
                     &= B_o \; (-y  \hat{\vb{i}},\, x \hat{\vb{j}},\, 0 \hat{\vb{k}})
\end{aligned}                     
\end{equation*}


while its \textit{curl} only differs from $\vb{B}$ by a factor of two:

\begin{equation*}
\begin{aligned}
\curl{\left( \vb{B} \cross \vb{r} \right) } &= B_o \; \curl{\left( \hat{\vb{k}} \cross \vb{r} \right)} \\
											&= B_o \; \curl{(-y  \hat{\vb{i}},\, x \hat{\vb{j}},\, 0 \hat{\vb{k}})} \\
                                            &= 2 B_o \hat{\vb{k}}
\end{aligned}                     
\end{equation*}

Therefore, a tentative choice for a vector potential producing a constant field $B_o \hat{\vb{k}}$ inside the cylinder is (with $\hat{\vb{\phi}}$ a unit vector orthogonal to $\hat{\vb{k}}$ and $\vb{r}$)

\begin{equation}
\vb{A}_{inside} = \frac{B_o}{2} \; \hat{\vb{k}} \cross \vb{r} = \frac{B_o}{2} \, r \, \hat{\phi} =  \frac{B_o}{2} \, (-y  \hat{\vb{i}},\, x \hat{\vb{j}},\, 0 \hat{\vb{k}})
\label{eq:solenoid_A_inside}
\end{equation}

This choice for $\vb{A}$ yelds a uniform field $B_o \hat{\vb{k}}$  through the whole space 
so while it is correct for $\norm{\vb{r}} \le a$ -- where $a$ is the cylinder \textit{radius} -- it fails to give the correct field outside the cylinder, where the magnetic field is null. A possible correction is suggested by Stokes theorem when applied to computing the flux of $\vb{B}$ across a disk of radius $r$ centered on the cylinder axis and orthogonal to it. Indeed, the potential \ref{eq:solenoid_A_inside} is consistent with Stokes theorem for $\norm{\vb{r}} \le a$, because\footnote{
In cylindrical coordinates, $x = r cos \phi$, $y = r sin \phi$, therefore $dx = -r sin \phi d\phi = -y d\phi$ and $dy = r cos \phi d\phi = x d\phi$. Therefore, $d\vb{r} = (dx, dy) = (-y, x) d\phi = (\hat{\vb{k}} \cross \vb{r})d\phi$. When computing line integrals involving $\vb{A} \vdot d\vb{r}$ we can thus replace the expression $( \hat{\vb{k}} \cross \vb{r}) \vdot d\vb{r}$ with $r^2 d\phi$.   
As a result, 
\begin{equation*}
\begin{aligned}
\oint_{\Gamma} \vb{A} \vdot d\vb{r} &= \frac{B_o}{2} \oint_{\Gamma} (\hat{\vb{k}} \cross \vb{r}) \vdot d\vb{r} \\
&= \frac{B_o r^2}{2} \oint_{\Gamma} d\phi \\
&= \pi r^2 B_o 
\end{aligned}                     
\end{equation*}
},	%end of footnote

\begin{equation}
\begin{aligned}
\Phi(r) &= \int\limits_{S} \vb{B}_o \vdot \vu{n} \; \mathrm{d}A \\
&= \pi r^2 B_{o} \\
&= \oint_{\Gamma} \vb{A} \vdot d\vb{r}
\end{aligned}
\label{eq:solenoid_Flux_inside} 
\end{equation}

We observe that $\Phi(r)$ must reach its maximum for $r = a$ (the cylinder radius) and it must retain the constant value $\Phi(a) = \pi a^2 B_o$ for $r \geq a$. Given that the line integral curve length in \ref{eq:solenoid_Flux_inside} is proportional to $r$, then outside the cylinder $\norm{\vb{A}}$ must be proportional to $1/r$. With these prescriptions an obvious candidate is 
\begin{equation}
\begin{aligned}
\vb{A}_{outside} &= \frac{a^2}{r^2} \vb{A}_{inside}  \\
				 &= \frac{B_o}{2} \frac{a^2}{x^2 + y^2} 
				  ( -y \;\hat{\vb{i}},\, x \;\hat{\vb{j}},\, 0  \; \hat{\vb{k}} ) 
\end{aligned}
\label{eq:solenoid_A_outside}
\end{equation}

Indeed, using the formula for the curl in cartesian coordinates it is immediate to verify that $\curl{\vb{A}_{outside}}$ is well defined and identically zero for $r > 0$. Moreover, the line integral around a curve external to the cylinder yelds the expected result

\begin{equation}
\begin{aligned}
\Phi(r) &= \int\limits_{S} \vb{B}_o \vdot \vu{n} \; \mathrm{d}A \\
&= \oint_{\Gamma} \vb{A}_{outside} \vdot d\vb{r} \\
&= \frac{B_o}{2} \oint_{\Gamma} \frac{a^2}{r^2} r^2 \mathrm{d}\phi \\
&= \pi a^2 B_{o} 
\end{aligned}
\label{eq:solenoid_Flux_outside} 
\end{equation}

This example shows that the potential $\vb{A}$ may be different from zero in vast regions were there is no magnetic field. In this example there actually exists a \textit{singular} gauge transformation by which it is possible to \textit{gauge away} the potential almost everywhere, in the region outside the solenoid, at the price of introducing a singolarity of $\vb{A}$ where the gauge potential $\chi$ associated with the transformation has a discontinuity\footnote{See Felsager, pp.17-18.}. 










  












 
