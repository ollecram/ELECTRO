\chapter{Felsager -- Electromagnetism}
\label{felsager_01} 

\section{The Electromagnetic Field}

The two fundamental quantities of the electromagnetic field are the field strenghts

\begin{equation*}
\vb{E}(\vb{r}, t); \:\:\: \vb{B}(\vb{r}, t)
\end{equation*}

$\vb{E}$ is an ordinary (i.e. \textit{polar}) vector field, and as such it can be defined before any choice of a reference system. Instead, $\vb{B}$ is an \textit{axial} vector field and thus it can only be specified with respect to a given reference system. Reference systems can be separated in two non-intersecting classes (a.k.a \textit{orientations}), according to the sign (positive or negative) of the determinant associated to the transformation between two such systems. 

Both polar and axial vectors transform in the same way under spatial \textit{rotations} (a rotation connects two systems with the same orientation). Different behaviors arise when dealing with a \textit{reflection} of the three coordinate axes. This one connects two systems with opposite orientations. Polar vectors are not affected at all by this transformation while any axial vector turns into its opposite.

A large set of experimental data can be interpreted as the manifestation of the \textit{Lorentz} force, defining the effects of the fields $\vb{E}$ and $\vb{B}$ on a particle possessing a \textit{charge} $q$

\begin{equation}
\vb{F}(\vb{r}, t) = q \left(\vb{E}(\vb{r}, t) + \frac{1}{c} \vb{v} \cross \vb{B}(\vb{r}, t) \right)
\label{eq:Lorentzforce1}
\end{equation}

It should be noted that $\vb{v} \cross \vb{B}$ is an ordinary (\textit{polar}) vector because $\vb{B}$ is an axial vector and $\vb{v}$ an ordinary vector. 

According to special relativity the particle's motion obeys the following equations 

\begin{equation}
\dv{\vb{p}}{t} = q \left(\vb{E} + \frac{1}{c} \vb{v} \cross \vb{B} \right) ; \:\: \vb{p} = \gamma m \vb{v} ; \:\: \gamma = \frac{1}{\sqrt{1 - \frac{v^2}{c^2}}}
\label{eq:Lorentzforce2}
\end{equation}

This relation is of extreme importance and it actually serves to define the electromagnetic field. The argument is that the charge being small allows one to neglect the influence of the particle on the electromagnetic field\footnote{The field generated by the charge is certainly very large in its proximity, so for the argument to work one must assume a charge not to be subject to its own field.}, so that by analyzing the motion of a swarm of particles through the electromagnetic field one can determine the field strenghts $\vb{E}$ and $\vb{B}$.

In 1860 Maxwell derived the equations governing the motion of the electromagnetic field:

\begin{subequations}
\label{eq:Maxwell}
\begin{align}
\div{\vb{B}} &= 0 \label{eq:maxwell_divB}\\
\curl{\vb{E}} + \frac{1}{c} \pdv{\vb{B}}{t} &= 0 \label{eq:maxwell_rotE}\\
\div{\vb{E}} &= 4 \pi \rho \label{eq:maxwell_divE}\\
\curl{\vb{B}} &= \frac{4 \pi}{c} \vb{J} + \frac{1}{c} \pdv{\vb{E}}{t} \label{eq:maxwell_rotB}
\end{align}
\end{subequations}

where $\rho = \rho(\vb{r}, t)$ is the \textit{charge}  density, $\vb{J} = \rho \vb{v}$  the \textit{current}  and $c$ a constant equal to the speed of light in empty space. 

\subsection*{Vector calculus}
%Some formulas of vector calculus are reproduced here to help in the following sections
 
\begin{subequations}
\label{eq:vcalc}
\begin{align}
\vb{A} \cross \vb{B} &= - \vb{B} \cross \vb{A} \label{eq:vcalc_a}\\
\vb{A} \vdot (\vb{B} \cross \vb{C}) &= (\vb{A} \cross (\vb{B}) \vdot \vb{C}) \label{eq:vcalc_b}\\
\vb{A} \cross (\vb{B} \cross \vb{C}) &= \vb{B} (\vb{A} \vdot \vb{C}) - (\vb{A} \vdot \vb{B})\vb{C} \label{eq:vcalc_c}\\
(\vb{A} \cross \vb{B}) \cross \vb{C} &= (\vb{A} \vdot \vb{C}) \vb{B} - \vb{A} (\vb{B} \vdot \vb{C}) \label{eq:vcalc_d}\\
\div{(\grad{\phi})} &= \laplacian{\phi} \label{eq:vcalc_e}\\
\curl{(\grad{\phi})} &= \vb{0} \label{eq:vcalc_f}\\
\div{(\curl{\vb{B}})} &= 0 \label{eq:vcalc_g}\\
\curl{(\curl{\vb{B}})} &= \grad(\div{\vb{B}}) - \laplacian{\vb{B}}  \label{eq:vcalc_h}
\end{align}
\end{subequations}

\textbf{Gauss' theorem}

\begin{equation}
\int\limits_{\Omega}\div{\vb{E}} \; \mathrm{d}V = \int\limits_{S}\vb{E} \vdot \vu{n} \; \mathrm{d}A \label{eq:Gauss}
\end{equation}

\textbf{Stokes' theorem}

\begin{equation}
\int\limits_{S}(\curl{\vb{B}}) \vdot \vu{n} \; \mathrm{d}A = \oint\limits_{\Gamma} \vb{B} \vdot \mathrm{d}\vb{r}  \label{eq:Stokes}
\end{equation}

\textbf{Theorem of line-integrals}\\
The integral of the gradient of a scalar function $\phi$ between points $P$ and $Q$ along a curve $\Gamma$ is
  
\begin{equation*}
\int\limits_{\Gamma} \grad{\phi} \vdot \mathrm{d}\vb{r} = \phi(Q) - \phi(P)  \label{eq:LineInt}
\end{equation*}


\subsection*{Charge conservation}
By taking the divergence of \ref{eq:maxwell_rotB} and considering calculus rule \ref{eq:vcalc_g}

\begin{align*}
0 &= \div{(\curl{\vb{B}})} = \frac{4 \pi}{c} \div{\vb{J}} + \frac{1}{c} \pdv{\div{\vb{E}}}{t}
\end{align*}

In force of equation \ref{eq:maxwell_divE} the divergence appearing in the last term can be replaced by $4\pi\rho$ yelding the so called \textit{equation of continuity}  

\begin{equation}
\pdv{\rho}{t} + \div{\vb{J}}= 0 \label{eq:continuity}
\end{equation}

The physical significance of the above equation becomes more evident by applying Gauss' theorem to a control volume $\Omega$ to determine the rate of change of the total charge $Q$ contained inside that volume. Indeed, by definition the total charge located within the control volume at a given point in time is

\begin{equation*}
Q(t) = \int\limits_{\Omega}\rho(t) \; \mathrm{d}V 
\end{equation*}

while the istantaneous \textit{flux} of charge (charge entering or leaving the control volume in a unit of time) is given by the surface integral

\begin{equation*}
\int\limits_{S}\rho \,\vb{v} \vdot \vu{n} \; \mathrm{d}A = \int\limits_{S} \vb{J} \vdot \vu{n} \; \mathrm{d}A 
\end{equation*}

The continuity equation \ref{eq:continuity} just links, via the Gauss' theorem, the total flux of charge to the rate of change of $Q$  

\begin{equation*}
\frac{\mathrm{d} \, Q(t)}{\mathrm{d} \, t}  = \int\limits_{\Omega} \pdv{\rho}{t} \; \mathrm{d}V = - \int\limits_{S}\vb{J} \vdot \vu{n} \; \mathrm{d}A 
\end{equation*}

thereby expressing the law of \textit{local conservation} of charge. 

Further assumptions are needed in order to establish a \textit{global conservation} law for the charge. A typical approach often presented in books starts by considering a finite volume yet large enough to enclose the whole universe and so that currents vanish at at the boundaries.

\section{Gauge Potentials in Electromagnetism}


