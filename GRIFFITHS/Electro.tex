%%%%%%%%%%%%%%%%%%%% book.tex %%%%%%%%%%%%%%%%%%%%%%%%%%%%%

\documentclass[english, 11pt, a4paper]{book}

% Some specific typographic conventions used in Griffiths I2QM   START 
\usepackage{mathtools}			% equation tag with [..] instead of (..)
\newtagform{brackets}{[}{]}		% equation tag with [..] instead of (..)
\usetagform{brackets}			% equation tag with [..] instead of (..)
% Some specific typographic conventions used in Griffiths I2QM   END 

%\usepackage[nomath]{lmodern}
\usepackage[T1]{fontenc}
%\usepackage[italian]{babel}
% The following changes the Chapter heading from 'Chapter' to 'Lecture'
%$\addto\captionsenglish{\renewcommand{\chaptername}{Lecture}}
%$%\usepackage{fancyhdr}
%$\newcommand\chap[1]{%
%$ \chapter*{#1}%
%$  \addcontentsline{toc}{chapter}{#1}}
%$\newcommand\sect[1]{%
%$  \section*{#1}%
%$  \addcontentsline{toc}{section}{#1}}

% The following for using the Euro symbol
\usepackage[utf8x]{inputenc}
\usepackage{lmodern, textcomp}
  
% choose options for [] as required from the list
% in the Reference Guide, Sect. 2.2

\usepackage{makeidx}         % allows index generation
\usepackage{graphicx}        % standard LaTeX graphics tool
\usepackage{subcaption}      % for subfigures environments 
                             % when including figure files
\usepackage{multicol}        % used for the two-column index
\usepackage[bottom]{footmisc}% places footnotes at page bottom
% etc.
% see the list of further useful packages
% in the Reference Guide, Sects. 2.3, 3.1-3.3
\usepackage[normalem]{ulem}

\usepackage[shortlabels]{enumitem}	% to be able to resume enumerated lists

\usepackage{amsmath}	% To be able to slash
\usepackage{bm}	        % To use bold greek letters in math mode with \bm{}
\usepackage{amsfonts}	% To be able to use \mathbb ... 
\usepackage{amssymb}	% To be able to use \nmid ... 
\usepackage{amsthm}		% \qed, \qedhere
\usepackage{slashed}	% any character (dirac)
\usepackage[title,toc,page]{appendix}

% See https://tex.stackexchange.com/questions/36524/how-to-put-a-framed-box-around-text-math-environment/36528
\usepackage{collectbox}	% To make box around formulas

% *** AFTER THIS LINE *** 
%     put \usepackage{} for shared packages kept under ~\Links\repos\git\LaTeX_Styles

% Physics package 
% https://tex.stackexchange.com/questions/38978/how-can-i-manually-install-a-latex-package-debian-ubuntu-linux
\usepackage[italicdiff]{/home/marcello/Links/repos/git/LaTeX_Styles/physics}	
% To put accents below letters
\usepackage{/home/marcello/Links/repos/git/LaTeX_Styles//accents}

% To control vertical white space above and below equations
% see https://tex.stackexchange.com/questions/69662/how-to-globally-change-the-spacing-around-equations
\expandafter\def\expandafter\normalsize\expandafter{%
    \normalsize
    \setlength\abovedisplayskip{16pt}
    \setlength\belowdisplayskip{16pt}
    \setlength\abovedisplayshortskip{16pt}
    \setlength\belowdisplayshortskip{16pt}
}

% FROM BOXED_TEXT_ETC.tex

% To write two equations side by side
\usepackage{multicol}

% To use PGF/TikZ https://tex.stackexchange.com/questions/3622/best-way-to-generate-nice-function-plots-in-latex
\usepackage{tikz}
\usetikzlibrary{datavisualization}
\usetikzlibrary{datavisualization.formats.functions}

% To create a placeholder paragraph with Latin text
\usepackage{lipsum}

% To create framed text boxes with custom defined styles 
%\usepackage[linewidth=1pt]{mdframed}
\usepackage[framemethod=TikZ]{mdframed}
\mdfdefinestyle{MyFrame}{%
    linecolor=brown,			% blue, orange, brown, ...
    outerlinewidth=1pt,
    roundcorner=10pt,
    innertopmargin=\baselineskip,
    innerbottommargin=\baselineskip,
    innerrightmargin=15pt,
    innerleftmargin=15pt,
    backgroundcolor=gray!5!white}

		%Use for creating boxed/framed parts of text with nice borders

% To use extra symbols like dagger and double dagger in numbering footnotes 
\usepackage{footmisc}

% Allows aligning numbers at decimal point within `tabular environment
%\usepackage{siunitx}
%\sisetup{
%  round-mode          = places, % Rounds numbers
%  round-precision     = 2, % to 2 places
%}

% Force chapter numbering to restart within each part
\makeatletter
%\@addtoreset{chapter}{part}
\makeatletter


\makeindex             % used for the subject index
                       % please use the style svind.ist with
                       % your makeindex program


%%%%%%%%%%%%%%%%%%%%%%%%%%%%%%%%%%%%%%%%%%%%%%%%%%%%%%%%%%%%%%%%%%%%%

\begin{document}

% Useful within \begin{equation*}...\end{equation*} to have ONE equation with number & label
\newcommand\numberthis{\addtocounter{equation}{1}\tag{\theequation}}

\newcommand{\umlaut}[1]{\"#1}
\newcommand{\quotes}[1]{``#1''}
\newcommand{\ovr}[1]{\overline{#1}}
\newcommand{\sfT}{$\mathsf{T}$}
\newcommand{\udT}{\rotatebox[origin=c]{180}{$\mathsf{T}$}}
\newcommand{\avg}[1]{\langle{#1}\rangle}

%Bold calligraphic letters 
\newcommand{\N}{\mathbb{N}}	% integers
\newcommand{\Z}{\mathbb{Z}}	% relative
\newcommand{\Q}{\mathbb{Q}}	% rationals
\newcommand{\R}{\mathbb{R}}	% reals
\newcommand{\C}{\mathbb{C}}	% complex
\newcommand{\F}{\mathbb{F}}	% generic field 1
\newcommand{\K}{\mathbb{K}}	% generic field 2
\newcommand{\V}{\mathbb{V}}	% Shankar's for vector space V

%Plain calligraphic letters 
\newcommand{\cC}{\mathcal{C}}    % space 1
\newcommand{\cF}{\mathcal{F}}    % space 2
\newcommand{\cH}{\mathcal{H}}    % Calligraphic H for Hilbert space
\newcommand{\cS}{\mathcal{S}}    % space 3, Flow of energy (e.g in electromagnetism)
\newcommand{\cT}{\mathcal{T}}    % space 4 

\newcommand{\cI}{\mathcal{I}}    % Moment of Inertia
\newcommand{\cU}{\mathcal{U}}    % sets 1
\newcommand{\cV}{\mathcal{V}}    % sets 2
\newcommand{\cW}{\mathcal{W}}    % sets 3
\newcommand{\cP}{\mathcal{P}}    % sets 4, Momentum density (e.g in electromagnetism) 
\newcommand{\cQ}{\mathcal{Q}}    % sets 5
\newcommand{\cR}{\mathcal{R}}    % sets 6

\newcommand{\cL}{\mathcal{L}}    % Lagrangian density
\newcommand{\cE}{\mathcal{E}}    % Energy density (e.g in electromagnetism)
\newcommand{\cY}{\mathcal{Y}}    % Y

% Quaternions
\newcommand{\Qt}{\mathbb{H}}	% Hamilton's quaternions ('\H' APPARENTLY defined elsewhere by LaTeX}
\newcommand{\qu}{\mathbf{1}}     % 1
\newcommand{\qi}{\mathbf{i}}     % i
\newcommand{\qj}{\mathbf{j}}     % j
\newcommand{\qk}{\mathbf{k}}     % k

% Fraktur (Gothic) font (e.g for algebras)
\newcommand{\frk}[1]{\mathfrak{#1}}  

% To show argument of the exponential function vertically, i.e., as a superscript 
\newcommand{\vexp}[1]{\,e^{#1}}

% To type an angle as a number of degrees like 45^\circ
\newcommand{\degree}[1]{{#1}^\circ}

% To create not-bold vectors with a hat or check accent 
\newcommand{\hatv}[1]{\hat{#1}}
\newcommand{\chkv}[1]{\check{#1}}

% To create boldface vectors with a hat or check accent 
\newcommand{\hatvb}[1]{\vb{\hat{#1}}}
\newcommand{\chkvb}[1]{\vb{\check{#1}}}

% To create boldface greek letters (e.g. for denoting vectors) 
\newcommand{\bmath}[1]{\bm{#1}}  				% SAME AS \bm{#1} - NOT WORTH USING 
\newcommand{\chkbm}[1]{\boldmath{\check{#1}}}	% bold-check
\newcommand{\hatbm}[1]{\boldmath{\hat{#1}}}		% bold-hat

% To create <x|, |x> and <x|y> with unit vectors inside
\newcommand{\ubra}[1]{\bra*{\vu{#1}}}
\newcommand{\uket}[1]{\ket*{\vu{#1}}}
\newcommand{\uip}[2]{\ip*{\vu{#1}}{\vu{#2}}}

% To put accents below letters. 
\newcommand{\ut}[1]{\underaccent{\tilde}{#1}}
\newcommand{\uh}[1]{\underaccent{\hat}{#1}}
\newcommand{\form}[1]{\uh{#1}}

% To create italic, bold, bolditalic text
\newcommand{\tit}[1]{\textit{#1}}
\newcommand{\tbf}[1]{\textbf{#1}}
\newcommand{\tbi}[1]{\textit{\textbf{#1}}}

% Latin Modern sans serif |OR| Helvetica (SELECT)
\newcommand{\textlmss}{\fontfamily{lmss}\selectfont}
\newcommand{\texthv}{\fontfamily{phv}\selectfont}

% Latin Modern sans serif |OR| Helvetica (USE, within OR outside MATH !)
\newcommand{\tlmss}[1]{\text{\textlmss{#1}}}
\newcommand{\thv}[1]{\text{\texthv{#1}}}

% To use \tlmss{T} symbol to denote transpose 
\newcommand{\transp}[1]{{#1}^{\tlmss{T}}}

% To use \dagger symbol to denote operator Adjoint
\newcommand{\Adj}[1]{{#1}^\dagger}

% To denote the Hermitian conjugate with a '+' superscript
\newcommand{\Hconj}[1]{{#1}^{+}}

% To use \tlmss{Ker}, \tlmss{Coker} and \tlmss{Img} to denote Kernel, Co-Kernel & Image 
\newcommand{\Ker}{\tlmss{Ker}\,}
\newcommand{\Coker}{\tlmss{Coker}\,}
\newcommand{\Img}{\tlmss{Im}\,}

% To use \tlmss{Alt} and \tlmss{alt} to denote alternation 
\newcommand{\Alt}{\tlmss{Alt}\,}
\newcommand{\alt}{\tlmss{alt}\,}

% To use \tlmss{Ann} to denote annulets 
\newcommand{\Ann}{\tlmss{Ann}\,}

% Misc abbreviations
\newcommand{\ora}[1]{\overrightarrow{#1}}

\DeclareRobustCommand{\rchi}{{\mathpalette\irchi\relax}}
\newcommand{\irchi}[2]{\raisebox{\depth}{$#1\chi$}} % inner command, used by \rchi

% See https://tex.stackexchange.com/questions/36524/how-to-put-a-framed-box-around-text-math-environment/36528
\makeatletter
\newcommand{\mybox}{%
    \collectbox{%
        \setlength{\fboxsep}{1pt}%
        \fbox{\BOXCONTENT}%
    }%
}
\makeatother

\author{Marcello Vitaletti}
\title{Electrodynamics}
\maketitle

\frontmatter%%%%%%%%%%%%%%%%%%%%%%%%%%%%%%%%%%%%%%%%%%%%%%%%%%%%%%

%\include{dedic}

%\chapter*{Plan}
\label{plan} 

In this book I am keeping notes about the theory of classical electromagnetism, 
as exposed in various books. In particular, I intend to cover the following materials:

\begin{itemize}

\item B. Felsager -- Geometry Particles and Fields
\begin{enumerate}
\setcounter{enumi}{0}
\item Electromagnetism (1.1 to 1.4)
\end{enumerate}

\item C. Cattaneo -- Teoria Einsteniana della Gravitazione
\begin{enumerate}
\setcounter{enumi}{0}
\item Elementi di Algebra e Analisi Lineare
\end{enumerate}

\item D.J. Griffiths -- Introduction to Electrodynamics
\begin{enumerate}
\setcounter{enumi}{0}
\item Vector Analysis
\item Electrostatics
\item Potentials
\item Electric Fields in Matter
\item Magnetostatics
\item Magnetic Fields in Matter
\item Electrodynamics
\item Conservation Laws
\item Electromagnetic Waves
\item Radiation
\item Electrodynamics and Relativity
\item Potentials and Fields
\item Helmoltz Theorem
\end{enumerate}

\item J.D. Jackson -- Classical Electrodynamics, 2nd Edition
\begin{enumerate}
\setcounter{enumi}{0}
\item Introduction to Electrostatics
\item Boundary Value Problems in Electrostatics - I
\item Boundary Value Problems in Electrostatics - II
\item Multipoles, Electrostatics of Macroscopic Media, Dielectrics
%\item Magnetostatics
%\item Time Varying Fields, Maxwell Equations, Conservation Laws
%\item Plane Electromagnetic Waves and Wave Propagation
%\item Wave Guides and Resonant Cavities
%\item Simple Radiating Systems, Scattering and Diffraction
%\item Magnetohydrodynamics and Plasma Physics
\end{enumerate}

\item J.D. Jackson -- Classical Electrodynamics, 3rd Edition
\begin{enumerate}
\setcounter{enumi}{4}
\item Magnetostatics, Faraday's Law, Quasi-Static Fields
\item Maxwell Equations, Macroscopic Electromagnetism, Conservation Laws
\item Plane Electromagnetic Waves and Wave Propagation
\item Wave Guides, Resonant Cavities and Optical Fibers
\item Radiating Systems, Multipole Fields and Radiation
\item Scattering and Diffraction
\item Special Theory of Relativity
\item Dynamics of Relativistic Particles and Electromagnetic Fields
\end{enumerate}

\item B. Felsager -- Geometry Particles and Fields
% Contacts with quantum theory of particles dynamics in EM fields
\begin{enumerate}
\setcounter{enumi}{1}
\item Interaction of Fields and Particles
\end{enumerate}

\item J. Franklin -- Advanced Mechanics and General Relativity
\begin{enumerate}
\setcounter{enumi}{1}
\item Relativistic Mechanics
\item Tensors
\item Curved Space
\item Scalar Field Theory
\item Tensor Field Theory (6.1 to 6.5)
\end{enumerate}

\item J.D. Jackson -- Classical Electrodynamics, 3rd Edition
\begin{enumerate}
\setcounter{enumi}{12}
\item Collisions, Energy Loss and Scattering of Charged Particles, Cherenkov and Transition Radiation
\item Radiation by Moving Charges
\item Bremsstrahlung, Method of Virtual Quanta, Radiative Beta Processes
\item Radiation Damping, Classical Models of Charged Particles
\end{enumerate}

\item B. Felsager -- Geometry Particles and Fields
% Contacts with quantum theory of fields dynamics + differential geometry math
\begin{enumerate}
\setcounter{enumi}{2}
\item Dynamics of Classical Fields
\end{enumerate}

\begin{enumerate}
\setcounter{enumi}{5}
\item Differentiable Manifolds, Tensor analysis
\item Differential Forms, Exterior Calculus
\item Integral Calculus on Manifolds
\end{enumerate}

\item C.W. Misner, K.S. Thorne, J.A. Wheeler -- Gravitation
\begin{enumerate}
\setcounter{enumi}{1}
\item Foundations of Special Relativity
\item The Electromagnetic Field
\item Electromagnetism and Differential Forms
\end{enumerate}

\item L.D. Landau, E.M. Lifshitz -- Teoria dei Campi
\begin{enumerate}
\setcounter{enumi}{0}
\item Principio di Relatività
\item Meccanica Relativistica
\item Carica in un Campo Elettromagnetico
\item Equazioni del Campo Elettromagnetico
\item Campo Elettromagnetico Costante
\item Onde Elettromagnetiche
\item Propagazione della Luce
\item Campo di Cariche in Moto
\item Radiazione Elettromagnetica
\end{enumerate}

\item L.D. Landau, E.M. Lifshitz -- Elettrodinamica dei Mezzi Continui
\begin{enumerate}
\setcounter{enumi}{0}
\item Elettrostatica dei Conduttori
\item Elettrostatica nei Dielettrici
\item Corrente Continua
\item Campo Magnetico Costante
\item Ferromgnetismo e Antiferromagnetismo
\item Superconduttività
\item Campo Magnetico Quasi Stazionario
\item Idrodinamica Magnetica
\item Equazioni delle Onde Elettromagnetiche
\item Propagazione delle Onde Elettromagnetiche
\item Onde Elettromagnetiche in Mezzi Anisotropi
\item Dispersione Spaziale
\item Ottica non Lineare
\item Passaggio delle Particelle Veloci attraverso la Materia
\item Diffusione delle Onde Elettromagnetiche
\item Diffrazione dei Raggi X nei Cristalli
\end{enumerate}

\end{itemize}
	
\setcounter{tocdepth}{1}	% Must appear BEFORE \tableofcontents!
\tableofcontents
%\addappheadtotoc

\mainmatter%%%%%%%%%%%%%%%%%%%%%%%%%%%%%%%%%%%%%%%%%%%%%%%%%%%%%%%
%\setcounter{chapter}{-1}	% To start with Chapter 0 !!  
%\input{../FANCYBOX}		% Example of boxed/framed parts of text with nice borders

\begin{flushright}
\tit{Quassù tutto vi appare regolato\\
dal sorgere e calare di una stella;\\
Si scambiano il pensiero con dei suoni,\\
movimenti del viso e delle mani;\\
Cercano l'armonia, ma spesso in guerra;\\
Eppur vorrei restarci sulla Terra!} 
\end{flushright} 
% A -- Vector Analysis
% B -- Electrostatics
\chapter{THE ELECTRIC FIELD}\label{ch:B1}

\section{Coulomb's Law}\label{sec:1.1} 
The starting problem of electrostatics is to determine the force caused by the presence of a \tit{source} charge $q$ at location $\vb{r}'$ on a \tit{test} charge $Q$  at $\vb{r}$ when both charges are \tit{at rest} in a given \tit{inertial} frame of reference. The answer to this problem is \tbf{Coulomb's law}\footnote{The absence of scaling factors in Coulomb's law is a distinctive feature of the Gaussian system of units, which is adopted in these notes.}
\begin{equation}
\vb{F}_{Gauss} = \frac{q Q}{s^2} \vu{s}\label{eq:B.1}
\end{equation}

where 

\begin{equation}\label{eq:B.2}
\vb{s} = \vb{r} - \vb{r}'
\end{equation}

is the \tbi{separation vector}  between the the two charges, directed from the source point $q$ at $\vb{r}'$ to the test point $Q$ at $\vb{r}$ , $s = \abs{\vb{r} - \vb{r}'}$ and 
$\vu{s} = \vb{s}/s$.

Based on the \tbi{superposition principle}, the effect of multiple source charges (at rest) $q_1, q_2, \ldots, q_n$ on the test charge $Q$ is simply the sum over all sources of the force generated by each one of them
\begin{equation}\label{eq:B.3}
\vb{F} =  \vb{F}_1 + \vb{F}_2 + \cdots + \vb{F}_n = Q \left( \frac{q_1}{s_1^2} \vu{s}_1 + \frac{q_2}{s_2^2} \vu{s}_2 + \cdots + \frac{q_n}{s_n^2} \vu{s}_n \right)
\end{equation}

Force has dimension $[m\,l\, t^{-2}]$. The \tbi{newton} is the unit of force in SI and MKSA, were the unit of mass (1 Kg) is $10^3$ times the unit of mass (1 g) in Gaussian units. The SI and MKSA unit of length (1 meter) is $10^2$ times the unit of length (1 cm) in Gaussian units. Therefore,\\
\tbi{1 newton} = $10^5$ \tbi{dynes}, where the \tbi{dyna} is the Gaussian unit of force.  

\section{Conversion between Gaussian and SI units}\label{sec:1.2}
Equation \ref{eq:B.1} assumes Gaussian units, where the unit of force is the \tit{dyna}: a force that causes the speed of a $1$ gram particle to increase by $1$ centimeter per second on every second. This equation can be seen as a way to fix the unit of charge, which in the Gaussian system is called \tbf{electrostatic unit} (\tbf{esu}). 
The following table lists the units of \tit{length}, \tit{mass}, \tit{force} and \tit{charge} in the Gaussian and SI (MKSA) units, respectively:\\ 

\begin{tabular}{r|c|c|r}
         & Gauss      &  SI   & Scale\\ \hline
   time  & second     &  second & $1$\\
  length & centimeter &  meter & $10^2$\\
  mass   & gram       &  kilogram & $10^3$\\
  force  & dyna       &  newton & $10^5$\\
  charge & esu        &  coulomb & $3 \times 10^9$\\
\end{tabular}\\ 

The adimensional scale factor $3 \times 10^9$ between the \tbf{esu} and \tbf{coulomb} units for the electric charge (which appears at the intersection of the rightmost column and last row in the above table) means that one coulomb contains \tit{three billions} esu's, which is why the Gaussian units are often chosen to discuss electromagnetic interactions at the atomic or molecular level, while SI units are preferred for engineering application involving macroscopic systems.


\subsection*{Problem 1}
Determine the ratio between $1$ \tbf{coulomb}, the SI unit of charge, and $1$ \tbf{esu} (aka \tbf{statcoulomb}), where the latter is the unit of charge in the Gauss electromagnetic units. 
Start from equation (\ref{eq:B.1}) where \tit{force} between two point charges $q$ and $Q$ is expressed in the Gauss units and make use of the corresponding equation where length, time and charge are expressed in the \tbf{SI (MKSA)} units, namely
\begin{equation}
\vb{F}_{SI}  = \frac{1}{4 \pi \epsilon_{0}} \frac{\tilde{q} \tilde{Q}}{\tilde{s}^2} \vu{s}\,,\label{eq:B.4}
\end{equation}

where $$\frac{1}{4 \pi \epsilon_{0}} = \frac{1}{4 \pi} \frac{4 \pi c^2}{10^7} = \frac{c^2}{10^7}\,.$$

\subsection*{Solution}

Let consider two charged, massive particles, namely a \tit{source} particle $p$ of charge $q$ (which for simplicity is supposed to be at rest) and a \tit{test} particle $P$ of mass $M$ and charge $Q$ which is free to move. 

The acceleration of the particle $P$ in the field of particle $p$ does not depend on the system of units which one adopts\footnote{Only its \tit{numerical value} does.}. The units of acceleration $a$ in the Gauss and SI units are $[cm \cdot sec^{-2}]$ and $[m \cdot sec^{-2}]$, respectively. Therefore, an acceleration of $1 m \cdot sec^{-2}$ in SI units corresponds to $10^2 cm \cdot sec^{-2}$ in Gauss units. 

In the following, $q$ and $Q$ denote the numerical value of \tit{electrical charge} for the particles $p$ and $P$ in the Gauss units (esu), while $\tilde{q}$ and $\tilde{Q}$ denote the numerical value of \tit{electrical charge} for the same particles $p$ and $P$ in the SI (MKSA) units (coulomb). 

Let also $m$ be the \tit{mass} of particle $p$ and $M$ be the mass of particle $P$ in the Gauss units (gram), while $\tilde{m}$ and $\tilde{M}$ denote the numerical value of \tit{mass} for the same particles $p$ and $P$ in the SI (MKSA) units (kilogram).

With these stipulations, the numerical values of the acceleration of particle $P$ in the Gauss and SI units are, respectively 
$$a = \frac{F_{Gauss}}{M}\,=\, \frac{qQ}{M\,s^2}\, \text{cm}\, \cdot\, \text{sec}^{-2}\,\text{, and}$$    
$$\tilde{a} = \frac{F_{SI}}{\tilde{M}}\,=\, \frac{1}{4 \pi \epsilon_{0}} \frac{\tilde{q}\tilde{Q}}{\tilde{M}\,\tilde{s}^2}\,
=\, \frac{\tilde{c}^2}{10^7} \frac{\tilde{q}\tilde{Q}}{\tilde{M}\,\tilde{s}^2}\,  \text{m}\, \cdot\, \text{sec}^{-2}\, \text{.}$$   

Taking the ratio of corresponding terms in the two equations, while denoting with $\tilde{c} = 3 \times 10^8$ the speed of light in the SI units, we get 
\begin{align*}
\frac{a}{\tilde{a}} = 10^2 &=  \frac{q}{\tilde{q}}  \frac{Q}{\tilde{Q}} \frac{\tilde{M}}{M} \frac{\tilde{s}^2}{s^2} \frac{10^7}{\tilde{c}^2}\\
                           &=  \frac{q}{\tilde{q}}  \frac{Q}{\tilde{Q}} \cdot       10^{-3} \cdot           10^{-4} \frac{10^7}{\tilde{c}^2}\\
                           &= \left( \frac{Q}{\tilde{Q}} \right)^2 \cdot \frac{1}{\tilde{c}^2}\,,
\end{align*}

which can be reduced to $$\left( \frac{Q}{\tilde{Q}} \right)^2 = (10 \tilde{c})^2\,\text{or, taking the square root }$$
$$Q = \tilde{Q} \cdot (3 \times 10^9)\,,$$
whereby $Q$, the charge of a particle $P$ in the Gauss unit (\tbf{esu}, \tit{aka} \tbf{statcoulomb}) is a number $3 \times 10^9$ larger than the same charge expressed in the SI (MKSA) unit of charge (\tbf{coulomb}). Of course this means that 
$$1 \text{ \tbf{coulomb} } = 3 \times 10^9 \text{ \tbf{esu} .} $$

\section{The Electric Field}\label{sec:1.3}
The \tbi{electrostatic field} $\vb{E}(\vb{r})$ is, by definition, the force exerted on a \tbi{unit test charge} located at the point $\vb{r}$ by all other charges (assumed at rest). With $\vb{s} = \vb{r} - \vb{r}'$, we therefore write

\begin{equation}
\label{eq:B.5}
\vb{E}(\vb{r}) = \frac{\vb{F}}{Q} =  \sum_{i=1}^n \frac{q_i}{s_i^2} \vu{s}_i
\end{equation}

In terms of $\vb{E}(\vb{r})$ the \tbf{Coulomb's law} \ref{eq:B.3} can be rewritten as 

\begin{equation}
\label{eq:B.6}
\vb{F}(\vb{r}) = Q \vb{E}(\vb{r})\,.
\end{equation}


\subsection*{Example 1}
Find the electric field a distance $z$ above the midpoint between two equal charges ($q$), a distance $d$ apart.

\subsection*{Solution}
Let $\vb{E}_1$ be the field of the left charge alone, and $\vb{E}_2$ that of the right charge alone. Adding them (vectorially), the horizontal ($\vu{i}$) components cancel and the vertical ($\vu{k}$) components conspire (see Griffiths Fig. 4)
$$E_z = 2 \frac{q}{s^2} \cos \theta\,,$$ where $s = \sqrt{z^2 + (d/2)^2}$  and $\cos \theta = z / s$, so

$$ \vb{E} = \frac{2 q z}{[z^2 + (d/2)^2]^{3/2}} \vu{k}\,.$$ 


\section{Continuous Charge Distributions}\label{sec:1.4}

Our definition of the electric field (Eq. \ref{eq:B.5}) assumed the source of the field to be a collection of discrete point charges $q_i$. If instead the charge is distributed over some region $\omega$, the sum becomes an integral:
 
\begin{equation}
\label{eq:B.7}
\vb{E}(\vb{r}) = \int_{\omega} \frac{\vu{s}}{s^2} \dd{q}\,.
\end{equation}

If the charge is spread out along a \tit{line}, with $\lambda$ the charge per unit length, then $\dd{q} = \lambda \dd{l'}$ 
(where $\dd{l'}$ is an element of length along the line);
if the charge is spread out over a \tit{surface}, with $\sigma$ the charge per unit area, then $\dd{q} = \sigma \dd{a'}$ 
(where $\dd{a'}$ is an element of area on the surface);
if the charge is spread out within a \tit{volume}, with $\rho$ the charge per unit volume, then $\dd{q} = \rho \dd{\tau'}$
(where $\dd{\tau'}$ is an element of volume):

$$ \dd{q} \rightarrow \lambda \dd{l'} \sim  \sigma \dd{a'} \sim \rho \dd{\tau'}\,. $$

Thus the electric field of a line charge is 
\begin{equation}\label{eq:B.8}
\vb{E}(\vb{r}) = \int_{\Lambda} \frac{\lambda(\vb{r'})}{s^2} \vu{s} \dd{l'}\,;
\end{equation}

for a surface charge,  
\begin{equation}\label{eq:B.9}
\vb{E}(\vb{r}) = \int_{\Sigma} \frac{\sigma(\vb{r'})}{s^2} \vu{s} \dd{a'}\,;
\end{equation}

and for a volume charge,  
\begin{equation}\label{eq:B.10}
\vb{E}(\vb{r}) = \int_{\Omega} \frac{\rho(\vb{r'})}{s^2} \vu{s} \dd{\tau'}\,.
\end{equation}

Equation \ref{eq:B.10} itself is often referred to as \quotes{Coulomb's law}, because it is such a short step from the original 
(\ref{eq:B.1}), and because a volume charge is in a sense the most general and realistic case.\\

Please note carefully the meaning of $\vb{s} = s \vu{s}$ in these formulas. Originally, in equation \ref{eq:B.3}, $\vb{s}$ stood for the vector from the source charge $q_i$ to the field point $\vb{r}$. Correspondingly, in Eqs. \ref{eq:B.7}-\ref{eq:B.10}, 
$\vb{s}$ is the vector from $\dd{q}$ (therefore from $\dd{l'}$, $\dd{a'}$, or $\dd{\tau'}$) to the field point $\vb{r}$.\\

\tbi{Warning:} The unit vector $\vu{s}$ is \tit{not} constant; its \tit{direction} depends on the source point $\vb{r'}$, hence it \tit{cannot be taken outside of the integrals} (Eqs. \ref{eq:B.7}-\ref{eq:B.10}). In practice, \tit{you must work with Cartesian components} even if you use curvilinear coordinates to perform the integration ($\vu{i}$, $\vu{j}$, $\vu{k}$ \tit{are} constant, and \tit{do} come out). 


\subsection*{Example 2}
Find the electric field a distance $z$ \tbi{above the midpoint} of a straight line segment of length $2L$ that carries a uniform charge $\lambda$. 

\subsection*{Solution}
The simplest method is to chop the line into symmetrically placed pairs (at $\pm x$), quote the result of Ex. 1 (with $d/2\,\rightarrow x$, $q\, \rightarrow \lambda \dd{x}$) and integrate ($x\,:\,0 \rightarrow L$).\\
But here's a more general approach:\footnote{Ordinarily I'll put a prime on the source coordinates, but where no confusion can arise I'll remove the prime to simplify the notation.} 

\begin{align*}
\vb{r} = z \vu{k}, \quad \vb{r}' &= x \vu{i}, \quad \dd{l}' = \dd{x}\\
\vb{s} = \vb{r} - \vb{r}' = z \vu{k} -  x \vu{i}, \quad s &= \sqrt{z^2 + x^2}, \quad \vu{s} = \frac{\vb{s}}{s} = \frac{z \vu{k} -  x \vu{i}}{\sqrt{z^2 + x^2}}\,.
\end{align*}

\begin{align}
\vb{E} &= \int_{-L}^{L} \frac{\lambda}{z^2 + x^2} \frac{z \vu{k} -  x \vu{i}}{\sqrt{z^2 + x^2}} \dd{x}\label{eq:B.11}\\
       &= \lambda \left[ z \vu{k}  \int_{-L}^{L} \frac{1}{(z^2 + x^2)^{3/2}} \dd{x}  - \vu{i} \int_{-L}^{L} \frac{x}{(z^2 + x^2)^{3/2}} \dd{x} \right]\label{eq:B.12}\\
       &= \lambda \left[ z \vu{k}  \left( \frac{x}{z^2 \sqrt{z^2 + x^2}}\right)\bigg\rvert_{-L}^{L} - \vu{i} \left(- \frac{1}{\sqrt{z^2 + x^2}}\right)\bigg\rvert_{-L}^{L} \right]\label{eq:B.13}\\
       &= \frac{2 \lambda L}{z \sqrt{z^2 + L^2}} \vu{k}\,.\label{eq:B.14}
\end{align}

Primitives of the above two integrals for $\vb{E}$ are found in Dwight as formulas 200.03 and 201.03.\\

For points far from the line ($z \gg L$),
$$ \lim_{L/z \rightarrow 0}  E \simeq \frac{2\lambda L}{z^2}\,. $$
This makes sense: From far away the line looks like a point charge $q = 2\lambda L$. 
In the limit $L \rightarrow \infty$, on the other hand, we obtain the field of an infinite straight wire: 
\begin{equation}\label{eq:B.15}
\lim_{L \rightarrow \infty} E \simeq \frac{2\lambda}{z}
\end{equation}




\subsection*{Problem 3}
Find the electric field a distance $z$ \tbi{above one end} of a straight line segment of length $L$ that carries a uniform line charge $\lambda$. Check that your formula is consistent with what you would expect for the case $z \gg L$. 

\subsection*{Solution}

Adapting a partial result (Eq. \ref{eq:B.13}) from the solution of Example 2 we get

\begin{align}
\frac{\vb{E}}{\lambda} &= \vu{k}  \left( \frac{x}{z \sqrt{z^2 + x^2}}\right)\bigg\rvert_{0}^{L} + \vu{i} \left(\frac{1}{\sqrt{z^2 + x^2}}\right)\bigg\rvert_{0}^{L}\label{eq:B.16} \\
       &= \frac{L}{z\,\sqrt{z^2 + L^2}} \vu{k} + \left( \frac{1}{\sqrt{z^2 + L^2}} - \frac{1}{z} \right)\,\vu{i}\,,\label{eq:B.17}
\end{align}

where $\vu{k}$ denotes any\footnote{We observe that symmetry requires the field behaviour to be the same along any segment starting at one end of the charged line and orthogonal to it.} direction \tbi{orthogonal} to the charged line segment of length $L$, which we assume to start at $x=0$ along the $\vu{i}$ axis.\\

In the limit of large distances from the $\vu{i}$ axis (when $z \gg L$) the field component \tit{orthogonal} to that axis is:
$$ \lim_{L/z \rightarrow 0}\frac{E_z}{\lambda L} = \lim_{L/z \rightarrow 0} \frac{1}{z^2\,\sqrt{(1 + L^2/z^2}} = \frac{1}{z^2}\,.$$

The above result is what should be expected: at large distances the charged segment has approximately the same effect as if the same amount of charge $q'=\lambda L$ would be concentrated at the origin ($\vb{r}_{q'} = \vb{0}$). 

Measured at the same point ($\vb{r} = (0,0,z)$, with $z \gg L$) the field component along $\vu{i}$ (the direction of the charged line segment) is 
$$\frac{E_x}{\lambda L} = \frac{1}{\sqrt{z^2 + L^2}} - \frac{1}{z}$$
In the limit where $z \gg L$ this term becomes 
$$\frac{E_x}{\lambda L} = \frac{1}{\sqrt{z^2 + L^2}} - \frac{1}{z}  \simeq \frac{1}{z} - \frac{1}{z} = 0\,.$$

Symmetry dictates $E_x$ to be exactly zero above the midpoint  of the segment charge, so at $(L/2,0,z)$. 
But of course in the limit where  $z \gg L$ the field component along $\vu{i}$ measured at $(0,0,z)$ or at $(L,0,z)$ must also be close to zero.

\subsection*{Problem 4}
Find the electric field a distance $z$ above one end of the center of a square loop (side $a$) carrying uniform line charge $\lambda$. [\tit{Hint}: Use the result of  Ex. 2.]

\subsection*{Solution}

The field produced at $\vb{r} = (0, 0, z)$ by each one of the four sides of the loop is derived from (Eq. \ref{eq:B.14}) by replacing  $z$ with $z' = \sqrt{z^2 + a^2/4}$, because Eq. \ref{eq:B.14} holds when the charged segment and the point  $\vb{r}$ are in the same plane. 
We ignore the field component $E_y$ generated by segments parallel to $\vu{i}$ and the field component $E_x$ generated by segments parallel to $\vu{j}$ because these cancel out within each pair of parallel sides. Therefore 

\begin{equation}\label{eq:B.18}
\frac{\vb{E}}{\lambda a} = \frac{4}{z' \sqrt{{z'}^2 + a^2/4}}\,\vu{k}
\end{equation}

When $z \gg a$ we can substitute $z$ for $z'$ in Eq. \ref{eq:B.18} and ignore the constant under the square root, whence

\begin{equation}\label{eq:B.19}
\lim_{a/z \rightarrow 0} \vb{E} = \frac{q}{z^2}\,  \vu{k}\,,  
\end{equation}
where $q = 4 \lambda a$ is the total charge of the loop.  

\subsection*{Problem 5}
Find the electric field a distance $z$ above the center of a circular loop of radius $R$ that carries a uniform line charge $\lambda$). 

\subsection*{Solution}

Let $P$ be the point of coordinates $\vb{r} = (0, 0, z)$ where the field is being evaluated. Let $p'$ denote the the infinitesimal charge element of length $R \dd{\theta}$ on the circular loop. Consider the right triangle of base $R$ and height $z$. Clearly $R^2 + z^2$ is the squared distance from the loop element at $p'$ (of charge $\lambda R \dd{\theta}$) and the point $P$. The element's contribution to the field at $P$ is $$\dd{\vb{E}} = \frac{\lambda R \dd{\theta}}{R^2 + z^2}\,. $$ 

The component of $\dd{\vb{E}}$ parallel to the $\vu{i}\:\vu{j}$ plane becomes negligible as $z \gg R$. In any case, we can apply a reasoning similar to the one used in solving Problem 4, namely we do ignore the fields component $E_x$ and $E_y$ parallel to the loop plane because for each segment $R \dd{\theta}$ there is a corresponding segment $R (\pi + \dd{\theta})$ that generates equal and \tit{opposite} components in the $\vu{i}\,\vu{j}\;$ plane. Therefore, knowing the angle $\phi$ between $\dd{\vb{E}}$ and the $\vu{k}$ axis, we obtain the exact  contribution of each element to the field $\vb{E}$ at $P$.

\begin{align*}
\vb{E} &= \vu{k}\: \int_0^{2\pi} \frac{\lambda R \dd{\theta}}{(R^2 + z^2)} \cdot \cos \phi \\
       &= \vu{k}\: \int_0^{2\pi} \frac{\lambda R \dd{\theta}}{(R^2 + z^2)} \cdot \frac{z}{\sqrt{R^2 + z^2}}\\
       &= \vu{k}\: \int_0^{2\pi} \frac{\lambda R z \dd{\theta}}{(R^2 + z^2)^{3/2}}\\
       &= \vu{k}\: \frac{q z}{(R^2 + z^2)^{3/2}}\,, \numberthis \label{eq:B.20}
\end{align*}

where $q =  2 \pi R \lambda$ is the loop total charge. In the limit of $z \gg R$, $E \rightarrow q\vu{k}/z^2 $, as expected.


\subsection*{Problem 6}
Find the electric field a distance $z$ above the center of a flat circular disk of radius $R$ that carries a uniform surface charge $\sigma$. What does your formula give in the limit $R \rightarrow \infty$? Also check the case $z \gg R$.  

\subsection*{Solution}
We build on top of  the solution of Problem 5 by transforming the line integrals leading to [\ref{eq:B.20}] into double integrals
\begin{align*}
\vb{E} &= \vu{k}\: \int_0^R \dd{r'} \int_0^{2\pi} \frac{\sigma r' \dd{\theta}}{({r'}^2 + z^2)} \cdot \cos \phi \\
       &= \vu{k}\: \int_0^R \dd{r'} \int_0^{2\pi} \frac{\sigma r' \dd{\theta}}{({r'}^2 + z^2)} \cdot \frac{z}{\sqrt{{r'}^2 + z^2}}\\
       &= \vu{k}\: \int_0^R \frac{2 \pi \sigma r' z }{({r'}^2 + z^2)^{3/2}} \dd{r'} \\
       &= \pi \sigma z \, \int_0^R \frac{2\,r' \dd{r'}}{({r'}^2 + z^2)^{3/2}}\:\vu{k}\,, \numberthis \label{eq:B.21}
\end{align*}
  
where in the above formulas $r' = \abs{\vb{r'}} = \abs{x \vu{i} + x \vu{j}} = \sqrt{x^2 + y^2}\;$ is the projection onto the $\vu{i}\;\vu{j}$ plane of the vector $\vb{r} = \vb{r'} + z \vu{k}$ which connects the charge element $r'\dd{\theta}$ to the point $P = z\,\vu{k}$. 

The integral (\ref{eq:B.21}) can be evaluated by using $\eta(r') = ({r'}^2 + z^2)$ as the integration variable instead of $r'$. 
Given that $\dv*{\eta}{r'} = 2 r'$, the integral in [\ref{eq:B.21}] becomes
\begin{align*}
\int_0^R \frac{2\,r' \dd{r'}}{({r'}^2 + z^2)^{3/2}} &= \int_{z^2}^{R^2 + z^2} \eta^{-3/2} \dd{\eta}\\
                                                    &= \frac{2}{\sqrt{\eta}} \bigg\rvert_{R^2 + z^2}^{z^2}\\
                                                    &= 2 \left( \frac{1}{z} - \frac{1}{\sqrt{R^2 + z^2}} \right)\,. \numberthis \label{eq:B.22}
\end{align*}

Putting the above result in (\ref{eq:B.21}) we obtain
\begin{equation}\label{eq:B.23}
\vb{E} =   2 \pi \sigma z \left( \frac{1}{z} - \frac{1}{\sqrt{R^2 + z^2}} \right) \vu{k}\,.
\end{equation}

For $R \gg z$ the second term can be neglected, hence the field close to the disk becomes\footnote{The corresponding formula in SI units is $\vb{E}_{disk} =   \sigma/(2 \epsilon_0)\,\vu{k}\,.$}
\begin{equation}\label{eq:B.24}
\vb{E}_{disk} =   2 \pi \sigma\; \vu{k}\,.
\end{equation}

For $z \gg R$ the second term can be approximated as follows
\begin{align*}
\frac{1}{\sqrt{R^2 + z^2}} &= \frac{1}{z} \frac{1}{\sqrt{R^2/z^2 + 1}}\\
						   &\approx \frac{1}{z} \left(1 - \frac{R^2}{2z^2}\right)\,,
\end{align*}

which substituted in [\ref{eq:B.23}] gives the expected result:
\begin{equation}\label{eq:B.25}
\vb{E} \approx  2 \pi \sigma z \left( \frac{1}{z} - \frac{1}{z} + \frac{R^2}{2z^3} \right) \vu{k} = \frac{\pi R^2 \sigma}{z^2} \vu{k} = \frac{Q}{z^2}  \vu{k}\,.
\end{equation}





\subsection*{Problem 7}

Find the electric field a distance $z$ from the center of a spherical surface of radius $R$ that carries a charge density $\sigma$. Treat the case $z < R$ (inside) as well as $z > R$ (outside). Express your answers in terms of the total charge $q$ on the sphere. [\tit{Hint:} Use the law of cosines to write $\vb{r}$\footnote{The vector from a source point to the field point.} in terms of $R$ and $\theta$. Be sure to take the \tit{positive} square root: $\sqrt{R^2 + z^2 - 2Rz} = (R - z)$ if $R > z$, but it's $z - R$ if $R < z$.]

\subsection*{Solution}









  % THE ELECTRIC FIELD
\chapter{DIVERGENCE AND CURL OF ELECTROSTATIC FIELDS}\label{ch:2}

\section{Field Lines, Flux, and Gauss's Law}\label{sec:2.1} 
In principle, we are \tit{done} with the subject of electrostatics. Equations \ref{eq:1.5} and \ref{eq:1.10} tell us how to compute the electrical field produced by a statical arrangement of point charges or charge distributions, respectively. Eq. \ref{eq:1.6} then tells us what the force on a charge $Q$ placed in this field will be. Unfortunately, the integrals involved in computing $\vb{E}$ can be formidable, even for reasonably simple charge distributions. Much of the rest of electrostatics is devoted to assembling a bag of tools and tricks for avoiding these integrals. It all begins with the divergence and curl of $\vb{E}$.

\begin{equation}\label{eq:B.11}
\end{equation}


\section{The Divergence of E}\label{sec:2.2}


\section{Applications of Gauss's Law}\label{sec:2.3}


\section{The Curl of E}\label{sec:2.4}

 
  % DIVERGENCE AND CURLS OF ELECTROSTATIC FIELDS
\chapter{ELECTRIC POTENTIAL}\label{ch:3}

\section{Introduction to Potential}\label{sec:3.1} 

In principle, we are \tit{done} with the subject of electrostatics. Equations \ref{eq:1.5} and \ref{eq:1.10} tell us how to compute the electrical field produced by a statical arrangement of point charges or charge distributions, respectively. Eq. \ref{eq:1.6} then tells us what the force on a charge $Q$ placed in this field will be. Unfortunately, the integrals involved in computing $\vb{E}$ can be formidable, even for reasonably simple charge distributions. Much of the rest of electrostatics is devoted to assembling a bag of tools and tricks for avoiding these integrals. It all begins with the divergence and curl of $\vb{E}$.





\section{Comments on Potential}\label{sec:3.2}

\section{Poisson's Equation and Laplace's Equation}\label{sec:3.3}

\section{The Potential of a Localized Charge Distribution}\label{sec:3.4}

\section{Boundary Conditions}\label{sec:3.5}

 
  % ELECTRIC POTENTIAL 
\chapter{WORK AND ENERGY IN ELECTROSTATICS}\label{ch:4}

\section{The Work It Takes to Move a Charge}\label{sec:4.1} 

In principle, we are \tit{done} with the subject of electrostatics. Equations \ref{eq:1.5} and \ref{eq:1.10} tell us how to compute the electrical field produced by a statical arrangement of point charges or charge distributions, respectively. Eq. \ref{eq:1.6} then tells us what the force on a charge $Q$ placed in this field will be. Unfortunately, the integrals involved in computing $\vb{E}$ can be formidable, even for reasonably simple charge distributions. Much of the rest of electrostatics is devoted to assembling a bag of tools and tricks for avoiding these integrals. It all begins with the divergence and curl of $\vb{E}$.





\section{The Energy of a Point Charge Distribution}\label{sec:4.2}

\section{The Energy of a Continuous Charge Distribution}\label{sec:4.3}

\section{Comments on Electrostatic Energy}\label{sec:4.4}


 
  % WORK AND ENERGY IN ELECTROSTATICS
\chapter{CONDUCTORS}\label{ch:5}

\section{Basic Properties}\label{sec:5.1} 

In principle, we are \tit{done} with the subject of electrostatics. Equations \ref{eq:1.5} and \ref{eq:1.10} tell us how to compute the electrical field produced by a statical arrangement of point charges or charge distributions, respectively. Eq. \ref{eq:1.6} then tells us what the force on a charge $Q$ placed in this field will be. Unfortunately, the integrals involved in computing $\vb{E}$ can be formidable, even for reasonably simple charge distributions. Much of the rest of electrostatics is devoted to assembling a bag of tools and tricks for avoiding these integrals. It all begins with the divergence and curl of $\vb{E}$.





\section{Induced Charges}\label{sec:5.2}

\section{Surface Charge and the Force on a Conductor}\label{sec:5.3}

\section{Capacitors}\label{sec:5.4}


 
  % CONDUCTORS
% C -- Potentials
% D -- Electric Fields in Matter
% E -- Magnetostatics
% F -- Magnetic-Fields-in-Matter
% G -- Electrodynamics
% H -- Conservation Laws
% I -- Electromagnetic Waves
% J -- Radiation
% K -- Electrodynamics and Relativity
% L -- Potentials and Fields

\appendixpage
\appendix
% M -- Helmoltz-Theorem

\backmatter%%%%%%%%%%%%%%%%%%%%%%%%%%%%%%%%%%%%%%%%%%%%%%%%%%%%%%%
%%%%%%%%%%%%%%%%%%%%%%%%% referenc.tex %%%%%%%%%%%%%%%%%%%%%%%%%%%%%%
% sample references
% 
% Use this file as a template for your own input.
%
%%%%%%%%%%%%%%%%%%%%%%%% Springer-Verlag %%%%%%%%%%%%%%%%%%%%%%%%%%

%
% BibTeX users please use
% \bibliographystyle{}
% \bibliography{}
%
% Non-BibTeX users please use
\begin{thebibliography}{99.}
%
% and use \bibitem to create references.
%
% Use the following syntax and markup for your references
%
% Monograph
\bibitem{Griffiths_4th} D.J. Griffiths (2017)
Introduction to Electrodynamics. Cambridge University Press, Cambridge

% Monograph
\bibitem{Felsager_1981} B. Felsager (1981)
Geometry, Particles and Fields. Odense University Press

% Monograph
\bibitem{BudakFomin_1973} B.M. Budak, S.V. Fomin (1973)
Multiple Integrals, Field Theory and Series. Mir Publishers, Moscow

% Monograph
\bibitem{Postnikov_II_1982} Mikhail Postnikov (1982)
Lectures in Geometry, Semester II. Linear Algebra and Differential Geometry. Mir Publishers, Moscow

\end{thebibliography}

\printindex

%%%%%%%%%%%%%%%%%%%%%%%%%%%%%%%%%%%%%%%%%%%%%%%%%%%%%%%%%%%%%%%%%%%%%%

\end{document}





