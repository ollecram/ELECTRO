\chapter{CONDUCTORS}\label{ch:5}

\section{Basic Properties}\label{sec:5.1} 

In principle, we are \tit{done} with the subject of electrostatics. Equations \ref{eq:1.5} and \ref{eq:1.10} tell us how to compute the electrical field produced by a statical arrangement of point charges or charge distributions, respectively. Eq. \ref{eq:1.6} then tells us what the force on a charge $Q$ placed in this field will be. Unfortunately, the integrals involved in computing $\vb{E}$ can be formidable, even for reasonably simple charge distributions. Much of the rest of electrostatics is devoted to assembling a bag of tools and tricks for avoiding these integrals. It all begins with the divergence and curl of $\vb{E}$.





\section{Induced Charges}\label{sec:5.2}

\section{Surface Charge and the Force on a Conductor}\label{sec:5.3}

\section{Capacitors}\label{sec:5.4}


 
