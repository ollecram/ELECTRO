\chapter{Simultaneous Events; Synchronized Clocks}
\label{ch:Morin_05}
The puzzlement we feel at the fact that a given pulse of light has the same speed in both the track frame and the train can be traced to a deeply ingrained misconception about the fundamental nature of time. Until we learn otherwise -- and prior to Einstein in 1905, nobody had learned otherwise -- we implicitly believe that there is an absolute meaning to the simultaneity of two events that happen in different places, independent of the frame of reference in which the events are described. This assumption is so pervasive in our view of the world that it is built into the very language we speak, making it difficult to reexamine the question of what it actually means to assert that two events in different places are simultaneous.

Before embarking on such a reexamination, it is necessary to take a careful look at the term \tit{event}, which plays a fundamental role in the relativistic description of the world. An event is something that happens at a definite place at a definite time. It is, if you like, the space-time generalization of the purely spatial geometric notion of point.

If you look at this a little more closely,you realize that, like the concept of a point, the concept of an event is an idealization. No object we can actually get our hands on has the property of zero spatial extension that characterizes a geometric point, and no process we will be talking about has zero extension in both space and time.

(Although zero extension in time is captured by the word \quotes{instantaneous,} I know of no English word -- other than \quotes{pointlike,} if that is a word -- that signifies zero extension in space.) Whether or not we wish to view something as an event depends on the spatial and temporal differences we want to discriminate between. If, for example, the relevant time scale is years and the relevant distance scale is hundreds of kilometers, then it makes sense to view the meeting of a class between 1:25 and 2:40 in room 115 of Rockefeller Hall on the Cornell University campus in Ithaca, New York, as an event (at least in frames of reference that are not moving too rapidly with respect to the Earth). But if the relevant scales are minutes and feet, it does not. So a phenomenon can be viewed as constituting a single event in a given frame of reference, it its temporal and spatial extension in that frame are both small compared with all other times and distances of interest. All of the events we will be examining will qualify as events in all the frames of reference we are interested in -- they can be viewed as space-time points.

How can we decide whether two different events, happening in different places, that are simultaneous in the train frame are also simultaneous in the track frame? To be concrete, suppose one event consists of quickly making a tiny mark on the tracks (as they speed past) from the rear of the train, and the other consists of doing the same from the front. The two events could be anything else you like $\cdots$ But since it will be useful to mark the spot along the tracks where each event occurs, it is simplest to take the two events to be nothing more that two acts of marking the tracks.

How does Alice, who used the train frame, persuade herself that the two marks on the tracks are made at the same time? Well, she could provide both ends of the train with accurate clocks and confirm that each mark was made when the clock at its end of the train read noon. But how can she be sure that the two clocks are properly \tit{synchronized}? How does she know that they both read noon \tit{at the same time}?

Trying to check the simultaneity of the two events with clocks gets Alice nowhere, since confirming that the clocks are properly synchronized requires her to have precisely what we're trying to construct: a way to confirm that two events in different places  -- in this case, each clock reading noon -- happen \tit{at the same time}.

This is a centrally important point. It is useful to have two clocks in different places only if they are properly synchronized. But \quotes{synchronized} means that the clocks have the same reading \tit{at the same time}. Therefore you need a way to check that two events in different places are simultaneous, if you want to check that the two clocks in different places are synchronized. The question of whether clocks in different places are synchronized, and the question of whether events in different places are simultaneous, are simply different aspects of the same fundamental puzzle. You can answer one question if and only if you can answer the other.

Let uis try again. Alice could bring the two clocks together, directly confirm that they read the same time when they're both in the same place, and then have them carried to the two ends of the train. But how does she know how fast each clock was running as it was carried to its end of the train?Faced with a phenomenon as peculiar as the constancy of the velocity of light, it would be rash to assume she knew anything about the rate at which a clock runs when it is moving. (We will learn how to deal with this in chapter \ref{ch:Morin_06}.) The straightforward way to check on whether the clocks have done anything peculiar while being carried to the two ends of the train would be to compare what each reads when it gets there with the reading of a stationary clock at that end of the train. But we can only do this if those two stationary clocks are properly synchronized, which brings us right back to the same problem.

Ah, but suppose, even though we don't know how it might affect their rates, the two clocks start at the exact middle of the train and are carried to the two ends in exactly the same way except, of course, that one of the two clock-transportation procedures is executed in the opposite direction from the other. Then, however erratically its motion caused one clock to behave during the journey, the other, having experienced just the same kind of trip, would have run erratically in exactly the same way. So even if they lost or gained time because of their motion, the two clocks would still agree when they arrived at the ends of the train. That method of providing both ends of the train with synchronized clocks ought to work. And it does! In the train frame.

But now we are faced with another problem Even if Alice did cleverly use two such centrally synchronized, symmetrically transported clocks to confirm that two events at the two ends of the train were simultaneous in the train frame, Bob, using the track frame, would not agree that the two clock-transportation procedures were identical, because in the track frame motion toward the front of the train is \tit{not} insignificantly different from motion toward the rear.

Bob \tit{would} agree with Alice that the reading of one of her clocks, the instant it arrived at its end of the train, was the same as the reading of the other clock, the instant it arrived at the other end, since Alice and Bob can't disagree about things that happen \tit{both} in the same place \tit{and} at the same time. But Bob need not agree with Alice that the identical readings of the clocks meant that it took an identical amount of time for each clock to get to its end, since the clocks were not moving symmetrically in the track frame and therefore might be running at different rates during their journeys from the center to the ends. So Bob will have to do a rather elaborate calculation to determine whether each clock reached its end of the train \tit{at the same time} as the other clock. That calculation would have to figure out how fast each of the clocks was moving in the track frame, and how far it had to go. It could get quite complicated. It can, howvere, be done, and it leads to a remarkable conclusion that we can extract by a much simpler stratagem.

Teh simpler stratagem, like our earlier method for finding the relativistic velocity addition law, avoids all worries about possibly misbehaving clocks by using in the train frame a nethod to check teh simultaneity of two events in different places that makes no use of clocks at all. This method can easily be analyzed in the track frame too. The analysis relies only on the fact that the speed of light is $c$ -- 1 foor per nanpsecond -- regardless of the direction the light is moving in and regardless of the frame of reference in which the speed is measured.

Why, you might ask, should we build such a strange fact into our procedure for determining whether two spatially separated events are simultaneous? If you do ask, then you have forgotten why we started worrying about whether simultaneity might depend on frame of reference. It was our hope that this might lead us to a clearer understanding of the constancy of the velocity of light. Whet we are doing is perfectly sensible. We \tit{start} from the strange fact of the constancy of the velocity of light, and see what it \tit{forces} us to conclude about the simultaneity of events. We shall find that it forces us to conclude that whether two events in different places are simultaneous does indeed depend on the frame of reference in a way that can be stated sim,ply and precisely.

Note first that Alice, on the train and using the train frame, can easily exploit the fact that light travels with a definite speed $c$ to arrange that the two marks on the tracks are made from the two ends of the train simultaneously. She places a lamp exactly halfway along the train, and then turns on the lamp. Light from the lamp races toward both ends of the train at the same speed $c$. Since the light has to travel the same distance (half the length of the train) in either direction, and moves at the same speed in either direction, it arrives at the two ends of the train \tit{at the same time}. So if the making of each mark on the tracks is triggered by the arrival of light, the marks will certainly be made at the same time. 

Alice has thus managed to produce a pair of events in different places that are simultaneous (in the train frame) without having to make any problematic use of clocks. This procedure, of course, works for any two signals that move from the center of the train to the two ends, as long as they move at the same speed. If the common speed of the signals is not the speed of light, however, the track frame analysis cannot be as concise as it is with light signals, because the two signals have different speeds in the track frame. The two speeds can be found with the help of the relativistic velocity addition law, and with considerably more effort one can generalize to arbitrary signals the analysis based on light signals. One then finds (see pp. 54-56) that this more general way to produce two simultaneous events in the train frame leads to a relation between the track-frame time and track frame distance between the two events that is exactly the same as the relation we now extract with much less effort using light signals. 

The question facing us is how this procedure, which convinced Alice that the two events are simultaneous in the train frame, will be interpreted by Bob, who uses the track frame. Bob will certainly agree that the lamp is indeed in the center of the train, for if the train is 100 cars long and the lamp is bolted down between cars 50 and 51, then there is no denying that it is indeed in the center\footnote{This is true even if the length of the train in the track frame is altered by its motion -- as we shall see in chapter \ref{ch:Morin_06} it is -- because whatever that alteration may be, it would be the same for both the front half and the rear half of the train.}. But in the track frame when the lamp is turned on and the light the light starts to move toward the two ends, the rear of the train moves toward the place where the light originated and the front moves away from it. Since the speed of the light in either direction is $c$ in the track frame -- remember we are using the strange fact that the speed of the light is 1f/sec in the track frame even though it is also 1 f/sec in the train frame -- in the track frame it will clearly take light less time to reach the rear of the train, which is heading toward the light to meet it, than it will take the light to reach the front of the train, which is running away as the light pursues it.

So Bob must conclude thatthe light reaches the rear of the train before it reaches the front, and therefore that the mark in the rear is made before the mark in the front. The very same evidence that convinces Alice, using the train frame, that the marks are made simultaneously, convinces Bob, using the track frame, that they are not made simultaneously. \tit{Whether or not two events at different places happen at the same time has no absolute meaning -- it depends on the frame of reference in which the events are described.} \footnote{Notice, in passing, that if you interchange the two words \quotes{place} and \quotes{time} in this sentence, the shocking assertion becomes quite humdrum: \tit{Whether or not two events at different times happen at the same place has no absolute meaning -- it depends on the frame of reference in which the events are described.}}

People using the train frame, for whom the marks are made simultaneously, cau use the arrival of the light signals to synchronize clocks at the front and rear of the train. Since people in the track frame maintain that the mark in the rear is made \tit{before} the mark in the front, the track people would also maintain that the synchronization procedure used by the train people had actually resulted in the clock in the front of the train being behind the clock in the rear. 

It is easy to find a precise quantitative measure of these disagreements, Let's analyze Alice's procedure for making marks simultaneously at both ends of the train, from the point of view of Bob in the track frame, where the train moves with speed $v$. It is convenient to call the length of the train $L$. I emphasize that by $L$ we mean the length of the train \tit{in the track frame}. Although we are used to assuming that the length of an object is independent of the frame in which it is measured, we can no longer take this for granted and, as noted earlier, we will indeed find it to be a false assumption. 

In part 1 of figure 5.1, the light is turned on in the middle of the train, and the two pulses of light -- which we shall again call photons -- start moving from the center toward the fromnt and the rear. 

Part 2 of figure 5.1 shows things a time $T_r$ later, just as the rearward moving photon meets the rear of the train, which has been moving toward it. At the instant of encounter a mark is made at the place on the tracks where the meeting takes place. During the time $T_r$ the photon (which moves with speed $c$) has covered a distance $cT_r$. That distance is just half the length of the train, reduced by the distance the rear of the train (which is moving with speed $v$) has moved toward the photon in the time $T_r$. So
\begin{equation}\label{eq:Morin_05.1}
c T_r = \frac{1}{2} L - v T_r \,.
\end{equation}

 

