\chapter{Morin -- The Speed of Light}
\label{ch:Morin_03}

When you turn on a light, how long does it take the light to get from the bulb to the things it illuminates?
Galileo apparently tried to answer this by stationing two people on top of two mountains, a large distance $D$ apart. Alice opens her lantern, Bob opens his the instant he sees Alice's, and Alice notes the time $T$ that passes between the moment she open hers and the moment she sees the light returning from Bob's.

To get the speed $c$ with which the light moves from her mountaintop to Bob's and back again, Alice just divides twice the distance between the mountains by the delay time $T$ to get
\begin{equation}\label{eq:Morin_03.1}
c = 2 D / T\,. 
\end{equation}

$\cdots$

Three centuries later, Galileo's unsuccessful attempt was realized by replacing the two mountains by the Earth and the Moon. The Moon is so far away that it takes radar more than $2$ seconds to get there and bounce back. But by then the speed of light was known to high precision by other methods. Note, by the way, that the speed of radar is the same as the speed of light. Both are forms of electromagnetic radiation and all forms of electromagnetic radiation (light, radar, radio, x-rays, gamma rays, TV signals, for example) have the same speed in empty space. 

Light travels so fast that to measure its speed either you have to let it travel an enormous distance, or you have to make very accurate measurements of extremely tiny intervals of time. The very first successful estimate of the speed of light came from using astronomical distances. Galileo, who plays many roles in the story of relativity, discovered the four major moons of Jupiter earlier in the 17th century. In 1676 careful observations by Ole Romer, of the regularly occurring moments when a moon disappeared within Jupiter's shadow, revealed that sometimes these Jovian lunar eclipses lagged behind schedule by about 10 minutes, and sometimes they came in 10 minutes ahead. It was noted that they were ahead of schedule when the Earth was closest to Jupiter and behind when the Earth was furthest away. One concludes that the time it takes light to cross the orbit of the Earth must be something like 20 minutes. This gives an estimate of several hundred thousand kilometers per second for the speed of light. 

$\cdots$

Today we have highly sophisticated ways to measure the speed of light and know that it is $299,792,458$ meters per second (m/sec). Furthermore,, that is what it always shall be, because as of 1983 the meter has been \tit{defined} to be not the distance between two scratches on a platinum-iridium bar lovingly cared for in Paris, but as the distance light travels in $1/299,792,458$ of a second. Our unit of length (the meter) now is tied to our unit of time (the second). You might think that since the speed of light is now fixed forever by definition of the meter, this means that there is no longer any point in striving to measure it more and more accurately. But such improved experiments now provide more and more accurate measurements of the length of a meter--better and better standards of length. The experiments remain just as important as they used to be. What has changed is how we describe what we have learned from them. 

There are two useful numerical near coincidences associated with the speed of light being 299,792,458 m/sec:

First, the number is extremely close to 300 million m/sec or 300,000 kilometers per second (km/sec). Physicists are very used to taking it to be $3 \times 10^8$ m/sec. 

Second, the corresponding English unit is about 186,000 miles per second. Since there are 5,280 feet in a mile, this works out to about 982,000,000 feet per second. Thus, within 2 percent accuracy the speed of light is 1 billion feet per second or, in more practical units, 1 foot per nanosecond. Note that people working with the metric system can take \tit{1 light nanosecond} (which amounts to 30 cm.) as an equivalent and convenient unit of length. Nanoseconds and feet are also relevant to the accuracy of the global positioning system (GPS), which uses satellites broadcasting time signals every nanosecond. The signals are therefore spaced a foot (actually, 30 cm.) apart as they arrive at the surface of the earth, establishing the foot as a measure of the accuracy of the system.

In thinking about relativity it is very convenient to measure speeds in units that assign an especially simple value to the speed of light. In 1959, the foot was officially defined to be exactly 0.3048 of a meter. Since the speed of light is exactly 299,792,458 m/sec, if only people in 1959 had defined the foot to be 0.299792458 of a meter, a mere 1.64 percent shorter, then the speed of light would now be \tit{exactly} 1 f/ns. This unit of length will prove to be so useful, that for the purposes of this book \tit{I hereby redefine the foot:}

Hencefort, by 1 \tit{foot} we shall mean \tit{1 light nanosecond} (the distance light travels in a nanosecond) which--for people used to the metric system--is extremely close to being \tbi{30 cm.}

For comparing with lesser speeds, it can sometimes be convenient to think of 1 foot per nanosecond as 1,000 feet per microsecond. Since the speed of sound in ordinary air is about 300 m/sec, hence 1,000 f/sec, we get
\begin{equation*}
\begin{aligned}
\text{Speed of light:}\: &\approx	1 f/\text{nanosecond} = 1,000 f/\text{microsecond}\,,	\text{and}	\\
\text{Speed of sound:}\: &\approx 300 m/\text{second} = 1,000 f/\text{second}\,, \text{hence}
\end{aligned}
\end{equation*}
\tit{light is a million times faster than sound}.  

There is something peculiar and, when you think about it, quite extraordinary about the unqualified assertion that the speed of light in empty space is 299,792,458 m/sec. Ordinarily, when you specify a speed to such high precision and indeed when you mention any speed at all, the question \quotes{with respect to what} comes irresistibly to mind. After all, the speed of an object depends on the frame of reference in which that speed is measured. As we have repeatedly noted, a ball that Alice throws while riding on a uniformly moving train has one speed with respect to the train but quite another speed with respect to the tracks. In the case of light there are two obvious possible answere to the question \quotes{with respect to what?}:
\\\\\tbf{First Obvious Answer}\\

The speed of light is 299,792,458 m/sec with respect to the source of light.

$\cdots$

This reasonable answer is contradicted by our current understanding of the electromagnetic character of light. In the 19th century there was a great unification of the laws of electricity and magnetism, completed by the Scottish physicist James Clerk Maxwell. Maxwell's equations led to the prediction that when electrically charged particles jiggle back and forth (as they do, for example, in a hot wire) they must emit radiant energy that travels at a speed of about 300,000,000 m/sec. Since this speed was numerically indistinguishable from the speed of light, it was natural to identify light with a particular form of such radiation (associated with a very rapid jiggling--almost a million billion times a second). Maxwell's equations imply quite unambiguosly that this speed does not depend on the speed of the source of the radiation. According to the theory, the speed of the light is the same whether the chunk of matter in which the charged particles are jiggling is stationary or moving toward or away from the direction in which the light is emitted.

People had also noted that the regularity of certain astronomical motions as observed from Earth was quite unaffected by whether the source of the light that enabled us to observe them was moving toward or away from us. So there was both theoretical and astronomical evidence that the speed of light did not depend on the speed of its source. 
\\\\\tbf{Second Obvious Answer}\\

With respect to a light medium (historically called the ether), 299,792,458 m/sec is the speed of light \tit{in vacuum}. Light goes significantly slower in transparent media like water or glass, and a little bit slower in air. The ether, then, would be a sort of irreducible residue of otherwise empty space--what remains after you've removed everything it is possible to remove.

The analogy now is not to bullets from a gun, but to sound, which is a wave in the air. Like the speed of light, the speed of sound does not depend on the speed of the source of the sound. Sound moves at a definite speed with respect to the air, whose vibrations constitute and transmit that sound. If light is a vibration of something called the ether, then the speed of light should be with respect to that ether. 

Since the Earth moves about the sun at a brisk clip of 30 km/sec in different directions, depending on the time of year, and the sun moves briskly about the center of our galaxy, it would be a remarkable coincidence if the Earth just happened to be stationary in the rest frame of the ether. One would expect there to be a kind of \quotes{ether wind} blowing past the Earth, leading to a dependence of the speed of light on Earth on the direction of that wind. The speed of light on Earth into the direction from which the ether wind was blowing ought to be a bit less than its speed along the direction of the wind. Efforts to detect such a difference failed to yield a clear-cut result, most famously in the Michelson-Morley esperiment of 1887. The measurement demonstrated that if the speed of light was fixed with respect to an ether, then the Earth, in spite of its complicated motion with respect to the galaxy, was improbably close to being at rest in the rest frame of that ether at the time the experiment was performed. Stubborn people considered the possibility that the Earth dragged the ether in its neighborhood along with it. But if that were so, then the apparent positions of the stars in the sky should shift through the year depending on the way in which the ether was being dragged by the Earth. No such shift was observed. 

The importance of the Michelson-Morley experiment in the historical development of relativity has been debated. Einstein apparently alludes to it in his famous 1905 paper setting forth relativity, but only once and then only in passing: \quotes{Examples of this sort, together with \tit{unsuccessful attempts to determine any motion of the earth relative to the `light medium,'} lead to the conjecture that...} (my italics). The reference is little more than parenthetical. Such attempts had to be mentioned, because had they been successful and unambiguosly demonstrated a significant direction dependence to the velocity of light on Earth, the theory of relativity would have been dead on arrival.

The \quotes{examples of this sort} that Einstein offered as the real motivation for his reexamination of the nature of time were all examples of the fact that the electric and magnetic behavior of matter is consistent with the principle of relativity, in spite of the then widespread view that there actually was a preferred inertial frame of reference for electromagnetic phenomena--the frame in which the ether was stationary. The equations of electromagnetic theory were held by many to be valid in that frame of reference and no other. Einstein noted, in effect, that even if this were so, a broad range of electromagnetic phenomena seemed to play out in exactly the same way in frames of reference other than the frame in which the ether was stationary. This led him to postulate that the laws of electromagnetism were, in fact, rigorously valid in arbitrary inertial frames of reference. If this postulate was valid, then, Einstein noted, \quotes{the introduction of a `luminiferous ether' will prove to be superfluous} because there would be no way to determine the rest frame of the ether by any physical experiment involving electromagnetic phenomena. It is this specific postulate--that what we now call the principle of relativity applies to electromagnetism as well as to Newtonian mechanics (where everybody agreed that it was indeed valid)--that Einstein named the \quotes{principle of relativity} (\tit{Prinzip der Relativitat}). 

Now if Maxwell equations are valid in any inertial frame of reference, and if they predict that electromagnetic radiation and light in particular propagate at a fixed speed that is independent of the speed of the source of the light, then light must propagate at the same speed in any inertial frame of reference. The answer to the question \quotes{with respect to what?} is, as we now know, \quotes{with respect to any inertial frame of reference}. The speed of light in vacuum is simply 299,792,458 m/sec in any inertial frame of reference, regardless of how fast the source of the light is moving, and regardless of the choice of frame of reference in which the measurement of the speed of light is made. If, for example, you race after the light in a rocket at 10 km/sec, you do not reduce its speed away from you to 299,782 km/sec. It still recedes from you at 299,792 km/sec. I emphasize that it is only the speed of light in \tit{vacuum} that has this special property. The speed of light in water \tit{does} depend on how fast you are moving with respect to the water, though not in an obvious way, as we shall see. Indeed, what is special here is not light, but the speed $c = 299,792,458$ m/sec. When one says \quotes{speed of light} without any qualification, one almost always means the speed of light in vacuum, 299,792,458 m/sec.

How can this be? How can there be a speed $c$ with the property that if something moves at speed $c$ then it must have speed $c$ in any inertial frame of reference? This fact--known as the \tit{constancy of the speed of light} is highly counterintuitive. Indeed, \quotes{counterintuitive} is too weak a word. It seems downright impossible. One of the central aims of this book is to remove this sense of impossibility and to see how it can, in fact, make perfect sense.

$\cdots$

To make sense of the constancy of the speed of light we must look very closely and critically at what it actually means to \quotes{have a speed} with respect to a particular frame of reference. When we say that an object moves uniformly with a certain speed $s$, we mean that it goes a certain distance $D$ in a certain time $T$ and that the distance and time are related by $D/T = s$. We are thus led to examine carefully how one actually measures such distances and how one actually measures such times.

Let $P$ be a valid procedure for carrying out the time and distance measurements that allow one to determine the speed of an object in a given inertial frame. Let Bob, carrying out the procedure $P$ in the frame of reference of a space station, measure the speed of a pulse of light as it zooms off into space. He finds that it moves at about 299,792 km/sec. Suppose Alice flies swiftly after the light at a speed Bob determines to be 792 km/sec. Bob will then (correctly) note that in each second the light gets an additional 299,792 km away from him and Alice gets an additional 792 km away, so that the distance between Alice and the light is growing at only 299,000 km/sec. But if Alice carries out the same procedure $P$ in the frame of reference of her rocket ship, she will find that the speed of light is 299,792 km/sec, so that in her own frame of reference the distance between her and the light is still growing at the full 299,792 km/sec. 

How are we to account for this discrepancy? Obviously the methods Alice uses to measure distances and times must be different from those used by Bob. But don't they use exactly the same procedure $P$? Yes, but you have to think about what \quotes{exactly the same} means. If Bob, for example, uses clocks that are stationary in the frame of his space station to measure times, then if Alice uses exactly the same procedure in her frame of reference, she must use clocks that are stationary in the frame of \tit{her} rocket ship. Thus in Bob's frame of reference Alice's clocks are moving, while his are not, and, of course, vice versa: in Alice's frame Bob's clocks are moving and hers are not. Similar considerations apply to the meter sticks they might use to measure distances. The not terribly subtle but easily ovelooked point is that Bob's procedure \tit{as described in Bob's frame of reference} must be exactly the same as Alice's procedure \tit{as described in Alice's frame of reference}. But Alice's procedure as described in \tit{Bob's} frame of reference is not exactly the same as Bob's procedure as described in \tit{Bob's} frame of reference.

It is this difference that makes it possible for either Bob or Alice to account, in an entirely rational way, for the discrepancy in their conclusions. The fact that Alice and Bob, using different frames of reference, both find exactly the same speed for one and the same pulse of light appears paradoxical only if you make several assumptions about the relation between the clocks and meter sticks used by Alice and Bob. Before 1905, everybody implicitly made all of these assumptions:
\begin{enumerate}[1.]
\item The procedure Alice uses to synchronize all the clocks in her frame of reference gives a set of clocks that Bob agrees are synchronized when he tests them against a set of clocks that he has synchronized using the same procedure in his own frame of reference. (\quotes{Same} here, as earlier, is to be taken to mean that what Bob does has the same description in his own frame of reference as Alice's procedure has in hers.)
\item The rate of a clock, as determined in Bob's frame of reference, is independent of how fast that clock moves with respect to Bob.
\item The length of a meter stick, as determined in Bob's frame of reference, is independent of how fast that meter stick moves with respect to Bob.
\end{enumerate}

If any of these assumptions is false, that we must reexamine the nonrelativistic velocity addition law--the rule specifying how the speed of an object changes as one changes the frame of reference in which its speed is measured. Today we know that \tit{all three} of these assumptions are false. The special theory of relativity gives a quantitative specification of how they fail, and how, when they are suitably corrected, one emerges with a simple and coherent picture of space and time measurements that is entirely in accord with the existence of an invariant speed--a speed that is the same in all inertial frames of reference.

The traditional (and simplest) way to arrive at this picture--the way we shall be taking and the way Einstein used--is simply to accept as a working hypothesis that in any inertial frame of reference, any procedure that correctly measures the speed of light in vacuum \tit{must} give 299,792,458 m/sec. We shall accept the strange fact that if Alice and Bob both measure the speed of one and the same pulse of light, they will both find it to be 299,792,458 m/sec, even though Alice and her measuring instruments may be moving in the same direction as the light with respect to Bob and his. By tentatively accepting this peculiar fact, and insisting that the principle of relativity must remain valid, we will be able to \tit{deduce} the precise way in which each of the three assumptions about the behavior of moving clocks and meter sticks must be modified. Once this is done and the corrected version of these three assumptions are identified and understood, the strange fact will cease to appear strange. More importantly, we will have acquired a firm understanding of the new and wonderful subtleties Einstein first realized about the nature of time. 

This remarkable property of light--that its speed does not depend on the frame of reference in which it is measured--is today called the \tit{principle of the constancy of the velocity of light.} The special theory of relativity is said to rest on two principles: the principle of relativity and the principle of the constancy of the velocity of light. In Einstein's great 1905 paper, he did not use the word \tit{Prinzip} for this second principle (as he did for the first). He characterized each principle as a \quotes{postulate} (Voraussetzung). His second postulate was that light in empty space moves with a velocity that is independent of the velocity of the body that emitted the light. This is tantamount to the second principle when it is conjoined with the first, which Einstein stated as the postulate that the concept of absolute rest has no more meaning for electromagnetic phenomena than it does for phenomena in ordinary Newtonian mechanics.




  

 