\chapter*{Prefazione}\label{Wald_EM_00}
\thv{From R. M. Wald -- Advanced Classical Electromagnetism, 2022}\\

Questo libro è nato dal corso avanzato del primo trimestre in Elettromagnetismo tenuto agli studenti dell'Università di Chicago nell'inverno del 2018. Erano trascorsi decenni da quando avevo precedentemente insegnato questa materia, per cui mi ci sono accostato con occhi nuovi, ed è stato naturale per me cercare di ripensare a come presentarla a livello universitario. Nel farlo, mi è risultato evidente che l'usuale metodo quasi-storico di presentare la materia induce modi non appropriati di concepire l'elettromagnetismo. Pertanto, per evitare di partire con il piede sbagliato, decisi di spendere le prime lezioni del corso per descrivere quelli che nel capitolo 1 di questo libro chiamo con il termine di \quotes{miti} riferiti all'elettromagnetismo. Constatai che partendo in questo modo,  divenne molto più facile presentare in modo diretto gli argomenti in modo chiaro e conciso, senza dover effettuare cambi di prospettiva nel  corso del loro sviluppo. Ho insegnato il corso nuovamente nel successivi 3 anni, fornendo appunti delle lezioni alla classe. Gli appunti di queste lezioni sono ora state sviluppate in questo libro.

Il primo capitolo di questo libro è perciò una introduzione all'elettromagnetismo alquanto fuori dall'ordinario. Invece di cominciare con la forza tra particelle cariche, e discutere come questo da luogo ad un concetto di \quotes{campo}, e così via, il mio proposito nel capitolo 1 è di spiegare agli studenti come essi dovrebbero pensare all'elettromagnetismo da una prospettiva moderna e matematicamente precisa. I punti principali esposti in questo capitolo sono che (i) i potenziali, non le intensità dei campi, sono le variabili dinamiche fondamentali nell'elettromagnetismo; (ii) le proprietà energia ed impulso associate al campo elettromagnetico sono parte essenziale della formulazione della teoria e non possono essere derivate da argomentazioni sul \quotes{\tit{lavoro fatto}} dal campo; (iii) non si deve pensare ai campi elettromagnetici come \tit{prodotti} da cariche; e (iv) a livello fondamentale nell'elettrodinamica classica la materia carica deve essere considerata come distribuita in modo continuo piuttosto che essere costituita da cariche puntiformi. Molti di questi punti non possono essere chiariti appieno se non negli ultimi capitoli del libro--particolarmente i capitoli 9 e 10--ma il mio intento è di esporre queste idee in modo sufficientemente chiaro ed esplicito nel capitolo 1 in modo da poter poi mantenere questi capisaldi nel prosieguo del libro senza ulteriori giustificazioni. 

Gli argomenti trattati nei capitoli 2-7 sono quelli che verrebbero normalmente esposti in ogni corso avanzato di Elettromagnetismo. L'Elettrostatica è trattata nel capitolo 2, ma a partire dall'equazione di Poisson, non dalla legge di Coulomb. I materiali dielettrici in elettrostatica sono trattati nel capitolo 3, con un considerevole grado di attenzione a come eseguire la media su scale macroscopiche e al trattamento dell'energia. La Magnetostatica è trattata nel capitolo 4, con una discussione completa della differenza di segno tra magnetostatica ed elettrostatica nella energia di interazione di un dipolo in un campo esterno--e come questa si collega al cambiamento della massa a riposo di un magnete quando esso si muove in modo quasi-statico in un campo magnetico esterno. Elettrodinamica e radiazione sono discusse a fondo nel capitolo 5. In aggiunta agli argomenti normalmente trattati nei testi di Elettromagnetismo, io derivo in quel capitolo la formulazione ai valori iniziali per le equazioni di Maxwell. L'Elettrodinamica nei mezzi viene trattata nel capitolo 6, inclusa una discussione della mgnetoidrodinamica. L'approssimazione dell'ottica geometrica nella dinamica ondulatoria è presentata nella prima sezione del capitolo 7, seguita da una discussione di interferenza e coerenza ed un'analisi di due problemi nella diffrazione: diffusione da una palla dielettrica e propagazione e radiazione attraverso un'apertura.

La relatività ristretta è trattata nel capitolo 8. La relatività ristretta è alla base della formulazione della teoria elettromagnetica, quindi dovrebbe propriamente essere presentata all'inizio di un libro sull'elettromagnetismo, piuttosto che essere relegata in un capitolo verso la fine del libro. Tuttavia, la relatività ristretta rimane un argomento così poco familiare a gran parte degli studenti che ciò non è possibile. Molte esposizioni della relatività ristretta si concentrano sulle regole per applicare trasformazioni di Lorentz a certe quantità, senza fornire molte informazioni sul contenuto geometrico sottostante della teoria. All'opposto sarebbe naturale, per me che mi occupo di relatività generale, introdurre un livello di maggiore astrazione matematica e apparato geometrico di quanto sarebbe necessario per fornire una chiara descrizione della relatività ristretta. Ho posto una considerevole cura nella scrittura della sezione 8.1 in modo tale da introdurre la relatività ristretta in modo concettualmente chiaro senza introdurre un livello di astrazione maggiore di quello che ritengo essenziale. Questa sezione può essere letta in modo indipendente dal resto del libro, e spero che possa fornire da sola una utile introduzione alla relatività ristretta. 
Viene quindi esposta la formulazione dell'elettromagnetismo nel contesto della relatività ristretta, seguita da una discussione del moto di particelle cariche e della radiazione da una carica puntiforme in moto arbitrario.

Il capitolo 9 tratta l'elettromagnetismo come una teoria di gauge, con ciò portando la formulazione dell'elettromagnetismo in questo libro al livello di comprensione concettuale che è stato raggiunto alla metà del ventesimo secolo. Vi è un gap considerevole tra il modo in cui viene normalmente descritto il campo elettromagnetico e la sua interazione con la materia carica nei corsi di elettromagnetismo classico e quello in cui esso appare di fatto come una delle interazioni fondamentali nel modello standard della fisica delle particelle. Questo capitolo dovrebbe essere di aiuto a colmare questo gap.    

Infine, la nozione di carica puntuale viene discussa in dettaglio nel capitolo 10. Si mostra che è possibile considerare il limite, in senso matematicamente ben definito, in cui l'estensione di un corpo carico si riduce a zero, purché anche la carica e la massa del corpo vengano ridotte in modo proporzionale alla sua estensione. In questo limite si ottiene la forza di Lorentz. Le correzioni dovute alla \quotes{self-force} si possono poi calcolare perturbativamente in modo matematicamente rigoroso. Nella parte finale di questo capitolo viene trattato il problema di come descrivere in modo consistente il moto di un corpo carico tenendo conto delle correzioni dovute alla \quotes{self-force} senza introdurre soluzioni spurie (\quotes{runaway}).

Nel libro, ho cercato di formulare tutti i principali concetti ed i risultati della teoria dell'elettromagnetismo in modo chiaro e conciso. Tuttavia, non ho cercato di presentare un'ampia collezione di esempi o applicazioni. Queste caratteristiche del libro spiegano il fatto che la sua lunghezza sia circa un terzo di alcuni altri testi di elettromagnetismo avanzato con una simile copertura di argomenti. 

Ho cercato di presentare ogni cosa con un alto livello di precisione matematica. Sebbene mi sia sforzato di evitare distrazioni dovute an un eccessivo dettaglio matematico, non ho ecceduto nella semplificazione di alcuna proposizione contenuta nel libro ed ho cercato di essere attento ad inserire dei caveat appropriati quando delle formule o altri risultati sono validi solo sotto determinate condizioni restrittive. In parecchi casi, nei primi capitoli, ho aggiunto dei \quotes{commenti a lato} per spiegare alcuni punti matematici potenzialmente interessanti e rilevanti per il lettore ma non strettamente necessari per la discussione.

Una estesa varietà di problemi viene fornita per i capitoli 2-8. Uno degli scopi di tali problemi è quello solito di fornire agli studenti l'opportunità di mettere alla prova la propria comprensione dei concetti base introdotti nel capitolo. Vi è comunque un altro importante scopo aggiuntivo per alcuni dei problemi: presentare argomenti che non sono essenziali per lo sviluppo delle idee fondamentali del libro ma che sono, nondimeno, di rilevante interesse ed importanza. Alcuni esempi di tali argomenti trattati nei problemi sono impulso nascosto, effetto Hall, fasci Gaussiani, diffusione Thompson, fibre ottiche, parametri di Stokes e radiazione Cherenkov. Ho scritto questi problemi in modo tale che nella formulazione del problema siano esposti tanto i concetti chiave che i risultati chiave. Il lettore potrebbe in tal modo trovare in questi problemi una utile introduzione a questi argomenti.

La platea di utenti che ho in mente per questo libro include gli studenti universitari in fisica teorica, sebbene io spero che anche studenti in fisica sperimentale, ed altri ancora possano trovare il libro di loro interesse. Questo libro è scritto con l'assunzione che i lettori abbiano seguito un corso introduttivo in elettromagnetismo e che pertanto abbiano già sviluppato un certo livello di comprensione intuitiva dei campi elettrico e magnetico. Mi aspetto anche che i lettori abbiano una solida conoscenza del calcolo vettoriale, ma non assumo che possiedano un bagaglio matematico molto al di là di questo.

Io uso le unità SI dall'inizio alla fine del libro. Sfortunatamente, le unità SI hanno la assai spiacevole caratteristica di introdurre due costanti, $\epsilon_0$ e $\mu_0$, legate dalla relazione $\epsilon_0 \mu_0 c^2 = 1$, in cui $c$ è la velocità della luce. Ci sono ottime ragioni storiche per questa scelta. E' naturale assegnare una permittività elettrica $\epsilon$ e una permeabilità magnetica $\mu$ a molti materiali, e dunque naturale assegnare valori corrispondenti, $\epsilon_0$ e $\mu_0$, al vuoto. Fù dunque un successo veramente grande quello di Maxwell di riconoscere che le proprie equazioni richiedevano che disturbi dei campi elettrico e magnetico si propagassero nel vuoto con velocità $c=\sqrt{\epsilon_0 \mu_0}$  e che questi disturbi dovessero identificarsi con la luce. Tuttavia, questa relazione tra $\epsilon_0$, $\mu_0$ e $c$ significa che in queste costanti c'è ridondanza. Di conseguenza, formule espresse nelle unità SI possono essere trasformate in modo non banale utilizzando questa ridondanza. Per esempio, nelle unità SI, una delle equazioni di Maxwell si scrive abitualmente come $\div{\vb{E}} = \rho / \epsilon_0 $. Tuttavia, questa equazione potrebbe essere espressa altrettanto bene come  $\div{\vb{E}} = \mu_0 c^2 \rho$. Quest'ultima forma può sembrare piuttosto stridente, poiché sembra suggerire che la permeabilità del vuoto e la velocità della luce entrino in una delle equazioni fondamentali dell'elettrostatica. In ogni caso, si deve scegliere quali tra tali costanti usare in ciascuna formula. La convenzione usuale è di usare $\epsilon_0$ nell'equazione di Maxwell di cui sopra e di usare $\mu_0$ nell'equazione di Maxwell ove compare la densità di corrente $\vb{J}$. Tuttavia, questa convenzione non può essere mantenuta quando si scrivono le equazioni di Maxwell nella forma covariante della relatività ristretta, poichè in queste equazioni figura la $4-$corrente $J^{\mu}$, e non è sensato usare convenzioni diverse per diverse componenti di questo $4-$vettore. Infatti, dal capitolo 8 in poi, io sospendo completamente l'uso di $\epsilon_0$ ed uso $\mu_0$  e $c$ in tutte le formule. Per evitare il fastidio connesso a questa ridondanza di $\epsilon_0$, $\mu_0$ e $c$, nella versione originale delle mie lezioni adottai le unità di Gauss. Tuttavia, sebbene decenni orsono le unità di Gauss fossero piuttosto prevalenti, le unità SI sono attualmente utilizzate in modo quasi esclusivo. Perciò, il fastidio delle unità SI è più che compensato dalla non familiarità degli studenti con le unità di Gauss--così come dalla possibilità che qualcuno possa essere indotto dal mio libro ad acquistare apparati elettromagnetici della taglia sbagliata se le formule fossero scritte in unità di Gauss. Perciò, ho scelto di usare unità SI. 

Vettori nell'ordinario spazio $3$-dimensionale saranno denotati in grassetto (p.es., il campo elettrico sarà indicato con $\vb{E}$, come nel precedente paragrafo). Componenti cartesiane di vettori saranno denotate con indici latini in basso e simbolo non in grassetto  (p.es., $E_i$, con $i=1,2,3$, denota le componenti di $\vb{E}$ in una base cartesiana). A partire dal capitolo 8, introduco la nozione di vettore spaziotemporale. Per le ragioni espresse alla sezione 8.1, sarà dunque essenziale introdurre la nozione di vettore duale e distinguere in modo chiaro nella nostra notazione tra vettori e vettori duali. Aderirò quindi alla notazione standard in relatività ristretta, in cui  vettori spaziotemporali vengono denotati con indice Greco in alto (p.es., $W^{\mu}$) e vettori spaziotemporali duali     
vengono denotati con indice Greco in basso (p.es., $U_{\mu}$). Alcune convenzioni aggiuntive connesse alla relatività ristretta vengono enunciate alla fine della sezione 8.1.

Sono in debito con numerosi colleghi per aver letto parti (e, in alcuni casi, tutte) del manoscritto e per avermi fornito un feedback prezioso. Tra questi Sam Gralla, Abe Harte, Jim Isenberg, Istvan Racz, e Gautam Satishchandran, così come numerosi studenti che hanno seguito il mio corso. Tra questi ultimi, Tixuan Tan merita un ringraziamento speciale per aver letto il manoscritto con grande cura e per aver sollevato molti punti riguardanti l'esposizione.\\
\\ 

\thv{Note aggiunte dal traduttore}\\
Le note a pié di pagina il cui numero è seguito dal simbolo $\ddagger$ non sono dell'autore (R. Wald) ma del traduttore.
