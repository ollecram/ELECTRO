\setcounter{chapter}{0}
\renewcommand{\thechapter}{8}
\chapter{Special Relativity}\label{ch:8}
\setcounter{equation}{0}	        % To start with Equation 1
\counterwithin{equation}{chapter}	% Equation numbering will be 8.1 8.2 8.3   ... 

Special relativity is the theory of spacetime structure formulated by Einstein in 1905. Properties of the electromagnetic field played a central role in motivating special relativity. Specifically, electromagnetism is not compatible with pre-relativity notions of spacetime structure unless there is a preferred rest frame (the \quotes{aether}), since, as we have seen, Maxwell's equations predict that electromagnetic waves propagate with a particular velocity $c$, which can only be true in some preferred rest frame if pre-relativity notions of spacetime structure are valid. The Michelson-Morley experiment failed to find such a preferred rest frame. Furthermore, as Einstein realized, some physical phenomena in electromagnetism appear to obey an invariance with respect to moving observers even if the description of these phenomena in terms of a preferred rest frame does not have such an invariance.

The theory of electromagnetism is far more elegant and simple when formulated in the framework of special relativity. It therefore is somewhat of a travesty that, well into the twenty-first century, special relativity is discussed here---as in other texts on electromagnetism---as a separate chapter toward the end of the book. The reason, of course, is that even though special relativity has been a well established theory for much more than a century, its basic concepts are still so unfamiliar to most physicists that it is not feasible to begin the treatment of electromagnetism by giving its formulation in the framework of special relativity. It is my hope that this situation will be rectified by the twenty-second century.

Einstein's original formulation of special relativity relied heavily on the transformations between the labeling of events by different inertial observers and the invariance of the laws of physics under such transformations. The theory was reformulated in a much more geometrical form by Minkowski in 1908, wherein it was recognized that the underlying structure of spacetime in special relativity is that of a spacetime metric.\footnote{Minkowsky introduced an imaginary time coordinate so as to obtain a Euclidean spacetime metric. However, although this approach remains in use in many treatments of special relativity, it does not generalize to curved spacetime and cannot be used in general relativity. We shall use a real time coordinate in our treatment, and our spacetime metric will therefore be of Lorentzian signature.} Our treatment of special relativity will emphasize the role of the spacetime metric. Although Einstein was initially unimpressed by Minkowski's reformulation, he soon incorporated it into his thinking about gravitation. This led him to the theory of general relativity, wherein the spacetime metric becomes a dynamical variable that describes not only spacetime but also the effects of gravity. However, we shall not discuss general relativity here. 

The framework of special relativity is presented in section 8.1. The formulation of electromagnetism in the framework of special relativity is then given in section 8.2. On section 8.3.1, we analyze the motion of a (relativistic) charged particle in an external electromagnetic field, including the solutions for motion in a nonuniform electric field and in a uniform magnetic field. The Lienard-Wiechert solution describing the retarded field of a point charge in arbitrary motion is given in section 8.3.2, and properties of the radiated power for this solution are analyzed there as well.  

\section{The Framework of Special Relativity}\label{sec:8.1}
It is useful to think of space and time as composed of \quotes{events}---where each event corresponds to a point of space at an instant of time. The collection of all events comprises a 4-dimensional continuum, which I refer to as \quotes{spacetime}.

I take as a starting point that there exist global families of inertial observers who \quotes{fill} all of spacetime (i.e., within each family, one and only one of these observers passes through each event in spacetime). I further assume that the observers in each family are all \quotes{at rest} with respect to one another, that they can consistently synchronize their clocks by some physical procedure, and that the spatial relationships between these observers are described by Euclidean geometry. Finally, I also assume that different families of such inertial observers all move at uniform velocity with respect to one another, so that the different families may be labeled by their velocity with respect to some reference family. These assumptions are true in both pre-relativity physics and in special relativity, so they make a good starting point for describing the differences between these theories of spacetime structure. However, these assumptions are \tit{not} true in general relativity, so they would make a very poor starting point from a fundamental viewpoint.

By the above assumptions, the inertial observers in a given family can uniquely label events by $(t, \vb{x})$, where $t$ denotes the time of the event on the synchronized clock of the observer who passes through the event, and $\vb{x} = (x, y, z)$ are the Cartesian coordinates of that observer. I refer to the labeling of events in this way as \tit{inertial coordinates}. The assumption that events can be labeled in this way is implicit in every physics text or other reference where a \quotes{$t$} or \quotes{$\vb{x}$} appears in an equation. However,   this labeling depends on (i) a choice of origin of time (i.e., what time is labeled as $t=0$); (ii) a choice of origin of space (i.e., what observer in the family is at $\vb{x} = 0$); (iii) a choice of orientation of axes (to define the $x-$, $y-$,  and $z-$directions); and, most importantly for our present purposes, (iv) a choice of which family of inertial observers to use (i.e., a choice of the velocity $\vb{v}$ of the family). Different choices of origin of $t$ and $\vb{x}$, orientation of axes, and $\vb{v}$ will give rise to different labelings ($t', \vb{x}'$) of events. 

Usually, treatments of special relativity focus entirely on the difference in the labeling of events between families of inertial observers who are moving with velocity $\vb{v}$ relative to one another. This is given by a Galilean transformation in pre-relativity physics and by a Lorentz transformation in special relativity. Although it certainly is useful to know the explicit form of this transformation, a nearly exclusive focus on this obscures the geometrical content of the theory. It is analogous to studying ordinary Euclidean geometry by focusing on how the Cartesian coordinates transform under rotations. 

I therefore focus on the \quotes{invariant structure} of spacetime. The four numbers $(t, \vb{x})$ associated with an event do not, by themselves, convey meaningful information about the event, since they depend as much, for example, on the choice of origin in $t$ and $\vb{x}$ as they do on the event itself. Even if we consider the differences $(\Delta t, \Delta \vb{x})$ in the labeling of two events by a given family of inertial observers so as to eliminate the origin dependence, the values of $(\Delta t, \Delta \vb{x})$ will depend on the choice of orientation of axes as well as on the choice of family of inertial observers. It is of great interest to determine what quantities constructed out of $(\Delta t, \Delta \vb{x})$ are invariant (i.e., independent of these choices). Such quantities are well defined without making any arbitrary choices and hence can be considered as attributable to the structure of spacetime itself. 

In pre-relativity physics, there are two such invariant quantities: (i) the time interval $\Delta t$ between events, and (ii) the space interval $\abs{\Delta \vb{x}}^2$ between simultaneous events (i.e., events with $\Delta t = 0$). The space interval between nonsimultaneous events is not invariant, because if the family $O'$ of inertial observers moves with velocity $\vb{v}$ with respect to the family $O$, then we have 
\begin{equation}\label{eq:8.1}
\Delta \vb{x}' = \Delta \vb{x} - \vb{v} \Delta t\,,
\end{equation}

so $\abs{\Delta \vb{x}'}^2 \neq \abs{\Delta \vb{x}}^2$ if $\Delta t \neq 0$.  In addition, the collection of \quotes{worldlines} of inertial observers (i.e., the possible paths in spacetime of inertial observers) also can be viewed as an additional aspect of spacetime structure. The worldlines of inertial observers contain additional information independent of (i) and (ii) in that they cannot be constructed from knowing only $\Delta t$ for all pairs of events and knowing $\abs{\Delta \vb{x}}^2$ for simultaneous events. 

The situation in special relativity is much simpler. In special relativity, there is a single invariant quantity, the spacetime interval $I$ between any pair of events, given by 
\begin{equation}\label{eq:8.2}
I = - c^2 (\Delta t)^2 + \abs{\Delta \vb{x}}^2\;.
\end{equation}
 Furthermore, it can be shown that the worldlines of inertial observers can be constructed from a knowledge of $I$ between all pairs of events. Thus, $I$ provides the complete description of spacetime structure in special relativity. 
 
To tie the previous paragraph to what people are usually taught in special relativity, note that, in special relativity, 
the labeling $(t, \vb{x})$ of events by a family $O$ of observers is related to the labeling $(t', \vb{x}')$ by a family $O'$ of observers moving with velocity $v$ in the $x-$direction relative to $O$ (and with the same origin event and the same orientation of axes as $O$) by a \tit{Lorentz transformation}:
\begin{equation}\label{eq:8.3}
\begin{aligned}
t' &= \gamma (t - vx/c^2)\,,\\
x' &= \gamma (x - vt)\,,\\
y' &= y\,,\\
z' &= z\,,
\end{aligned}
\end{equation}
where   
\begin{equation}\label{eq:8.4}
\gamma \equiv \frac{1}{\sqrt{1 - v^2/c^2}}\,.
\end{equation}

\newpage

% BOX 1 **START** -- << Give Lorentz transformation in terms of parallel and normal components wrt the relative velocity >>
\subsubsection*{BOX 8.1 -- Lorentz transformations in vector form}\label{box:8.1}
\parindent=0pt  % Set the paragraph indentation to 0 (normal = 10pt) before the box 
\parbox{\textwidth}{\begin{mdframed}[style=MyFrame] %Added "\parbox{\textwidth}{"
%\lipsum[1]
A Lorentz transformation does only change time and the component of the spatial position vector $\vb{r}$ in the direction of relative motion; it does not change any component of $\vb{r}$ perpendicular to that direction. With this in mind, split $\vb{r}$ as measured in a frame $F$, and $\vb{r}'$ as measured in $F'$, each into components perpendicular ($\perp$) and parallel ($\parallel$) to the direction of the relative velocity $\vb{v}$:
\begin{equation*}
\vb{r} = \vb{r}_\perp + \vb{r}_\parallel\,,\quad \vb{r'} = \vb{r'}_\perp + \vb{r'}_\parallel 
\end{equation*}
then the transformations are 
\begin{align*}
t' &= \gamma \left( t - \frac{\vb{r}_\parallel \cdot \vb{v}}{c^2} \right)\\
\vb{r'}_\parallel &= \gamma (\vb{r}_\parallel - \vb{v}t)\\
\vb{r'}_\perp &= \vb{r}_\perp
\end{align*}
%\lipsum[2]
\end{mdframed}} %Added a closing "}" here
\parindent=10pt % Set the paragraph indentation to normal (10pt) 
% BOX 1 ** END ** -- << Give Lorentz transformation in terms of parallel and normal components wrt the relative velocity >>

The interval $I$ (Eq. \ref{eq:8.2}) is invariant under Lorentz transformations. 

% BOX 2 **START** -- << Check that I is invariant under Lorentz transformations >>
\subsubsection*{BOX 8.2 -- Invariance of $I$}\label{box:8.2}
\parindent=0pt  % Set the paragraph indentation to 0 (normal = 10pt) before the box 
\parbox{\textwidth}{\begin{mdframed}[style=MyFrame] %Added "\parbox{\textwidth}{"
%\lipsum[1]
Making use of vector notations introduced in Box \ref{box:8.1} we have 
\begin{align*}
{I'}^2 &=  - \left(c \Delta t'\right)^2 + \abs{\Delta \vb{r}'}^2\\
       &= -c^2 \gamma^2 \left(\Delta t - \Delta \vb{r}_\parallel \cdot \vb{v} /c^2 \right)^2
          +    \gamma^2 \abs{\Delta \vb{r}_\parallel - \vb{v} \Delta t}^2 \quad + \abs{\Delta \vb{r}_\perp}^2\\
       &=\;    -\gamma^2 \left( {(c \Delta t)}^2 + v^2/c^2 \abs{\Delta \vb{r}_\parallel}^2      -2 \Delta \vb{r}_\parallel \cdot \vb{v} \Delta t \right)\\
       &+ \quad \gamma^2 \left( \abs{\Delta \vb{r}_\parallel}^2 + v^2 (\Delta t)^2 \quad\quad   -2 \Delta \vb{r}_\parallel \cdot \vb{v} \Delta t \right)\quad\:\: + \abs{\Delta \vb{r}_\perp}^2\\
       &= - c^2 (\Delta t)^2  + \gamma^2 (1 - v^2/c^2) \abs{\Delta \vb{r}_\parallel}^2 + \abs{\Delta \vb{r}_\perp}^2\\
       &= - c^2 (\Delta t)^2  + \abs{\Delta \vb{r}_\parallel}^2 + \abs{\Delta \vb{r}_\perp}^2\\
       &= - c^2 (\Delta t)^2  + \abs{\Delta \vb{r}}^2\\ 
       &= I^2\qed
\end{align*}

%\lipsum[2]
\end{mdframed}} %Added a closing "}" here
\parindent=10pt % Set the paragraph indentation to normal (10pt) 
% BOX 2 ** END ** -- << Check that I is invariant under Lorentz transformations >>

\newpage
Furthermore, it can be shown that the most general transformation that preserves $I$ is a \tit{Poincaré transformation} (i.e., a composition of Lorentz transformations, rotations, and translations, as well as parity and time reversal transformations). Thus, Lorentz transformations naturally arise as (part of) the symmetry group of $I$.

The spacetime interval $I$ has the same mathematical form as the squared distance in Euclidean geometry except for the minus sign in front of the contribution coming from $(\Delta t)^2$. To pursue this further, we switch notation from $(t, \vb{x})$ to $x^\mu$ with $\mu=0, 1, 2, 3\,,$ where 
\begin{equation}\label{eq:8.5}
x^0 = ct,\quad x^1 = x,\quad x^2 = y,\quad x^3 = z\,.
\end{equation}

Note the superscript position of $\mu$ in $x^\mu$, which will be important in order to align with notational conventions explained further below. We view $x^\mu$ as representing a spacetime displacement vector (relative to some origin in spacetime) in much the same way as we normally view $\vb{x}$ as representing a spatial displacement vector (relative to some origin in space). We view $I$, eq. (\ref{eq:8.2}), as arising from an \tit{inner product} on spacetime displacement vectors, where the inner product of $x^\mu_1$ and $x^\mu_2$ is given in any inertial coordinates by 
\begin{equation}\label{eq:8.6}
I(x^\mu_1, x^\mu_2)  = \sum_{\mu,\nu=0}^{3} \eta_{\mu \nu} x^\mu_1 x^\nu_2\,,
\end{equation}

where\footnote{Many authors define $\eta_{\mu \nu}$ with an opposite sign convention, which results in sign changes in some formulas. The reader is advised to check the sign convention used for $\eta_{\mu \nu}$ when comparing formulas in different references. As mentioned in footnote 1 in this chapter, some authors use an imaginary time coordinate, in which case $\eta_{\mu \nu}$ would take a Euclidean form and normally would not be written down explicitly at all.}
\begin{equation}\label{eq:8.7}
\eta_{\mu \nu} \equiv \mqty(-1 &0 &0 &0\\ 0 &1 &0 &0\\ 0 &0 &1 &0\\ 0 &0 &0 &1)\;.
\end{equation}

I put \quotes{inner product} in quotes, because although $I$ is linear in each variable, symmetric, and nondegenerate (i.e., $I(x^\mu_1, x^\mu_2) = 0$ for all $x^\mu_2$ if and only if $x^\mu_1 = 0$), it fails to be positive definite. Nevertheless, it is closely analogous to the inner product on vectors in ordinary Euclidean geometry, 
\begin{equation}\label{eq:8.8}
(\vb{x}_1, \vb{x}_2) = \vb{x}_1 \cdot \vb{x}_2 = \sum_{i,j=1}^3 e_{ij} x_1^i x_2^j\,,  
\end{equation}

where 

\begin{equation}\label{eq:8.9}
e_{i j} \equiv \mqty(1 &0 &0\\ 0 &1 &0\\ 0 &0 &1)\;.
\end{equation}

We refer to $e_{ij}$ as the \tit{metric of space} in Euclidean geometry. Similarly, werefer to $\eta_{\mu \nu}$ as the 
\tit{metric of spacetime} in special relativity.

It should be mentioned that the ability to give finite spatial or spacetime displacements a vector space structure, as implicitly assumed in the discussion above, is very special to a \tit{flat} geometry. In a curved geometry---such as the $2-$dimensional surface of a potato---there is no natural notion of adding two finite displacements about a point. Nevertheless, in a curved geometry, a notion of 
\quotes{infinitesimal displacements} about $p$ is referred to as the \tit{tangent space} at $p$. 
In differential geometry, a metric would be defined as a (not necessarily positive-definite) inner product defined on the tangent space at $p$ for all $p$. However, the spacetime geometry of special relativity is flat, so we may treat \tit{finite} spacetime displacements 
$\Delta x^\mu \equiv x^\mu - x^\mu(p)$ about a point $p$ as \quotes{vectors}---and we can treat the metric as an inner product on these finite displacement vectors---as we have done above.

A striking feature of the spacetime metric eq. (\ref{eq:8.7}) is that there are nonzero spacetime displacement vectors $\Delta x^\mu$
about any event $p$ that are \tit{null}, in other words, such that 
\begin{equation}\label{eq:8.10}
\sum_{\mu \nu} \eta_{\mu \nu} \Delta x^\mu \Delta x^\nu = 0 \,.
\end{equation}

The collection of all null spacetime displacement vectors about $p$ comprise a cone with vertex at $p$, as illustrated in figure 8.1. The portion with $\Delta x \leq 0$ is referred to as its \tit{past light cone}. In the precise sense discussed in chapter 5, electromagnetic radiation emitted at $p$ propagates along the future light cone of $p$ (see section 5.2), whereas electromagnetic radiation observed at $p$ propagated to $p$ along its past ligt cone (see section 5.4). The interior of the future light cone of $p$ is referred to as the \tit{future} of $p$. It is composed of events that, in principle, can be reached by an observer initially present at $p$. Similarly, the interior of the past light cone of $p$ is referred to as the \tit{past} of $p$. It is composed of events with the property that an observer starting at that event can, in principle, arrive at $p$. The events lying outside the light cone of $p$ are said to be \tit{spacelike related} to $p$. No observer can be present both at $p$ and at an event spacelike related to $p$. In other words, in special relativity, nothing can travel faster than light. 

If $q$ is an event that lies in the future of $p$, then there is a unique inertial observer who is present at both $p$ and $q$. The proper time $\Delta \tau$ that elapses on a clock carried by this observer between $p$ and $q$ is given in any inertial cordinates by\footnote{This can be seen by noting that the right side of eq. (\ref{eq:8.11}) is the spacetime interval between $p$ and $q$ and does not depend on the choice of inertial coordinates. In the frame of the inertial observers who goes from $p$ to $q$, we have $\Delta \vb{x} = 0$ so $\sum\eta_{\mu \nu} \Delta x^\mu \Delta x^\nu = - c^2 (\Delta t)^2$.}
\begin{equation}\label{eq:8.11}
\Delta \tau = \frac{1}{c}\sqrt{-\sum_{\mu, \nu} \eta_{\mu \nu} \Delta x^\mu \Delta x^\nu} \,,
\end{equation}

where $\Delta x^\mu = x^\mu(q) - x^\mu(p)$. A general, noninertial observer will trace out a curve in spacetime, called the \tit{worldline} of the observer. Any curve in spacetime may be specified by giving $x^\mu(\lambda)$, where $\lambda$ 
is an arbitrary parametrization of the curve. The tangent $T^\mu$ to the curve in this parametrization is defined by 
\begin{equation}\label{eq:8.12}
T^\mu = \dv{x^\mu}{\lambda}\,.
\end{equation}

The tangent $T^\mu$ to the worldline of any observer must be timelike (i.e, $\sum_{\mu \nu} \eta_{\mu \nu} T^\mu T^\nu < 0$), since the observer must stay within the light cone of any event that he/she passes through. The proper time elapsed on the clock of an arbitrary noninertial observer going between events $p$ and $q$ is given by
\begin{equation}\label{eq:8.13}
\Delta \tau = \frac{1}{c} \bigint_{\!\!\!\!\!\!\!\!\lambda(p)}^{\!\!\!\!\!\!\!\!\!\lambda(q)} {\sqrt{-\sum_{\mu, \nu} \eta_{\mu \nu} T^\mu T^\nu}} \dd{\lambda}\,.
\end{equation}

It is not difficult to show that the inertial observer who passes through events $p$ and $q$ maximizes the elapsed proper time relative to all observers who pass between $p$ and $q$. This fact is often referred to as the \tit{twin paradox} (see problem 1 at the end of the chapter).

It is convenient to parametrize the worldline of an arbitrary observer by the proper time $\tau$ elapsed along the worldline. The tangent to the worldline in this parametrization is called the 4-\tit{velocity} of the observer:
\begin{equation}\label{eq:8.14}
u^\mu = \dv{x^\mu}{\tau}\,.
\end{equation}

It follows from eq. (\ref{eq:8.13}) with $\lambda = \tau$ that $$\sum_{\mu\, \nu} \eta_{\mu \nu} u^\mu u^\nu = -c^2\,.$$
In any inertial coordinates, the velocity $\vb{v}$ of this observer is given by
\begin{equation}\label{eq:8.15}
\vb{v} = \dv{\vb{x}}{t} = c \dv{\vb{x}}{x^0} = c \frac{\dv*{\vb{x}}{\tau}}{\dv*{x^0}{\tau}} = c \frac{\vb{u}}{u^0}\,.
\end{equation}

It follows from this relation and the normalization condition on $u^\mu$ that, in any inertial coordinates, the components of $u^\mu$ are given in terms of $\vb{v}$ by 
\begin{equation}\label{eq:8.16}
u^\mu = (c \gamma, \gamma \vb{v})\,,
\end{equation}
where $\gamma$ is given by eq. (\ref{eq:8.4}).

As discussed above, because of the flat spacetime geometry, the finite displacements 
$\Delta x^\mu = x^\mu - x^\mu(p)$ about $p$ comprise a 4-dimensional vector space, which we denote by $V_p$. The tangent $T^\mu$ to a curve passing through $p$ can naturally be viewed as a vector in $V_p$. For an arbitrary spacetime vector, we will strictly adhere to the notational convention that an \tit{upper} Greek letter index will be attached to the letter representing the vector (e.g: when a symbol such as $W^\mu$ appears below, one can immediately tell from the notation that $W^\mu$ is a spacetime vector). 

Spacetime vectors in special relativity have a geometrical status in exactly the same sense that ordinary vectors have a geometrical status in ordinary, 3-dimensional space. Although the representation of a spacetime vector in terms of its components depends on a choice of inertial coordinate system, one may view the spacetime vector as having a geometrical meaning that does not depend on a choice of coordinates---in exactly the same manner as an ordinary vector can be viewed as having a geometrical meaning, independent of the representation of its components in a particular Cartesian coordinate system. 

Another class of objects that have a similar geometrical status are linear maps taking vectors to numbers. A linear map taking $V_p$ into $\R$ is called a \tit{dual vector}. The collection of all dual vectors at $p$ comprises a vector space of the same dimension as $V_p$, known as the \tit{dual space} to $V_p$ and denoted by $V^*_p$. Given any vector $W^\mu$ in $V_p$, we can use the spacetime metric $\eta_{\mu \nu}$  to construct a dual vector $L_W$ taking 
$V_p$ into $\R$ by

\begin{equation}\label{eq:8.17}
S^\mu \rightarrow \sum_{\mu, \nu} \eta_{\mu \nu} S^\mu W^\nu\,,
\end{equation}

that is, $L_W$ maps the arbitrary vector $S^\mu$ into the number given by the right side of this equation. It is not difficult to show that, in the presence of a metric, all dual vectors arise in this manner. It is natural to denote the dual vector $L_W$ by 
\begin{equation}\label{eq:8.18}
W_\mu \equiv \sum_{\nu = 0}^3 \eta_{\mu \nu} W^\nu\,,
\end{equation}
so that the linear map $L_W$ is given by $S^\mu \rightarrow \sum_\nu S^\nu W_\nu$. In this notation, the lowered index on $W_\mu$ on the left side of eq. (\ref{eq:8.18}) indicates that this quantity is the dual vector obtained from $W^\mu$ rather than $W^\mu$ itself. For an arbitrary dual vector we will strictly adhere to the notational convention that a \tit{lower} Greek letter index will be attached to the letter representing the dual vector (e.g., when a symbol such as $U_\mu$ appears below, one can immediately tell from the notation that $U_\mu$ is a spacetime dual vector).

Since the spacetime metric is not degenerate, eq. (\ref{eq:8.18}) can be inverted to give
\begin{equation}\label{eq:8.19}
W^\mu = \sum_{\nu = 0}^3 \eta^{\mu \nu} W_\nu\,,
\end{equation}
where $\eta^{\mu \nu}$ denotes the inverse spacetime metric, whose matrix of components is given by the inverse matrix\footnote{In an inertial coordinate system, the components of $\eta_{\mu \nu}$ are given by eq. (\ref{eq:8.7}), and the inverse of eq. (\ref{eq:8.7}) then has exactly the same components as eq. (\ref{eq:8.7}) (see eq. (5.121)). However, $\eta^{\mu \nu}$ and $\eta_{\mu \nu}$ are fundamentally very different objects, and this equality of their components would not hold in a general, noninertial coordinate system.} of $\eta_{\mu \nu}$. Following standard conventions, if $U_\mu$ is a dual vector, we denote the corresponding vector given by eq. (\ref{eq:8.19}) as $U^\mu$. Thus, we use $\eta^{\mu \nu}$ and $\eta_{\mu \nu}$ to raise and lower indices in the manner given by eq. (\ref{eq:8.18}) and eq. (\ref{eq:8.19}).

Vectors and dual vectors are fundamentally very different objects. Nevertheless, when a metric is present, vectors and dual vectors can be identified via the correspondence eq. (\ref{eq:8.18}) or equivalently, eq. (\ref{eq:8.19}). When dealing with a vector $\vb{v}$ in ordinary Euclidean space, the components $(v^1, v^2, v^3)$ of the vector Cartesian coordinates are equal to the components $(v_1, v_2, v_3)$ of the corresponding dual vector on account of the trivial form, eq. (\ref{eq:8.9}), of the Euclidean metric. Thus, in this case, if one treats a vector as a 3-tuple of numbers, one can get away with not making a distinction between a vector and its corresponding dual vector.\footnote{In this regard, it should be noted that in the previous chapters of this book, we denoted the Cartesian components of vectors such as the electric field $\vb{E}$ as $E_i$ rather than $E^i$. This is not because we were working with the corresponding dual vector but instead because there was no need to make a distinction between vectors and dual vectors and, correspondingly, no need to adhere to the conventions on the index positions that we have just introduced. We wrote \quotes{$E_i$} simply because it was typographically more convenient to write $E_i$ rather than $E^i$. However, from this point forward in this book, we will strictly adhere to our conventions on index positions.} 
This accounts for why many students of physics have never explicitly encountered the notion of a dual vector. However, in special relativity, in any inertial coordinates, the dual vector $W_\mu$ corresponding to $W^\mu$ has $W_0 = - W^0$, and we cannot get away with ignoring the distinction between vectors and dual vectors.  

A prime example of a dual vector at $p$ is the spacetime gradient $\partial_\mu f = \pdv*{f}{x^\mu}$ of a function $f$ evaluated at $p$. This is seen to be a dual vector as follows. For any curve with tangent $T^\mu$ at $p$, the quantity $\sum_\mu T^\mu \partial_\mu f$ represents the derivative of $\pdv*{f}{\lambda}$ of $f$ along the curve and thus is well defined, independently of any choice of coordinates. Thus, $\partial_\mu f$ at $p$ naturally can be viewed as a linear map taking tangent vectors $T^\mu$ at $p$ into numbers (i.e., a dual vector). This is because the correspondence between vectors and dual vectors that is provided by the Euclidean metric is being used. Fundamentally, however, the gradient of $f$ is a dual vector.  


 




 
 
  
\section{The Formulation of Electromagnetic Theory in the Framework of Special Relativity}\label{sec:8.2}

\section{Charged Particle Motion and Radiation}\label{sec:8.3}

\subsection{Charged Particle Motion}\label{ssec:8.3.1}

\subsection{Radiation from a Point Charge in Arbitrary Motion}\label{ssec:8.3.2}

%===================================================================================

\section*{Problems}


