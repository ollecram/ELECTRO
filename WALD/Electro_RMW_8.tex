\setcounter{chapter}{0}
\renewcommand{\thechapter}{8}
\chapter{Special Relativity}
\setcounter{equation}{0}	        % To start with Equation 1
\counterwithin{equation}{chapter}	% Equation numbering will be 8.1 8.2 8.3   ... 

Special relativity is the theory of spacetime structure formulated by Einstein in 1905. Properties of the electromagnetic field played a central role in motivating special relativity. Specifically, electromagnetism is not compatible with pre-relativity notions of spacetime structure unless there is a preferred rest frame (the \quotes{aether}), since, as we have seen, Maxwell's equations predict that electromagnetic waves propagate with a particular velocity $c$, which can only be true in some preferred rest frame if pre-relativity notions of spacetime structure are valid. The Michelson-Morley experiment failed to find such a preferred rest frame. Furthermore, as Einstein realized, some physical phenomena in electromagnetism appear to obey an invariance with respect to moving observers even if the description of these phenomena in terms of a preferred rest frame does not have such an invariance.

The theory of electromagnetism is far more elegant and simple when formulated in the framework of special relativity. It therefore is somewhat of a travesty that, well into the twenty-first century, special relativity is discussed here---as in other texts on electromagnetism---as a separate chapter toward the end of the book. The reason, of course, is that even though special relativity has been a well established theory for much more than a century, its basic concepts are still so unfamiliar to most physicists that it is not feasible to begin the treatment of electromagnetism by giving its formulation in the framework of special relativity. It is my hope that this situation will be rectified by the twenty-second century.

Einstein's original formulation

\section{The Framework of Special Relativity}

\section{The Formulation of Electromagnetic Theory in the Framework of Special Relativity}

\section{Charged Particle Motion and Radiation}

\subsection{Charged Particle Motion}

\subsection{Radiation from a Point Charge in Arbitrary Motion}

%===================================================================================

\section*{Problems}


