%%%%%%%%%%%%%%%%%%%% book.tex %%%%%%%%%%%%%%%%%%%%%%%%%%%%%

\documentclass[english, 11pt, a4paper]{book}
%\usepackage[nomath]{lmodern}
\usepackage[T1]{fontenc}
\usepackage[italian]{babel}
% The following changes the Chapter heading from 'Chapter' to 'Lecture'
%$\addto\captionsenglish{\renewcommand{\chaptername}{Lecture}}
%$%\usepackage{fancyhdr}
%$\newcommand\chap[1]{%
%$ \chapter*{#1}%
%$  \addcontentsline{toc}{chapter}{#1}}
%$\newcommand\sect[1]{%
%$  \section*{#1}%
%$  \addcontentsline{toc}{section}{#1}}
  
  
% choose options for [] as required from the list
% in the Reference Guide, Sect. 2.2

\usepackage{makeidx}         % allows index generation
\usepackage{graphicx}        % standard LaTeX graphics tool
\usepackage{subcaption}      % for subfigures environments 
                             % when including figure files
\usepackage{multicol}        % used for the two-column index
\usepackage[bottom]{footmisc}% places footnotes at page bottom
% etc.
% see the list of further useful packages
% in the Reference Guide, Sects. 2.3, 3.1-3.3
\usepackage[normalem]{ulem}

\usepackage[shortlabels]{enumitem}	% to be able to resume enumerated lists

\usepackage{amsmath}	% To be able to slash
\usepackage{bm}	        % To use bold greek letters in math mode with \bm{}
\usepackage{amsfonts}	% To be able to use \mathbb ... 
\usepackage{amssymb}	% To be able to use \nmid ... 
\usepackage{amsthm}		% \qed, \qedhere
\usepackage{slashed}	% any character (dirac)
\usepackage[title,toc,page]{appendix}

% See https://tex.stackexchange.com/questions/36524/how-to-put-a-framed-box-around-text-math-environment/36528
\usepackage{collectbox}	% To make box around formulas

% *** AFTER THIS LINE *** 
%     put \usepackage{} for shared packages kept under ~\Links\repos\git\LaTeX_Styles

% Physics package 
% https://tex.stackexchange.com/questions/38978/how-can-i-manually-install-a-latex-package-debian-ubuntu-linux
\usepackage{/home/marcello/Links/repos/git/LaTeX_Styles/physics}	
% To put accents below letters
\usepackage{/home/marcello/Links/repos/git/LaTeX_Styles//accents}

% To control vertical white space above and below equations
% see https://tex.stackexchange.com/questions/69662/how-to-globally-change-the-spacing-around-equations
\expandafter\def\expandafter\normalsize\expandafter{%
    \normalsize
    \setlength\abovedisplayskip{16pt}
    \setlength\belowdisplayskip{16pt}
    \setlength\abovedisplayshortskip{16pt}
    \setlength\belowdisplayshortskip{16pt}
}

% To use extra symbols like dagger and double dagger in numbering footnotes 
\usepackage{footmisc}

% Force chapter numbering to restart within each part
\makeatletter
%\@addtoreset{chapter}{part}
\makeatletter


\makeindex             % used for the subject index
                       % please use the style svind.ist with
                       % your makeindex program


%%%%%%%%%%%%%%%%%%%%%%%%%%%%%%%%%%%%%%%%%%%%%%%%%%%%%%%%%%%%%%%%%%%%%

\begin{document}

\newcommand{\quotes}[1]{``#1''}
\newcommand{\sfT}{$\mathsf{T}$}
\newcommand{\udT}{\rotatebox[origin=c]{180}{$\mathsf{T}$}}

%Bold calligraphic letters 
\newcommand{\N}{\mathbb{N}}	% integers
\newcommand{\Z}{\mathbb{Z}}	% relative
\newcommand{\Q}{\mathbb{Q}}	% rationals
\newcommand{\R}{\mathbb{R}}	% reals
\newcommand{\C}{\mathbb{C}}	% complex
\newcommand{\F}{\mathbb{F}}	% generic field 1
\newcommand{\K}{\mathbb{K}}	% generic field 2
\newcommand{\V}{\mathbb{V}}	% Shankar's for vector space V

%Plain calligraphic letters 
\newcommand{\cC}{\mathcal{C}}    % space 1
\newcommand{\cF}{\mathcal{F}}    % space 2
\newcommand{\cS}{\mathcal{S}}    % space 3 
\newcommand{\cT}{\mathcal{T}}    % space 4 

\newcommand{\cU}{\mathcal{U}}    % sets 1
\newcommand{\cV}{\mathcal{V}}    % sets 2
\newcommand{\cW}{\mathcal{W}}    % sets 3
\newcommand{\cP}{\mathcal{P}}    % sets 4 
\newcommand{\cQ}{\mathcal{Q}}    % sets 5
\newcommand{\cR}{\mathcal{R}}    % sets 6

\newcommand{\cY}{\mathcal{Y}}    % Y


% Calligraphic H for Hilbert space
\newcommand{\cH}{\mathcal{H}}    

% To show argument of the exponential function vertically, i.e., as a superscript 
\newcommand{\vexp}[1]{\,e^{#1}}

% To type an angle as a number of degrees like 45^\circ
\newcommand{\degree}[1]{{#1}^\circ}

% To create boldface vectors with a hat or check accent 
\newcommand{\hatvb}[1]{\vb{\hat{#1}}}
\newcommand{\chkvb}[1]{\vb{\check{#1}}}

% To create not-bold vectors with a hat or check accent 
\newcommand{\hatv}[1]{\hat{#1}}
\newcommand{\chkv}[1]{\check{#1}}

% To create <x|, |x> and <x|y> with unit vectors inside
\newcommand{\ubra}[1]{\bra*{\vu{#1}}}
\newcommand{\uket}[1]{\ket*{\vu{#1}}}
\newcommand{\uip}[2]{\ip*{\vu{#1}}{\vu{#2}}}

% To put accents below letters. 
\newcommand{\ut}[1]{\underaccent{\tilde}{#1}}
\newcommand{\uh}[1]{\underaccent{\hat}{#1}}
\newcommand{\form}[1]{\uh{#1}}

% To create italic, bold, bolditalic text
\newcommand{\tit}[1]{\textit{#1}}
\newcommand{\tbf}[1]{\textbf{#1}}
\newcommand{\tbi}[1]{\textit{\textbf{#1}}}

% Latin Modern sans serif |OR| Helvetica (SELECT)
\newcommand{\textlmss}{\fontfamily{lmss}\selectfont}
\newcommand{\texthv}{\fontfamily{phv}\selectfont}

% Latin Modern sans serif |OR| Helvetica (USE, within OR outside MATH !)
\newcommand{\tlmss}[1]{\text{\textlmss{#1}}}
\newcommand{\thv}[1]{\text{\texthv{#1}}}

% To use \tlmss{T} symbol to denote transpose 
\newcommand{\transp}[1]{{#1}^{\tlmss{T}}}

% To use \dagger symbol to denote operator Adjoint
\newcommand{\Adj}[1]{{#1}^\dagger}

% To denote the Hermitian conjugate with a '+' superscript
\newcommand{\Hconj}[1]{{#1}^{+}}

% To use \tlmss{Ker}, \tlmss{Coker} and \tlmss{Img} to denote Kernel, Co-Kernel & Image 
\newcommand{\Ker}{\tlmss{Ker}\,}
\newcommand{\Coker}{\tlmss{Coker}\,}
\newcommand{\Img}{\tlmss{Im}\,}

% To use \tlmss{Alt} and \tlmss{alt} to denote alternation 
\newcommand{\Alt}{\tlmss{Alt}\,}
\newcommand{\alt}{\tlmss{alt}\,}

% To use \tlmss{Ann} to denote annulets 
\newcommand{\Ann}{\tlmss{Ann}\,}

% Misc abbreviations
\newcommand{\ora}[1]{\overrightarrow{#1}}

\DeclareRobustCommand{\rchi}{{\mathpalette\irchi\relax}}
\newcommand{\irchi}[2]{\raisebox{\depth}{$#1\chi$}} % inner command, used by \rchi

% See https://tex.stackexchange.com/questions/36524/how-to-put-a-framed-box-around-text-math-environment/36528
\makeatletter
\newcommand{\mybox}{%
    \collectbox{%
        \setlength{\fboxsep}{1pt}%
        \fbox{\BOXCONTENT}%
    }%
}
\makeatother

\author{Robert M. Wald}
\title{Advanced Classical Electromagnetism}
%\subtitle{Italian translation of the Princeton University Press, 2022 edition.}
\maketitle

\frontmatter%%%%%%%%%%%%%%%%%%%%%%%%%%%%%%%%%%%%%%%%%%%%%%%%%%%%%%

%\include{dedic}

%\chapter*{Plan}
\label{plan} 

In this book I am keeping notes about the theory of classical electromagnetism, 
as exposed in various books. In particular, I intend to cover the following materials:

\begin{itemize}

\item B. Felsager -- Geometry Particles and Fields
\begin{enumerate}
\setcounter{enumi}{0}
\item Electromagnetism (1.1 to 1.4)
\end{enumerate}

\item C. Cattaneo -- Teoria Einsteniana della Gravitazione
\begin{enumerate}
\setcounter{enumi}{0}
\item Elementi di Algebra e Analisi Lineare
\end{enumerate}

\item D.J. Griffiths -- Introduction to Electrodynamics
\begin{enumerate}
\setcounter{enumi}{0}
\item Vector Analysis
\item Electrostatics
\item Potentials
\item Electric Fields in Matter
\item Magnetostatics
\item Magnetic Fields in Matter
\item Electrodynamics
\item Conservation Laws
\item Electromagnetic Waves
\item Radiation
\item Electrodynamics and Relativity
\item Potentials and Fields
\item Helmoltz Theorem
\end{enumerate}

\item J.D. Jackson -- Classical Electrodynamics, 2nd Edition
\begin{enumerate}
\setcounter{enumi}{0}
\item Introduction to Electrostatics
\item Boundary Value Problems in Electrostatics - I
\item Boundary Value Problems in Electrostatics - II
\item Multipoles, Electrostatics of Macroscopic Media, Dielectrics
%\item Magnetostatics
%\item Time Varying Fields, Maxwell Equations, Conservation Laws
%\item Plane Electromagnetic Waves and Wave Propagation
%\item Wave Guides and Resonant Cavities
%\item Simple Radiating Systems, Scattering and Diffraction
%\item Magnetohydrodynamics and Plasma Physics
\end{enumerate}

\item J.D. Jackson -- Classical Electrodynamics, 3rd Edition
\begin{enumerate}
\setcounter{enumi}{4}
\item Magnetostatics, Faraday's Law, Quasi-Static Fields
\item Maxwell Equations, Macroscopic Electromagnetism, Conservation Laws
\item Plane Electromagnetic Waves and Wave Propagation
\item Wave Guides, Resonant Cavities and Optical Fibers
\item Radiating Systems, Multipole Fields and Radiation
\item Scattering and Diffraction
\item Special Theory of Relativity
\item Dynamics of Relativistic Particles and Electromagnetic Fields
\end{enumerate}

\item B. Felsager -- Geometry Particles and Fields
% Contacts with quantum theory of particles dynamics in EM fields
\begin{enumerate}
\setcounter{enumi}{1}
\item Interaction of Fields and Particles
\end{enumerate}

\item J. Franklin -- Advanced Mechanics and General Relativity
\begin{enumerate}
\setcounter{enumi}{1}
\item Relativistic Mechanics
\item Tensors
\item Curved Space
\item Scalar Field Theory
\item Tensor Field Theory (6.1 to 6.5)
\end{enumerate}

\item J.D. Jackson -- Classical Electrodynamics, 3rd Edition
\begin{enumerate}
\setcounter{enumi}{12}
\item Collisions, Energy Loss and Scattering of Charged Particles, Cherenkov and Transition Radiation
\item Radiation by Moving Charges
\item Bremsstrahlung, Method of Virtual Quanta, Radiative Beta Processes
\item Radiation Damping, Classical Models of Charged Particles
\end{enumerate}

\item B. Felsager -- Geometry Particles and Fields
% Contacts with quantum theory of fields dynamics + differential geometry math
\begin{enumerate}
\setcounter{enumi}{2}
\item Dynamics of Classical Fields
\end{enumerate}

\begin{enumerate}
\setcounter{enumi}{5}
\item Differentiable Manifolds, Tensor analysis
\item Differential Forms, Exterior Calculus
\item Integral Calculus on Manifolds
\end{enumerate}

\item C.W. Misner, K.S. Thorne, J.A. Wheeler -- Gravitation
\begin{enumerate}
\setcounter{enumi}{1}
\item Foundations of Special Relativity
\item The Electromagnetic Field
\item Electromagnetism and Differential Forms
\end{enumerate}

\item L.D. Landau, E.M. Lifshitz -- Teoria dei Campi
\begin{enumerate}
\setcounter{enumi}{0}
\item Principio di Relatività
\item Meccanica Relativistica
\item Carica in un Campo Elettromagnetico
\item Equazioni del Campo Elettromagnetico
\item Campo Elettromagnetico Costante
\item Onde Elettromagnetiche
\item Propagazione della Luce
\item Campo di Cariche in Moto
\item Radiazione Elettromagnetica
\end{enumerate}

\item L.D. Landau, E.M. Lifshitz -- Elettrodinamica dei Mezzi Continui
\begin{enumerate}
\setcounter{enumi}{0}
\item Elettrostatica dei Conduttori
\item Elettrostatica nei Dielettrici
\item Corrente Continua
\item Campo Magnetico Costante
\item Ferromgnetismo e Antiferromagnetismo
\item Superconduttività
\item Campo Magnetico Quasi Stazionario
\item Idrodinamica Magnetica
\item Equazioni delle Onde Elettromagnetiche
\item Propagazione delle Onde Elettromagnetiche
\item Onde Elettromagnetiche in Mezzi Anisotropi
\item Dispersione Spaziale
\item Ottica non Lineare
\item Passaggio delle Particelle Veloci attraverso la Materia
\item Diffusione delle Onde Elettromagnetiche
\item Diffrazione dei Raggi X nei Cristalli
\end{enumerate}

\end{itemize}
	
\tableofcontents
%\addappheadtotoc

\mainmatter%%%%%%%%%%%%%%%%%%%%%%%%%%%%%%%%%%%%%%%%%%%%%%%%%%%%%%%
\chapter*{Prefazione}\label{Wald_EM_00}
\thv{From R. M. Wald -- Advanced Classical Electromagnetism, 2022}\\

Questo libro è nato dal corso avanzato del primo trimestre in Elettromagnetismo tenuto agli studenti dell'Università di Chicago nell'inverno del 2018. Erano trascorsi decenni da quando avevo precedentemente insegnato questa materia, per cui mi ci sono accostato con occhi nuovi, ed è stato naturale per me cercare di ripensare a come presentarla a livello universitario. Nel farlo, mi è risultato evidente che l'usuale metodo quasi-storico di presentare la materia induce modi non appropriati di concepire l'elettromagnetismo. Pertanto, per evitare di partire con il piede sbagliato, decisi di spendere le prime lezioni del corso per descrivere quelli che nel capitolo 1 di questo libro chiamo con il termine di \quotes{miti} riferiti all'elettromagnetismo. Constatai che partendo in questo modo,  divenne molto più facile presentare in modo diretto gli argomenti in modo chiaro e conciso, senza dover effettuare cambi di prospettiva nel  corso del loro sviluppo. Ho insegnato il corso nuovamente nel successivi 3 anni, fornendo appunti delle lezioni alla classe. Gli appunti di queste lezioni sono ora state sviluppate in questo libro.

Il primo capitolo di questo libro è perciò una introduzione all'elettromagnetismo alquanto fuori dall'ordinario. Invece di cominciare con la forza tra particelle cariche, e discutere come questo da luogo ad un concetto di \quotes{campo}, e così via, il mio proposito nel capitolo 1 è di spiegare agli studenti come essi dovrebbero pensare all'elettromagnetismo da una prospettiva moderna e matematicamente precisa. I punti principali esposti in questo capitolo sono che (i) i potenziali, non le intensità dei campi, sono le variabili dinamiche fondamentali nell'elettromagnetismo; (ii) le proprietà energia ed impulso associate al campo elettromagnetico sono parte essenziale della formulazione della teoria e non possono essere derivate da argomentazioni sul \quotes{\tit{lavoro fatto}} dal campo; (iii) non si deve pensare ai campi elettromagnetici come \tit{prodotti} da cariche; e (iv) a livello fondamentale nell'elettrodinamica classica la materia carica deve essere considerata come distribuita in modo continuo piuttosto che essere costituita da cariche puntiformi. Molti di questi punti non possono essere chiariti appieno se non negli ultimi capitoli del libro--particolarmente i capitoli 9 e 10--ma il mio intento è di esporre queste idee in modo sufficientemente chiaro ed esplicito nel capitolo 1 in modo da poter poi mantenere questi capisaldi nel prosieguo del libro senza ulteriori giustificazioni. 

Gli argomenti trattati nei capitoli 2-7 sono quelli che verrebbero normalmente esposti in ogni corso avanzato di Elettromagnetismo. L'Elettrostatica è trattata nel capitolo 2, ma a partire dall'equazione di Poisson, non dalla legge di Coulomb. I materiali dielettrici in elettrostatica sono trattati nel capitolo 3, con un considerevole grado di attenzione a come eseguire la media su scale macroscopiche e al trattamento dell'energia. La Magnetostatica è trattata nel capitolo 4, con una discussione completa della differenza di segno tra magnetostatica ed elettrostatica nella energia di interazione di un dipolo in un campo esterno--e come questa si collega al cambiamento della massa a riposo di un magnete quando esso si muove in modo quasi-statico in un campo magnetico esterno. Elettrodinamica e radiazione sono discusse a fondo nel capitolo 5. In aggiunta agli argomenti normalmente trattati nei testi di Elettromagnetismo, io derivo in quel capitolo la formulazione ai valori iniziali per le equazioni di Maxwell. L'Elettrodinamica nei mezzi viene trattata nel capitolo 6, inclusa una discussione della mgnetoidrodinamica. L'approssimazione dell'ottica geometrica nella dinamica ondulatoria è presentata nella prima sezione del capitolo 7, seguita da una discussione di interferenza e coerenza ed un'analisi di due problemi nella diffrazione: diffusione da una palla dielettrica e propagazione e radiazione attraverso un'apertura.

La relatività ristretta è trattata nel capitolo 8. La relatività ristretta è alla base della formulazione della teoria elettromagnetica, quindi dovrebbe propriamente essere presentata all'inizio di un libro sull'elettromagnetismo, piuttosto che essere relegata in un capitolo verso la fine del libro. Tuttavia, la relatività ristretta rimane un argomento così poco familiare a gran parte degli studenti che ciò non è possibile. Molte esposizioni della relatività ristretta si concentrano sulle regole per applicare trasformazioni di Lorentz a certe quantità, senza fornire molte informazioni sul contenuto geometrico sottostante della teoria. All'opposto sarebbe naturale, per me che mi occupo di relatività generale, introdurre un livello di maggiore astrazione matematica e apparato geometrico di quanto sarebbe necessario per fornire una chiara descrizione della relatività ristretta. Ho posto una considerevole cura nella scrittura della sezione 8.1 in modo tale da introdurre la relatività ristretta in modo concettualmente chiaro senza introdurre un livello di astrazione maggiore di quello che ritengo essenziale. Questa sezione può essere letta in modo indipendente dal resto del libro, e spero che possa fornire da sola una utile introduzione alla relatività ristretta. 
Viene quindi esposta la formulazione dell'elettromagnetismo nel contesto della relatività ristretta, seguita da una discussione del moto di particelle cariche e della radiazione da una carica puntiforme in moto arbitrario.

Il capitolo 9 tratta l'elettromagnetismo come una teoria di gauge, con ciò portando la formulazione dell'elettromagnetismo in questo libro al livello di comprensione concettuale che è stato raggiunto alla metà del ventesimo secolo. Vi è un gap considerevole tra il modo in cui viene normalmente descritto il campo elettromagnetico e la sua interazione con la materia carica nei corsi di elettromagnetismo classico e quello in cui esso appare di fatto come una delle interazioni fondamentali nel modello standard della fisica delle particelle. Questo capitolo dovrebbe essere di aiuto a colmare questo gap.    

Infine, la nozione di carica puntuale viene discussa in dettaglio nel capitolo 10. Si mostra che è possibile considerare il limite, in senso matematicamente ben definito, in cui l'estensione di un corpo carico si riduce a zero, purché anche la carica e la massa del corpo vengano ridotte in modo proporzionale alla sua estensione. In questo limite si ottiene la forza di Lorentz. Le correzioni dovute alla \quotes{self-force} si possono poi calcolare perturbativamente in modo matematicamente rigoroso. Nella parte finale di questo capitolo viene trattato il problema di come descrivere in modo consistente il moto di un corpo carico tenendo conto delle correzioni dovute alla \quotes{self-force} senza introdurre soluzioni spurie (\quotes{runaway}).

Nel libro, ho cercato di formulare tutti i principali concetti ed i risultati della teoria dell'elettromagnetismo in modo chiaro e conciso. Tuttavia, non ho cercato di presentare un'ampia collezione di esempi o applicazioni. Queste caratteristiche del libro spiegano il fatto che la sua lunghezza sia circa un terzo di alcuni altri testi di elettromagnetismo avanzato con una simile copertura di argomenti. 

Ho cercato di presentare ogni cosa con un alto livello di precisione matematica. Sebbene mi sia sforzato di evitare distrazioni dovute an un eccessivo dettaglio matematico, non ho ecceduto nella semplificazione di alcuna proposizione contenuta nel libro ed ho cercato di essere attento ad inserire dei caveat appropriati quando delle formule o altri risultati sono validi solo sotto determinate condizioni restrittive. In parecchi casi, nei primi capitoli, ho aggiunto dei \quotes{commenti a lato} per spiegare alcuni punti matematici potenzialmente interessanti e rilevanti per il lettore ma non strettamente necessari per la discussione.

Una estesa varietà di problemi viene fornita per i capitoli 2-8. Uno degli scopi di tali problemi è quello solito di fornire agli studenti l'opportunità di mettere alla prova la propria comprensione dei concetti base introdotti nel capitolo. Vi è comunque un altro importante scopo aggiuntivo per alcuni dei problemi: presentare argomenti che non sono essenziali per lo sviluppo delle idee fondamentali del libro ma che sono, nondimeno, di rilevante interesse ed importanza. Alcuni esempi di tali argomenti trattati nei problemi sono impulso nascosto, effetto Hall, fasci Gaussiani, diffusione Thompson, fibre ottiche, parametri di Stokes e radiazione Cherenkov. Ho scritto questi problemi in modo tale che nella formulazione del problema siano esposti tanto i concetti chiave che i risultati chiave. Il lettore potrebbe in tal modo trovare in questi problemi una utile introduzione a questi argomenti.

La platea di utenti che ho in mente per questo libro include gli studenti universitari in fisica teorica, sebbene io spero che anche studenti in fisica sperimentale, ed altri ancora possano trovare il libro di loro interesse. Questo libro è scritto con l'assunzione che i lettori abbiano seguito un corso introduttivo in elettromagnetismo e che pertanto abbiano già sviluppato un certo livello di comprensione intuitiva dei campi elettrico e magnetico. Mi aspetto anche che i lettori abbiano una solida conoscenza del calcolo vettoriale, ma non assumo che possiedano un bagaglio matematico molto al di là di questo.

Io uso le unità SI dall'inizio alla fine del libro. Sfortunatamente, le unità SI hanno la assai spiacevole caratteristica di introdurre due costanti, $\epsilon_0$ e $\mu_0$, legate dalla relazione $\epsilon_0 \mu_0 c^2 = 1$, in cui $c$ è la velocità della luce. Ci sono ottime ragioni storiche per questa scelta. E' naturale assegnare una permittività elettrica $\epsilon$ e una permeabilità magnetica $\mu$ a molti materiali, e dunque naturale assegnare valori corrispondenti, $\epsilon_0$ e $\mu_0$, al vuoto. Fù dunque un successo veramente grande quello di Maxwell di riconoscere che le proprie equazioni richiedevano che disturbi dei campi elettrico e magnetico si propagassero nel vuoto con velocità $c=\sqrt{\epsilon_0 \mu_0}$  e che questi disturbi dovessero identificarsi con la luce. Tuttavia, questa relazione tra $\epsilon_0$, $\mu_0$ e $c$ significa che in queste costanti c'è ridondanza. Di conseguenza, formule espresse nelle unità SI possono essere trasformate in modo non banale utilizzando questa ridondanza. Per esempio, nelle unità SI, una delle equazioni di Maxwell si scrive abitualmente come $\div{\vb{E}} = \rho / \epsilon_0 $. Tuttavia, questa equazione potrebbe essere espressa altrettanto bene come  $\div{\vb{E}} = \mu_0 c^2 \rho$. Quest'ultima forma può sembrare piuttosto stridente, poiché sembra suggerire che la permeabilità del vuoto e la velocità della luce entrino in una delle equazioni fondamentali dell'elettrostatica. In ogni caso, si deve scegliere quali tra tali costanti usare in ciascuna formula. La convenzione usuale è di usare $\epsilon_0$ nell'equazione di Maxwell di cui sopra e di usare $\mu_0$ nell'equazione di Maxwell ove compare la densità di corrente $\vb{J}$. Tuttavia, questa convenzione non può essere mantenuta quando si scrivono le equazioni di Maxwell nella forma covariante della relatività ristretta, poichè in queste equazioni figura la $4-$corrente $J^{\mu}$, e non è sensato usare convenzioni diverse per diverse componenti di questo $4-$vettore. Infatti, dal capitolo 8 in poi, io sospendo completamente l'uso di $\epsilon_0$ ed uso $\mu_0$  e $c$ in tutte le formule. Per evitare il fastidio connesso a questa ridondanza di $\epsilon_0$, $\mu_0$ e $c$, nella versione originale delle mie lezioni adottai le unità di Gauss. Tuttavia, sebbene decenni orsono le unità di Gauss fossero piuttosto prevalenti, le unità SI sono attualmente utilizzate in modo quasi esclusivo. Perciò, il fastidio delle unità SI è più che compensato dalla non familiarità degli studenti con le unità di Gauss--così come dalla possibilità che qualcuno possa essere indotto dal mio libro ad acquistare apparati elettromagnetici della taglia sbagliata se le formule fossero scritte in unità di Gauss. Perciò, ho scelto di usare unità SI. 

Vettori nell'ordinario spazio $3$-dimensionale saranno denotati in grassetto (p.es., il campo elettrico sarà indicato con $\vb{E}$, come nel precedente paragrafo). Componenti cartesiane di vettori saranno denotate con indici latini in basso e simbolo non in grassetto  (p.es., $E_i$, con $i=1,2,3$, denota le componenti di $\vb{E}$ in una base cartesiana). A partire dal capitolo 8, introduco la nozione di vettore spaziotemporale. Per le ragioni espresse alla sezione 8.1, sarà dunque essenziale introdurre la nozione di vettore duale e distinguere in modo chiaro nella nostra notazione tra vettori e vettori duali. Aderirò quindi alla notazione standard in relatività ristretta, in cui  vettori spaziotemporali vengono denotati con indice Greco in alto (p.es., $W^{\mu}$) e vettori spaziotemporali duali     
vengono denotati con indice Greco in basso (p.es., $U_{\mu}$). Alcune convenzioni aggiuntive connesse alla relatività ristretta vengono enunciate alla fine della sezione 8.1.

Sono in debito con numerosi colleghi per aver letto parti (e, in alcuni casi, tutte) del manoscritto e per avermi fornito un feedback prezioso. Tra questi Sam Gralla, Abe Harte, Jim Isenberg, Istvan Racz, e Gautam Satishchandran, così come numerosi studenti che hanno seguito il mio corso. Tra questi ultimi, Tixuan Tan merita un ringraziamento speciale per aver letto il manoscritto con grande cura e per aver sollevato molti punti riguardanti l'esposizione.\\
\\ 

\thv{Note aggiunte dal traduttore}\\
Le note a pié di pagina il cui numero è seguito dal simbolo $\ddagger$ non sono dell'autore (R. Wald) ma del traduttore.

\chapter{Introduzione: Teoria Elettromagnetica senza Miti}\label{Wald_EM_01}
\thv{From R. M. Wald -- Advanced Classical Electromagnetism, 2022}\\

Il pieno sviluppo della teoria dell'elettromagnetismo nel diciannovesimo secolo rappresenta una delle più grandi conquiste nella storia della fisica. La teoria dell'elettromagnetismo formulata da Maxwell è una teoria matematicamente coerente che fornisce un'eccellente descrizione di una gamma estremamente ampia di fenomeni fisici. Naturalmente, l’elettromagnetismo di Maxwell è una teoria classica che non può descrivere adeguatamente fenomeni in cui le proprietà quantistiche del campo elettromagnetico svolgono un ruolo importante, ma la teoria quantistica del campo elettromagnetico è costruita sulle fondamenta della teoria classica.

Le equazioni di Maxwell mettono in relazione i campi elettrico e magnetico, 
$E$ e $B$, tra loro e con la densità di carica, $\rho$, e la densità di corrente, $\vb*{J}$. 
Cioè, $\rho(\vb{x})$ è la carica elettrica per unità di volume in $\vb{x}$, e per qualsiasi 
vettore unitario $\vu{n}$ in $\vb{x}$, $\vb*{J}(\vb{x}) \cdot \vu{n}$ 
fornisce il flusso di carica per unità di area attraverso un elemento di area perpendicolare 
a $\vu{n}$. 
Le equazioni di Maxwell in unità\footnote{Come discusso nella prefazione, le unità SI hanno la sfortunata caratteristica che le tre costanti $\epsilon_0 \approx 8,85 \times 10^{-12} \text{F/m}$  (la permittività del vuoto), $\mu_0 \approx 1,26 \times 10^{-6} \text{H/m}$ (la permeabilità del vuoto), e $c \approx 3,00 \times 10^8 \text{m/s}$ (la velocità della luce) che compaiono nelle equazioni dell'elettromagnetismo non sono indipendenti ma soddisfano $\epsilon_0 \mu_0 c^2 = 1$. Di conseguenza, l'aspetto delle formule nelle unità SI può essere modificato in modi non banali utilizzando questa identità.} SI sono le seguenti:

\begin{align}
\div{\vb*{E}}  &= \frac{\rho}{\epsilon_0}\,, \label{eq:1.1} \\
\curl{\vb*{B}} - \frac{1}{c^2} \pdv{\vb*{E}}{t} &= \mu_0 \vb*{J}\,,\label{eq:1.2} \\
\div{\vb*{B}}  &= 0\,, \label{eq:1.3} \\
\curl{\vb*{E}} + \pdv{\vb*{B}}{t} &= 0\,.\label{eq:1.4}
\end{align}

Le sorgenti $\rho$ e $\vb*{J}$ debbono soddisfare l'equazione di conservazione carica-corrente
\begin{equation}\label{eq:1.5}
\pdv{\rho}{t} + \div{\vb*{J}} = 0\,,
\end{equation}
poichè altrimenti non esisterebbero soluzioni per le equazioni \ref{eq:1.1} e \ref{eq:1.2}. 
A parte questa restrizione, le quantità $\rho(t, \vb(x))$ e $\vb*{J}(t, \vb(x))$ possono essere assegnate arbitrariamente. 

Le equazioni di Maxwell sono sopravvissute senza modifiche per più di un secolo e mezzo (vale a dire, le equazioni che ho scritto sopra sono equivalenti a quelle fornite da Maxwell). Tuttavia, la nostra comprensione dell’elettromagnetismo a livello fondamentale è progredita notevolmente dai tempi di Maxwell. Nonostante questo, molti modi obsoleti di pensare all’elettromagnetismo rimangono prevalenti. Ciò è fortemente rafforzato dal modo quasi storico in cui l’elettromagnetismo viene solitamente insegnato, anche a livello universitario: normalmente si inizia con la legge di Coulomb in elettrostatica, con le cariche puntiformi considerate \quotes{fondamentali}.

Ciò motiva l'introduzione di un campo elettrico $\vb*{E}$ che soddisfa l'eq. (\ref{eq:1.1}) così come $\curl{\vb*{E}} = 0$ (cioè, l’equazione (\ref{eq:1.4}) con $\pdv*{\vb*{B}}{t} = 0$ ). 
L'energia viene assegnata all'interazione elettrostatica tramite un'analisi del lavoro meccanico svolto durante lo spostamento quasi statico delle cariche puntiformi. Allo stesso modo, in magnetostatica, si inizia normalmente con la legge di Biot-Savart per la forza tra gli elementi di corrente. 
Ciò motiva l'introduzione di un campo magnetico $\vb*{B}$ che soddisfi l'eq. (\ref{eq:1.2}) con $\pdv*{\vb*{E}}{t} = 0$ così come l'eq. (\ref{eq:1.3}). 
Vengono quindi introdotti i termini dinamici in $\vb*{E}$ e $\vb*{B}$ per ottenere le equazioni di Maxwell complete nella forma fornita sopra. 

Ad un certo punto vengono inoltre introdotti, come un modo conveniente per risolvere le equazioni di Maxwell (\ref{eq:1.3}) e (\ref{eq:1.4}), un potenziale scalare, $\phi$, ed un potenziale vettoriale, $\vb*{A}$, che soddisfano\footnote{Entrambe le (\ref{eq:1.3}) ed (\ref{eq:1.4}) sono soddisfatte come conseguenza delle identità 
$\div{(\curl \vb{V})} = 0$ e $\curl{(\grad{f})} = \vb{0}$, a loro volta conseguenza della commutatività delle derivate parziali \tit{miste} del secondo ordine, cioè dell'identità $\pdv*{f}{x}{y} = \pdv*{f}{y}{x}$ valida 
per qualunque funzione $f$ di due o più variabili.}

\begin{align}
\vb*{E}  &= - \grad{\phi} - \pdv{\vb*{A}}{t}\,, \label{eq:1.6} \\
\vb*{B}  &= \curl{\vb*{A}}\,. \label{eq:1.7} 
\end{align}

Questo modo di presentare la teoria dell'elettromagnetismo incoraggia una serie di modi malsani di pensare alla teoria, che ho definito \quotes{miti} nel titolo di questo capitolo. I più perniciosi tra questi miti sono i seguenti: 
(i) Le intensità di campo, $\vb*{E}$ e $\vb*{B}$, sono considerate fondamentali, mentre i potenziali, $\phi$ e $\vb*{A}$, sono visti come quantità introdotte semplicemente per comodità. 
(ii) Le proprietà di energia, quantità di moto e stress del campo elettromagnetico sono considerate proprietà derivate o ipotizzate dalle interazioni con la materia carica e leggi di conservazione piuttosto che proprietà del campo elettromagnetico aventi uno stato fondamentale paragonabile a quello delle stesse equazioni di Maxwell. Ad esempio, a questo proposito, si afferma spesso che la densità dell'impulso del campo elettromagnetico è definita a meno del rotore di un campo vettoriale, poiché non è determinata univocamente dalla conservazione dell'energia. 
(iii) I campi elettromagnetici sono considerati \tit{prodotti} dalla materia carica (in opposizione al fatto che i campi elettromagnetici \tit{interagiscono} con la materia carica). 
(iv) Le cariche puntiformi sono considerate una descrizione fondamentale della materia carica, nonostante le evidenti incongruenze matematiche ad esse associate, come l'autoenergia infinita. 
Nelle sezioni seguenti, farò del mio meglio per sfatare questi miti. 
C'è, ovviamente, un serio problema pedagogico nel fare questo, poiché per seguire pienamente tutta la discussione di questo capitolo, i lettori dovranno avere una notevole conoscenza della teoria elettromagnetica.
Mentre sarebbe ragionevole sperare che i lettori abbiano una conoscenza
considerevole della teoria 
elettromagnetica una volta arrivati 
alla fine di questo libro non è ragionevole presumere tale conoscenza all'inizio.
In effetti, molti dei punti qui discussi verranno adeguatamente spiegati in dettaglio solo negli ultimi due capitoli di questo libro.
Non è necessario che il lettore segua tutti i dettagli della discussione in questo capitolo -- poiché tutto ciò che viene detto in questo capitolo sarà chiarito nel resto del libro -- ma è importante che il lettore acquisisca un’idea del punto di vista sull’elettromagnetismo classico che propongo. Ritengo che sia altamente preferibile iniziare questo libro in questo modo piuttosto che iniziare con il piede sbagliato, seguendo il consueto percorso quasi storico. Nei capitoli successivi svilupperò l'argomento in modo largamente convenzionale, iniziando con l'elettrostatica e la magnetostatica prima di passare all'elettrodinamica completa, ma il punto di vista adottato sarà sempre pienamente compatibile con la discussione di questo capitolo. Prima di discutere i miti di cui sopra, desidero fare alcuni commenti sulla relazione tra l’elettrodinamica classica e la relatività ristretta. Le equazioni di Maxwell non sono compatibili con la struttura dello spaziotempo della fisica pre-relatività a meno che non si abbia un \quotes{sistema di quiete preferenziale}. Questo, di per sé, non era preoccupante nel diciannovesimo secolo, poiché si credeva che esistesse un mezzo meccanico – l’\tit{etere luminifero} – attraverso il quale si propagavano i campi elettromagnetici. Un tale etere fornirebbe naturalmente una struttura di quiete preferita. Tuttavia, la mancanza di prove nell'esperimento di Michelson-Morley per un sistema di quiete preferenziale, così come altri problemi con la teoria dell'etere, hanno portato a gravi difficoltà che alla fine sono state risolte dalla teoria della relatività speciale. Nella teoria della relatività speciale, la funzione temporale newtoniana $t$ (che definisce una “nozione assoluta” di simultaneità) e la metrica dello spazio sono sostituite da un’unica quantità: la metrica dello spaziotempo. L’elettrodinamica classica è pienamente compatibile con la struttura dello spaziotempo della relatività ristretta, senza la necessità dell’etere. La struttura dell'elettrodinamica classica è considerevolmente più semplice se formulata nel quadro della relatività ristretta. Aspetto fino al capitolo 8 per discutere adeguatamente la formulazione dell'elettromagnetismo nell'ambito della relatività ristretta, ma desidero fare qui alcune osservazioni, in modo che il lettore possa avere un'idea di come appare questa formulazione senza aspettare fino alla fine del capitolo. Nella relatività speciale, il potenziale scalare, $\phi$, e il potenziale vettoriale, $\vb*{A}$, sono considerati le componenti temporali e spaziali di un singolo \tit{potenziale quadri-(duale)vettoriale}
\begin{equation}\label{eq:1.8}
A_\mu = (-\phi/c, \vb*{A})\,.
\end{equation}
I campi elettrico e magnetico si immaginano prodotti da un singolo tensore intensità di campo
\begin{equation}\label{eq:1.9}
F_{\mu \nu} = \pdv{A_\nu}{x^\mu} - \pdv{A_\mu}{x^\nu}\,,
\end{equation}
con $x^\mu = (x^0 = ct,  x^1, x^2, x^3)$. Poiché $F_{\mu \nu} = - F_{\nu \mu}$, esso ha 6 componenti indipendenti. Per un osservatore in quiete in questo sistema di coordinate, il campo elettrico corrisponde alle 3 componenti tempo-spazio di $F_{\mu \nu}$ 
\begin{equation}\label{eq:1.10}
E_i = c F_{i0}\,, \quad i=1,2,3,
\end{equation}
mentre il campo magnetico corrisponde alle 3 componenti indipendenti spazio-spazio di $F_{\mu \nu}$
\begin{equation}\label{eq:1.11}
B_i = F_{jk}\,, \quad i=1,2,3,
\end{equation}
ove $(i,j,k)$ è una permutazione ciclica di $(1,2,3)$. In particolare, 
poiché gli osservatori che si muovono l’uno rispetto all’altro definiscono diverse \quotes{direzioni temporali} nello spaziotempo, 
quello che un osservatore affermerebbe essere un \quotes{campo elettrico puro} sarà visto 
da un altro osservatore come una combinazione di campi elettrici e magnetici. La \tit{descrizione invariante} delle intensità di campo è data da $F_{\mu \nu}$. Le equazioni di Maxwell possono quindi essere scritte in termini di $F_{\mu \nu}$, la metrica dello spaziotempo, ed il 4-vettore carica-corrente:
\begin{equation}\label{eq:1.12}
J^\mu = (c \rho, \vb*{J})\,.
\end{equation}
Sebbene la formulazione relativistica speciale dell'elettrodinamica  classica presenti il principale vantaggio  della semplicità, presenta il principale svantaggio della non familiarità. È improbabile che la maggior parte dei lettori abbia familiarità con la distinzione tra, ad esempio, vettori e vettori duali, e con il ruolo svolto dalla metrica dello spaziotempo nelle equazioni della fisica. Sebbene questi concetti non siano eccessivamente difficili da spiegare – e li spiegherò nel capitolo 8 – sarebbe una distrazione eccessiva farlo prima di presentare la teoria dell’elettromagnetismo. Pertanto, rimanderò la discussione sulla relatività speciale al capitolo 8 e, con l'eccezione di alcuni commenti collaterali, non utilizzerò la notazione relativistica speciale per l'elettrodinamica classica fino a quel momento. 

Tuttavia, è importante che il lettore sia consapevole del fatto che l'elettrodinamica classica è compatibile con la struttura dello spaziotempo della relatività ristretta, anche se usiamo una notazione che non la rende manifestamente tale.

\section{Le variabili Elettromagnetiche Fondamentali Sono i Potenziali, Non le Intensità dei Campi}\label{sec:1.1}
Il campo eletromagnetico è un costituente fondamentale della natura. La sua esistenza non richiede di essere giustificata o spiegata più  di quanto richieda di essere giustificata o spiegata l'esistenza, diciamo, degli elettroni. Il campo elettromagnetico è un \quotes{campo di gauge,} lo stesso fondamentale tipo di campo che descrive anche la $W$, la $Z$, bosoni e gluoni. In effetti, il campo elettromagnetico insieme con i campi $W$ e $Z$ costituisce un \quotes{campo di gauge elettrodebole} che descrive tanto le interazioni elettromagnetiche che le interazioni deboli. Tuttavia, per i fenomeni di (\quotes{bassa-energia}) cui ci interessiamo in questo libro, il campo elettromagnetico si disaccoppia dai suoi partner elettrodeboli e può essere preso in considerazione separatamente.
Rimando al capitolo 9 per fornire una discussione matematicamente completa dell'elettromagnetismo come campo di gauge.   
Ciò di cui è necessario che il lettore sia consapevole fin da ora è che la descrizione fondamentale del campo elettromagnetico è data in termini dei potenziali $\phi$ e $\vb*{A}$, non delle intensità di campo $\vb*{E}$ e $\vb*{B}$. Come spiegato in seguito, ci sono situazioni in cui i potenziali contengono più informazioni di quelle che possono essere ottenute dalle intensità del campo. Tuttavia i potenziali $\phi$ e $\vb*{A}$ non descrivono univocamente il campo elettromagnetico: i potenziali $\phi', \vb*{A}'$ e $\phi, \vb*{A}$ sono considerati fisicamente equivalenti (cioè rappresentano lo stesso campo elettromagnetico) se differiscono per una trasformazione di gauge, 
cioè se per qualche funzione $\chi(t, \vb{x})$, abbiamo
\footnote{Nella notazione della relatività ristretta, una trasformazione di gauge può essere espressa più semplicemente come $A_\mu  \longrightarrow A_\mu + \pdv*{\chi}{x^\mu}$.}

\begin{equation}\label{eq:1.13}
\phi' = \phi - \pdv{\chi}{t}\,, \quad \vb*{A}' = \vb*{A} + \grad{\chi}\,.
\end{equation}

In altre parole, un campo elettromagnetico è una classe di equivalenza di potenziali 
$\phi, \vb*{A}$ definita dalla trasformazione (\ref{eq:1.13}). 

Si dimostra facilmente che le intensità di campo, $\vb*{E}$ e $\vb*{B}$, definite dalle equazioni (\ref{eq:1.6}) e (\ref{eq:1.7}), sono gauge invarianti. Inoltre, non è difficile mostrare che in ogni regione semplicemente connessa dello spaziotempo, se 
$\phi_1, \vb*{A}_1$ e  $\phi_2, \vb*{A}_2$ danno origine alle stesse intensità di campo $\vb*{E}$ e $\vb*{B}$, allora $\phi_1, \vb*{A}_1$ e  $\phi_2, \vb*{A}_2$ differiscono al più per una trasformazione di gauge. Pertanto, in una regione semplicemente connessa, 
$\vb*{E}$ e $\vb*{B}$ contengono tutta l'informazione contenuta in $\phi$ ed $\vb*{A}$. 
Poichè tutte le quantità fisicamente misurabili debbono essere gauge invarianti, è molto conveniente in molte circostanze lavorare con $\vb*{E}$ e $\vb*{B}$ piuttosto che con $\phi$ ed $\vb*{A}$. In molti contesti, i fenomeni elettromagnetici possono essere completamente descritti in termini di $\vb*{E}$ e $\vb*{B}$.

Tuttavia, come vedremo nel capitolo 9, l'accoppiamento del campo elettromagnetico alla materia carica fondamentale (vale a dire, campi di carica) può essere descritta solamente in termini di potenziali, non delle intensità di campo. In aggiunta a quanto esposto, vi sono situazioni fisicamente rilevanti in cui $\vb*{E}$ e $\vb*{B}$ non contengono tutta l'informazione associata al campo elettromagnetico. Ad esempio, si consideri la regione esterna ad un solenoide infinito. Supponiamo che all'interno del solenoide vi sia un campo magnetico uniforme non nullo, ma all'esterno del solenoide, abbiamo $\vb*{E} = \vb*{B} = 0$. Poichè la regione esterna al solenoide non è semplicemente connessa, il fatto che $\vb*{E}$ e $\vb*{B}$ si annullino in quella regione non implica che lì i potenziali siano gauge equivalenti a zero.  
Infatti, per il teorema di Stokes, l'eq. (\ref{eq:1.7}) implica che quando $\vb*{B} \neq 0$ all'\tit{interno} del solenoide si abbia 
$\oint \vb*{A} \vdot \dd \vb*{l} \neq 0$ per ogni circuito chiuso che lo racchiuda  dall'\tit{esterno}.
(Si noti che $\oint \vb*{A} \vdot \dd \vb*{l} \neq 0$ è gauge invariante, cioè il suo valore non cambia per effetto di una trasformazione del tipo (\ref{eq:1.13}).)

Una particella quantomeccanica carica che rimane completamente all'esterno del solenoide sarà influenzata dal potenziale vettore in questa zona, ove esso causerà uno spostamento di fase relativo della funzione d'onda attorno al solenoide, producendo uno spostamento fisicamente misurabile nel risultante schema di interferenza. Questo fenomeno, noto come effetto Aharonov-Bohm, è talvolta attribuito alle stranezze della meccanica quantistica. Tuttavia, l'effetto non ha nulla a che fare con la meccanica quantistica: lo stesso effetto si verificherebbe per un campo carico classico\footnote{Per i dettagli si veda B. Felsager, \tit{Geometry, Particles and Fields}, sezione 1.4 .}. 
E non c'è niente di strano nell'effetto, una volta che si riconosca che il campo elettromagnetico è rappresentato, a livello fondamentale, dai potenziali $\phi, \vb*{A}$ (modulo gauge), non dalle intensità di campo $\vb*{E}$ e $\vb*{B}$. Pertanto, mentre per molti scopi è conveniente introdurre e lavorare con le intensitò di campo $\vb*{E}$ e $\vb*{B}$, è importante riconoscere che la descrizione fondamentale del campo elettromagnetico è data dai potenziali $\phi$ ed $\vb*{A}$ . Le equazioni di Maxwell (\ref{eq:1.3}) e (\ref{eq:1.4}) dovrebbero essere viste come date dalle equazioni (\ref{eq:1.6}) e (\ref{eq:1.7}).


\section{Energia, Impulso e Stress del Campo Sono una Parte Integrante della Teoria}\label{sec:1.2}
Al campo elettromagnetico, come a tutte le altre forme di materia, sono associate le proprietà energia, impulso e stress. Queste proprietà, come le equazioni di Maxwell (\ref{eq:1.1})--(\ref{eq:1.4}), sono parte integrante della teoria. Come discusso in modo più approfondito nel capitolo 9, l'elettrodinamica classica può essere formulata sulla base della densità Lagrangiana  
\begin{equation}\label{eq:1.14}
\cL = \frac{1}{2} \left(\epsilon_0 {\abs{\vb*{E}}}^2 - \frac{1}{\mu_0} {\abs{\vb*{B}}}^2 \right) - \phi \rho + \vb*{A} \vdot \vb*{J}\,.
\end{equation}

Qui, come discusso nella sezione \ref{sec:1.1}, le variabili dinamiche sono $\phi$ ed $\vb*{A}$, e le equazioni di Eulero-Lagrange si ottengono variando $\cL$ rispetto a queste variabili; $\vb*{E}$ e $\vb*{B}$ sono trattate come funzioni di $\phi$ ed $\vb*{A}$ definite dalle equazioni (\ref{eq:1.6}) e (\ref{eq:1.7}). Nell'equazione (\ref{eq:1.14}), la densità di carica $\rho$ e la densità di corrente $\vb*{J}$ sono trattate come quantità dinamiche prescritte esternamente
\footnote{Naturalmente, la materia carica dovrebbe realmente avere i propri gradi di libertà dinamici, e dovrebbero esserci termini aggiuntivi nella Lagrangiana che coinvolgono i campi che rappresentano la materia carica. I termini di accoppiamento tra la materia carica ed il campo elettromagnetico dovrebbero quindi essere rappresentati in termini di $\phi$, $\vb*{A}$ ed i campi dinamici che descrivono la materia carica. Ciò verrà visto esplicitamente nel capitolo 9.}. 

Le equazioni di Eulero-Lagrange derivanti dalla variazione dell'eq. (\ref{eq:1.14}) rispetto a $\phi$ ed $\vb*{A}$ sono proprio le equazioni di Maxwell (\ref{eq:1.1})--(\ref{eq:1.2}). Le equazioni di Maxwell aggiuntive (\ref{eq:1.3})--(\ref{eq:1.4}) derivano dalle definizioni (\ref{eq:1.6}) e (\ref{eq:1.7}) di $\vb*{E}$ e $\vb*{B}$, rispettivamente. Il fatto che la Lagrangiana debba essere vista come una funzione di $\phi$ ed $\vb*{A}$ -- e che i termini che rappresentano l'accoppiamento del campo elettromagnetico alla materia carica non possono nemmeno essere scritti in termini di $\vb*{E}$ e $\vb*{B}$ -- è un'ulteriore manifestazione del fatto che le variabili dinamiche fondamentali in elettromagnetismo sono $\phi$ ed $\vb*{A}$.

le proprietà di energia, impulso e stress del campo elettromagnetico sono determinate dal suo accoppiamento con la gravità. L'accoppiamento alla gravità si ottiene generalizzando la Lagrangiana (\ref{eq:1.14}) per lo spaziotempo della relatività speciale allo spaziotempo curvo. Ciò può essere fatto in un modo molto semplice e naturale, che è anche unico se non si ammette che derivate della metrica appaiano nella lagrangiana di Maxwell. Il tensore stress-energia-impulso del campo elettromagnetico si ottiene quindi per differenziazione funzionale della lagrangiana rispetto alla metrica dello spaziotempo, poichè questo è ciò che appare come termine sorgente per la gravità nell'equazione della relatività generale di Einstein. Indico brevemente come ciò funziona nella sezione 9.1. L'unico punto che qui desidero sottolineare è che, proprio come la Lagrangiana (\ref{eq:1.14}) dà origine alle equazioni di Maxwell, la sua naturale generalizzazione allo spaziotempo curvo dà origine alle seguenti formule per la densità di energia $\cE$, la densità di impulso $\cP$ e per il tensore degli stress $\Theta_{ij}$:
\begin{equation}\label{eq:1.15}
\cE = \frac{1}{2} \left(\epsilon_0 {\abs{\vb*{E}}}^2 + \frac{1}{\mu_0} {\abs{\vb*{B}}}^2 \right)\,,
\end{equation}
\begin{equation}\label{eq:1.16}
\cP = \epsilon_0 \vb*{E} \cross \vb*{B}\,,
\end{equation}
\begin{equation}\label{eq:1.17}
\Theta_{ij} =  \frac{1}{2} E_iE_j + \frac{1}{\mu_0} B_iB_j
- \frac{1}{2} \delta_{ij} \left(\epsilon_0 {\abs{\vb*{E}}}^2 + \frac{1}{\mu_0} {\abs{\vb*{B}}}^2 \right)\,.
\end{equation}
Queste formule dovrebbero essere considerate come aventi uno status fondamentale nella teoria dell'elettromagnetismo, paragonabile a quello delle equazioni di Maxwell.
In linea di principio, la validità delle eq. (\ref{eq:1.15})-(\ref{eq:1.17}) potrebbe essere verificata osservando gli effetti gravitazionali dei campi elettromagnetici. I campi elettromagnetici forniscono contributi non trascurabili all'energa di massa della materia ordinaria, certamente abbastanza grandi da produrre effetti gravitazionali osservabili per corpi macroscopici. Tuttavia, non c'è modo di osservare questi effetti separatamente dagli effetti gravitazionali dei costituenti non elettromagnetici della materia. Pertanto, se uno volesse verificare le equazioni (\ref{eq:1.15})-(\ref{eq:1.17}), sarebbe necessario osservare gli effetti gravitazionali dei campi elettromagnetici \tit{liberi}. Gli effetti gravitazionali dei campi elettromagnetici liberi sono troppo piccoli per essere misurati in esperimenti di laboratorio. Tuttavia, nell'universo primordiale, la radiazione elettromagnetica distribuita termicamente che attualmente costituisce lo sfondo cosmico a microonde ha dato un contributo dominante alla densità di energia e alla pressione nell'universo, che influenzano entrambe l'espansione dell'universo. Si osserva che la storia dell'espansione dell'universo è in accordo con la densità di energia elettromagnetica e la pressione della radiazione termica ottenute dalle formule di cui sopra. 
Vi sono importanti leggi di conservazione associate alle eq. (\ref{eq:1.15})-(\ref{eq:1.17}). Nella relatività ristretta, il \quotes{flusso di massa} (impulso) e il \quotes{flusso di energia} rappresentano la stessa quantità, a parte un fattore $c^2$, quindi 
\begin{equation}\label{eq:1.18}
\vb*{\cS} \equiv c^2 \vb*{\cP} =  c^2 \epsilon_0 \vb*{E} \cross \vb*{B} = \frac{1}{\mu_0}\vb*{E} \cross \vb*{B}
\end{equation}
rappresenta il flusso di energia per unità di volume del campo elettromagnetico. Un calcolo basato sulle equazioni di Maxwell produce (per i dettagli, vedi la sezione \ref{sec:5.1})
\begin{equation}\label{eq:1.19}
\pdv{\cE}{t} + \div{\vb*{\cS}} = - \vb*{J} \vdot \vb*{E}\,,
\end{equation}
\begin{equation}\label{eq:1.20}
\pdv{\cP_i}{t} - \sum_{j=1}^{3} \partial_j \Theta_{ij} = -[\rho E_i + {(\vb*{J} \cross \vb*{B})}_i] \,.
\end{equation}
In assenza di cariche e correnti (cioè, quando $\rho = 0$ e $\vb*{J} = \vb{0}$), i termini a destra delle equazioni (\ref{eq:1.19}) e (\ref{eq:1.20}) si annullano. In questo caso l'interpretazione delle equazioni (\ref{eq:1.19}) e (\ref{eq:1.20}) è che esse esprimano la conservazione locale della energia e dell'impulso del campo elettromagnetico. Per chiarire ciò in modo più esplicito, si noti che in un piccolo volume $\delta V$ attorno ad $\vb{x}$, la quantità $\delta V \div{\vb*{\cS}}$ rappresenta il flusso netto dell'energia in uscita da $\delta V$. Per l'eq. (\ref{eq:1.19}), questo è uguale a $-\delta V \pdv*{\cE}{t}$ quando $\rho = 0$ e $\vb*{J} = \vb{0}$, esprimendo quindi la conservazione locale dell'energia. La conservazione dell'energia globale del campo elettromagnetico si ottiene integrando l'eq. (\ref{eq:1.19}) su tutto lo spazio, assumento che  $\vb*{E}$ e $\vb*{B}$ vadano a zero all'infinito in modo sufficientemente rapido. In tal caso, l'integrale su tutto lo spazio di $\div{\vb*{\cS}}$ si annulla per il teorema di Gauss (vedi il capitolo 2), e si ottiene
\begin{equation}\label{eq:1.21}
\dv{t} \int{\cE \dd[3] x} = 0\,,
\end{equation}
posto che si annullino ovunque densità di carica ($\rho = 0$) e corrente ($\vb*{J} = \vb{0}$). Analogamente, quando $\rho = 0$ e $\vb*{J} = \vb{0}$, l'eq. (\ref{eq:1.20}) esprime la conservazione locale dell'impulso e l'integrale dell'eq. (\ref{eq:1.20}) su tutto lo spazio esprime la legge della conservazione globale dell'impulso
\begin{equation}\label{eq:1.22}
\dv{t} \int{\vb*{\cP} \dd[3] x} = 0\,.
\end{equation}

Quando $\rho$ e $\vb*{J}$ non si annullano, in generale non si annullano neanche i termini a destra delle equazioni (\ref{eq:1.19}) e (\ref{eq:1.20}) quindi energia ed impulso elettromagnetici, considerati separatamente dal resto, non si conservano. Ciò è dovuto al fatto che energia ed impulso possono essere scambiati con energia ed impulso della materia carica. Affinché l'energia totale (elettromagnetica e della materia) possa essere conservata localmente, il campo elettromagnetico deve trasferire energia alla materia al tasso 
\begin{equation}\label{eq:1.23}
\pdv{\cE_{matter}}{t} = \vb*{J} \vdot \vb*{E}\,.
\end{equation}
Analogamente, affinché si conservi l'impulso totale, il campo elettromagnetico deve trasferire impulso alla materia al tasso che si ottiene prendendo con il segno meno il termine destro dell'eq. (\ref{eq:1.20}); cioè esso deve esercitare, sulla materia, una forza per unità di volume pari a 
\begin{equation}\label{eq:1.24}
\vb*{f} = \rho \vb*{E} + \vb*{J} \cross \vb*{B}\,.
\end{equation}

Nelle trattazioni standard dell'elettromagnetismo, l'ordine degli argomenti qui presentati è invertito, dando luogo ad alcune gravi difficoltà. Invece di iniziare con le eq. (\ref{eq:1.15})-(\ref{eq:1.17}) per densità di energia, densità di impulso e stress, e quindi derivare la forza di Lorentz, eq. (\ref{eq:1.24}), i trattamenti standard iniziano con la forza di Lorentz, o meglio, la versione della legge di Coulomb di questa espressione per le cariche puntiformi statiche in elettrostatica. Viene quindi calcolato il \quotes{lavoro svolto} per portare le cariche dall'infinito alle loro posizioni e questo lavoro è associato all'energia contenuta nel campo elettromagnetico. Questo argomento alla fine porta alla formula corretta $\frac{\epsilon_0}{2} \int{\abs{\vb*{E}}^2 \dd[3] x} $ per l'energia del campo elettromagnetico in elettrostatica. Tuttavia, questo argomento funziona in elettrostatica perché è possibile spostare un corpo carico in un campo elettrico in modo tale che la sua massa a riposo (cioè l'energia interna) non cambi. Anche se questo può sembrare ovvio, il risultato corrispondente non vale in magnetostatica, perché per mantenere le correnti in un corpo è necessaria energia. Questo vale sia per i magneti permanenti che per i circuiti di corrente. La massa a riposo di un dipolo magnetico cambierà mentre si muove in un campo magnetico non uniforme, come vedremo esplicitamente nella sezione \ref{sec:4.3} e ancora nella sezione \ref{sec:10.3.2}. Come mostro nella sezione \ref{sec:4.3}, l'energia di interazione elettromagnetica di un dipolo magnetico $\vb*{\mu}$ in un campo magnetico esterno $\vb*{B}^{ext}$ può essere derivata direttamente dall'eq. (\ref{eq:1.15}) e dà il valore $+ \vb*{\mu} \vdot \vb*{B}^{ext}$. Tuttavia, molte fonti forniscono la formula errata $- \vb*{\mu} \vdot \vb*{B}^{ext}$ con argomentazioni che si basano sul \quotes{lavoro svolto}, non tenendo conto della variazione della massa a riposo. 

Le formule (\ref{eq:1.15}) per la densità di energia elettromagnetica $\cE$ e (\ref{eq:1.16}) per la densità dell'impulso elettromagnetico $\vb*{\cP}$ sono giustificate in molti trattamenti standard prendendo come punto di partenza l'eq. (\ref{eq:1.19}) (che deriva direttamente dalle equazioni di Maxwell) come espressione della conservazione locale dell'energia. Uno può quindi identificare $\cE$ ed $\vb*{\cS} \equiv c^2 \vb*{\cP}$ con la densità di energia elettromagnetica e, rispettivamente, il flusso di energia. Tuttavia, questo argomento presenta il grave inconveniente che $\vb*{\cP}$ appare nell'eq. (\ref{eq:1.19}) solo nella forma $\div{\vb*{\cP}}$. Ciò porta molti autori a suggerire che $\vb*{\cP}$ sia definito a meno dell'aggiunta del rotore di un campo vettoriale. Questo non è corretto; formule per $\vb*{\cP}$ che differiscono per il rotore di un campo vettoriale avranno conseguenze gravitazionali diverse, quindi se si hanno due formule per $\vb*{\cP}$ che differiscono per un rotore, al più una di esse può essere valida. 

In sintesi, piuttosto che derivare le eq. (\ref{eq:1.15})-(\ref{eq:1.17}) dalle equazioni di Maxwell assumendo che valgano l'eq. (\ref{eq:1.23}) e l'eq. (\ref{eq:1.24}), è molto più sano considerare le formule (\ref{eq:1.15})-(\ref{eq:1.17}) come parte integrale della specificazione della teoria, con l'eq. (\ref{eq:1.23}) e l'eq. (\ref{eq:1.24}) che seguono poi come conseguenze. Le leggi di conservazione (\ref{eq:1.19}) e (\ref{eq:1.20}) costituiscono importanti relazioni di coerenza tra le equazioni di Maxwell (\ref{eq:1.15})-(\ref{eq:1.17}) ma non consentono di dedurre le eq. (\ref{eq:1.15})-(\ref{eq:1.17}) dalle equazioni di Maxwell. Le eq. (\ref{eq:1.15})-(\ref{eq:1.17}) dovrebbero essere viste come aspetti fondamentali della teoria elettromagnetica, con uno status simile a quello delle equazioni di Maxwell.

\section{I Campi Elettromagnetici Non Dovrebbero Essere Visti come Prodotti dalla Materia Carica}\label{sec:1.3}
Le equazioni di Maxwell (\ref{eq:1.1})-(\ref{eq:1.4}) insieme alle eq. (\ref{eq:1.23})-(\ref{eq:1.24}) descrivono l'interazione del campo elettromagnetico con la materia. In questa interazione, il campo elettromagnetico gioca un ruolo assolutamente non subordinato. Il campo elettromagnetico ha i suoi gradi di libertà dinamici indipendenti, e questi dovrebbero essere considerati alla pari con i gradi di libertà dinamici della materia carica. Il campo elettromagnetico non deve essere pensato come prodotto da cariche e correnti, nonostante il fatto che $\rho$ e $\vb*{J}$ siano comunemente indicati come \quotes{termini sorgente} nelle equazioni di Maxwell (e io utilizzi questa terminologia in questo libro). I gradi di libertà dinamici indipendenti del campo elettromagnetico sono identificati dalla formulazione ai valori iniziali delle equazioni di Maxwell, discussa nella sezione \ref{sec:5.4}. Il teorema alla fine di quella sezione afferma quanto segue: 
Si assumano specificati $\rho(t, \vb*{x})$ e $\vb*{J}(t, \vb*{x})$ nello \tit{spaziotempo}, per i quali sia soddisfatta l'equazione di conservazione (\ref{eq:1.5}). 
Siano $\vb*{E}_0(\vb*{x})$ e $\vb*{B}_0(\vb*{x})$ campi vettoriali arbitrari nello spazio tali che $\div{\vb*{E}_0} = \rho(t=0, \vb*{x})/\epsilon_0$, and $\div{\vb*{B}_0} = 0$. 
Allora esiste un'unica soluzione $\vb*{E}(t, \vb*{x}), \vb*{B}(t, \vb*{x})$ delle equazioni di Maxwell (\ref{eq:1.1})-(\ref{eq:1.4}) tale che 
$\vb*{E}(t=0, \vb*{x}) = \vb*{E}_0(\vb*{x})$ e $\vb*{B}(t=0, \vb*{x}) = \vb*{B}_0(\vb*{x})$. Esistono tante soluzioni delle equazioni di Maxwell con 
$\rho$ e $\vb*{J}$ fissati, quante sono le possibili scelte dei campi $\vb*{E}_0(\vb*{x}), \vb*{B}_0(\vb*{x})$ la cui divergenza soddisfa le condizioni di cui sopra. Il fatto che questa informazione iniziale del campo elettromagnetico possa essere specificata liberamente dimostra che il campo eletrromagnetico ha i suoi propri gradi di libertà dinamici indipendenti. Le soluzioni alle equazioni di Maxwell \tit{non} sono determinate da $\rho$ e $\vb*{J}$.

I gradi di libertà dinamici del campo elettromagnetico non sono visibili in elettrostatica e magnetostatica, poichè nessuna soluzione indipendente dal tempo delle equazioni di Maxwell con $\rho = 0$ e $\vb*{J} = \vb{0}$ va a zero all'infinito. Pertanto, se si specificano sorgenti indipendenti dal tempo, 
$\pdv*{\rho}{t} = 0$ e $\pdv*{\vb*{J}}{t} = \vb{0}$, allora le soluzioni delle equazioni di Maxwell per $\vb*{E}$ e $\vb*{B}$ con $\pdv*{\vb*{E}}{t} = \pdv*{\vb*{B}}{t} = 0$ e con $\vb*{E}$ e $\vb*{B}$ che vanno a zero all'infinito sono determinati in modo univoco da $\rho$ e $\vb*{J}$. Di conseguenza, si può associare in modo univoco un campo elettrico stazionario $\vb*{E}$ con una distribuzione di carica stazionaria $\rho$, e si può associare in modo univoco un campo magnetico stazionario $\vb*{B}$ con una distribuzione di corrente stazionaria $\vb*{J}$. Pertanto, è possibile vedere il campo elettrico in elettrostatica come \quotes{prodotto} da cariche, ed è possibile vedere il campo magnetico in magnetostatica come \quotes{prodotto} da correnti. In elettrostatica si può anche farla franca dicendo, come si fa spesso, che le cariche esercitano forze le une sulle altre. Questo, ovviamente, non è il caso: la forza elettromagnetica (\ref{eq:1.24}) su un corpo carico è esercitata dal campo elettromagnetico presente nella posizione del corpo, non da altre cariche distanti.

In elettrodinamica, si è spesso interessati a considerare situazioni in cui \quotes{non c'è radiazione elettromagnetica in arrivo}. Come discusso in modo approfondito nella sezione \ref{sec:5.2}, le soluzioni senza radiazione entrante sono date dalla funzione di Green ritardata applicata a $\rho$ e $\vb*{J}$, e queste soluzioni sono determinate univocamente da $\rho$ e $\vb*{J}$. Ancora una volta, questo rende possibile ritenere che, in assenza di radiazioni in arrivo, i campi elettromagnetici sono \quotes{prodotti} dalle cariche e dalle correnti. Tuttavia, sebbene la condizione di \quotes{nessuna radiazione in arrivo} sia un'utile idealizzazione applicabile a molti problemi, non dovrebbe essere presa sul serio come condizione iniziale per il nostro universo. Anche se certamente non conosciamo le precise condizioni del \quotes{big bang}, sappiamo che la materia nell'universo primordiale era un plasma estremamente caldo e denso. In un plasma così caldo e denso, il campo elettromagnetico \quotes{produce} cariche (ad esempio coppie elettrone-positrone) più o meno nella stessa misura in cui le cariche \quotes{producono} campi elettromagnetici. Certamente non ha alcun senso pensare che prima arrivino le cariche e poi producano il campo elettromagnetico.

Pertanto, sebbene vi siano circostanze in cui si potrebbe ritenere che i campi elettromagnetici siano prodotti da cariche, è molto più sano pensare al campo elettromagnetico e alla materia carica come entità indipendenti che interagiscono tramite le equazioni di Maxwell (\ref{eq:1.23}) e (\ref{eq:1.24}). In effetti, l'idea che i campi elettromagnetici siano prodotti da cariche è particolarmente insostenibile nella teoria quantistica dei campi, poiché è essenziale per la comprensione di fenomeni come le fluttuazioni del campo elettromagnetico nel vuoto che il campo elettromagnetico abbia i propri gradi di libertà dinamici, indipendentemente dall'esistenza di materia carica.

\section{A Livello Fondamentale, la Materia Carica Classica Deve Essere vista come Continua, Piuttosto che Puntiforme}\label{sec:1.4}

Le equazioni di Maxwell (\ref{eq:1.1})-(\ref{eq:1.4}) sono state formulate utilizzando una nozione di continuo per la densità di carica $\rho$ e per la densità di corrente $\vb*{J}$; cioè, $\rho$ e  $\vb*{J}$ sono state considerate funzioni regolari di $(t, \vb*{x})$. Queste equazioni costituiscono la formulazione matematicamente ben posta di un problema ai valori iniziali, come già menzionato nella sezione \ref{sec:1.3} e come discusso in modo approfondito nella sezione \ref{sec:5.4}. Tuttavia, in una teoria completa, è necessario specificare anche un modello della materia carica e le sue equazioni del moto. A livello fondamentale, come discusso più avanti nel capitolo 9, si ritiene che la materia carica sia costituita da campi (quantistici) carichi.  Si possono tuttavia anche considerare \quotes{modelli fenomenologici} della materia carica, come un fluido carico. In ogni caso, le equazioni del moto della materia carica insieme alle equazioni di Maxwell costituiscono un sistema accoppiato che deve essere risolto simultaneamente, poichè il moto della materia carica dipende dal campo elettromagnetico, ma l'evoluzione dinamica del campo elettrodinamico dipende dalle cariche e correnti della materia. E' essenziale che il sistema accoppiato Maxwell--materia-carica abbia una formulazione ben posta del problema ai valori inizialiin modo che non vi siano difficoltà in linea di principio, nell'ottenere soluzioni per il sistema accoppiato per date condizioni iniziali. Tuttavia, almeno il 99\% di ciò che viene normalmente trattato nei corsi di elettromagnetismo non considera il sistema Maxwell-materia completo e accoppiato, ma considera invece i seguenti due problemi idealizzati:

\begin{itemize}
\item Type I.  Per dati $\rho$ e  $\vb*{J}$ specificati esternamente, trovare i campi elettromagnetici corrispondenti (vale a dire, l'unica soluzione stazionaria in elettrostatica e magnetostatica e/o la soluzione ritardata in elettrodinamica)
\item Type II. Trovare il movomento di un corpo carico per dati campi $\vb*{E}$ e $\vb*{B}$ specificati esternamente (cioè trascurando i campi propri associati alla presenza del corpo carico). 
\end{itemize}

Per questi problemi idealizzati è molto utile introdurre la nozione di carica puntiforme. Per \quotes{particella puntiforme di carica $q$} che si muove sulla linea d'universo $\vb*{X}(t)$ (con $\abs{\dv*{\vb*{X}}{t}} < c$ per ogni $t$) si intende la carica-corrente 
\begin{equation}\label{eq:1.25}
\rho(t, \vb*{x}) = q \delta(\vb*{x} - \vb*{X}(t))\,, \quad \vb*{J}(t, \vb*{x}) = q \dv{\vb*{X}(t)}{t}(t) \delta(\vb*{x} - \vb*{X}(t))\,,
\end{equation}
ove $\delta$ denota la funzione delta di Dirac 3-dimensionale. Questa può essere pensata come il limite di una distribuzione di carica che ad ogni $t$ diventa sempre più concentrata nel punto $\vb*{X}(t)$. Questo limite non definisce una funzione, ma ha un significato ben definito come distribuzione\footnote{Una distribuzione è una mappa lineare da \quotes{funzioni test} (ovvero funzioni regolari che non sono nulle solo in una regione delimitata) in numeri che dipendono in modo continuo dalla funzione test in un senso appropriato. La funzione delta di Dirac è semplicemente la mappa di valutazione sulle funzioni di test; cioè $\delta(\vb*{x} - \vb*{X}(t))$ mappa la funzione di test $f$ nel numero $f(\vb*{X})$}.
L'eq. (\ref{eq:1.25}) soddisfa l'eq. (\ref{eq:1.5}) in un senso distributivo ben definito.  

Si può vedere dalla legge di Gauss che per le soluzioni dell'eq. (\ref{eq:1.1}) con $\rho$ dato  dall'eq. (\ref{eq:1.25}), il campo eletrrico $\vb*{E}$ deve divergere vicino alla carica come $1/\abs{\vb*{x} - \vb*{X}}^2$. Di conseguenza per l'eq. (\ref{eq:1.15}), la densità di energia elettromagnetica diverge come $1/\abs{\vb*{x} - \vb*{X}}^4$, che non è integrabile. Pertanto l'energia elettromagnetica totale di una carica puntiforme è infinita, e quindi le cariche puntiformi non possono essere considerate oggetti fisici nell'elettrodinamica classica. Tuttavia possono essere introdotte nel contesto di problemi del tipo I o del tipo II di cui sopra.    
    
\chapter{Elettrostatica}\label{Wald_EM_02}
\thv{Da R. M. Wald -- Advanced Classical Electromagnetism, 2022}\\

Prima di considerare l'elettrodinamica nella sua interezza, è molto istruttivo fornire un'analisi completa del caso in cui la densità di carica $\rho$ e la densità di corrente $\vb*{J}$ sono indipendenti dal tempo ($\pdv*{\rho}{t} = 0$ e $\pdv*{\vb*{J}}{t} = \vb{0}$), e anche i potenziali 
$\phi$ ed $\vb*{A}$ sono indipendenti dal tempo ($\pdv*{\phi}{t} = 0$ e $\pdv*{\vb*{A}}{t} = \vb{0}$). In tal caso, l'equazione (\ref{eq:1.1}) per 
$\vb*{E} = - \grad{\phi}$ si disaccoppia completamente dall'equazione (\ref{eq:1.2}) per $\vb*{B} = \curl{\vb*{A}}$. Basta quindi considerare separatamente i casi in cui, oltre alla stazionarietà, o abbiamo $\vb*{J} = \vb*{A} = \vb{0}$ (elettrostatica) oppure abbiamo 
$\rho = \phi = 0$ (magnetostatica). Tratteremo la magnetostatica nel capitolo 4.

La maggior parte delle trattazioni dell'elettrostatica iniziano con le cariche puntiformi e la legge di Coulomb ed infine arrivano all'equazione di Poisson. Noi iniziamo con le equazioni di Maxwell, che si riducono immediatamente all'equazione di Poisson. Introduco le cariche puntiformi nella sezione \ref{sec:2.2} e ottengo la legge di Coulomb alla fine della sezione \ref{sec:2.3}. La sezione \ref{sec:2.1} stabilisce le proprietà chiave delle soluzioni. 

\section[Unicità delle Soluzioni]{Unicità delle Soluzioni in Elettrostatica}\label{sec:2.1}
Assumiamo $\vb*{J} = \vb*{A} = \vb{0}$ e $\pdv*{\rho}{t} = \pdv*{\phi}{t} = 0$. Le uniche equazioni non banali dell'elettromagnetismo, in questo caso, sono la prima equazione di Maxwell (\ref{eq:1.1}),  
\begin{equation}\label{eq:2.1}
\div{\vb*{E}}  = \frac{\rho}{\epsilon_0}\,, 
\end{equation}

e la relazione tra $\vb*{E}$ e $\phi$, eq. (\ref{eq:1.6}),
\begin{equation}\label{eq:2.2}
\vb*{E}  = - \grad{\phi}\,. 
\end{equation}

Queste equazioni si possono combinare in un'unica equazione
\begin{equation}
\laplacian{\phi}  = - \frac{\rho}{\epsilon_0}\,, 
\end{equation}\label{eq:2.3}
in cui l'\tit{operatore Laplaciano}, $\laplacian$, è definito in coordinate Cartesiane come
\begin{equation}\label{eq:2.4}
\laplacian \equiv \div{\grad} = \pdv[2]{}{x} + \pdv[2]{}{y} + \pdv[2]{}{z} \,.  
\end{equation}

L'equazione (\ref{eq:2.3}) è conosciuta come l'\tit{equazione di Poisson}.

Si noti che la libertà di gauge eq. (\ref{eq:1.13}) è fortemente limitata in elettrostatica dal requisito
che $\phi$ sia indipendente dal tempo e $\vb*{A} = 0$. Pertanto, le uniche trasformazioni di gauge consentite sono generate da 
$\chi(t, \vb*{x}) = t \times \mathrm{costante}$, quindi l'unica libertà di gauge in $\phi$ è
\begin{equation}\label{eq:2.5}
\phi \longrightarrow \phi' = \phi + \mathrm{costante}\,. 
\end{equation}

Il seguente teorema è alla base di molti risultati in elettrostatica.

\tbi{Teorema (Teorema di Gauss):} \tit{Sia $\vb*{v}$ un campo vettoriale arbitrario differenziabile in $\R^3$.
Sia $\cV \subset \R^3$ una regione limitata la cui frontiera, $S = \partial \cV$, è una superficie bi-dimensionale (vedi il commento di seguito a questo teorema). Sotto queste condizioni si ha} 
\begin{equation}\label{eq:2.6}
\int_{\cV} \div{\vb*{v}} \dd[3]{x} = \int_S \vb*{v} \vdot \vu{n} \dd{S}
\end{equation}

\tit{ove $\vu{n}$ è il vettore unitario normale ad $S$ \quotes{diretto all'esterno} (cioè al di fuori di $\cV$) e $dS$ denota l'elemento di area su $S$.}


\section{Cariche Puntiformi e Funzioni di Green}\label{sec:2.2}


\section{Energia di Interazione e Forza}\label{sec:2.3}


\section[Espansione Multipolo]{Espansione Multipolo della Funzione di Green}\label{sec:2.4}


\section[Cavità Conduttrici]{Cavità Conduttrici; Funzioni di Green di Dirichlet e Neumann}\label{sec:2.5}

\section*{Problemi}


\appendixpage
\appendix
%\include{Wald_EM_Notes_A1}
%\include{Wald_EM_Notes_A2}
%\include{Wald_EM_Notes_A3}
%\include{Wald_EM_Notes_A4}

\backmatter%%%%%%%%%%%%%%%%%%%%%%%%%%%%%%%%%%%%%%%%%%%%%%%%%%%%%%%
%%%%%%%%%%%%%%%%%%%%%%%%% referenc.tex %%%%%%%%%%%%%%%%%%%%%%%%%%%%%%
% sample references
% 
% Use this file as a template for your own input.
%
%%%%%%%%%%%%%%%%%%%%%%%% Springer-Verlag %%%%%%%%%%%%%%%%%%%%%%%%%%

%
% BibTeX users please use
% \bibliographystyle{}
% \bibliography{}
%
% Non-BibTeX users please use
\begin{thebibliography}{99.}
%
% and use \bibitem to create references.
%
% Use the following syntax and markup for your references
%
% Monograph
\bibitem{Griffiths_4th} D.J. Griffiths (2017)
Introduction to Electrodynamics. Cambridge University Press, Cambridge

% Monograph
\bibitem{Felsager_1981} B. Felsager (1981)
Geometry, Particles and Fields. Odense University Press

% Monograph
\bibitem{BudakFomin_1973} B.M. Budak, S.V. Fomin (1973)
Multiple Integrals, Field Theory and Series. Mir Publishers, Moscow

% Monograph
\bibitem{Postnikov_II_1982} Mikhail Postnikov (1982)
Lectures in Geometry, Semester II. Linear Algebra and Differential Geometry. Mir Publishers, Moscow

\end{thebibliography}

\printindex

%%%%%%%%%%%%%%%%%%%%%%%%%%%%%%%%%%%%%%%%%%%%%%%%%%%%%%%%%%%%%%%%%%%%%%

\end{document}





