%%%%%%%%%%%%%%%%%%%% book.tex %%%%%%%%%%%%%%%%%%%%%%%%%%%%%

\documentclass[english, 11pt, a4paper]{book}
%\usepackage[nomath]{lmodern}
\usepackage[T1]{fontenc}
\usepackage[italian]{babel}
% The following changes the Chapter heading from 'Chapter' to 'Lecture'
%$\addto\captionsenglish{\renewcommand{\chaptername}{Lecture}}
%$%\usepackage{fancyhdr}
%$\newcommand\chap[1]{%
%$ \chapter*{#1}%
%$  \addcontentsline{toc}{chapter}{#1}}
%$\newcommand\sect[1]{%
%$  \section*{#1}%
%$  \addcontentsline{toc}{section}{#1}}
  
  
% choose options for [] as required from the list
% in the Reference Guide, Sect. 2.2

\usepackage{makeidx}         % allows index generation
\usepackage{graphicx}        % standard LaTeX graphics tool
\usepackage{subcaption}      % for subfigures environments 
                             % when including figure files
\usepackage{multicol}        % used for the two-column index
\usepackage[bottom]{footmisc}% places footnotes at page bottom
% etc.
% see the list of further useful packages
% in the Reference Guide, Sects. 2.3, 3.1-3.3
\usepackage[normalem]{ulem}

\usepackage[shortlabels]{enumitem}	% to be able to resume enumerated lists

\usepackage{amsmath}	% To be able to slash
\usepackage{bm}	        % To use bold greek letters in math mode with \bm{}
\usepackage{amsfonts}	% To be able to use \mathbb ... 
\usepackage{amssymb}	% To be able to use \nmid ... 
\usepackage{amsthm}		% \qed, \qedhere
\usepackage{slashed}	% any character (dirac)
\usepackage[title,toc,page]{appendix}

% See https://tex.stackexchange.com/questions/36524/how-to-put-a-framed-box-around-text-math-environment/36528
\usepackage{collectbox}	% To make box around formulas

% *** AFTER THIS LINE *** 
%     put \usepackage{} for shared packages kept under ~\Links\repos\git\LaTeX_Styles

% Physics package 
% https://tex.stackexchange.com/questions/38978/how-can-i-manually-install-a-latex-package-debian-ubuntu-linux
\usepackage{/home/marcello/Links/repos/git/LaTeX_Styles/physics}	
% To put accents below letters
\usepackage{/home/marcello/Links/repos/git/LaTeX_Styles//accents}

% To control vertical white space above and below equations
% see https://tex.stackexchange.com/questions/69662/how-to-globally-change-the-spacing-around-equations
\expandafter\def\expandafter\normalsize\expandafter{%
    \normalsize
    \setlength\abovedisplayskip{16pt}
    \setlength\belowdisplayskip{16pt}
    \setlength\abovedisplayshortskip{16pt}
    \setlength\belowdisplayshortskip{16pt}
}

% To use extra symbols like dagger and double dagger in numbering footnotes 
\usepackage{footmisc}

% Force chapter numbering to restart within each part
\makeatletter
%\@addtoreset{chapter}{part}
\makeatletter


\makeindex             % used for the subject index
                       % please use the style svind.ist with
                       % your makeindex program


%%%%%%%%%%%%%%%%%%%%%%%%%%%%%%%%%%%%%%%%%%%%%%%%%%%%%%%%%%%%%%%%%%%%%

\begin{document}

\newcommand{\quotes}[1]{``#1''}
\newcommand{\sfT}{$\mathsf{T}$}
\newcommand{\udT}{\rotatebox[origin=c]{180}{$\mathsf{T}$}}

%Bold calligraphic letters 
\newcommand{\N}{\mathbb{N}}	% integers
\newcommand{\Z}{\mathbb{Z}}	% relative
\newcommand{\Q}{\mathbb{Q}}	% rationals
\newcommand{\R}{\mathbb{R}}	% reals
\newcommand{\C}{\mathbb{C}}	% complex
\newcommand{\F}{\mathbb{F}}	% generic field 1
\newcommand{\K}{\mathbb{K}}	% generic field 2
\newcommand{\V}{\mathbb{V}}	% Shankar's for vector space V

%Plain calligraphic letters 
\newcommand{\cC}{\mathcal{C}}    % space 1
\newcommand{\cF}{\mathcal{F}}    % space 2
\newcommand{\cS}{\mathcal{S}}    % space 3 
\newcommand{\cT}{\mathcal{T}}    % space 4 

\newcommand{\cU}{\mathcal{U}}    % sets 1
\newcommand{\cV}{\mathcal{V}}    % sets 2
\newcommand{\cW}{\mathcal{W}}    % sets 3
\newcommand{\cP}{\mathcal{P}}    % sets 4 
\newcommand{\cQ}{\mathcal{Q}}    % sets 5
\newcommand{\cR}{\mathcal{R}}    % sets 6

\newcommand{\cY}{\mathcal{Y}}    % Y

% Calligraphic H for Hilbert space
\newcommand{\cH}{\mathcal{H}}    

% To show argument of the exponential function vertically, i.e., as a superscript 
\newcommand{\vexp}[1]{\,e^{#1}}

% To type an angle as a number of degrees like 45^\circ
\newcommand{\degree}[1]{{#1}^\circ}

% To create boldface vectors with a hat or check accent 
\newcommand{\hatvb}[1]{\vb{\hat{#1}}}
\newcommand{\chkvb}[1]{\vb{\check{#1}}}

% To create not-bold vectors with a hat or check accent 
\newcommand{\hatv}[1]{\hat{#1}}
\newcommand{\chkv}[1]{\check{#1}}

% To create <x|, |x> and <x|y> with unit vectors inside
\newcommand{\ubra}[1]{\bra*{\vu{#1}}}
\newcommand{\uket}[1]{\ket*{\vu{#1}}}
\newcommand{\uip}[2]{\ip*{\vu{#1}}{\vu{#2}}}

% To put accents below letters. 
\newcommand{\ut}[1]{\underaccent{\tilde}{#1}}
\newcommand{\uh}[1]{\underaccent{\hat}{#1}}
\newcommand{\form}[1]{\uh{#1}}

% To create italic, bold, bolditalic text
\newcommand{\tit}[1]{\textit{#1}}
\newcommand{\tbf}[1]{\textbf{#1}}
\newcommand{\tbi}[1]{\textit{\textbf{#1}}}

% Latin Modern sans serif |OR| Helvetica (SELECT)
\newcommand{\textlmss}{\fontfamily{lmss}\selectfont}
\newcommand{\texthv}{\fontfamily{phv}\selectfont}

% Latin Modern sans serif |OR| Helvetica (USE, within OR outside MATH !)
\newcommand{\tlmss}[1]{\text{\textlmss{#1}}}
\newcommand{\thv}[1]{\text{\texthv{#1}}}

% To use \tlmss{T} symbol to denote transpose 
\newcommand{\transp}[1]{{#1}^{\tlmss{T}}}

% To use \dagger symbol to denote operator Adjoint
\newcommand{\Adj}[1]{{#1}^\dagger}

% To denote the Hermitian conjugate with a '+' superscript
\newcommand{\Hconj}[1]{{#1}^{+}}

% To use \tlmss{Ker}, \tlmss{Coker} and \tlmss{Img} to denote Kernel, Co-Kernel & Image 
\newcommand{\Ker}{\tlmss{Ker}\,}
\newcommand{\Coker}{\tlmss{Coker}\,}
\newcommand{\Img}{\tlmss{Im}\,}

% To use \tlmss{Alt} and \tlmss{alt} to denote alternation 
\newcommand{\Alt}{\tlmss{Alt}\,}
\newcommand{\alt}{\tlmss{alt}\,}

% To use \tlmss{Ann} to denote annulets 
\newcommand{\Ann}{\tlmss{Ann}\,}

% Misc abbreviations
\newcommand{\ora}[1]{\overrightarrow{#1}}

\DeclareRobustCommand{\rchi}{{\mathpalette\irchi\relax}}
\newcommand{\irchi}[2]{\raisebox{\depth}{$#1\chi$}} % inner command, used by \rchi

% See https://tex.stackexchange.com/questions/36524/how-to-put-a-framed-box-around-text-math-environment/36528
\makeatletter
\newcommand{\mybox}{%
    \collectbox{%
        \setlength{\fboxsep}{1pt}%
        \fbox{\BOXCONTENT}%
    }%
}
\makeatother

\author{Robert M. Wald}
\title{Advanced Classical Electromagnetism}
%\subtitle{Italian translation of the Princeton University Press, 2022 edition.}
\maketitle

\frontmatter%%%%%%%%%%%%%%%%%%%%%%%%%%%%%%%%%%%%%%%%%%%%%%%%%%%%%%

%\include{dedic}

%\chapter*{Plan}
\label{plan} 

In this book I keep notes about the classical theory of fields, as exposed in several books. 

The presentation is made homogeneous by adoption of the Gauss's units. 
Each topic is often covered in several books, as it will be evident by a cursory glance at the following list of titles taken from the index of those sources. This is deliberate, as different sources may present the same topic from different angles, or provide different illuminating examples. Introductory and more pedagogically oriented presentations are generally put ahead of the more technical and more advanced, thus avoiding a useless duplication of content. Each chapter in these notes \textit{adopts} a corresponding chapter in one of the listed sources as a \textit{track} so that there should be no \textit{holes}. However, this is by no means a pedantic replica, because (i) all content is cast into a uniform notation regardless of notation and units used in each source and (ii) additional content or examples are included when deemed necessary. 

Focus is on the classic theory of the electromagnetic fields. However, selected content from sources like Felsager, Franklin and Mister-Thorne-Weeler, are meant to establish strong connections with contiguous areas of theoretical physics, namely the Lagrangian and Hamiltonian formalism, Quantum Mechanics and General Relativity. 

When completed, these notes could become the ideal companion to a person willing to teach (or to rehearse) the Classical Theory of Fields, at different levels of depth. 

Included in these notes is a set of fully developed examples and problems (some proposed in the source books, some invented by myself). These I consider essential for a good understanding of the matter. The reason to work them out explicitly is that it is seldom the case for a person to have the time to work each one of them out from scratch without loosing moment in such endeavour.  \\\\      

\begin{itemize}

\item D.J. Griffiths -- Introduction to Electrodynamics
\begin{enumerate}
\setcounter{enumi}{0}
\item Vector Analysis
\end{enumerate}

\item B. Felsager -- Geometry Particles and Fields
\begin{enumerate}
\setcounter{enumi}{0}
\item Electromagnetism (1.1 to 1.4)
\end{enumerate}

\item D.J. Griffiths -- Introduction to Electrodynamics
\begin{enumerate}
\setcounter{enumi}{1}
\item Electrostatics
\item Potentials
\item Electric Fields in Matter
\item Magnetostatics
\item Magnetic Fields in Matter
\item Electrodynamics
\item Conservation Laws
\item Electromagnetic Waves
\item Radiation
\item Electrodynamics and Relativity
\item Potentials and Fields
\end{enumerate}

\item J.D. Jackson -- Classical Electrodynamics, 2nd Edition
\begin{enumerate}
\setcounter{enumi}{0}
\item Introduction to Electrostatics
\item Boundary Value Problems in Electrostatics - I
\item Boundary Value Problems in Electrostatics - II
\item Multipoles, Electrostatics of Macroscopic Media, Dielectrics
%\item Magnetostatics
%\item Time Varying Fields, Maxwell Equations, Conservation Laws
%\item Plane Electromagnetic Waves and Wave Propagation
%\item Wave Guides and Resonant Cavities
%\item Simple Radiating Systems, Scattering and Diffraction
%\item Magnetohydrodynamics and Plasma Physics
\end{enumerate}

\item J.D. Jackson -- Classical Electrodynamics, 3rd Edition
\begin{enumerate}
\setcounter{enumi}{4}
\item Magnetostatics, Faraday's Law, Quasi-Static Fields
\item Maxwell Equations, Macroscopic Electromagnetism, Conservation Laws
\item Plane Electromagnetic Waves and Wave Propagation
\item Wave Guides, Resonant Cavities and Optical Fibers
\item Radiating Systems, Multipole Fields and Radiation
\item Scattering and Diffraction
\end{enumerate}

% --------------- LAGRANGIAN, HAMILTONIAN, RELATIVITY, FIELD-PARTICLES INTERACTION --------  

\item J. Franklin -- Advanced Mechanics and General Relativity
\begin{enumerate}
\setcounter{enumi}{0}
\item Newtonian Gravity
\item Relativistic Mechanics
\end{enumerate}

\item B. Felsager -- Geometry Particles and Fields
% Contacts with quantum theory of particles dynamics in EM fields
\begin{enumerate}
\setcounter{enumi}{1}
\item Interaction of Fields and Particles
\end{enumerate}

\item J.D. Jackson -- Classical Electrodynamics, 3rd Edition
\begin{enumerate}
\setcounter{enumi}{10}
\item Special Theory of Relativity
\item Dynamics of Relativistic Particles and Electromagnetic Fields
\end{enumerate}

\item J.D. Jackson -- Classical Electrodynamics, 3rd Edition
\begin{enumerate}
\setcounter{enumi}{12}
\item Collisions, Energy Loss and Scattering of Charged Particles, Cherenkov and Transition Radiation
\item Radiation by Moving Charges
\item Bremsstrahlung, Method of Virtual Quanta, Radiative Beta Processes
\item Radiation Damping, Classical Models of Charged Particles
\end{enumerate}

% --------------- CONTACTS with QUANTUM THEORY of FIELD DYNAMICS --------  

\item B. Felsager -- Geometry Particles and Fields
\begin{enumerate}
\setcounter{enumi}{2}
\item Dynamics of Classical Fields
\end{enumerate}

% --------------- CONTACTS with DIFFERENTIAL GEOMETRY MATH & GENERAL RELATIVITY --------  

\item J. Franklin -- Advanced Mechanics and General Relativity
\begin{enumerate}
\setcounter{enumi}{2}
\item Tensors
\item Curved Space
\item Scalar Field Theory
\item Tensor Field Theory (6.1 to 6.5)
\end{enumerate}

\item B. Felsager -- Geometry Particles and Fields
\begin{enumerate}
\setcounter{enumi}{5}
\item Differentiable Manifolds, Tensor analysis
\item Differential Forms, Exterior Calculus
\item Integral Calculus on Manifolds
\end{enumerate}

\item C.W. Misner, K.S. Thorne, J.A. Wheeler -- Gravitation
\begin{enumerate}
\setcounter{enumi}{1}
\item Foundations of Special Relativity
\item The Electromagnetic Field
\item Electromagnetism and Differential Forms
\end{enumerate}


% --------------- REVIEW IT ALL AGAINST LANDAU-LIFSHITZ Volume II --------  

\item L.D. Landau, E.M. Lifshitz -- Teoria dei Campi
\begin{enumerate}
\setcounter{enumi}{0}
\item Principio di Relatività
\item Meccanica Relativistica
\item Carica in un Campo Elettromagnetico
\item Equazioni del Campo Elettromagnetico
\item Campo Elettromagnetico Costante
\item Onde Elettromagnetiche
\item Propagazione della Luce
\item Campo di Cariche in Moto
\item Radiazione Elettromagnetica
\end{enumerate}

% --------------- EXPAND ELECTRODYNAMICS IN CONTINUOUS MEDIA ALONG LANDAU-LIFSHITZ Volume VIII --------  

\item L.D. Landau, E.M. Lifshitz -- Elettrodinamica dei Mezzi Continui
\begin{enumerate}
\setcounter{enumi}{0}
\item Elettrostatica dei Conduttori
\item Elettrostatica nei Dielettrici
\item Corrente Continua
\item Campo Magnetico Costante
\item Ferromgnetismo e Antiferromagnetismo
\item Superconduttività
\item Campo Magnetico Quasi Stazionario
\item Idrodinamica Magnetica
\item Equazioni delle Onde Elettromagnetiche
\item Propagazione delle Onde Elettromagnetiche
\item Onde Elettromagnetiche in Mezzi Anisotropi
\item Dispersione Spaziale
\item Ottica non Lineare
\item Passaggio delle Particelle Veloci attraverso la Materia
\item Diffusione delle Onde Elettromagnetiche
\item Diffrazione dei Raggi X nei Cristalli
\end{enumerate}

\end{itemize}
	
\tableofcontents
%\addappheadtotoc

\mainmatter%%%%%%%%%%%%%%%%%%%%%%%%%%%%%%%%%%%%%%%%%%%%%%%%%%%%%%%
\chapter*{Prefazione}\label{Wald_EM_00}
\thv{From R. M. Wald -- Advanced Classical Electromagnetism, 2022}\\

Questo libro è nato dal corso avanzato del primo trimestre in elettromagnetismo tenuto ai laureati all'Università di Chicago nell'inverno del 2018. Erano trascorsi decenni da quando avevo precedentemente insegnato questa materia, per cui mi ci sono accostato con occhi nuovi, ed è stato naturale per me cercare di ripensare a come presentarla a studenti laureati. Nel farlo, mi è risultato evidente che l'usuale metodo quasi-storico di presentare la materia induce modi non appropriati di concepire l'elettromagnetismo. Pertanto, per evitare di partire con il piede sbagliato, decisi di spendere le prime lezioni del corso per descrivere quelli che nel capitolo 1 di questo libro chiamo con il termine di \quotes{miti} riferiti all'elettromagnetismo. Constatai che partendo in questo modo,  divenne molto più facile presentare in modo diretto gli argomenti in modo chiaro e conciso, senza dover effettuare cambi di prospettiva nel  corso del loro sviluppo. Ho insegnato il corso nuovamente nel successivi 3 anni, fornendo appunti delle lezioni alla classe. Gli appunti di queste lezioni sono ora state sviluppate in questo libro.

Il primo capitolo di questo libro è perciò una introduzione all'elettromagnetismo alquanto fuori dall'ordinario. Invece di cominciare con la forza tra particelle cariche, e discutere come questo da luogo ad un concetto di \quotes{campo}, e così via, il mio proposito nel capitolo 1 è di spiegare agli studenti come essi dovrebbero pensare all'elettromagnetismo da una prospettiva moderna e matematicamente precisa. I punti principali esposti in questo capitolo sono che (i) i potenziali, non le intensità dei campi, sono le variabili dinamiche fondamentali nell'elettromagnetismo; (ii) le proprietà energia ed impulso associate al campo elettromagnetico sono parte essenziale della formulazione della teoria e non possono essere derivate da argomentazioni sul \quotes{\tit{lavoro fatto}} dal campo; (iii) non si deve pensare ai campi elettromagnetici come \tit{prodotti} da cariche; e (iv) a livello fondamentale nell'elettrodinamica classica la materia carica deve essere considerata come distribuita in modo continuo piuttosto che essere costituita da cariche puntiformi. Molti di questi punti non possono essere chiariti appieno se non negli ultimi capitoli del libro--particolarmente i capitoli 9 e 10--ma il mio intento è di esporre queste idee in modo sufficientemente chiaro ed esplicito nel capitolo 1 in modo da poter poi mantenere questi capisaldi nel prosieguo del libro senza ulteriori giustificazioni. 

Gli argomenti trattati nei capitoli 2-7 sono quelli che verrebbero normalmente esposti in ogni corso avanzato di elettromagnetismo. L'Elettrostatica è trattata nel capitolo 2, ma a partire dall'equazione di Poisson, non dalla legge di Coulomb. I materiali dielettrici in elettrostatica sono trattati nel capitolo 3, con un considerevole grado di attenzione a come eseguire la media su scale macroscopiche e al trattamento dell'energia. La Magnetostatica è trattata nel capitolo 4, con una discussione completa della differenza di segno tra magnetostatica ed elettrostatica nella energia di interazione di un dipolo in un campo esterno--e come questa si collega al cambiamento della massa a riposo di un magnete quando esso si muove in modo quasi-statico in un campo magnetico esterno. Elettrodinamica e radiazione sono discusse a fondo nel capitolo 5. In aggiunta agli argomenti normalmente trattati nei testi di elettromagnetismo, io derivo in quel capitolo la formulazione ai valori iniziali per le equazioni di Maxwell. L'Elettrodinamica nei mezzi viene trattata nel capitolo 6, inclusa una discussione della mgnetoidrodinamica. L'approssimazione dell'ottica geometrica nella dinamica ondulatoria è presentata nella prima sezione del capitolo 7, seguita da una discussione di interferenza e coerenza ed un'analisi di due problemi nella diffrazione: diffusione da una palla dielettrica e propagazione e radiazione attraverso un'apertura.

La relatività ristretta è trattata nel capitolo 8. La relatività ristretta è alla base della formulazione della teoria elettromagnetica, quindi dovrebbe propriamente essere presentata all'inizio di un libro sull'elettromagnetismo, piuttosto che essere relegata in un capitolo verso la fine del libro. Tuttavia, la relatività ristretta rimane un argomento così poco familiare a gran parte degli studenti che ciò non è possibile. Molte esposizioni della relatività ristretta si concentrano sulle regole per applicare trasformazioni di Lorentz a certe quantità, senza fornire molte informazioni sul contenuto geometrico sottostante della teoria. All'opposto sarebbe naturale, per me che mi occupo di relatività generale, introdurre un livello di maggiore astrazione matematica e apparato geometrico di quanto sarebbe necessario per fornire una chiara descrizione della relatività ristretta. Ho posto una considerevole cura nella scrittura della sezione 8.1 in modo tale da introdurre la relatività ristretta in modo concettualmente chiaro senza introdurre un livello di astrazione maggiore di quello che ritengo essenziale. Questa sezione può essere letta in modo indipendente dal resto del libro, e spero che possa fornire da sola una utile introduzione alla relatività ristretta. 
Viene quindi esposta la formulazione dell'elettromagnetismo nel contesto della relatività ristretta, seguita da una discussione del moto di particelle cariche e della radiazione da una carica puntiforme in moto arbitrario.

Il capitolo 9 tratta l'elettromagnetismo come una teoria di gauge, con ciò portando la formulazione dell'elettromagnetismo in questo libro al livello di comprensione concettuale che è stato raggiunto alla metà del ventesimo secolo. Vi è un gap considerevole tra il modo in cui viene normalmente descritto il campo elettromagnetico e la sua interazione con la materia carica nei corsi di elettromagnetismo classico e quello in cui esso appare di fatto come una interazione della natura nel modello standard della fisica delle particelle. Questo capitolo dovrebbe essere di aiuto a colmare questo gap.    

Infine, la nozione di carica puntuale viene discusso in dettaglio nel capitolo 10. Si mostra che è possibile considerare il limite, in senso matematicamente ben definito, in cui l'estensione di un corpo carico si riduce a zero, purché anche la carica e la massa del corpo vengano ridotte in modo proporzionale alla sua estensione. In questo limite si ottiene la forza di Lorentz. Le correzioni dovute alla \quotes{self-force} si possono poi calcolare perturbativamente in modo matematicamente rigoroso. Nella parte finale di questo capitolo viene trattato il problema di come descrivere in modo consistente il moto di un corpo carico tenendo conto delle correzioni dovute alla \quotes{self-force} senza introdurre soluzioni spurie (\quotes{runaway}).

Nel libro, ho cercato di formulare tutti i principali concetti ed i risultati della teoria dell'elettromagnetismo in modo chiaro e conciso. Tuttavia, non ho cercato di presentare un'ampia collezione di esempi o applicazioni. Queste caratteristiche del libro spiegano il fatto che la sua lunghezza sia circa un terzo di alcuni altri testi di elettromagnetismo avanzato con una simile copertura di argomenti. 

Ho cercato di presentare ogni cosa con un alto livello di precisione matematica. Sebbene mi sia sforzato di evitare distrazioni dovute an un eccessivo dettaglio matematico, non ho ecceduto nella semplificazione di alcuna proposizione contenuta nel libro ed ho cercato di essere attento ad inserire dei caveat appropriati quando delle formule o altri risultati sono validi solo sotto determinate condizioni restrittive. In parecchi casi, nei primi capitoli, ho aggiunto dei \quotes{commenti a lato} per spiegare alcuni punti matematici potenzialmente interessanti e rilevanti per il lettore ma non strettamente necessari per la discussione.

Una estesa varietà di problemi viene fornita per i capitoli 2-8. Uno degli scopi di tali problemi è quello solito di fornire agli studenti l'opportunità di mettere alla prova la propria comprensione dei concetti base introdotti nel capitolo. Vi è comunque un altro importante scopo aggiuntivo per alcuni dei problemi: presentare argomenti che non sono essenziali per lo sviluppo delle idee fondamentali del libro ma che sono, nondimeno, di rilevante interesse ed importanza. Alcuni esempi di tali argomenti trattati nei problemi sono impulso nascosto, effetto Hall, fasci Gaussiani, diffusione Thompson, fibre ottiche, parametri di Stokes e radiazione Cherenkov. Ho scritto questi problemi in modo tale che nella formulazione del problema siano esposti tanto i concetti chiave che i risultati chiave. Il lettore potrebbe in tal modo trovare in questi problemi una utile introduzione a questi argomenti.

La platea di utenti che ho in mente per questo libro sono studenti laureati in fisica teorica, sebbene io spero che anche studenti in fisica sperimentale, non laureati ed altri ancora possano trovare il libro di loro interesse. Questo libro è scritto con l'assunzione che i lettori abbiano seguito un corso introduttivo in elettromagnetismo e che pertanto abbiano già sviluppato un certo livello di comprensione intuitiva dei campi elettrico e magnetico. Mi aspetto anche che i lettori abbiano una solida conoscenza del calcolo vettoriale, ma non assumo che possiedano un bagaglio matematico molto al di là di questo.

Io uso le unità SI dall'inizio alla fine del libro. Sfortunatamente, le unità SI hanno la assai spiacevole caratteristica di introdurre due costanti, $\epsilon_0$ e $\mu_0$, legate dalla relazione $\epsilon_0 \mu_0 c^2 = 1$, in cui $c$ è la velocità della luce. Ci sono ottime ragioni storiche per questa scelta. E' naturale assegnare una permittività elettrica $\epsilon$ e una permeabilità magnetica $\mu$ a molti materiali, e dunque naturale assegnare valori corrispondenti, $\epsilon_0$ e $\mu_0$, al vuoto. Fù dunque un successo veramente grande quello di Maxwell di riconoscere che le proprie equazioni richiedevano che disturbi dei campi elettrico e magnetico si propagassero nel vuoto con velocità $c=\sqrt{\epsilon_0 \mu_0}$  e che questi disturbi dovesero identificarsi con la luce. Tuttavia, questa relazione tra $\epsilon_0$, $\mu_0$ e $c$ significa che in queste costanti c'è ridondanza. Di conseguenza, formule espresse nelle unità SI possono essere trasformate in modo non banale utilizzando questa ridondanza. Per esempio, nelle unità SI, una delle equazioni di Maxwell si scrive abitualmente come $\div{\vb{E}} = \rho / \epsilon_0 $. Tuttavia, questa equazione potrebbe essere espressa altrettanto bene come  $\div{\vb{E}} = \mu_0 c^2 \rho$. Quest'ultima forma può sembrare piuttosto stridente, poiché sembra suggerire che la permeabilità del vuoto e la velocità della luce entrino in una delle equazioni fondamentali dell'elettrostatica. In ogni caso, si deve scegliere quali tra tali costanti usare in ciascuna formula. La convenzione usuale è di usare $\epsilon_0$ nell'equazione di Maxwell di cui sopra e di usare $\mu_0$ nell'equazione di Maxwell ove compare la densità di corrente $\vb{J}$. Tuttavia, questa convenzione non può essere mantenuta quando si scrivono le equazioni di Maxwell nella forma covariante della relatività ristretta, poichè in queste equazioni figura la $4-$corrente $J^{\mu}$, e non è sensato usare convenzioni diverse per diverse componenti di questo $4-$vettore. Infatti, dal capitolo 8 in poi, io sospendo completamente l'uso di $\epsilon_0$ ed uso $\mu_0$  e $c$ in tutte le formule. Per evitare il fastidio connesso a questa ridondanza di $\epsilon_0$, $\mu_0$ e $c$, nella versione originale delle mie lezioni adottai le unità di Gauss. Tuttavia, sebbene decenni orsono le unità di Gauss fossero piuttosto prevalenti, le unità SI sono attualmente utilizzate in modo quasi esclusivo. Perciò, il fastidio delle unità SI è più che compensato dalla non familiarità degli studenti con le unità di Gauss--così come dalla possibilità che qualcuno possa essere indotto dal mio libro ad acquistare apparati elettromagnetici della taglia sbagliata se le formule fossero scritte in unità di Gauss. Perciò, ho scelto di usare unità SI. 

Vettori nell'ordinario spazio $3$-dimensionale saranno denotati in grassetto (p.es., il campo elettrico sarà indicato con $\vb{E}$, come nel precedente paragrafo). Componenti cartesiane di vettori saranno denotate con indici latini in basso e simbolo non in grassetto  (p.es., $E_i$, con $i=1,2,3$, denota le componenti di $\vb{E}$ in una base cartesiana). A partire dal capitolo 8, introduco la nozione di vettore spaziotemporale. Per le ragioni espresse alla sezione 8.1, sarà dunque essenziale introdurre la nozione di vettore duale e distinguere in modo chiaro nella nostra notazione tra vettori e vettori duali. Aderirò quindi alla notazione standard in relatività ristretta, in cui  vettori spaziotemporali vengono denotati con indice Greco in alto (p.es., $W^{\mu}$) e vettori spaziotemporali duali     
vengono denotati con indice Greco in basso (p.es., $U_{\mu}$). Alcune convenzioni aggiuntive connesse alla relatività ristretta vengono enunciate alla fine della sezione 8.1.

Sono in debito con numerosi colleghi per aver letto parti (e, in alcuni casi, tutte) del manoscritto e per avermi fornito un feedback prezioso. Tra questi Sam Gralla, Abe Harte, Jim Isenberg, Istvan Racz, e Gautam Satishchandran, così come numerosi studenti che hanno seguito il mio corso. Tra questi ultimi, Tixuan Tan merita un ringraziamento speciale per aver letto il manoscritto con grande cura e per aver sollevato molti punti riguardanti l'esposizione. 


\chapter{Introduzione: \\Teoria Elettromagnetica senza Miti}\label{Wald_EM_01}
\thv{From R. M. Wald -- Advanced Classical Electromagnetism, 2022}\\

\chapter{Elettrostatica}\label{Wald_EM_02}
\thv{Da R. M. Wald -- Advanced Classical Electromagnetism, 2022}\\

Prima di considerare l'elettrodinamica nella sua interezza, è molto istruttivo fornire un'analisi completa del caso in cui la densità di carica $\rho$ e la densità di corrente $\vb*{J}$ sono indipendenti dal tempo ($\pdv*{\rho}{t} = 0$ e $\pdv*{\vb*{J}}{t} = \vb{0}$), e anche i potenziali 
$\phi$ ed $\vb*{A}$ sono indipendenti dal tempo ($\pdv*{\phi}{t} = 0$ e $\pdv*{\vb*{A}}{t} = \vb{0}$). In tal caso, l'equazione (\ref{eq:1.1}) per 
$\vb*{E} = - \grad{\phi}$ si disaccoppia completamente dall'equazione (\ref{eq:1.2}) per $\vb*{B} = \curl{\vb*{A}}$. Basta quindi considerare separatamente i casi in cui, oltre alla stazionarietà, o abbiamo $\vb*{J} = \vb*{A} = \vb{0}$ (elettrostatica) oppure abbiamo 
$\rho = \phi = 0$ (magnetostatica). Tratteremo la magnetostatica nel capitolo 4.

La maggior parte delle trattazioni dell'elettrostatica iniziano con le cariche puntiformi e la legge di Coulomb ed infine arrivano all'equazione di Poisson. Noi iniziamo con le equazioni di Maxwell, che si riducono immediatamente all'equazione di Poisson. Introduco le cariche puntiformi nella sezione \ref{sec:2.2} e ottengo la legge di Coulomb alla fine della sezione \ref{sec:2.3}. La sezione \ref{sec:2.1} stabilisce le proprietà chiave delle soluzioni. 

\section[Unicità delle Soluzioni]{Unicità delle Soluzioni in Elettrostatica}\label{sec:2.1}
Assumiamo $\vb*{J} = \vb*{A} = \vb{0}$ e $\pdv*{\rho}{t} = \pdv*{\phi}{t} = 0$. Le uniche equazioni non banali dell'elettromagnetismo, in questo caso, sono la prima equazione di Maxwell (\ref{eq:1.1}),  
\begin{equation}\label{eq:2.1}
\div{\vb*{E}}  = \frac{\rho}{\epsilon_0}\,, 
\end{equation}

e la relazione tra $\vb*{E}$ e $\phi$, eq. (\ref{eq:1.6}),
\begin{equation}\label{eq:2.2}
\vb*{E}  = - \grad{\phi}\,. 
\end{equation}

Queste equazioni si possono combinare in un'unica equazione
\begin{equation}
\laplacian{\phi}  = - \frac{\rho}{\epsilon_0}\,, 
\end{equation}\label{eq:2.3}
in cui l'\tit{operatore Laplaciano}, $\laplacian$, è definito in coordinate Cartesiane come
\begin{equation}\label{eq:2.4}
\laplacian \equiv \div{\grad} = \pdv[2]{}{x} + \pdv[2]{}{y} + \pdv[2]{}{z} \,.  
\end{equation}

L'equazione (\ref{eq:2.3}) è conosciuta come l'\tit{equazione di Poisson}.

Si noti che la libertà di gauge eq. (\ref{eq:1.13}) è fortemente limitata in elettrostatica dal requisito
che $\phi$ sia indipendente dal tempo e $\vb*{A} = 0$. Pertanto, le uniche trasformazioni di gauge consentite sono generate da 
$\chi(t, \vb*{x}) = t \times \mathrm{costante}$, quindi l'unica libertà di gauge in $\phi$ è
\begin{equation}\label{eq:2.5}
\phi \longrightarrow \phi' = \phi + \mathrm{costante}\,. 
\end{equation}

Il seguente teorema è alla base di molti risultati in elettrostatica.

\tbi{Teorema (Teorema di Gauss):} \tit{Sia $\vb*{v}$ un campo vettoriale arbitrario differenziabile in $\R^3$.
Sia $\cV \subset \R^3$ una regione limitata la cui frontiera, $S = \partial \cV$, è una superficie bi-dimensionale (vedi il commento di seguito a questo teorema). Sotto queste condizioni si ha} 
\begin{equation}\label{eq:2.6}
\int_{\cV} \div{\vb*{v}} \dd[3]{x} = \int_S \vb*{v} \vdot \vu{n} \dd{S}
\end{equation}

\tit{ove $\vu{n}$ è il vettore unitario normale ad $S$ \quotes{diretto all'esterno} (cioè al di fuori di $\cV$) e $dS$ denota l'elemento di area su $S$.}


\section{Cariche Puntiformi e Funzioni di Green}\label{sec:2.2}


\section{Energia di Interazione e Forza}\label{sec:2.3}


\section[Espansione Multipolo]{Espansione Multipolo della Funzione di Green}\label{sec:2.4}


\section[Cavità Conduttrici]{Cavità Conduttrici; Funzioni di Green di Dirichlet e Neumann}\label{sec:2.5}

\section*{Problemi}


\appendixpage
\appendix
%\include{Wald_EM_Notes_A1}
%\include{Wald_EM_Notes_A2}
%\include{Wald_EM_Notes_A3}
%\include{Wald_EM_Notes_A4}

\backmatter%%%%%%%%%%%%%%%%%%%%%%%%%%%%%%%%%%%%%%%%%%%%%%%%%%%%%%%
%%%%%%%%%%%%%%%%%%%%%%%%% referenc.tex %%%%%%%%%%%%%%%%%%%%%%%%%%%%%%
% sample references
% 
% Use this file as a template for your own input.
%
%%%%%%%%%%%%%%%%%%%%%%%% Springer-Verlag %%%%%%%%%%%%%%%%%%%%%%%%%%

%
% BibTeX users please use
% \bibliographystyle{}
% \bibliography{}
%
% Non-BibTeX users please use
\begin{thebibliography}{99.}
%
% and use \bibitem to create references.
%
% Use the following syntax and markup for your references
%
% Monograph
\bibitem{Griffiths_4th} D.J. Griffiths (2017)
Introduction to Electrodynamics. Cambridge University Press, Cambridge

% Monograph
\bibitem{Felsager_1981} B. Felsager (1981)
Geometry, Particles and Fields. Odense University Press

% Monograph
\bibitem{monograph} Joel Franklin (2010)
Advanced Mechanics and General Relativity. Cambridge University Press, Cambridge, UK

% Monograph
\bibitem{Boas_2006} M.L. Boas (2006)
Mathematical Methods in the Physical Sciences. John Wiley \& Sons

% Monograph
\bibitem{BudakFomin_1973} B.M. Budak, S.V. Fomin (1973)
Multiple Integrals, Field Theory and Series. Mir Publishers, Moscow

% Monograph
\bibitem{Postnikov_II_1982} Mikhail Postnikov (1982)
Lectures in Geometry, Semester II. Linear Algebra and Differential Geometry. Mir Publishers, Moscow

% Monograph
\bibitem{Choquet-Bruhat_1982} Y. Choquet-Bruhat, M. Dillard-Bleick (1982)
Analysis, Manifolds and Physics - Part I: Basics. North-Holland, Amsterdam

\end{thebibliography}

\printindex

%%%%%%%%%%%%%%%%%%%%%%%%%%%%%%%%%%%%%%%%%%%%%%%%%%%%%%%%%%%%%%%%%%%%%%

\end{document}





