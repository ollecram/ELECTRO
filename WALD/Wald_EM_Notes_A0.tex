\chapter{Matematica per l'aspirante Fisico Teorico}\label{app:A0}

Mi sembra interessante concepire la produzione (o raccolta) organica di contenuti formativi
destinati a studiosi che non sono matematici o fisici teorici, ma sono tuttavia motivati ad approfondire ed estendere il proprio livello di maturità matematica per una personale aspirazione a comprendere le basi su cui poggia la formulazione di parti fondamentali della Fisica Teorica contemporanea.

Per comprenderci: qui non si tratta di spiegare la fisica \tit{a parole}, cioè senza formule matematiche, come tentano di fare le decine di (anche ottimi) libri di divulgazione. Si tratta invece di delineare un percorso che partendo da una esposizione fortemente integrata delle  fondamenta dell'Algebra, della Geometria e  dell'Analisi Matematica, fornisca allo studioso la padronanza di un linguaggio espressivo e potente, capace di stimolare l'intuizione e permettere la comprensione dei modelli teorici con cui la Fisica contemporanea cerca di descrivere la Natura. 

La cosa più sorprendente, come sottolineato da A. Einstein, è che questi modelli teorici della realtà fisica siano esprimibili in termini di concetti matematici estremamente profondi e tra loro fortemente interconnessi. Come conseguenza di questa constatazione e dell'esperienza finora maturata, siamo indotti a pensare che la strada più fruttuosa per sostituire un modello teorico che appare inadeguato a rendere conto di nuovi fenomeni, sia quella di generalizzarne la sua formulazione matematica in modo che il nuovo modello renda conto anche di questi ultimi.

In questa sede non ci proponiamo di dare alcun contributo originale alla Fisica Teorica, quindi il nostro scopo è unicamente quello di presentare, in lingua Italiana, nell'ordine ed al livello più appropriato, tutta la matematica necessaria alla formulazione di alcune parti fondative della fisica teorica contemporanea, tra le quali elenchiamo:\\

\begin{enumerate}[I]
\item Teoria classica dei campi (Elettromagnetismo)
\item Teoria relativistica del campo gravitazionale (Relatività generale),
\item Meccanica quantistica non relativistica (Fisica atomica),
\item Teoria quantistica (relativistica) del campo elettromagnetico. 
\end{enumerate} 

Tra le omissioni più significative della lista precedente elenchiamo: 
\begin{enumerate}[(1)]
\item Termodinamica,
\item Meccanica statistica,
\item Fisica dei solidi,
\item Fisica nucleare, 
\item Fisica delle particelle. 
\end{enumerate} 

Il motivo delle esclusioni di cui sopra è soprattutto quello di definire un perimetro entro cui lo sforzo di apprendimento richiesto sia ancora contenibile nell'arco di tempo di pochi anni. Tuttavia la scelta di cosa includere nella prima lista è principalmente dovuta al fatto che la trattazione dei relativi argomenti poggia su basi matematiche comuni, che sono poi le basi della cosiddetta \quotes{Teoria dei Campi} \tit{classica} e \tit{quantistica}. 

Nella esposizione, in Italiano, sarà conveniente mantenere la terminologia dei testi in lingua Inglese per termini come \tit{Manifold} (\tit{Varietà} nella letteratura matematica in Italiano), mentre per altri come \tit{limit point} sembra preferibile la traduzione letterale (\tit{punto limite}) piuttosto che il termine (\tit{punto di accumulazione}) spesso usato nei testi di Analisi Matematica in Italiano.    

\section{Linee guida generali}\label{app:A0.1}

Per ciascuna delle 4 aree tematiche (parti fondative) sopra elencate, verranno indicati uno o più testi di riferimento, generalmente in lingua Inglese, che ne espongano la formulazione fisico-teorica nel modo ritenuto più moderno ed efficace.\\
Verranno poi indicati, sempre per area tematica, uno o più testi di riferimento con cui fornire preliminarmente allo studioso le basi matematiche necessarie per una piena comprensione della esposizione fisico-teorica. Nella struttura seguente, un certo numero di testi di matematica viene elencato come propedeutico al tema fisico-teorico che li precede: questo implica il contenuto di tali risorse matematiche dovrebbe essere assimilato prima dello (o parallelamente allo) studio dellla esposizione fisico-teorica.    

\begin{enumerate}[I]
%========================================================================
\item Teoria classica dei campi (Elettromagnetismo)
%------------------------------------------------------------------------
\begin{enumerate}[(i)]
\item Advanced Classical Electromagnetism, [R. Wald, 2023]
\item Classical Covariant Fields, [M. Burgess, 2022]
\end{enumerate} 
%------------------------------------------------------------------------
\begin{enumerate}[(a)]
\item Principles of Mathematical Analysis, 3rd Ed, [W. Rudin, 1964]
\item Introduction to Manifolds, 2nd Edition, [Loring W. Tu, 2010]
\end{enumerate} 
%========================================================================
\item Teoria relativistica del campo gravitazionale (Relatività generale),
%------------------------------------------------------------------------
\begin{enumerate}[(i)]
\item Some GR book, [A. Bbbbb, 1666]
\item Some other GR book, [C. Ddddd, 1777]
\end{enumerate} 
%------------------------------------------------------------------------
\begin{enumerate}[(a)]
\item Some Differential Geometry book, [A. Bbbbb, 1666]
\item Some other Differential Geometry book, [C. Ddddd, 1777]
\end{enumerate} 
%========================================================================
\item Meccanica quantistica non relativistica (Fisica atomica),
%------------------------------------------------------------------------
\begin{enumerate}[(i)]
\item Some QM book, [A. Bbbbb, 1666]
\item Some other QM book, [C. Ddddd, 1777]
\end{enumerate} 
%------------------------------------------------------------------------
\begin{enumerate}[(a)]
\item Some Hilbert Space book, [A. Bbbbb, 1666]
\item Some Functional Analysis book, [C. Ddddd, 1777]
\end{enumerate} 
%========================================================================
\item Teoria quantistica (relativistica) del campo elettromagnetico. 
%------------------------------------------------------------------------
\begin{enumerate}[(i)]
\item Some QFT book, [A. Bbbbb, 1666]
\item Some other QFT book, [C. Ddddd, 1777]
\end{enumerate} 
%------------------------------------------------------------------------
\begin{enumerate}[(a)]
\item Some Fancy Math book for QFT, [A. Bbbbb, 1666]
\item Some other Fancy Math book for QFT, [C. Ddddd, 1777]
\end{enumerate} 
\end{enumerate} 
 
