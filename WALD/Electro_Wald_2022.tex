%%%%%%%%%%%%%%%%%%%% book.tex %%%%%%%%%%%%%%%%%%%%%%%%%%%%%

\documentclass[english, 11pt, a4paper]{book}

% Some specific typographic conventions used in Griffiths I2QM   START 
\usepackage{mathtools}			% equation tag with [..] instead of (..)
\newtagform{brackets}{[}{]}		% equation tag with [..] instead of (..)
\usetagform{brackets}			% equation tag with [..] instead of (..)
% Some specific typographic conventions used in Griffiths I2QM   END 

%\usepackage[nomath]{lmodern}
\usepackage[T1]{fontenc}
%\usepackage[italian]{babel}
% The following changes the Chapter heading from 'Chapter' to 'Lecture'
%$\addto\captionsenglish{\renewcommand{\chaptername}{Lecture}}
%$%\usepackage{fancyhdr}
%$\newcommand\chap[1]{%
%$ \chapter*{#1}%
%$  \addcontentsline{toc}{chapter}{#1}}
%$\newcommand\sect[1]{%
%$  \section*{#1}%
%$  \addcontentsline{toc}{section}{#1}}

% The following for using the Euro symbol
\usepackage[utf8x]{inputenc}
\usepackage{lmodern, textcomp}
  
% choose options for [] as required from the list
% in the Reference Guide, Sect. 2.2

\usepackage{makeidx}         % allows index generation
\usepackage{graphicx}        % standard LaTeX graphics tool
\usepackage{subcaption}      % for subfigures environments 
                             % when including figure files
\usepackage{multicol}        % used for the two-column index
\usepackage[bottom, symbol]{footmisc} % See https://tex.stackexchange.com/questions/826/symbols-instead-of-numbers-as-footnote-markers
% etc.
% see the list of further useful packages
% in the Reference Guide, Sects. 2.3, 3.1-3.3
\usepackage[normalem]{ulem}

\usepackage[shortlabels]{enumitem}	% to be able to resume enumerated lists

\usepackage{amsmath}	% To be able to slash
\usepackage{bm}	        % To use bold greek letters in math mode with \bm{}
\usepackage{amsfonts}	% To be able to use \mathbb ... 
\usepackage{amssymb}	% To be able to use \nmid ... 
\usepackage{amsthm}		% \qed, \qedhere
\usepackage{slashed}	% any character (dirac)
\usepackage[title,toc,page]{appendix}

% See https://tex.stackexchange.com/questions/36524/how-to-put-a-framed-box-around-text-math-environment/36528
\usepackage{collectbox}	% To make box around formulas

% *** AFTER THIS LINE *** 
%     put \usepackage{} for shared packages kept under ~\Links\repos\git\LaTeX_Styles

% Physics package 
% https://tex.stackexchange.com/questions/38978/how-can-i-manually-install-a-latex-package-debian-ubuntu-linux
\usepackage[italicdiff]{/home/marcello/Links/repos/git/LaTeX_Styles/physics}	
% To put accents below letters
\usepackage{/home/marcello/Links/repos/git/LaTeX_Styles//accents}

% To attach a symbols (instead of a number) to footnotes 
%\usepackage[symbol]{footmisc}

% To control vertical white space above and below equations
% see https://tex.stackexchange.com/questions/69662/how-to-globally-change-the-spacing-around-equations
\expandafter\def\expandafter\normalsize\expandafter{%
    \normalsize
    \setlength\abovedisplayskip{16pt}
    \setlength\belowdisplayskip{16pt}
    \setlength\abovedisplayshortskip{16pt}
    \setlength\belowdisplayshortskip{16pt}
}

% FROM BOXED_TEXT_ETC.tex

% To write two equations side by side
\usepackage{multicol}

% To use PGF/TikZ https://tex.stackexchange.com/questions/3622/best-way-to-generate-nice-function-plots-in-latex
\usepackage{tikz}
\usetikzlibrary{datavisualization}
\usetikzlibrary{datavisualization.formats.functions}

% To create a placeholder paragraph with Latin text
\usepackage{lipsum}

% To create framed text boxes with custom defined styles 
%\usepackage[linewidth=1pt]{mdframed}
\usepackage[framemethod=TikZ]{mdframed}
\mdfdefinestyle{MyFrame}{%
    linecolor=brown,			% blue, orange, brown, ...
    outerlinewidth=1pt,
    roundcorner=10pt,
    innertopmargin=\baselineskip,
    innerbottommargin=\baselineskip,
    innerrightmargin=15pt,
    innerleftmargin=15pt,
    backgroundcolor=gray!5!white}

		%Use for creating boxed/framed parts of text with nice borders

% To use extra symbols like dagger and double dagger in numbering footnotes 
\usepackage{footmisc}

% Allows aligning numbers at decimal point within `tabular environment
%\usepackage{siunitx}
%\sisetup{
%  round-mode          = places, % Rounds numbers
%  round-precision     = 2, % to 2 places
%}

% Force chapter numbering to restart within each part
\makeatletter
%\@addtoreset{chapter}{part}
\makeatletter


\makeindex             % used for the subject index
                       % please use the style svind.ist with
                       % your makeindex program


%%%%%%%%%%%%%%%%%%%%%%%%%%%%%%%%%%%%%%%%%%%%%%%%%%%%%%%%%%%%%%%%%%%%%

\begin{document}

% Useful within \begin{equation*}...\end{equation*} to have ONE equation with number & label
\newcommand\numberthis{\addtocounter{equation}{1}\tag{\theequation}}

\newcommand{\umlaut}[1]{\"#1}
\newcommand{\quotes}[1]{``#1''}
\newcommand{\ovr}[1]{\overline{#1}}
\newcommand{\sfT}{$\mathsf{T}$}
\newcommand{\udT}{\rotatebox[origin=c]{180}{$\mathsf{T}$}}
\newcommand{\avg}[1]{\langle{#1}\rangle}

%Bold calligraphic letters 
\newcommand{\N}{\mathbb{N}}	% integers
\newcommand{\Z}{\mathbb{Z}}	% relative
\newcommand{\Q}{\mathbb{Q}}	% rationals
\newcommand{\R}{\mathbb{R}}	% reals
\newcommand{\C}{\mathbb{C}}	% complex
\newcommand{\F}{\mathbb{F}}	% generic field 1
\newcommand{\K}{\mathbb{K}}	% generic field 2
\newcommand{\V}{\mathbb{V}}	% Shankar's for vector space V

%Plain calligraphic letters 
\newcommand{\cC}{\mathcal{C}}    % space 1
\newcommand{\cF}{\mathcal{F}}    % space 2
\newcommand{\cH}{\mathcal{H}}    % Calligraphic H for Hilbert space
\newcommand{\cS}{\mathcal{S}}    % space 3, Flow of energy (e.g in electromagnetism)
\newcommand{\cT}{\mathcal{T}}    % space 4 

\newcommand{\cI}{\mathcal{I}}    % Moment of Inertia
\newcommand{\cU}{\mathcal{U}}    % sets 1
\newcommand{\cV}{\mathcal{V}}    % sets 2
\newcommand{\cW}{\mathcal{W}}    % sets 3
\newcommand{\cP}{\mathcal{P}}    % sets 4, Momentum density (e.g in electromagnetism) 
\newcommand{\cQ}{\mathcal{Q}}    % sets 5
\newcommand{\cR}{\mathcal{R}}    % sets 6

\newcommand{\cL}{\mathcal{L}}    % Lagrangian density
\newcommand{\cE}{\mathcal{E}}    % Energy density (e.g in electromagnetism)
\newcommand{\cY}{\mathcal{Y}}    % Y

% Quaternions
\newcommand{\Qt}{\mathbb{H}}	% Hamilton's quaternions ('\H' APPARENTLY defined elsewhere by LaTeX}
\newcommand{\qu}{\mathbf{1}}     % 1
\newcommand{\qi}{\mathbf{i}}     % i
\newcommand{\qj}{\mathbf{j}}     % j
\newcommand{\qk}{\mathbf{k}}     % k

% Fraktur (Gothic) font (e.g for algebras)
\newcommand{\frk}[1]{\mathfrak{#1}}  

% To show argument of the exponential function vertically, i.e., as a superscript 
\newcommand{\vexp}[1]{\,e^{#1}}

% To type an angle as a number of degrees like 45^\circ
\newcommand{\degree}[1]{{#1}^\circ}

% To create not-bold vectors with a hat or check accent 
\newcommand{\hatv}[1]{\hat{#1}}
\newcommand{\chkv}[1]{\check{#1}}

% To create boldface vectors with a hat or check accent 
\newcommand{\hatvb}[1]{\vb{\hat{#1}}}
\newcommand{\chkvb}[1]{\vb{\check{#1}}}

% To create boldface greek letters (e.g. for denoting vectors) 
\newcommand{\bmath}[1]{\bm{#1}}  				% SAME AS \bm{#1} - NOT WORTH USING 
\newcommand{\chkbm}[1]{\boldmath{\check{#1}}}	% bold-check
\newcommand{\hatbm}[1]{\boldmath{\hat{#1}}}		% bold-hat

% To create <x|, |x> and <x|y> with unit vectors inside
\newcommand{\ubra}[1]{\bra*{\vu{#1}}}
\newcommand{\uket}[1]{\ket*{\vu{#1}}}
\newcommand{\uip}[2]{\ip*{\vu{#1}}{\vu{#2}}}

% To put accents below letters. 
\newcommand{\ut}[1]{\underaccent{\tilde}{#1}}
\newcommand{\uh}[1]{\underaccent{\hat}{#1}}
\newcommand{\form}[1]{\uh{#1}}

% To create italic, bold, bolditalic text
\newcommand{\tit}[1]{\textit{#1}}
\newcommand{\tbf}[1]{\textbf{#1}}
\newcommand{\tbi}[1]{\textit{\textbf{#1}}}

% Latin Modern sans serif |OR| Helvetica (SELECT)
\newcommand{\textlmss}{\fontfamily{lmss}\selectfont}
\newcommand{\texthv}{\fontfamily{phv}\selectfont}

% Latin Modern sans serif |OR| Helvetica (USE, within OR outside MATH !)
\newcommand{\tlmss}[1]{\text{\textlmss{#1}}}
\newcommand{\thv}[1]{\text{\texthv{#1}}}

% To use \tlmss{T} symbol to denote transpose 
\newcommand{\transp}[1]{{#1}^{\tlmss{T}}}

% To use \dagger symbol to denote operator Adjoint
\newcommand{\Adj}[1]{{#1}^\dagger}

% To denote the Hermitian conjugate with a '+' superscript
\newcommand{\Hconj}[1]{{#1}^{+}}

% To use \tlmss{Ker}, \tlmss{Coker} and \tlmss{Img} to denote Kernel, Co-Kernel & Image 
\newcommand{\Ker}{\tlmss{Ker}\,}
\newcommand{\Coker}{\tlmss{Coker}\,}
\newcommand{\Img}{\tlmss{Im}\,}

% To use \tlmss{Alt} and \tlmss{alt} to denote alternation 
\newcommand{\Alt}{\tlmss{Alt}\,}
\newcommand{\alt}{\tlmss{alt}\,}

% To use \tlmss{Ann} to denote annulets 
\newcommand{\Ann}{\tlmss{Ann}\,}

% Misc abbreviations
\newcommand{\ora}[1]{\overrightarrow{#1}}

\DeclareRobustCommand{\rchi}{{\mathpalette\irchi\relax}}
\newcommand{\irchi}[2]{\raisebox{\depth}{$#1\chi$}} % inner command, used by \rchi

% See https://tex.stackexchange.com/questions/36524/how-to-put-a-framed-box-around-text-math-environment/36528
\makeatletter
\newcommand{\mybox}{%
    \collectbox{%
        \setlength{\fboxsep}{1pt}%
        \fbox{\BOXCONTENT}%
    }%
}
\makeatother

\author{Robert M. Wald}
\title{Advanced Classical Electromagnetism\\
\small{Princeton University Press, 2022}}
\maketitle

\frontmatter%%%%%%%%%%%%%%%%%%%%%%%%%%%%%%%%%%%%%%%%%%%%%%%%%%%%%%

%\include{dedic}

%\chapter*{Plan}
\label{plan} 

In this book I am keeping notes about the theory of classical electromagnetism, 
as exposed in various books. In particular, I intend to cover the following materials:

\begin{itemize}

\item B. Felsager -- Geometry Particles and Fields
\begin{enumerate}
\setcounter{enumi}{0}
\item Electromagnetism (1.1 to 1.4)
\end{enumerate}

\item C. Cattaneo -- Teoria Einsteniana della Gravitazione
\begin{enumerate}
\setcounter{enumi}{0}
\item Elementi di Algebra e Analisi Lineare
\end{enumerate}

\item D.J. Griffiths -- Introduction to Electrodynamics
\begin{enumerate}
\setcounter{enumi}{0}
\item Vector Analysis
\item Electrostatics
\item Potentials
\item Electric Fields in Matter
\item Magnetostatics
\item Magnetic Fields in Matter
\item Electrodynamics
\item Conservation Laws
\item Electromagnetic Waves
\item Radiation
\item Electrodynamics and Relativity
\item Potentials and Fields
\item Helmoltz Theorem
\end{enumerate}

\item J.D. Jackson -- Classical Electrodynamics, 2nd Edition
\begin{enumerate}
\setcounter{enumi}{0}
\item Introduction to Electrostatics
\item Boundary Value Problems in Electrostatics - I
\item Boundary Value Problems in Electrostatics - II
\item Multipoles, Electrostatics of Macroscopic Media, Dielectrics
%\item Magnetostatics
%\item Time Varying Fields, Maxwell Equations, Conservation Laws
%\item Plane Electromagnetic Waves and Wave Propagation
%\item Wave Guides and Resonant Cavities
%\item Simple Radiating Systems, Scattering and Diffraction
%\item Magnetohydrodynamics and Plasma Physics
\end{enumerate}

\item J.D. Jackson -- Classical Electrodynamics, 3rd Edition
\begin{enumerate}
\setcounter{enumi}{4}
\item Magnetostatics, Faraday's Law, Quasi-Static Fields
\item Maxwell Equations, Macroscopic Electromagnetism, Conservation Laws
\item Plane Electromagnetic Waves and Wave Propagation
\item Wave Guides, Resonant Cavities and Optical Fibers
\item Radiating Systems, Multipole Fields and Radiation
\item Scattering and Diffraction
\item Special Theory of Relativity
\item Dynamics of Relativistic Particles and Electromagnetic Fields
\end{enumerate}

\item B. Felsager -- Geometry Particles and Fields
% Contacts with quantum theory of particles dynamics in EM fields
\begin{enumerate}
\setcounter{enumi}{1}
\item Interaction of Fields and Particles
\end{enumerate}

\item J. Franklin -- Advanced Mechanics and General Relativity
\begin{enumerate}
\setcounter{enumi}{1}
\item Relativistic Mechanics
\item Tensors
\item Curved Space
\item Scalar Field Theory
\item Tensor Field Theory (6.1 to 6.5)
\end{enumerate}

\item J.D. Jackson -- Classical Electrodynamics, 3rd Edition
\begin{enumerate}
\setcounter{enumi}{12}
\item Collisions, Energy Loss and Scattering of Charged Particles, Cherenkov and Transition Radiation
\item Radiation by Moving Charges
\item Bremsstrahlung, Method of Virtual Quanta, Radiative Beta Processes
\item Radiation Damping, Classical Models of Charged Particles
\end{enumerate}

\item B. Felsager -- Geometry Particles and Fields
% Contacts with quantum theory of fields dynamics + differential geometry math
\begin{enumerate}
\setcounter{enumi}{2}
\item Dynamics of Classical Fields
\end{enumerate}

\begin{enumerate}
\setcounter{enumi}{5}
\item Differentiable Manifolds, Tensor analysis
\item Differential Forms, Exterior Calculus
\item Integral Calculus on Manifolds
\end{enumerate}

\item C.W. Misner, K.S. Thorne, J.A. Wheeler -- Gravitation
\begin{enumerate}
\setcounter{enumi}{1}
\item Foundations of Special Relativity
\item The Electromagnetic Field
\item Electromagnetism and Differential Forms
\end{enumerate}

\item L.D. Landau, E.M. Lifshitz -- Teoria dei Campi
\begin{enumerate}
\setcounter{enumi}{0}
\item Principio di Relatività
\item Meccanica Relativistica
\item Carica in un Campo Elettromagnetico
\item Equazioni del Campo Elettromagnetico
\item Campo Elettromagnetico Costante
\item Onde Elettromagnetiche
\item Propagazione della Luce
\item Campo di Cariche in Moto
\item Radiazione Elettromagnetica
\end{enumerate}

\item L.D. Landau, E.M. Lifshitz -- Elettrodinamica dei Mezzi Continui
\begin{enumerate}
\setcounter{enumi}{0}
\item Elettrostatica dei Conduttori
\item Elettrostatica nei Dielettrici
\item Corrente Continua
\item Campo Magnetico Costante
\item Ferromgnetismo e Antiferromagnetismo
\item Superconduttività
\item Campo Magnetico Quasi Stazionario
\item Idrodinamica Magnetica
\item Equazioni delle Onde Elettromagnetiche
\item Propagazione delle Onde Elettromagnetiche
\item Onde Elettromagnetiche in Mezzi Anisotropi
\item Dispersione Spaziale
\item Ottica non Lineare
\item Passaggio delle Particelle Veloci attraverso la Materia
\item Diffusione delle Onde Elettromagnetiche
\item Diffrazione dei Raggi X nei Cristalli
\end{enumerate}

\end{itemize}
	
\setcounter{tocdepth}{1}	% Must appear BEFORE \tableofcontents!
\tableofcontents
%\addappheadtotoc

\mainmatter%%%%%%%%%%%%%%%%%%%%%%%%%%%%%%%%%%%%%%%%%%%%%%%%%%%%%%%
%\setcounter{chapter}{-1}	% To start with Chapter 0 !!  
%\input{../FANCYBOX}		% Example of boxed/framed parts of text with nice borders

\begin{flushright}
\tit{Quassù tutto vi appare regolato\\
dal sorgere e calare di una stella;\\
Si scambiano il pensiero con dei suoni,\\
movimenti del viso e delle mani;\\
Cercano l'armonia, ma spesso in guerra;\\
Eppur vorrei restarci sulla Terra!} 
\end{flushright} 

\counterwithin{equation}{chapter}	% Equation numbering will be N.1 N.2   N.3   ... 

\section*{Preface}

This book arose from my teaching the first quarter of the standard graduate course in electromagnetism at the University of Chicago in the winter of 2018. It had been decades since I had previously taught this course, so I approached it with fresh eyes, and it was natural for me to try to rethink how the subject of electromagnetism should be presented at the graduate level. When I did so, it became clear to me that the usual quasi-historical way of presenting the subject promotes some very unhealthy ways of thinking about electromagnetism. Therefore, to avoid starting off on the wrong foot, I decided to spend the first few lectures of the course describing what I now refer to in chapter 1 of this book as \quotes{myths} concerning electromagnetism. I found that by starting out in this way, it became much easier to straightforwardly present the subject in a clear and concise manner, without having to make shifts in perspective as the subject is developed. I taught the course again in the following 3 years and provided lecture notes to the class. These lecture notes have now evolved into this book.

The first chapter of this book is thus a quite unconventional introduction to electromagnetism. Instead of beginning with the force between charged particles, discussing how this gives rise to a \quotes{field} concept, and so forth, my aim in chapter 1 is to explain to students how they should think about electromagnetism from a modern and mathematically precise perspective. The major points made in this chapter are that (i) the potentials, not the field strenghts, are the fundamental dynamical variables in electromagnetism; (ii) the energy and momentum properties of the electromagnetic field are an essential part of the formulation of the theory and cannot properly be derived by \quotes{work done} arguments; (iii) electromagnetic fields should not be thought of as being \tit{produced} by charges; and (iv) at a fundamental level, the charged matter in classical electrodynamics must be viewed as continuosly distributed rather than consisting of point charges. Many of these points cannot be fully elucidated until the later chapters in the book---particularly chapters 9 and 10---but my intent is to lay out these ideas in a sufficiently clear and explicit way in chapter 1 that I can take these perspectives unapologetically in the remainder of the book.

The topics treated in chapters 2-7 are ones that normally would be covered in any graduate course in electromagnetism. Electrostatics is treated in chapter 2, but starting with Poisson's equation, not Coulomb's law. Dielectric materials in electrostatics are treated in chapter 3, with considerable care given to how the macroscopic averaging is done and to the treatment of energy. Magnetostatics is treated in chapter 4, with a full discussion of the sign difference between magnetostatics and electrostatics in the field interaction energy of a dipole in an external field--and how this relates to the change in the rest mass of a magnet when it is quasy-statically moved in an external magnetic field. Electrodynamics and radiation are discussed in depth in chapter 5. In addition to topics normally found in electromagnetism texts, I derive the initial value formulation for Maxwell's equations in that chapter. Electrodynamics in media is treated in chapter 6, including a discussion of magnetohydrodynamics. The geometric optics approximation to wave dynamics is presented in the first section of chapter 7, followed by a discussion of interference and coherence and an analysis of two problems in diffraction: scattering by a dielectric ball and the propagation of radiation through an aperture.

Special relativity is discussed in chapter 8. Special relativity underlies the fornulation of electromagnetic theory, so it really should be presented at the outset of a book on electromagnetism, rather than be relegated to a chapter near the end of the book. However, special relativity remains such an unfamiliar topic for most students that it is not feasible to do this. Many treatments of special relativity focus on the rules for applying Lorentz transformations to quantities, without providing much insight into the underlying geometrical content of the theory. In contrast, it would be natural for a general relativist like me to introduce more mathematical abstraction and geometrical machinery than would be strictly needed to provide a clear description of special relativity. I have put considerable care into writing section 8.1 in such a way that it introduces special relativity in a conceptually clear way without introducing more abstraction than I believe to be essential. This section can be read independently of the rest of the book, and I hope it will provide a useful introduction to special relativity on its own. The formulation of electromagnetism in the framework of special relativity is then given, followed by a discussion of charged particle motion and the radiation from a point charge in arbitrary motion.

Finally, the notion of a point charge is discussed in depth in chapter 10. It is shown that a mathematically well-defined limit of a charged body as it shrinks down to  zero size can be taken, provided that one also takes the charge and mass of the body to scale to zero proportionally to its size. Lorentz force motion is obtained in this limit. Self-force corrections can then be computed perturbatively in a mathematically rigorous manner. The issue of how to self-consistently describe the motion of a charged body taking the self-force corrections into account---without introducing spurious \quotes{runaway} solutions---is addressed in the final section of this chapter.

Throughout this book, I have attempted to formulate all key conceptual ideas and results in the theory of electromagnetism in a clear and concise manner. However, I have not attempted to present everything in this book with a high level of mathematical precision. Although I have made an effort to avoid getting sidetracked with unnecessary mathematical detail, I have not knowingly oversimplified any statements in the book and have tried to be careful to insert appropriate caveats when formulas or other results hold only under restricted conditions. In several instances in the early chapters, I have added boxed \quotes{side comments} to explain some mathematical points that may be of potential interest and relevance to the reader but are not strictly needed for the discussion.

An extensive collection of problems is provided for chapters 2-8. One purpose of these problems is the usual one of providing students with an opportunity to test their understanding of the basic concepts introduced in the chapter. However, there is an additional important purpose for some of the problems: to present topics that are not essential to the development of the core ideas of the book but are, nevertheless, of considerable interest and importance. Some examples of such topics treated in the problems are hidden momentum, the Hall effect, Gaussian beams, Thomson scattering, optical fibers, Stokes parameters, and Cherenkov radiation. I have written these problems in such a way that the key concepts are explained---and the key results are given---in the statement of the problem. Thus, a reader may find these problems to be a useful introduction to the topics.

The main audience that I have in mind for this book are graduate students in theoretical physics, although I hope that graduate students in experimental physics, undergraduates, and others will also find the book to be of interest. This book is written under the assumption that readers have had an introductory course in electromagnetism and thereby already have some intuition about electric and magnetic fields. I also assume the readers have a solid knowledge of vector calculus, but I do not assume much mathematical background beyond this.

I use SI units throughout the book. Unfortunately, SI units have the highly unpleasant feature of introducing two constants, $\epsilon_0$ and $\mu_0$, that satisfy the relation $\epsilon_0 \mu_0 c^2 = 1$, where $c$ is the speed of light. There are good historical reasons that this is the case. It is natural to assign an electric permittivity $\epsilon$ and a magnetic permeability $\mu$ to many materials, and it therefore was natural to assign corresponding values, $\epsilon_0$ and $\mu_0$, to the vacuum. It was then a truly great achievement of Maxwell to recognize that his equations implied that disturbances of the electric and magnetic fields in vacuum propagate with speed $c = 1/\sqrt{\epsilon_0 \mu_0}$ and that these disturbances could be identified with light. However, this relation between $\epsilon_0$, $\mu_0$ and $c$ means that there is a redundancy in these constants. Consequently, the appearance of formulas in SI units can be changed in nontrivial-looking ways by using this redundancy. For example, in SI units, one of the Maxwell equations is usually written as $\div \vb{E} = \rho/\epsilon_{0}$. However, this equation could equally well be written as $\div \vb{E} = \mu_0 c^2 \rho$. The latter form may seem rather jarring, as it seems to suggest that the magnetic permeability of the vacuum and the speed of light enter a basic equation of electrostatics. In any case, one must take a choice of which of these constants to use in any formula. The usual convention is to use $\epsilon_0$ in the above Maxwell equation and use $\mu_0$ in the Maxwell equation involving the current density $\vb{J}$. However, this convention cannot be maintained when one writes Maxwell's equations in special relativistically covariant form, since the 4-current $J^\mu$ enters these equations, and it makes no sense to use different conventions for different components of this 4-vector. Indeed, from chapter 8 on, I dispense entirely with $\epsilon_0$, $\mu_0$ and $c$ in all formulas. To avoid the unpleasantness associated with this redundancy of      
$\epsilon_0$, $\mu_0$ and $c$, I used Gaussian units in the original versions of my lecture notes. However, although Gaussian units were in quite prevalent use decades ago, SI units are used nearly universally now. Thus, the unpleasantness of SI units is outweighed by the unfamiliarity of students with Gaussian units---as well as the possibility that someone using my book may be led to purchase the wrong size of electromagnetic equipment if the formulas were written in Gaussian units. So, I have chosen to use SI units. 

Ordinary vectors in 3-dimensional space will be denoted in boldface (e.g., the electric field will be denoted as $\vb{E}$, as in the previous paragraph). Cartesian components of vectors will be denoted with Latin subscripts and without boldface symbols (e.g., $E_i$, with $i=1,2,3,$) denotes the components of $\vb{E}$ in a Cartesian basis). Beginning in chapter 8, I introduce the notion of spacetime vectors. For the reason explained in section 8.1, it then will be essential to explicitly introduce the notion of dual vectors and to distinguish clearly between vectors and dual vectors in our notation, wherein spacetime vectors are denoted with Greek superscripts (e.g., $W^\mu$) and spacetime dual vectors are denoted with Greek subscripts (e.g., $U_\mu$). Some additional special relativistic notational conventions are stated at the end of section 8.1.

I am greatly indebted to numerous colleagues for reading parts (and, in some cases, all) of the manuscript and providing me with valuable feedback. These include Sam Gralla, Abe Harte, Jim Isenberg, Istvan Racz, and Gautam Satishchandran, as well as numerous students who took my course. Among the latter, Tixuan Tan deserves special thanks for reading the manuscript with great care and asking many penetrating questions about the exposition.    % Preface  

\setcounter{chapter}{0}
\renewcommand{\thechapter}{1}
\chapter{Introduction: Electromagnetic Theory without Myths}\label{ch:1}
\setcounter{equation}{0}	        % To start with Equation 1
\counterwithin{equation}{chapter}	% Equation numbering will be 1.1 1.2 1.3   ... 

\section{The Fundamental Electromagnetic Variables\\ Are the Potentials, Not the Field Strengths}\label{sec:1.1}

\section{Electromagnetic Energy, Momentum, and Stress\\ Are an Integral Part of the Theory}\label{sec:1.2}

\section{Electromagnetic Fields Should Not Be Viewed\\ as Being Produced by Charged Matter}\label{sec:1.3}

\section{At a Fundamental Level, Classical Charged\\ Matter Must Be Viewed as Continuous\\ Rather Than Point-Like}\label{sec:1.4}
   % Introduction: Electromagnetic Theory without Myths  
\setcounter{chapter}{0}
\renewcommand{\thechapter}{2}
\chapter{Electrostatics}\label{ch:2}
\setcounter{equation}{0}	        % To start with Equation 1
\counterwithin{equation}{chapter}	% Equation numbering will be 2.1 2.2 2.3   ... 

\section{Uniqueness of Solutions in Electrostatics}

\section{Point Charges and Green's Functions}

\section{Interaction Energy and Force}

\section{Multipole Expansion of the Green's Function}

\section{Conducting Cavities; Dirichlet and Neumann Green's Functions} 

%===================================================================================

\section*{Problems}


   % Electrostatics
\setcounter{chapter}{0}
\renewcommand{\thechapter}{3}
\chapter{Dielectrics}\label{ch:3}
\setcounter{equation}{0}	        % To start with Equation 1
\counterwithin{equation}{chapter}	% Equation numbering will be 2.1 2.2 2.3   ... 

\section{Macroscopic Description of Dielectrics}

\section{Force and Interaction Energy}

%===================================================================================

\section*{Problems}


   % Dielectrics
\setcounter{chapter}{0}
\renewcommand{\thechapter}{4}
\chapter{Magnetostatics}\label{ch:4}
\setcounter{equation}{0}	        % To start with Equation 1
\counterwithin{equation}{chapter}	% Equation numbering will be 2.1 2.2 2.3   ... 

\section{The Equations of Magnetostatics}

\section{Multipole Expansion}

\section{Interaction Energy and Force}

\section{Magnetic Materials}

%===================================================================================

\section*{Problems}


   % Magnetostatics
\setcounter{chapter}{0}
\renewcommand{\thechapter}{5}
\chapter{Electrodynamics}\label{ch:5}
\setcounter{equation}{0}	        % To start with Equation 1
\counterwithin{equation}{chapter}	% Equation numbering will be 2.1 2.2 2.3   ... 

In this chapter, we consider the general case where $\phi$, $\vb{A}$ and $\rho$, $\vb{J}$ are time dependent. The equations of electrodynamics are discussed in section \ref{sec:5.1}, and the Lorenz gauge is introduced. We solve for the retarded Green's function in section \ref{sec:5.2}, from which the solution to Maxwell's equations for general $\rho$, $\vb{J}$ with no incoming radiation can be obtained. Multipole expansions of the electromagnetic field of the retarded solution are obtained in section \ref{sec:5.3}. The retarded Green's function is then used in section \ref{sec:5.4} to obtain the solution to Maxwell's equations with prescribed values of the electromagnetic field at an initial time. We discuss plane wave solutions in section \ref{sec:5.5}. The chapter concludes with a discussion of electrodynamics in conducting cavities and waveguides in section \ref{sec:5.6}.

\section{The Equations of Electrodynamics}\label{sec:5.1}
The equations of electrodynamics have been presented in chapter 1. As discussed in section \ref{sec:1.1}, the fundamental dynamical variables are the potentials $\phi$ and $\vb{A}$, which are considered to be equivalent if and only if they differ by a gauge transformation:
\begin{equation}\label{eq:5.1}
\phi \rightarrow \phi' = \phi - \dv{\chi}{t}, \quad \quad \vb{A} \rightarrow \vb{A'} = \vb{A} + \grad \chi\:.
\end{equation}

% BOX 1 **START** -- << Maxwell equations in Gaussian units >>

\parindent=0pt  % Set the paragraph indentation to 0 (normal = 10pt) before the box 
\parbox{\textwidth}{\begin{mdframed}[style=MyFrame] %Added "\parbox{\textwidth}{"
%\lipsum[1]
\subsubsection*{BOX 5.1 -- Gauge invariance in Gaussian units}\label{box:8.1}
\begin{equation*}
\phi \rightarrow \phi' = \phi - \frac{1}{c}\dv{\chi}{t}, \quad \quad 
\vb{A} \rightarrow \vb{A'} = \vb{A} + \grad \chi\;.\quad \quad\quad \quad [G5.1]
\end{equation*}
\end{mdframed}} %Added a closing "}" here
\parindent=10pt % Set the paragraph indentation to normal (10pt) 
% BOX 1 ** END ** -- << Maxwell equations in Gaussian units >>


The equations of electrodynamics (i.e. Maxwell's equations) are
\begin{align}
\vb{E} &= - \grad \phi - \pdv{\vb{A}}{t}						\,,\label{eq:5.2}\\
\vb{B} &= \curl{\vb{A}}                  						\,,\label{eq:5.3}\\
\div{\vb{E}} &= \frac{\rho}{\epsilon_0} 						\,,\label{eq:5.4}\\
\curl{\vb{B}} &= \frac{1}{c^2} \pdv{\vb{E}}{t} + \mu_0 \vb{J}	\,.\label{eq:5.5}
\end{align}
The first two of these equations define $\vb{E}$ and $\vb{B}$ in terms of $\phi$ and $\vb{A}$. They imply that 
\begin{align}
\div{\vb{B}} = 0                                                \,,\label{eq:5.6}\\
\curl{\vb{E}} + \pdv{\vb{B}}{t}= 0                              \,,\label{eq:5.7}
\end{align}

% BOX 2 **START** -- << Maxwell equations in Gaussian units >>

\parindent=0pt  % Set the paragraph indentation to 0 (normal = 10pt) before the box 
\parbox{\textwidth}{\begin{mdframed}[style=MyFrame] %Added "\parbox{\textwidth}{"
%\lipsum[1]
\subsubsection*{BOX 5.2 -- Maxwell equations in Gaussian units}\label{box:8.2}
\begin{align*}
\vb{E} &= - \grad \phi - \frac{1}{c}\pdv{\vb{A}}{t}\,,\quad\quad\quad &[G5.2]\\
\vb{B} &= \curl{\vb{A}}\,,&[G5.3]\\
\div{\vb{E}} &= 4 \pi \rho\,,&[G5.4]\\
\curl{\vb{B}} &= \frac{1}{c} \pdv{\vb{E}}{t} + \frac{4\pi}{c} \vb{J}\,,&[G5.5]\\
\div{\vb{B}} &= 0\,,&[G5.6]\\
\curl{\vb{E}} + \frac{1}{c} \pdv{\vb{B}}{t }&= 0\,.&[G5.7]\\
\end{align*}
\end{mdframed}} %Added a closing "}" here
\parindent=10pt % Set the paragraph indentation to normal (10pt) 
% BOX 2 ** END ** -- << Maxwell equations in Gaussian units >>

Equations (\ref{eq:5.6}) and (\ref{eq:5.7}) are equivalent to eqs. (\ref{eq:5.2}) and (\ref{eq:5.3}) in a topologically trivial region\footnote{More precisely, by a \quotes{topologically trivial region} in the present context, I mean a region in which any closed 2-dimensional surface $S$ is the boundary od a 3-dimensional (compact) volume, and every closed loop $\mathscr{C}$ is the boundary of a 2-dimensional (compact) surface. The necessary and sufficient condition for the existence of a vector potential $\vb{A}$ such that $\vb{B} = \curl{\vb{A}}$ is that $\int_S \vb{B} \cdot \vu{n} = 0$ for any closed 2-dimensional surface $S$. The necessary and sufficient condition for the existence of a scalar potential $\phi$ satisfying eq. (\ref{eq:5.2}) is $\int_{\mathscr{C}} [\vb{E} + \pdv*{\vb{A}}{t}] \cdot \dd{\vb{l} = 0}$ for any closed loop $\mathscr{C}$. If eqs. (\ref{eq:5.6}) and (\ref{eq:5.7}) hold in a topologically trivial region, then the necessary and sufficient condition for the existence of $\vb{A}$ will automatically hold by Gauss's theorem, and the necessary and sufficient condition for the existence of $\phi$ will automatically hold by Stokes's theorem.} 
(i.e., if eqs. (\ref{eq:5.6}) and (\ref{eq:5.7}) hold in a topologically trivial region, they imply the existence of potentials $\phi$ and $\vb{A}$ in that region and satisfying eqs. (\ref{eq:5.2}) and (\ref{eq:5.3})). 
Taking the time derivative of eq. (\ref{eq:5.4}) and adding it to $c^2$ times the divergence of eq. (\ref{eq:5.4})---using the fact that $c^2 = \flatfrac{1}{\epsilon_0 \mu_0}$---we obtain the charge-current conservation law:
\begin{equation}\label{eq:5.8}
\pdv{\rho}{t} + \div{\vb{J}} = 0\,.
\end{equation}
It is worth noting that in the source-free case (i.e., when $\rho = \vb{J} = 0$), except for one sign difference and factors of $c$, eqs. (\ref{eq:5.6}) and (\ref{eq:5.7}) take the same form as eqs. (\ref{eq:5.4}) and (\ref{eq:5.5}) with $\vb{E}$ and $\vb{B}$ interchanged. It follows immediately that in any source-free, topologically trivial region, maxwell's equations are invariant under a \tit{duality transformation}:
\begin{equation}\label{eq:5.9}
\vb{E} \rightarrow c \vb{B}\,, \quad\quad c \vb{B} \rightarrow \vb{E}\,,
\end{equation}
that is, if $\vb{E}, \vb{B}$ solve eqs. (\ref{eq:5.4})-(\ref{eq:5.7}) with  $\rho = 0$, $\vb{J} = 0$, then so do 
$\vb{E'}, \vb{B'}$ with  $\vb{E'},= c \vb{B}$, $\vb{B'} = - \flatfrac{\vb{E}}{c}$. More generally, for any real number $\alpha$, the source-free Maxwell equations (\ref{eq:5.4})-(\ref{eq:5.7}) are invariant under the \tit{duality rotation}:
\begin{equation}\label{eq:5.10}
\vb{E} \rightarrow \cos\alpha \vb{E} + \sin\alpha (c \vb{B})\,, \quad\quad \vb{B} \rightarrow \cos\alpha \vb{B} - \sin\alpha ( \vb{E}/c)\,,
\end{equation}
As already discussed in section (\ref{sec:1.2}), the energy density, momentum density, and stress tensor of the electromagnetic field are, respectively,
\begin{equation}\label{eq:5.11}
\mathscr{E} = \frac{1}{2} \left(\epsilon_0 \abs{\vb{E}}^2 + \frac{1}{\mu_0} \abs{\vb{B}}^2 \right)\,,  
\end{equation}
\begin{equation}\label{eq:5.12}
\pmb{\mathscr{P}} = \epsilon_0 \vb{E} \cross \vb{B}\,,
\end{equation}
\begin{equation}\label{eq:5.13}
\Theta_{ij} = \epsilon_{0} E_i E_j + \frac{1}{\mu_0} B_i B_j  -\frac{1}{2} \delta_{ij} \left(\epsilon_0 \abs{\vb{E}}^2 + \frac{1}{\mu_0} \abs{\vb{B}}^2 \right)   \,.
\end{equation}

The total energy and momentum of the electromagnetic field are obtained by integrating eq. (\ref{eq:5.11}) and q. (\ref{eq:5.12}) over all space.

In special relativity, there is no distinction between a \quotes{flow of mass} (i.e., momentum) and a \quotes{flow of energy}---apart from a factor of $c^2$ needed to give these quantities conventional units. Thus, $c^2$ times the momentum density of the electromagnetic field gives the electromagnetic energy flux,  
\begin{equation}\label{eq:5.14}
\pmb{\mathscr{S}} = c^2 \pmb{\mathscr{P}} = c^2 \epsilon_0 \vb{E} \cross \vb{B} = \frac{1}{\mu_0} \vb{E} \cross \vb{B}\,,
\end{equation}
where, again, we use the fact that $c^2 = 1/(\epsilon_0 \mu_0)$. $\pmb{\mathscr{S}}$ is known as the \tit{Poynting vector}. For a surface $S$, the flux $\mathscr{F}$ of electromagnetic energy through $S$ (i.e., the electromagnetic energy that flows through $S$ per unit area per unit time) is given by 
\begin{equation}\label{eq:5.15}
\mathscr{F} =\pmb{\mathscr{S}} \cdot \vu{n}\,,
\end{equation}

where $\vu{n}$ is the unit normal to $S$.  The flow of electromagnetic energy out of a volume $\mathscr{V}$ bounded by surface $S$ is thus given by
\begin{equation}\label{eq:5.16}
\int_{S} \mathscr{F} \dd{S} = \int_{S} \pmb{\mathscr{S}} \cdot \vu{n}  \dd{S} = \int_{\mathscr{V}} \div{\pmb{\mathscr{S}}} \dd{V} \,.
\end{equation}

If electromagnetic energy were conserved by itself, we would have 
$\dv*{}{t} \int_{\mathscr{V}} \mathscr{E} \dd{V} = \int_{\mathscr{V}} \pdv*{\mathscr{E}}{t} \dd{V} = - \int_{\mathscr{V}} \div{\pmb{\mathscr{S}}} \dd{V}$, and therefore (since $\mathscr{V}$ is arbitrary), $\pdv*{\mathscr{E}}{t} = - \div{\pmb{\mathscr{S}}}$. However, a direct computation yields
\begin{equation}\label{eq:5.17}
\begin{aligned}
\pdv{\mathscr{E}}{t} &= 
\epsilon_0 \vb{E} \cdot \pdv{\vb{E}}{t} + \frac{1}{\mu_0} \vb{B} \cdot \pdv{\vb{B}}{t} + \frac{1}{\mu_0} \div{(\vb{E} \cross \vb{B})} \\
&= \epsilon_0 \vb{E} \cdot  ( c^2 \curl{\vb{B}} - c^2 \mu_0 \vb{J} ) + \frac{1}{\mu_0} \vb{B}  \cdot (- \curl{\vb{E}}) + \frac{1}{\mu_0} \div{(\vb{E} \cross \vb{B})} \\
&= - \vb{E} \cdot \vb{J}  + \frac{1}{\mu_0} [\vb{E} \cdot (\curl{\vb{B}})  - \vb{B} \cdot (\curl{\vb{E}}) + \div{(\vb{E} \cross \vb{B})}]    \\
&= - \vb{E} \cdot \vb{J}\,,
\end{aligned}
\end{equation}
where eqs. (\ref{eq:5.6}) and (\ref{eq:5.7}) were used in the second line, the relation $c^2 \epsilon_0 \mu_0 = 1$ was used in the third line, and the identity 
\begin{equation}\label{eq:5.18}
\begin{aligned}
\div{(\vb{E} \cross \vb{B})} &=  \sum_{i,j,k} \epsilon_{ijk}\partial_i (E_j B_k) = 
\sum_{i,j,k} \epsilon_{ijk} \left[(\partial_i E_j) B_k + E_j (\partial_i B_k)) \right]\\ 
                           &= \vb{B} \cdot (\curl{\vb{E}}) - \vb{E} \cdot (\curl{\vb{B}}) 
\end{aligned}
\end{equation}
was used to get the last line. This means that if total energy (i.e., the energy of the electromagnetic field plus the energy of matter) is to be conserved, then the electromagnetic field must be adding energy density to the matter at the rate
\begin{equation}\label{eq:5.19}
\pdv{\mathscr{E_{matter}}}{t} =  \vb{J} \cdot \vb{E}\,.
\end{equation}

Thus $\vb{J} \cdot \vb{E}$ \tit{is the rate at which energy per unit volume is transferred from the electromagnetic field to matter.} 

In a similar manner and by a similar calculation, the failure of momentum conservation to hold for the electromagnetic field momentum alone is given by
\begin{equation}\label{eq:5.20}
\pdv{\mathscr{P}_i}{t} - \sum_{j=1}^3 \partial_j \Theta_{ij} = - \left[ \rho E_i + (\vb{J} \cross \vb{B})_i   \right]\;.
\end{equation}

If total momentum is to be conserved, then the rate of change of the momentum density of matter must be given by minus the right side of eq. (\ref{eq:5.20}). In other words, the electromagnetic field must exert a force per unit volume, $\vb{f}$, on matter given by 
\begin{equation}\label{eq:5.21}
\vb{f} = \rho \vb{E} + \vb{J} \cross \vb{B}\;.
\end{equation}
which is referred to as the \tit{Lorentz force}.

It should be emphasized that the entire content of electromagnetic theory is expressed by Maxwell's equations (\ref{eq:5.2})--(\ref{eq:5.5}); equations (\ref{eq:5.11})--(\ref{eq:5.13}) for energy density, momentum density, and stress of the electromagnetic field; and equations (\ref{eq:5.19}) and (\ref{eq:5.21}), which express that energy and momentum are conserved for the total system composed of electromagnetic field and matter.

We now substitute eqs.  (\ref{eq:5.2}) and (\ref{eq:5.3}) into equations  (\ref{eq:5.4}) and (\ref{eq:5.5}) to write Maxwell's equation purely in terms of $\phi$ and $\vb{A}$:
\begin{equation}\label{eq:5.22}
- \laplacian \phi - \pdv{}{t} \div{\vb{A}} = \frac{\rho}{\epsilon_0}\,,
\end{equation}
\begin{equation}\label{eq:5.23}
- \laplacian{\vb{A}} - \grad{ (\div{\vb{A}}) }+ \frac{1}{c^2} \pdv{}{t} \grad{\phi} + \frac{1}{c^2}\pdv[2]{\vb{A}}{t} = \mu_0 \vb{J}\,,
\end{equation}
where the identity eq. (\ref{eq:4.16}) was used to get eq. (\ref{eq:5.23}). A tremendous simplification of these equations can be made by transforming the potentials to a new gauge $(\phi', \vb{A}')$ such that
\begin{equation}\label{eq:5.24}
\frac{1}{c^2} \pdv{\phi'}{t} + \div{\vb{A}'} = 0\,.
\end{equation}

This condition is known as the \tit{Lorenz}\footnote{Note that there is no \quotes{t} in Lorenz. The gauge condition eq. (\ref{eq:5.24}) is named after Ludvig Lorenz, a nineteenth century Danish physicist/mathematician, not Hendrik Lorentz, the Dutch physicist after whom the Lorentz transformation is named.} \tit{gauge condition}. By eq. (\ref{eq:5.1}), such a gauge can be chosen if we can find a function $\chi$ that satisfies
\begin{equation}\label{eq:5.25}
\frac{1}{c^2} \pdv{\phi}{t} - \frac{1}{c^2} \pdv[2]{\chi}{t} + \div{\vb{A}} + \laplacian{\chi} = 0\,.
\end{equation}
that is, if we can solve the equation
\begin{equation}\label{eq:5.26}
\Box \chi = -s\,,
\end{equation}
with $s= (1/c^2) \pdv{\phi}{t} + \div{\vb{A}}$, where  
\begin{equation}\label{eq:5.27}
\Box \equiv -\frac{1}{c^2} \pdv[2]{}{t} + \laplacian\;.
\end{equation}
The operator $\Box$ is known as the d'\tit{Alembertian} or \tit{wave operator}, and equation (\ref{eq:5.26}) is known as the 
\tit{wave equation with source} $s$. It is well known that this equation can be solved for any smooth $s$, and indeed, this will follow from results we obtain in section \ref{sec:5.4}. Thus, without loss of generality, we may always use the gauge freedom eq. (\ref{eq:5.1}) to put the potentials $\phi$ and $\vb{A}$ in the Lorenz gauge. It is important to note that the Lorenz gauge is \tit{not} unique. If $\chi$ satisfies 
eq. (\ref{eq:5.25}), then so does $\chi + \psi$, where $\psi$ is any solution to the homogeneous wave equation $\Box \psi = 0$. As we shall see in section (\ref{sec:5.4}), there are many solutions to this equation.

We now transform our potentials  $\phi$ and $\vb{A}$ to the Lorenz gauge potentials satisfying eq. (\ref{eq:5.24}). For notational simplicity, we drop the primes and denote the transformed potentials as  $\phi$ and $\vb{A}$ rather than  $\phi'$ and $\vb{A}'$. 
Maxwell'sequations (\ref{eq:5.22}) and (\ref{eq:5.23}) then become simply
  
\begin{align}
\Box \phi   &= - \frac{\rho}{\epsilon_0}\label{eq:5.28}\,,\\
\Box \vb{A} &= - \mu_0 \vb{J}\,.\label{eq:5.29}
\end{align}

Thus the full content of Maxwell's equations is expressed by the wave equations (\ref{eq:5.28}) and (\ref{eq:5.29}) together with the Lorenz gauge condition 
\begin{equation}\label{eq:5.30}
\frac{1}{c^2} \pdv{\phi}{t} + \div{\vb{A}} = 0\;.
\end{equation}

\section{Retarded Green's Function}\label{sec:5.2}
We see from eqs. (\ref{eq:5.28}) and (\ref{eq:5.29}) that the key to being able to solve Maxwell's equations is to be able to solve the wave equation with source
\begin{equation}\label{eq:5.31}
\Box \psi = -f\;.
\end{equation}

If we know how to obtain solutions $\psi$ to eq. (\ref{eq:5.31}) for a given $f$, then we can immediately solve eqs.  (\ref{eq:5.28}) and (\ref{eq:5.29}). Of course, we must still solve eq. (\ref{eq:5.30}) as well, but we will see that this equation is automatically satisfied for the retarded solution to eqs. (\ref{eq:5.28}) and (\ref{eq:5.29}) for sources with suitable fall-off.

As in electrostatics, we will be able to obtain a solution to eq. (\ref{eq:5.31}) if we can find a Green's function, that is, a solution to
\begin{equation}\label{eq:5.32}
\Box_{(t, \vb{x})} G(t, \vb{x};t', \vb{x}') = -\delta(\vb{x} - \vb{x}')(t - t')\,,
\end{equation}

where the subscript $(t, \vb{x})$ on $\Box$ indicates that the derivatives appearing in the d'Alembertian operator $\Box$ are taken with respect to the unprimed variables, not the primed variables. Note that, in contrast to eq. (2.67), G depends on $t$ and $t'$ as well as $\vb{x}$ and $\vb{x}'$, and the right side of eq. (\ref{eq:5.32}) has a delta function in $t - t'$ as well as in $\vb{x} - \vb{x}'$. As in electrostatics, given $G$, a solution $\psi$  to eq. (\ref{eq:5.31}) can then be obtained via  
\begin{equation}\label{eq:5.33}
\psi(t, \vb{x}) = \bigint G(t, \vb{x};t', \vb{x}') f(t', \vb{x}') \dd[3]x' \dd{t'}\,,
\end{equation}
provided that this integral converges. Note that I have used the expression \quotes{a Green's function} and \quotes{a solution} because neither the Green's function (\ref{eq:5.32}) nor solutions to eq. (\ref{eq:5.31}) are unique. Indeed, we will explicitly encounter the nonuniqueness below when attempting to solve for $G$. However, we will see that there is a unique \quotes{retarded} Green's function, for which $G$ vanishes for $t < t'$. In section \ref{sec:5.4}, we use the retarded Green's function to characterize the general solution to eq. (\ref{eq:5.31}).

We seek to solve eq.  (\ref{eq:5.32}) by using Fourier transforms. For an integrable function $F : \R \rightarrow \R$, we define its Fourier transform $\hat{F}$ by 
\begin{equation}\label{eq:5.34}
\hat{F}(k) = \frac{1}{\sqrt{2\pi}} \int_{- \infty}^{\infty} F(x) e^{-ikx}\dd{x}\,.
\end{equation}

The notion of a Fourier transform can be extended to distributions, in which case the Fourier transform yields a distribution. In particular, the Fourier transform of the delta function $\delta_{x_0} = \delta(x - x_0)$ is well defined by the function
\begin{equation}\label{eq:5.35}
\hat{\delta}_{x_0}(k) = \frac{1}{\sqrt{2\pi}} e^{-ik x_0}\;,
\end{equation}
as would be expected by formally replacing $F(x)$ by $\delta(x - x_0)$ in eq. (\ref{eq:5.34}). 

For a smooth function $F$, with suitably fast fall-off at infinity, it can be shown that the inverse of the Fourier transform is given by
\begin{equation}\label{eq:5.36}
F(x) = \frac{1}{\sqrt{2\pi}} \int_{- \infty}^{\infty} \hat{F}(k) e^{-ikx} e^{+ikx_0}\dd{k}\,.
\end{equation}

that is, one can recover $F$ from its Fourier transform $\hat{F}$ by the same formula as eq. (\ref{eq:5.34}) except for the sign change in the exponential. Equation (\ref{eq:5.36}) also applies to distributions when suitably interpreted. In particular, the delta function is given by 
\begin{equation}\label{eq:5.37}
\delta_{x_0}(x) = \frac{1}{\sqrt{2\pi}} \int_{-\infty}^{\infty} \hat{\delta}_{x_0}(k) e^{+ikx}\dd{k} =  
\frac{1}{2\pi} \int_{-\infty}^{\infty} e^{-ikx_0} e^{+ikx} \dd{k}\,.
\end{equation}

This equation---which appears very commonly in physics texts---may not look very sensible, since the integral on the right side of eq. (\ref{eq:5.37}) clearly does not converge. However, what this equation is really supposed to mean is that for any smooth function $f$ with suitably fall-off at infinity, we have
\begin{equation}\label{eq:5.38}
\delta_{x_0}(f) = \frac{1}{2 \pi} \int_{-\infty}^{\infty} \dd{k} e^{-ikx_0} \int_{-\infty}^{\infty} \dd{x} e^{+ikx} f(x)\,.
\end{equation}
This statement is correct, because 
\begin{equation}\label{eq:5.39}
\frac{1}{2 \pi} \int_{-\infty}^{\infty} \dd{k} e^{-ikx_0} \int_{-\infty}^{\infty} \dd{x} e^{+ikx} f(x) = 
\frac{1}{\sqrt{2\pi}} \int_{-\infty}^{\infty} \dd{k} e^{-ikx_0} \hat{f}(-k) = f(x_0)\,.
\end{equation}

A major reason Fourier transforms are so useful is that differentiation in physical space corresponds to multiplication by $ik$ in Fourier transform space. To see this, let $F$ be a smooth function that falls off sufficiently rapidly at infinity. Then the Fourier transform 
of $\dv*{F}{x}$ is given by
\begin{equation}\label{eq:5.40}
\begin{aligned}
\hat{\dv{F}{x}}(k) &= \frac{1}{\sqrt{2\pi}} \int_{-\infty}^{\infty} \dv{F}{x}(x) e^{-ikx} \dd{x}\\
                   &= \frac{1}{\sqrt{2\pi}} \int_{-\infty}^{\infty} (ik) F(x) e^{-ikx} \dd{x}\\
                   &= ik \hat{F}(k)\,,
\end{aligned}
\end{equation}
where we integrated by parts in the second line, disregarding the boundary term at infinity because of the fall-off of $F$. 
On account of eq. (\ref{eq:5.40}), any partial differential equation with constant coefficients can be converted to an algebraic equation in Fourier transform space.  
 
We now attemp to solve for $G$, eq. (\ref{eq:5.32}), by means of Fourier transforms. To simplify the notation we set $t' = \vb{x}' = 0$, and we also set $c = 1$. (We will restore  $t', \vb{x}', \text{ and } c$ at the end of the calculation.) Define the (4-dimensional) Fourier transform of $G$ by
\begin{equation}\label{eq:5.41}
\hat{G}(\omega, \vb{k}) = \frac{1}{(2 \pi)^2} \int_{-\infty}^{\infty} G(t, \vb{x}) e^{+i \omega t} e^{-i \vb{k}\cdot\vb{x}} \dd{t} \dd[3]x\,.
\end{equation}
Note that, by standard convention, the time Fourier transform is defined\footnote{This is done so that $(\omega/c, \vb{k})$ are the components of a 4-vector in special relativity (see chapter 8).} by integrating with $e^{+i \omega t}$  rather than $e^{-i \omega t}$. Taking the Fourier transform of eq. (\ref{eq:5.32}) with respect to $t$ and $\vb{x}$ and using eq. (\ref{eq:5.35}) (with $x_0 = 0$) and eq. (\ref{eq:5.40}), we obtain
\begin{equation}\label{eq:5.42}
(\omega^2 - k^2) \hat{G}(\omega, \vb{k}) = - {\left( \frac{1}{\sqrt{2 \pi}} \right)}^4 = - \frac{1}{4\pi^2}\,,
\end{equation} 

where we have written $k = \abs{\vb{k}}$. One may think that the solution to this equation is simply  
\begin{equation}\label{eq:5.43}
\hat{G}(\omega, \vb{k}) = - \frac{1}{4\pi^2} \frac{1}{(\omega^2 - k^2)} = - \frac{1}{4\pi^2} \, \frac{1}{(\omega - k)} \, \frac{1}{(\omega + k)}\,.
\end{equation}
However, dividing by $(\omega^2 - k^2)$ is actually an illegal step, because this quantity can be zero. The difficulty arising from this can be seen if we attempt to take the inverse Fourier transform of $\hat{G}$ with respect to $\omega$ (but not $\vb{k}$), so as to obtain the Fourier transform of $G$ with respect to space but not time, which we denote as $\tilde{G}$:
\begin{equation}\label{eq:5.44}
\tilde{G}(t, \vb{k}) \equiv \frac{1}{\sqrt{2 \pi}} \int_{-\infty}^{\infty} \hat{G}(\omega, \vb{k}) e^{-i \omega t} \dd\omega\,.
\end{equation}

Then, we have
\begin{equation}\label{eq:5.45}
\tilde{G}(t, \vb{k}) = - \frac{1}{4 \pi^2 \sqrt{2 \pi}} \int_{-\infty}^{\infty} \frac{e^{-i \omega t}}{(\omega - k) (\omega + k)} \dd\omega\,.
\end{equation}


However, there are logarithmic divergences in the integral on the right side of eq. (\ref{eq:5.45}) at 
$\omega = \pm k$, so the right side is ill defined (even as a distribution). This reflects the fact that many Green's functions satisfy eq. (\ref{eq:5.32}), so we cannot be expected to be able to solve for $G$ without providing additional input as to which Green's function we seek. 

This difficulty can be dealt with by suitably regularizing eq. (\ref{eq:5.45}) in such a way that the right side is well defined and eq. (\ref{eq:5.42}) continues to hold. A simple way of doing this is to infinitesimally displace the poles at $\omega = \pm k$ in eq. (\ref{eq:5.45}) into the complex $\omega$-plane. The case of most interest for us is to displace both poles infinitesimally into the lower half of the $\omega$-plane, thereby defining the \tit{retarded Green's function}
\footnote{As noted below, displacement of both poles into the upper half of the $\omega$-plane yields the advanced Green's function. Displacement of the pole at $\omega = k$ into the lower half-plane and the pole at $\omega = -k$ into the upper half-plane yields the Feynman propagator.} 
$\tilde{G}_{ret}$, given by
\begin{equation}\label{eq:5.46}
\tilde{G}(t, \vb{k}) = - \frac{1}{4 \pi^2 \sqrt{2 \pi}} \int_{-\infty}^{\infty} \frac{e^{-i \omega t}}{(\omega - k + i \epsilon) (\omega + k + i \epsilon)} \dd\omega\,,
\end{equation}
where $\epsilon > 0$ and the limit $\epsilon \rightarrow 0$ is to be taken after the integral is performed. We can view eq. (\ref{eq:5.46}) as a contour integral in the complex $\omega$-plane. For $t < 0$, the integral is exponentially damped in the upper half of the $\omega$-plane, and we may \quotes{close the contour} in that half-plane. The resulting closed contour encloses no singularities---since the poles have been pushed into the lower half-plane--so, by Cauchy's theorem, we obtain
\begin{equation}\label{eq:5.47}
\tilde{G}_{ret}(t, \vb{k}) = 0, \quad\quad\quad \text{for } t < 0\,.
\end{equation}
This condition uniquely characterizes the retarded Green's function. The fact that $\tilde{G}_{ret}$ vanishes prior to the \quotes{turn-on} of the delta function source at $t = 0$ can be interpreted as saying that it is providing the solution with \quotes{no incoming radiation.} This is the solution of physical relevancein problems where no radiation is present prior to the presence of the source.

For $t > 0$, we can similarly close the contour in the lower half of the $\omega$-plane. However, the resulting closed contour now contains poles at $\omega = \pm k -i\epsilon$. By Cauchy's theorem, we obtain for $t > 0$
\begin{equation}\label{eq:5.48}
\begin{aligned}
\tilde{G}(t, \vb{k}) &= + \frac{1}{4 \pi^2 \sqrt{2 \pi}} 2\pi i \left[ \frac{e^{-ikt}}{2k} - \frac{e^{+ikt}}{2k}  \right]\\
                     &= \frac{1}{2 \pi \sqrt{2 \pi}} \frac{\sin kt}{k}\,,
\end{aligned}
\end{equation}
where the sign change in the first line as compared with eq. (\ref{eq:5.46}) results from the contour running the \quotes{wrong way}.
  
We now take the inverse Fourier transform of eq. (\ref{eq:5.48}) with respect to $\vb{k}$ to obtain the retarded Green's function for $t > 0$ in position space:  

\section{Multipole Expansion}\label{sec:5.3}

\subsection{Cartesian Multipole Expansion of the Radiation Field for a Nonrelativistic Source}

\subsection{General Multipole Expansion for a Relativistic Source}

\section{The Initial Value Formulation for Maxwell's Equations}\label{sec:5.4}

\section{Plane Waves}\label{sec:5.5}

\section{Conducting Cavities and Waveguides}\label{sec:5.6}

\subsection{Conducting Cavities}

\subsection{Waveguides}

%===================================================================================

\section*{Problems}


   % Electrodynamics
\setcounter{chapter}{0}
\renewcommand{\thechapter}{6}
\chapter{Electrodynamics in Media}
\setcounter{equation}{0}	        % To start with Equation 1
\counterwithin{equation}{chapter}	% Equation numbering will be 2.1 2.2 2.3   ... 

\section{Linear, Homogeneous, Isotropic Medium with an Instantaneous Response}

\section{Linear, Homogeneous, Isotropic Medium with a Delayed Response}

\section{The Lorentz Model for $\epsilon(\omega)$ }

\subsection{Nonconducting Medium}

\subsection{Plasma or Conducting Medium}

\section{Magnetohydrodynamics}

%===================================================================================

\section*{Problems}


   % Electrodynamics in Media
\setcounter{chapter}{0}
\renewcommand{\thechapter}{7}
\chapter{Geometric Optics, Interference, Diffraction}
\setcounter{equation}{0}	        % To start with Equation 1
\counterwithin{equation}{chapter}	% Equation numbering will be 2.1 2.2 2.3   ... 

\section{Geometric Optics}

\section{Interference and Coherence}

\section{Diffraction}

\subsection{Scattering by a Dielectric BallNonconducting Medium}

\subsection{Propagation of Waves through an Aperture}

%===================================================================================

\section*{Problems}


   % Geometric Optics, Interference, Diffraction
\setcounter{chapter}{0}
\renewcommand{\thechapter}{8}
\chapter{Special Relativity}\label{ch:8}
\setcounter{equation}{0}	        % To start with Equation 1
\counterwithin{equation}{chapter}	% Equation numbering will be 8.1 8.2 8.3   ... 

Special relativity is the theory of spacetime structure formulated by Einstein in 1905. Properties of the electromagnetic field played a central role in motivating special relativity. Specifically, electromagnetism is not compatible with pre-relativity notions of spacetime structure unless there is a preferred rest frame (the \quotes{aether}), since, as we have seen, Maxwell's equations predict that electromagnetic waves propagate with a particular velocity $c$, which can only be true in some preferred rest frame if pre-relativity notions of spacetime structure are valid. The Michelson-Morley experiment failed to find such a preferred rest frame. Furthermore, as Einstein realized, some physical phenomena in electromagnetism appear to obey an invariance with respect to moving observers even if the description of these phenomena in terms of a preferred rest frame does not have such an invariance.

The theory of electromagnetism is far more elegant and simple when formulated in the framework of special relativity. It therefore is somewhat of a travesty that, well into the twenty-first century, special relativity is discussed here---as in other texts on electromagnetism---as a separate chapter toward the end of the book. The reason, of course, is that even though special relativity has been a well established theory for much more than a century, its basic concepts are still so unfamiliar to most physicists that it is not feasible to begin the treatment of electromagnetism by giving its formulation in the framework of special relativity. It is my hope that this situation will be rectified by the twenty-second century.

Einstein's original formulation of special relativity relied heavily on the transformations between the labeling of events by different inertial observers and the invariance of the laws of physics under such transformations. The theory was reformulated in a much more geometrical form by Minkowski in 1908, wherein it was recognized that the underlying structure of spacetime in special relativity is that of a spacetime metric.\footnote{Minkowsky introduced an imaginary time coordinate so as to obtain a Euclidean spacetime metric. However, although this approach remains in use in many treatments of special relativity, it does not generalize to curved spacetime and cannot be used in general relativity. We shall use a real time coordinate in our treatment, and our spacetime metric will therefore be of Lorentzian signature.} Our treatment of special relativity will emphasize the role of the spacetime metric. Although Einstein was initially unimpressed by Minkowski's reformulation, he soon incorporated it into his thinking about gravitation. This led him to the theory of general relativity, wherein the spacetime metric becomes a dynamical variable that describes not only spacetime but also the effects of gravity. However, we shall not discuss general relativity here. 

The framework of special relativity is presented in section 8.1. The formulation of electromagnetism in the framework of special relativity is then given in section 8.2. On section 8.3.1, we analyze the motion of a (relativistic) charged particle in an external electromagnetic field, including the solutions for motion in a nonuniform electric field and in a uniform magnetic field. The Lienard-Wiechert solution describing the retarded field of a point charge in arbitrary motion is given in section 8.3.2, and properties of the radiated power for this solution are analyzed there as well.  

\section{The Framework of Special Relativity}\label{sec:8.1}
It is useful to think of space and time as composed of \quotes{events}---where each event corresponds to a point of space at an instant of time. The collection of all events comprises a 4-dimensional continuum, which I refer to as \quotes{spacetime}.

I take as a starting point that there exist global families of inertial observers who \quotes{fill} all of spacetime (i.e., within each family, one and only one of these observers passes through each event in spacetime). I further assume that the observers in each family are all \quotes{at rest} with respect to one another, that they can consistently synchronize their clocks by some physical procedure, and that the spatial relationships between these observers are described by Euclidean geometry. Finally, I also assume that different families of such inertial observers all move at uniform velocity with respect to one another, so that the different families may be labeled by their velocity with respect to some reference family. These assumptions are true in both pre-relativity physics and in special relativity, so they make a good starting point for describing the differences between these theories of spacetime structure. However, these assumptions are \tit{not} true in general relativity, so they would make a very poor starting point from a fundamental viewpoint.

By the above assumptions, the inertial observers in a given family can uniquely label events by $(t, \vb{x})$, where $t$ denotes the time of the event on the synchronized clock of the observer who passes through the event, and $\vb{x} = (x, y, z)$ are the Cartesian coordinates of that observer. I refer to the labeling of events in this way as \tit{inertial coordinates}. The assumption that events can be labeled in this way is implicit in every physics text or other reference where a \quotes{$t$} or \quotes{$\vb{x}$} appears in an equation. However,   this labeling depends on (i) a choice of origin of time (i.e., what time is labeled as $t=0$); (ii) a choice of origin of space (i.e., what observer in the family is at $\vb{x} = 0$); (iii) a choice of orientation of axes (to define the $x-$, $y-$,  and $z-$directions); and, most importantly for our present purposes, (iv) a choice of which family of inertial observers to use (i.e., a choice of the velocity $\vb{v}$ of the family). Different choices of origin of $t$ and $\vb{x}$, orientation of axes, and $\vb{v}$ will give rise to different labelings ($t', \vb{x}'$) of events. 

Usually, treatments of special relativity focus entirely on the difference in the labeling of events between families of inertial observers who are moving with velocity $\vb{v}$ relative to one another. This is given by a Galilean transformation in pre-relativity physics and by a Lorentz transformation in special relativity. Although it certainly is useful to know the explicit form of this transformation, a nearly exclusive focus on this obscures the geometrical content of the theory. It is analogous to studying ordinary Euclidean geometry by focusing on how the Cartesian coordinates transform under rotations. 

I therefore focus on the \quotes{invariant structure} of spacetime. The four numbers $(t, \vb{x})$ associated with an event do not, by themselves, convey meaningful information about the event, since they depend as much, for example, on the choice of origin in $t$ and $\vb{x}$ as they do on the event itself. Even if we consider the differences $(\Delta t, \Delta \vb{x})$ in the labeling of two events by a given family of inertial observers so as to eliminate the origin dependence, the values of $(\Delta t, \Delta \vb{x})$ will depend on the choice of orientation of axes as well as on the choice of family of inertial observers. It is of great interest to determine what quantities constructed out of $(\Delta t, \Delta \vb{x})$ are invariant (i.e., independent of these choices). Such quantities are well defined without making any arbitrary choices and hence can be considered as attributable to the structure of spacetime itself. 

In pre-relativity physics, there are two such invariant quantities: (i) the time interval $\Delta t$ between events, and (ii) the space interval $\abs{\Delta \vb{x}}^2$ between simultaneous events (i.e., events with $\Delta t = 0$). The space interval between nonsimultaneous events is not invariant, because if the family $O'$ of inertial observers moves with velocity $\vb{v}$ with respect to the family $O$, then we have 
\begin{equation}\label{eq:8.1}
\Delta \vb{x}' = \Delta \vb{x} - \vb{v} \Delta t\,,
\end{equation}

so $\abs{\Delta \vb{x}'}^2 \neq \abs{\Delta \vb{x}}^2$ if $\Delta t \neq 0$.  In addition, the collection of \quotes{worldlines} of inertial observers (i.e., the possible paths in spacetime of inertial observers) also can be viewed as an additional aspect of spacetime structure. The worldlines of inertial observers contain additional information independent of (i) and (ii) in that they cannot be constructed from knowing only $\Delta t$ for all pairs of events and knowing $\abs{\Delta \vb{x}}^2$ for simultaneous events. 

The situation in special relativity is much simpler. In special relativity, there is a single invariant quantity, the spacetime interval $I$ between any pair of events, given by 
\begin{equation}\label{eq:8.2}
I = - c^2 (\Delta t)^2 + \abs{\Delta \vb{x}}^2\;.
\end{equation}
 Furthermore, it can be shown that the worldlines of inertial observers can be constructed from a knowledge of $I$ between all pairs of events. Thus, $I$ provides the complete description of spacetime structure in special relativity. 
 
To tie the previous paragraph to what people are usually taught in special relativity, note that, in special relativity, 
the labeling $(t, \vb{x})$ of events by a family $O$ of observers is related to the labeling $(t', \vb{x}')$ by a family $O'$ of observers moving with velocity $v$ in the $x-$direction relative to $O$ (and with the same origin event and the same orientation of axes as $O$) by a \tit{Lorentz transformation}:
\begin{equation}\label{eq:8.3}
\begin{aligned}
t' &= \gamma (t - vx/c^2)\,,\\
x' &= \gamma (x - vt)\,,\\
y' &= y\,,\\
z' &= z\,,
\end{aligned}
\end{equation}
where   
\begin{equation}\label{eq:8.4}
\gamma \equiv \frac{1}{\sqrt{1 - v^2/c^2}}\,.
\end{equation}

\newpage

% BOX 1 **START** -- << Give Lorentz transformation in terms of parallel and normal components wrt the relative velocity >>
\subsubsection*{BOX 8.1 -- Lorentz transformations in vector form}\label{box:8.1}
\parindent=0pt  % Set the paragraph indentation to 0 (normal = 10pt) before the box 
\parbox{\textwidth}{\begin{mdframed}[style=MyFrame] %Added "\parbox{\textwidth}{"
%\lipsum[1]
A Lorentz transformation does only change time and the component of the spatial position vector $\vb{r}$ in the direction of relative motion; it does not change any component of $\vb{r}$ perpendicular to that direction. With this in mind, split $\vb{r}$ as measured in a frame $F$, and $\vb{r}'$ as measured in $F'$, each into components perpendicular ($\perp$) and parallel ($\parallel$) to the direction of the relative velocity $\vb{v}$:
\begin{equation*}
\vb{r} = \vb{r}_\perp + \vb{r}_\parallel\,,\quad \vb{r'} = \vb{r'}_\perp + \vb{r'}_\parallel 
\end{equation*}
then the transformations are 
\begin{align*}
t' &= \gamma \left( t - \frac{\vb{r}_\parallel \cdot \vb{v}}{c^2} \right)\\
\vb{r'}_\parallel &= \gamma (\vb{r}_\parallel - \vb{v}t)\\
\vb{r'}_\perp &= \vb{r}_\perp
\end{align*}
%\lipsum[2]
\end{mdframed}} %Added a closing "}" here
\parindent=10pt % Set the paragraph indentation to normal (10pt) 
% BOX 1 ** END ** -- << Give Lorentz transformation in terms of parallel and normal components wrt the relative velocity >>

The interval $I$ (Eq. \ref{eq:8.2}) is invariant under Lorentz transformations. 

% BOX 2 **START** -- << Check that I is invariant under Lorentz transformations >>
\subsubsection*{BOX 8.2 -- Invariance of $I$}\label{box:8.2}
\parindent=0pt  % Set the paragraph indentation to 0 (normal = 10pt) before the box 
\parbox{\textwidth}{\begin{mdframed}[style=MyFrame] %Added "\parbox{\textwidth}{"
%\lipsum[1]
Making use of vector notations introduced in Box \ref{box:8.1} we have 
\begin{align*}
{I'}^2 &=  - \left(c \Delta t'\right)^2 + \abs{\Delta \vb{r}'}^2\\
       &= -c^2 \gamma^2 \left(\Delta t - \Delta \vb{r}_\parallel \cdot \vb{v} /c^2 \right)^2
          +    \gamma^2 \abs{\Delta \vb{r}_\parallel - \vb{v} \Delta t}^2 \quad + \abs{\Delta \vb{r}_\perp}^2\\
       &=\;    -\gamma^2 \left( {(c \Delta t)}^2 + v^2/c^2 \abs{\Delta \vb{r}_\parallel}^2      -2 \Delta \vb{r}_\parallel \cdot \vb{v} \Delta t \right)\\
       &+ \quad \gamma^2 \left( \abs{\Delta \vb{r}_\parallel}^2 + v^2 (\Delta t)^2 \quad\quad   -2 \Delta \vb{r}_\parallel \cdot \vb{v} \Delta t \right)\quad\:\: + \abs{\Delta \vb{r}_\perp}^2\\
       &= - c^2 (\Delta t)^2  + \gamma^2 (1 - v^2/c^2) \abs{\Delta \vb{r}_\parallel}^2 + \abs{\Delta \vb{r}_\perp}^2\\
       &= - c^2 (\Delta t)^2  + \abs{\Delta \vb{r}_\parallel}^2 + \abs{\Delta \vb{r}_\perp}^2\\
       &= - c^2 (\Delta t)^2  + \abs{\Delta \vb{r}}^2\\ 
       &= I^2\qed
\end{align*}

%\lipsum[2]
\end{mdframed}} %Added a closing "}" here
\parindent=10pt % Set the paragraph indentation to normal (10pt) 
% BOX 2 ** END ** -- << Check that I is invariant under Lorentz transformations >>

\newpage
Furthermore, it can be shown that the most general transformation that preserves $I$ is a \tit{Poincaré transformation} (i.e., a composition of Lorentz transformations, rotations, and translations, as well as parity and time reversal transformations). Thus, Lorentz transformations naturally arise as (part of) the symmetry group of $I$.

The spacetime interval $I$ has the same mathematical form as the squared distance in Euclidean geometry except for the minus sign in front of the contribution coming from $(\Delta t)^2$. To pursue this further, we switch notation from $(t, \vb{x})$ to $x^\mu$ with $\mu=0, 1, 2, 3\,,$ where 
\begin{equation}\label{eq:8.5}
x^0 = ct,\quad x^1 = x,\quad x^2 = y,\quad x^3 = z\,.
\end{equation}

Note the superscript position of $\mu$ in $x^\mu$, which will be important in order to align with notational conventions explained further below. We view $x^\mu$ as representing a spacetime displacement vector (relative to some origin in spacetime) in much the same way as we normally view $\vb{x}$ as representing a spatial displacement vector (relative to some origin in space). We view $I$, eq. (\ref{eq:8.2}), as arising from an \tit{inner product} on spacetime displacement vectors, where the inner product of $x^\mu_1$ and $x^\mu_2$ is given in any inertial coordinates by 
\begin{equation}\label{eq:8.6}
I(x^\mu_1, x^\mu_2)  = \sum_{\mu,\nu=0}^{3} \eta_{\mu \nu} x^\mu_1 x^\nu_2\,,
\end{equation}

where\footnote{Many authors define $\eta_{\mu \nu}$ with an opposite sign convention, which results in sign changes in some formulas. The reader is advised to check the sign convention used for $\eta_{\mu \nu}$ when comparing formulas in different references. As mentioned in footnote 1 in this chapter, some authors use an imaginary time coordinate, in which case $\eta_{\mu \nu}$ would take a Euclidean form and normally would not be written down explicitly at all.}
\begin{equation}\label{eq:8.7}
\eta_{\mu \nu} \equiv \mqty(-1 &0 &0 &0\\ 0 &1 &0 &0\\ 0 &0 &1 &0\\ 0 &0 &0 &1)\;.
\end{equation}

I put \quotes{inner product} in quotes, because although $I$ is linear in each variable, symmetric, and nondegenerate (i.e., $I(x^\mu_1, x^\mu_2) = 0$ for all $x^\mu_2$ if and only if $x^\mu_1 = 0$), it fails to be positive definite. Nevertheless, it is closely analogous to the inner product on vectors in ordinary Euclidean geometry, 
\begin{equation}\label{eq:8.8}
(\vb{x}_1, \vb{x}_2) = \vb{x}_1 \cdot \vb{x}_2 = \sum_{i,j=1}^3 e_{ij} x_1^i x_2^j\,,  
\end{equation}

where 

\begin{equation}\label{eq:8.9}
e_{i j} \equiv \mqty(1 &0 &0\\ 0 &1 &0\\ 0 &0 &1)\;.
\end{equation}

We refer to $e_{ij}$ as the \tit{metric of space} in Euclidean geometry. Similarly, werefer to $\eta_{\mu \nu}$ as the 
\tit{metric of spacetime} in special relativity.

It should be mentioned that the ability to give finite spatial or spacetime displacements a vector space structure, as implicitly assumed in the discussion above, is very special to a \tit{flat} geometry. In a curved geometry---such as the $2-$dimensional surface of a potato---there is no natural notion of adding two finite displacements about a point. Nevertheless, in a curved geometry, a notion of 
\quotes{infinitesimal displacements} about $p$ is referred to as the \tit{tangent space} at $p$. 
In differential geometry, a metric would be defined as a (not necessarily positive-definite) inner product defined on the tangent space at $p$ for all $p$. However, the spacetime geometry of special relativity is flat, so we may treat \tit{finite} spacetime displacements 
$\Delta x^\mu \equiv x^\mu - x^\mu(p)$ about a point $p$ as \quotes{vectors}---and we can treat the metric as an inner product on these finite displacement vectors---as we have done above.

A striking feature of the spacetime metric eq. (\ref{eq:8.7}) is that there are nonzero spacetime displacement vectors $\Delta x^\mu$
about any event $p$ that are \tit{null}, in other words, such that 
\begin{equation}\label{eq:8.10}
\sum_{\mu \nu} \eta_{\mu \nu} \Delta x^\mu \Delta x^\nu = 0 \,.
\end{equation}

The collection of all null spacetime displacement vectors about $p$ comprise a cone with vertex at $p$, as illustrated in figure 8.1. The portion with $\Delta x \leq 0$ is referred to as its \tit{past light cone}. In the precise sense discussed in chapter 5, electromagnetic radiation emitted at $p$ propagates along the future light cone of $p$ (see section 5.2), whereas electromagnetic radiation observed at $p$ propagated to $p$ along its past ligt cone (see section 5.4). The interior of the future light cone of $p$ is referred to as the \tit{future} of $p$. It is composed of events that, in principle, can be reached by an observer initially present at $p$. Similarly, the interior of the past light cone of $p$ is referred to as the \tit{past} of $p$. It is composed of events with the property that an observer starting at that event can, in principle, arrive at $p$. The events lying outside the light cone of $p$ are said to be \tit{spacelike related} to $p$. No observer can be present both at $p$ and at an event spacelike related to $p$. In other words, in special relativity, nothing can travel faster than light. 

If $q$ is an event that lies in the future of $p$, then there is a unique inertial observer who is present at both $p$ and $q$. The proper time $\Delta \tau$ that elapses on a clock carried by this observer between $p$ and $q$ is given in any inertial cordinates by\footnote{This can be seen by noting that the right side of eq. (\ref{eq:8.11}) is the spacetime interval between $p$ and $q$ and does not depend on the choice of inertial coordinates. In the frame of the inertial observers who goes from $p$ to $q$, we have $\Delta \vb{x} = 0$ so $\sum\eta_{\mu \nu} \Delta x^\mu \Delta x^\nu = - c^2 (\Delta t)^2$.}
\begin{equation}\label{eq:8.11}
\Delta \tau = \frac{1}{c}\sqrt{-\sum_{\mu, \nu} \eta_{\mu \nu} \Delta x^\mu \Delta x^\nu} \,,
\end{equation}

where $\Delta x^\mu = x^\mu(q) - x^\mu(p)$. A general, noninertial observer will trace out a curve in spacetime, called the \tit{worldline} of the observer. Any curve in spacetime may be specified by giving $x^\mu(\lambda)$, where $\lambda$ 
is an arbitrary parametrization of the curve. The tangent $T^\mu$ to the curve in this parametrization is defined by 
\begin{equation}\label{eq:8.12}
T^\mu = \dv{x^\mu}{\lambda}\,.
\end{equation}

The tangent $T^\mu$ to the worldline of any observer must be timelike (i.e, $\sum_{\mu \nu} \eta_{\mu \nu} T^\mu T^\nu < 0$), since the observer must stay within the light cone of any event that he/she passes through. The proper time elapsed on the clock of an arbitrary noninertial observer going between events $p$ and $q$ is given by
\begin{equation}\label{eq:8.13}
\Delta \tau = \frac{1}{c} \bigint_{\!\!\!\!\!\!\!\!\lambda(p)}^{\!\!\!\!\!\!\!\!\!\lambda(q)} {\sqrt{-\sum_{\mu, \nu} \eta_{\mu \nu} T^\mu T^\nu}} \dd{\lambda}\,.
\end{equation}

It is not difficult to show that the inertial observer who passes through events $p$ and $q$ maximizes the elapsed proper time relative to all observers who pass between $p$ and $q$. This fact is often referred to as the \tit{twin paradox} (see problem 1 at the end of the chapter).

It is convenient to parametrize the worldline of an arbitrary observer by the proper time $\tau$ elapsed along the worldline. The tangent to the worldline in this parametrization is called the 4-\tit{velocity} of the observer:
\begin{equation}\label{eq:8.14}
u^\mu = \dv{x^\mu}{\tau}\,.
\end{equation}

It follows from eq. (\ref{eq:8.13}) with $\lambda = \tau$ that $$\sum_{\mu\, \nu} \eta_{\mu \nu} u^\mu u^\nu = -c^2\,.$$
In any inertial coordinates, the velocity $\vb{v}$ of this observer is given by
\begin{equation}\label{eq:8.15}
\vb{v} = \dv{\vb{x}}{t} = c \dv{\vb{x}}{x^0} = c \frac{\dv*{\vb{x}}{\tau}}{\dv*{x^0}{\tau}} = c \frac{\vb{u}}{u^0}\,.
\end{equation}

It follows from this relation and the normalization condition on $u^\mu$ that, in any inertial coordinates, the components of $u^\mu$ are given in terms of $\vb{v}$ by 
\begin{equation}\label{eq:8.16}
u^\mu = (c \gamma, \gamma \vb{v})\,,
\end{equation}
where $\gamma$ is given by eq. (\ref{eq:8.4}).

As discussed above, because of the flat spacetime geometry, the finite displacements 
$\Delta x^\mu = x^\mu - x^\mu(p)$ about $p$ comprise a 4-dimensional vector space, which we denote by $V_p$. The tangent $T^\mu$ to a curve passing through $p$ can naturally be viewed as a vector in $V_p$. For an arbitrary spacetime vector, we will strictly adhere to the notational convention that an \tit{upper} Greek letter index will be attached to the letter representing the vector (e.g: when a symbol such as $W^\mu$ appears below, one can immediately tell from the notation that $W^\mu$ is a spacetime vector). 

Spacetime vectors in special relativity have a geometrical status in exactly the same sense that ordinary vectors have a geometrical status in ordinary, 3-dimensional space. Although the representation of a spacetime vector in terms of its components depends on a choice of inertial coordinate system, one may view the spacetime vector as having a geometrical meaning that does not depend on a choice of coordinates---in exactly the same manner as an ordinary vector can be viewed as having a geometrical meaning, independent of the representation of its components in a particular Cartesian coordinate system. 

Another class of objects that have a similar geometrical status are linear maps taking vectors to numbers. A linear map taking $V_p$ into $\R$ is called a \tit{dual vector}. The collection of all dual vectors at $p$ comprises a vector space of the same dimension as $V_p$, known as the \tit{dual space} to $V_p$ and denoted by $V^*_p$. Given any vector $W^\mu$ in $V_p$, we can use the spacetime metric $\eta_{\mu \nu}$  to construct a dual vector $L_W$ taking 
$V_p$ into $\R$ by

\begin{equation}\label{eq:8.17}
S^\mu \rightarrow \sum_{\mu, \nu} \eta_{\mu \nu} S^\mu W^\nu\,,
\end{equation}

that is, $L_W$ maps the arbitrary vector $S^\mu$ into the number given by the right side of this equation. It is not difficult to show that, in the presence of a metric, all dual vectors arise in this manner. It is natural to denote the dual vector $L_W$ by 
\begin{equation}\label{eq:8.18}
W_\mu \equiv \sum_{\nu = 0}^3 \eta_{\mu \nu} W^\nu\,,
\end{equation}
so that the linear map $L_W$ is given by $S^\mu \rightarrow \sum_\nu S^\nu W_\nu$. In this notation, the lowered index on $W_\mu$ on the left side of eq. (\ref{eq:8.18}) indicates that this quantity is the dual vector obtained from $W^\mu$ rather than $W^\mu$ itself. For an arbitrary dual vector we will strictly adhere to the notational convention that a \tit{lower} Greek letter index will be attached to the letter representing the dual vector (e.g., when a symbol such as $U_\mu$ appears below, one can immediately tell from the notation that $U_\mu$ is a spacetime dual vector).

Since the spacetime metric is not degenerate, eq. (\ref{eq:8.18}) can be inverted to give
\begin{equation}\label{eq:8.19}
W^\mu = \sum_{\nu = 0}^3 \eta^{\mu \nu} W_\nu\,,
\end{equation}
where $\eta^{\mu \nu}$ denotes the inverse spacetime metric, whose matrix of components is given by the inverse matrix\footnote{In an inertial coordinate system, the components of $\eta_{\mu \nu}$ are given by eq. (\ref{eq:8.7}), and the inverse of eq. (\ref{eq:8.7}) then has exactly the same components as eq. (\ref{eq:8.7}) (see eq. (5.121)). However, $\eta^{\mu \nu}$ and $\eta_{\mu \nu}$ are fundamentally very different objects, and this equality of their components would not hold in a general, noninertial coordinate system.} of $\eta_{\mu \nu}$. Following standard conventions, if $U_\mu$ is a dual vector, we denote the corresponding vector given by eq. (\ref{eq:8.19}) as $U^\mu$. Thus, we use $\eta^{\mu \nu}$ and $\eta_{\mu \nu}$ to raise and lower indices in the manner given by eq. (\ref{eq:8.18}) and eq. (\ref{eq:8.19}).

Vectors and dual vectors are fundamentally very different objects. Nevertheless, when a metric is present, vectors and dual vectors can be identified via the correspondence eq. (\ref{eq:8.18}) or equivalently, eq. (\ref{eq:8.19}). When dealing with a vector $\vb{v}$ in ordinary Euclidean space, the components $(v^1, v^2, v^3)$ of the vector Cartesian coordinates are equal to the components $(v_1, v_2, v_3)$ of the corresponding dual vector on account of the trivial form, eq. (\ref{eq:8.9}), of the Euclidean metric. Thus, in this case, if one treats a vector as a 3-tuple of numbers, one can get away with not making a distinction between a vector and its corresponding dual vector.\footnote{In this regard, it should be noted that in the previous chapters of this book, we denoted the Cartesian components of vectors such as the electric field $\vb{E}$ as $E_i$ rather than $E^i$. This is not because we were working with the corresponding dual vector but instead because there was no need to make a distinction between vectors and dual vectors and, correspondingly, no need to adhere to the conventions on the index positions that we have just introduced. We wrote \quotes{$E_i$} simply because it was typographically more convenient to write $E_i$ rather than $E^i$. However, from this point forward in this book, we will strictly adhere to our conventions on index positions.} 
This accounts for why many students of physics have never explicitly encountered the notion of a dual vector. However, in special relativity, in any inertial coordinates, the dual vector $W_\mu$ corresponding to $W^\mu$ has $W_0 = - W^0$, and we cannot get away with ignoring the distinction between vectors and dual vectors.  

A prime example of a dual vector at $p$ is the spacetime gradient $\partial_\mu f = \pdv*{f}{x^\mu}$ of a function $f$ evaluated at $p$. This is seen to be a dual vector as follows. For any curve with tangent $T^\mu$ at $p$, the quantity $\sum_\mu T^\mu \partial_\mu f$ represents the derivative of $\pdv*{f}{\lambda}$ of $f$ along the curve and thus is well defined, independently of any choice of coordinates. Thus, $\partial_\mu f$ at $p$ naturally can be viewed as a linear map taking tangent vectors $T^\mu$ at $p$ into numbers (i.e., a dual vector). This is because the correspondence between vectors and dual vectors that is provided by the Euclidean metric is being used. Fundamentally, however, the gradient of $f$ is a dual vector.  


 




 
 
  
\section{The Formulation of Electromagnetic Theory in the Framework of Special Relativity}\label{sec:8.2}

\section{Charged Particle Motion and Radiation}\label{sec:8.3}

\subsection{Charged Particle Motion}\label{ssec:8.3.1}

\subsection{Radiation from a Point Charge in Arbitrary Motion}\label{ssec:8.3.2}

%===================================================================================

\section*{Problems}


   % Special Relativity
\setcounter{chapter}{0}
\renewcommand{\thechapter}{9}
\chapter{Electromagnetism as a Gauge Theory}
\setcounter{equation}{0}	        % To start with Equation 1
\counterwithin{equation}{chapter}	% Equation numbering will be 2.1 2.2 2.3   ... 

\section{Lagrangian for the Electromagnetic Field and Its Interactions}

\section{Gauge Invariance and the Reinterpretation of the Electromagnetic Field as a Connection}

\section{Dirac Magnetic Monopoles}

%===================================================================================

% \section*{Problems}


   % Electromagnetism as a Gauge Theory
\setcounter{chapter}{0}
\renewcommand{\thechapter}{10}
\chapter{Point Charges and Self-Force}
\setcounter{equation}{0}	        % To start with Equation 1
\counterwithin{equation}{chapter}	% Equation numbering will be 2.1 2.2 2.3   ... 

\section{The Point Particle Limit}

\section{Lorentz Force}

\section{Corrections to Lorentz Force Motion}

\subsection{Self-force Corrections}

\subsection{Spin and Dipole Effects}

\section{Self-Consistent Motion}
%===================================================================================

% \section*{Problems}


  % Point Charges and Self-Force



%\appendixpage
% Appendix on Units and Dimensions
%\include{Electro_RMW_A1}	% This book has no appendices   



\backmatter%%%%%%%%%%%%%%%%%%%%%%%%%%%%%%%%%%%%%%%%%%%%%%%%%%%%%%%
%%%%%%%%%%%%%%%%%%%%%%%%% referenc.tex %%%%%%%%%%%%%%%%%%%%%%%%%%%%%%
% sample references
% 
% Use this file as a template for your own input.
%
%%%%%%%%%%%%%%%%%%%%%%%% Springer-Verlag %%%%%%%%%%%%%%%%%%%%%%%%%%

%
% BibTeX users please use
% \bibliographystyle{}
% \bibliography{}
%
% Non-BibTeX users please use
\begin{thebibliography}{99.}
%
% and use \bibitem to create references.
%
% Use the following syntax and markup for your references
%
% Monograph
\bibitem{Griffiths_4th} D.J. Griffiths (2017)
Introduction to Electrodynamics. Cambridge University Press, Cambridge

% Monograph
\bibitem{Felsager_1981} B. Felsager (1981)
Geometry, Particles and Fields. Odense University Press

% Monograph
\bibitem{BudakFomin_1973} B.M. Budak, S.V. Fomin (1973)
Multiple Integrals, Field Theory and Series. Mir Publishers, Moscow

% Monograph
\bibitem{Postnikov_II_1982} Mikhail Postnikov (1982)
Lectures in Geometry, Semester II. Linear Algebra and Differential Geometry. Mir Publishers, Moscow

\end{thebibliography}

\printindex

%%%%%%%%%%%%%%%%%%%%%%%%%%%%%%%%%%%%%%%%%%%%%%%%%%%%%%%%%%%%%%%%%%%%%%

\end{document}





