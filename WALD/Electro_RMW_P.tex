\section*{Preface}

This book arose from my teaching the first quarter of the standard graduate course in electromagnetism at the University of Chicago in the winter of 2018. It had been decades since I had previously taught this course, so I approached it with fresh eyes, and it was natural for me to try to rethink how the subject of electromagnetism should be presented at the graduate level. When I did so, it became clear to me that the usual quasi-historical way of presenting the subject promotes some very unhealthy ways of thinking about electromagnetism. Therefore, to avoid starting off on the wrong foot, I decided to spend the first few lectures of the course describing what I now refer to in chapter 1 of this book as \quotes{myths} concerning electromagnetism. I found that by starting out in this way, it became much easier to straightforwardly present the subject in a clear and concise manner, without having to make shifts in perspective as the subject is developed. I taught the course again in the following 3 years and provided lecture notes to the class. These lecture notes have now evolved into this book.

The first chapter of this book is thus a quite unconventional introduction to electromagnetism. Instead of beginning with the force between charged particles, discussing how this gives rise to a \quotes{field} concept, and so forth, my aim in chapter 1 is to explain to students how they should think about electromagnetism from a modern and mathematically precise perspective. The major points made in this chapter are that (i) the potentials, not the field strenghts, are the fundamental dynamical variables in electromagnetism; (ii) the energy and momentum properties of the electromagnetic field are an essential part of the formulation of the theory and cannot properly be derived by \quotes{work done} arguments; (iii) electromagnetic fields should not be thought of as being \tit{produced} by charges; and (iv) at a fundamental level, the charged matter in classical electrodynamics must be viewed as continuosly distributed rather than consisting of point charges. Many of these points cannot be fully elucidated until the later chapters in the book---particularly chapters 9 and 10---but my intent is to lay out these ideas in a sufficiently clear and explicit way in chapter 1 that I can take these perspectives unapologetically in the remainder of the book.

The topics treated in chapters 2-7 are ones that normally would be covered in any graduate course in electromagnetism. Electrostatics is treated in chapter 2, but starting with Poisson's equation, not Coulomb's law. Dielectric materials in electrostatics are treated in chapter 3, with considerable care given to how the macroscopic averaging is done and to the treatment of energy. Magnetostatics is treated in chapter 4, with a full discussion of the sign difference between magnetostatics and electrostatics in the field interaction energy of a dipole in an external field--and how this relates to the change in the rest mass of a magnet when it is quasy-statically moved in an external magnetic field. Electrodynamics and radiation are discussed in depth in chapter 5. In addition to topics normally found in electromagnetism texts, I derive the initial value formulation for Maxwell's equations in that chapter. Electrodynamics in media is treated in chapter 6, including a discussion of magnetohydrodynamics. The geometric optics approximation to wave dynamics is presented in the first section of chapter 7, followed by a discussion of interference and coherence and an analysis of two problems in diffraction: scattering by a dielectric ball and the propagation of radiation through an aperture.

Special relativity is discussed in chapter 8. Special relativity underlies the fornulation of electromagnetic theory, so it really should be presented at the outset of a book on electromagnetism, rather than be relegated to a chapter near the end of the book. However, special relativity remains such an unfamiliar topic for most students that it is not feasible to do this. Many treatments of special relativity focus on the rules for applying Lorentz transformations to quantities, without providing much insight into the underlying geometrical content of the theory. In contrast, it would be natural for a general relativist like me to introduce more mathematical abstraction and geometrical machinery than would be strictly needed to provide a clear description of special relativity. I have put considerable care into writing section 8.1 in such a way that it introduces special relativity in a conceptually clear way without introducing more abstraction than I believe to be essential. This section can be read independently of the rest of the book, and I hope it will provide a useful introduction to special relativity on its own. The formulation of electromagnetism in the framework of special relativity is then given, followed by a discussion of charged particle motion and the radiation from a point charge in arbitrary motion.

Finally, the notion of a point charge is discussed in depth in chapter 10. It is shown that a mathematically well-defined limit of a charged body as it shrinks down to  zero size can be taken, provided that one also takes the charge and mass of the body to scale to zero proportionally to its size. Lorentz force motion is obtained in this limit. Self-force corrections can then be computed perturbatively in a mathematically rigorous manner. The issue of how to self-consistently describe the motion of a charged body taking the self-force corrections into account---without introducing spurious \quotes{runaway} solutions---is addressed in the final section of this chapter.

Throughout this book, I have attempted to formulate all key conceptual ideas and results in the theory of electromagnetism in a clear and concise manner. However, I have not attempted to present everything in this book with a high level of mathematical precision. Although I have made an effort to avoid getting sidetracked with unnecessary mathematical detail, I have not knowingly oversimplified any statements in the book and have tried to be careful to insert appropriate caveats when formulas or other results hold only under restricted conditions. In several instances in the early chapters, I have added boxed \quotes{side comments} to explain some mathematical points that may be of potential interest and relevance to the reader but are not strictly needed for the discussion.

An extensive collection of problems is provided for chapters 2-8. One purpose of these problems is the usual one of providing students with an opportunity to test their understanding of the basic concepts introduced in the chapter. However, there is an additional important purpose for some of the problems: to present topics that are not essential to the development of the core ideas of the book but are, nevertheless, of considerable interest and importance. Some examples of such topics treated in the problems are hidden momentum, the Hall effect, Gaussian beams, Thomson scattering, optical fibers, Stokes parameters, and Cherenkov radiation. I have written these problems in such a way that the key concepts are explained---and the key results are given---in the statement of the problem. Thus, a reader may find these problems to be a useful introduction to the topics.

The main audience that I have in mind for this book are graduate students in theoretical physics, although I hope that graduate students in experimental physics, undergraduates, and others will also find the book to be of interest. This book is written under the assumption that readers have had an introductory course in electromagnetism and thereby already have some intuition about electric and magnetic fields. I also assume the readers have a solid knowledge of vector calculus, but I do not assume much mathematical background beyond this.

I use SI units throughout the book. Unfortunately, SI units have the highly unpleasant feature of introducing two constants, $\epsilon_0$ and $\mu_0$, that satisfy the relation $\epsilon_0 \mu_0 c^2 = 1$, where $c$ is the speed of light. There are good historical reasons that this is the case. It is natural to assign an electric permittivity $\epsilon$ and a magnetic permeability $\mu$ to many materials, and it therefore was natural to assign corresponding values, $\epsilon_0$ and $\mu_0$, to the vacuum. It was then a truly great achievement of Maxwell to recognize that his equations implied that disturbances of the electric and magnetic fields in vacuum propagate with speed $c = 1/\sqrt{\epsilon_0 \mu_0}$ and that these disturbances could be identified with light. However, this relation between $\epsilon_0$, $\mu_0$ and $c$ means that there is a redundancy in these constants. Consequently, the appearance of formulas in SI units can be changed in nontrivial-looking ways by using this redundancy. For example, in SI units, one of the Maxwell equations is usually written as $\div \vb{E} = \rho/\epsilon_{0}$. However, this equation could equally well be written as $\div \vb{E} = \mu_0 c^2 \rho$. The latter form may seem rather jarring, as it seems to suggest that the magnetic permeability of the vacuum and the speed of light enter a basic equation of electrostatics. In any case, one must take a choice of which of these constants to use in any formula. The usual convention is to use $\epsilon_0$ in the above Maxwell equation and use $\mu_0$ in the Maxwell equation involving the current density $\vb{J}$. However, this convention cannot be maintained when one writes Maxwell's equations in special relativistically covariant form, since the 4-current $J^\mu$ enters these equations, and it makes no sense to use different conventions for different components of this 4-vector. Indeed, from chapter 8 on, I dispense entirely with $\epsilon_0$, $\mu_0$ and $c$ in all formulas. To avoid the unpleasantness associated with this redundancy of      
$\epsilon_0$, $\mu_0$ and $c$, I used Gaussian units in the original versions of my lecture notes. However, although Gaussian units were in quite prevalent use decades ago, SI units are used nearly universally now. Thus, the unpleasantness of SI units is outweighed by the unfamiliarity of students with Gaussian units---as well as the possibility that someone using my book may be led to purchase the wrong size of electromagnetic equipment if the formulas were written in Gaussian units. So, I have chosen to use SI units. 

Ordinary vectors in 3-dimensional space will be denoted in boldface (e.g., the electric field will be denoted as $\vb{E}$, as in the previous paragraph). Cartesian components of vectors will be denoted with Latin subscripts and without boldface symbols (e.g., $E_i$, with $i=1,2,3,$) denotes the components of $\vb{E}$ in a Cartesian basis). Beginning in chapter 8, I introduce the notion of spacetime vectors. For the reason explained in section 8.1, it then will be essential to explicitly introduce the notion of dual vectors and to distinguish clearly between vectors and dual vectors in our notation, wherein spacetime vectors are denoted with Greek superscripts (e.g., $W^\mu$) and spacetime dual vectors are denoted with Greek subscripts (e.g., $U_\mu$). Some additional special relativistic notational conventions are stated at the end of section 8.1.

I am greatly indebted to numerous colleagues for reading parts (and, in some cases, all) of the manuscript and providing me with valuable feedback. These include Sam Gralla, Abe Harte, Jim Isenberg, Istvan Racz, and Gautam Satishchandran, as well as numerous students who took my course. Among the latter, Tixuan Tan deserves special thanks for reading the manuscript with great care and asking many penetrating questions about the exposition. 