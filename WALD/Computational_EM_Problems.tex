%%%%%%%%%%%%%%%%%%%% book.tex %%%%%%%%%%%%%%%%%%%%%%%%%%%%%

\documentclass[english, 11pt, a4paper]{book}
%\usepackage[nomath]{lmodern}
\usepackage[T1]{fontenc}
\usepackage[italian]{babel}
% The following changes the Chapter heading from 'Chapter' to 'Lecture'
%$\addto\captionsenglish{\renewcommand{\chaptername}{Lecture}}
%$%\usepackage{fancyhdr}
%$\newcommand\chap[1]{%
%$ \chapter*{#1}%
%$  \addcontentsline{toc}{chapter}{#1}}
%$\newcommand\sect[1]{%
%$  \section*{#1}%
%$  \addcontentsline{toc}{section}{#1}} 
  
% choose options for [] as required from the list
% in the Reference Guide, Sect. 2.2

\usepackage{makeidx}         % allows index generation
\usepackage{graphicx}        % standard LaTeX graphics tool
\usepackage{subcaption}      % for subfigures environments 
                             % when including figure files
\usepackage{multicol}        % used for the two-column index
\usepackage[bottom]{footmisc}% places footnotes at page bottom
% etc.
% see the list of further useful packages
% in the Reference Guide, Sects. 2.3, 3.1-3.3
\usepackage[normalem]{ulem}

\usepackage[shortlabels]{enumitem}	% to be able to resume enumerated lists

\usepackage{amsmath}	% To be able to slash
\usepackage{bm}	        % To use bold greek letters in math mode with \bm{}
\usepackage{amsfonts}	% To be able to use \mathbb ... 
\usepackage{amssymb}	% To be able to use \nmid ... 
\usepackage{amsthm}		% \qed, \qedhere
\usepackage{slashed}	% any character (dirac)
\usepackage[title,toc,page]{appendix}

% See https://tex.stackexchange.com/questions/36524/how-to-put-a-framed-box-around-text-math-environment/36528
\usepackage{collectbox}	% To make box around formulas

% *** AFTER THIS LINE *** 
%     put \usepackage{} for shared packages kept under ~\Links\repos\git\LaTeX_Styles

% Physics package 
% https://tex.stackexchange.com/questions/38978/how-can-i-manually-install-a-latex-package-debian-ubuntu-linux
\usepackage[italicdiff]{/home/marcello/Links/repos/git/LaTeX_Styles/physics}	
% To put accents below letters
\usepackage{/home/marcello/Links/repos/git/LaTeX_Styles//accents}

% To control vertical white space above and below equations
% see https://tex.stackexchange.com/questions/69662/how-to-globally-change-the-spacing-around-equations
\expandafter\def\expandafter\normalsize\expandafter{%
    \normalsize
    \setlength\abovedisplayskip{16pt}
    \setlength\belowdisplayskip{16pt}
    \setlength\abovedisplayshortskip{16pt}
    \setlength\belowdisplayshortskip{16pt}
}

% To use extra symbols like dagger and double dagger in numbering footnotes 
\usepackage{footmisc}


% Force chapter numbering to restart within each part
\makeatletter
%\@addtoreset{chapter}{part}
\makeatletter

\makeindex             % used for the subject index
                       % please use the style svind.ist with
                       % your makeindex program


%%%%%%%%%%%%%%%%%%%%%%%%%%%%%%%%%%%%%%%%%%%%%%%%%%%%%%%%%%%%%%%%%%%%%

\begin{document}

\newcommand{\quotes}[1]{``#1''}
\newcommand{\sfT}{$\mathsf{T}$}
\newcommand{\udT}{\rotatebox[origin=c]{180}{$\mathsf{T}$}}

%Bold calligraphic letters 
\newcommand{\N}{\mathbb{N}}	% integers
\newcommand{\Z}{\mathbb{Z}}	% relative
\newcommand{\Q}{\mathbb{Q}}	% rationals
\newcommand{\R}{\mathbb{R}}	% reals
\newcommand{\C}{\mathbb{C}}	% complex
\newcommand{\F}{\mathbb{F}}	% generic field 1
\newcommand{\K}{\mathbb{K}}	% generic field 2
\newcommand{\V}{\mathbb{V}}	% Shankar's for vector space V

%Plain calligraphic letters 
\newcommand{\cC}{\mathcal{C}}    % space 1
\newcommand{\cF}{\mathcal{F}}    % space 2
\newcommand{\cS}{\mathcal{S}}    % space 3, Flow of energy (e.g in electromagnetism)
\newcommand{\cT}{\mathcal{T}}    % space 4 

\newcommand{\cU}{\mathcal{U}}    % sets 1
\newcommand{\cV}{\mathcal{V}}    % sets 2
\newcommand{\cW}{\mathcal{W}}    % sets 3
\newcommand{\cP}{\mathcal{P}}    % sets 4, Momentum density (e.g in electromagnetism) 
\newcommand{\cQ}{\mathcal{Q}}    % sets 5
\newcommand{\cR}{\mathcal{R}}    % sets 6

\newcommand{\cL}{\mathcal{L}}    % Lagrangian density
\newcommand{\cE}{\mathcal{E}}    % Energy density (e.g in electromagnetism)
\newcommand{\cY}{\mathcal{Y}}    % Y

% Calligraphic H for Hilbert space
\newcommand{\cH}{\mathcal{H}}    

% To show argument of the exponential function vertically, i.e., as a superscript 
\newcommand{\vexp}[1]{\,e^{#1}}

% To type an angle as a number of degrees like 45^\circ
\newcommand{\degree}[1]{{#1}^\circ}

% To create boldface vectors with a hat or check accent 
\newcommand{\hatvb}[1]{\vb{\hat{#1}}}
\newcommand{\chkvb}[1]{\vb{\check{#1}}}

% To create not-bold vectors with a hat or check accent 
\newcommand{\hatv}[1]{\hat{#1}}
\newcommand{\chkv}[1]{\check{#1}}

% To create <x|, |x> and <x|y> with unit vectors inside
\newcommand{\ubra}[1]{\bra*{\vu{#1}}}
\newcommand{\uket}[1]{\ket*{\vu{#1}}}
\newcommand{\uip}[2]{\ip*{\vu{#1}}{\vu{#2}}}

% To put accents below letters. 
\newcommand{\ut}[1]{\underaccent{\tilde}{#1}}
\newcommand{\uh}[1]{\underaccent{\hat}{#1}}
\newcommand{\form}[1]{\uh{#1}}

% To create italic, bold, bolditalic text
\newcommand{\tit}[1]{\textit{#1}}
\newcommand{\tbf}[1]{\textbf{#1}}
\newcommand{\tbi}[1]{\textit{\textbf{#1}}}

% Latin Modern sans serif |OR| Helvetica (SELECT)
\newcommand{\textlmss}{\fontfamily{lmss}\selectfont}
\newcommand{\texthv}{\fontfamily{phv}\selectfont}

% Latin Modern sans serif |OR| Helvetica (USE, within OR outside MATH !)
\newcommand{\tlmss}[1]{\text{\textlmss{#1}}}
\newcommand{\thv}[1]{\text{\texthv{#1}}}

% To use \tlmss{T} symbol to denote transpose 
\newcommand{\transp}[1]{{#1}^{\tlmss{T}}}

% To use \dagger symbol to denote operator Adjoint
\newcommand{\Adj}[1]{{#1}^\dagger}

% To denote the Hermitian conjugate with a '+' superscript
\newcommand{\Hconj}[1]{{#1}^{+}}

% To use \tlmss{Ker}, \tlmss{Coker} and \tlmss{Img} to denote Kernel, Co-Kernel & Image 
\newcommand{\Ker}{\tlmss{Ker}\,}
\newcommand{\Coker}{\tlmss{Coker}\,}
\newcommand{\Img}{\tlmss{Im}\,}

% To use \tlmss{Alt} and \tlmss{alt} to denote alternation 
\newcommand{\Alt}{\tlmss{Alt}\,}
\newcommand{\alt}{\tlmss{alt}\,}

% To use \tlmss{Ann} to denote annulets 
\newcommand{\Ann}{\tlmss{Ann}\,}

% Misc abbreviations
\newcommand{\ora}[1]{\overrightarrow{#1}}

\DeclareRobustCommand{\rchi}{{\mathpalette\irchi\relax}}
\newcommand{\irchi}[2]{\raisebox{\depth}{$#1\chi$}} % inner command, used by \rchi

% See https://tex.stackexchange.com/questions/36524/how-to-put-a-framed-box-around-text-math-environment/36528
\makeatletter
\newcommand{\mybox}{%
    \collectbox{%
        \setlength{\fboxsep}{1pt}%
        \fbox{\BOXCONTENT}%
    }%
}
\makeatother

\author{Marcello Vitaletti}
\title{Problemi Computazionali\\ nell'Elettromagnetismo Classico\\
\large{con riferimenti ad R. M. Wald\\Advanced Classical Electromagnetism, 2022.}}
\maketitle

\frontmatter%%%%%%%%%%%%%%%%%%%%%%%%%%%%%%%%%%%%%%%%%%%%%%%%%%%%%%

%\include{dedic}

%\chapter*{Plan}
\label{plan} 

In this book I am keeping notes about the theory of classical electromagnetism, 
as exposed in various books. In particular, I intend to cover the following materials:

\begin{itemize}

\item B. Felsager -- Geometry Particles and Fields
\begin{enumerate}
\setcounter{enumi}{0}
\item Electromagnetism (1.1 to 1.4)
\end{enumerate}

\item C. Cattaneo -- Teoria Einsteniana della Gravitazione
\begin{enumerate}
\setcounter{enumi}{0}
\item Elementi di Algebra e Analisi Lineare
\end{enumerate}

\item D.J. Griffiths -- Introduction to Electrodynamics
\begin{enumerate}
\setcounter{enumi}{0}
\item Vector Analysis
\item Electrostatics
\item Potentials
\item Electric Fields in Matter
\item Magnetostatics
\item Magnetic Fields in Matter
\item Electrodynamics
\item Conservation Laws
\item Electromagnetic Waves
\item Radiation
\item Electrodynamics and Relativity
\item Potentials and Fields
\item Helmoltz Theorem
\end{enumerate}

\item J.D. Jackson -- Classical Electrodynamics, 2nd Edition
\begin{enumerate}
\setcounter{enumi}{0}
\item Introduction to Electrostatics
\item Boundary Value Problems in Electrostatics - I
\item Boundary Value Problems in Electrostatics - II
\item Multipoles, Electrostatics of Macroscopic Media, Dielectrics
%\item Magnetostatics
%\item Time Varying Fields, Maxwell Equations, Conservation Laws
%\item Plane Electromagnetic Waves and Wave Propagation
%\item Wave Guides and Resonant Cavities
%\item Simple Radiating Systems, Scattering and Diffraction
%\item Magnetohydrodynamics and Plasma Physics
\end{enumerate}

\item J.D. Jackson -- Classical Electrodynamics, 3rd Edition
\begin{enumerate}
\setcounter{enumi}{4}
\item Magnetostatics, Faraday's Law, Quasi-Static Fields
\item Maxwell Equations, Macroscopic Electromagnetism, Conservation Laws
\item Plane Electromagnetic Waves and Wave Propagation
\item Wave Guides, Resonant Cavities and Optical Fibers
\item Radiating Systems, Multipole Fields and Radiation
\item Scattering and Diffraction
\item Special Theory of Relativity
\item Dynamics of Relativistic Particles and Electromagnetic Fields
\end{enumerate}

\item B. Felsager -- Geometry Particles and Fields
% Contacts with quantum theory of particles dynamics in EM fields
\begin{enumerate}
\setcounter{enumi}{1}
\item Interaction of Fields and Particles
\end{enumerate}

\item J. Franklin -- Advanced Mechanics and General Relativity
\begin{enumerate}
\setcounter{enumi}{1}
\item Relativistic Mechanics
\item Tensors
\item Curved Space
\item Scalar Field Theory
\item Tensor Field Theory (6.1 to 6.5)
\end{enumerate}

\item J.D. Jackson -- Classical Electrodynamics, 3rd Edition
\begin{enumerate}
\setcounter{enumi}{12}
\item Collisions, Energy Loss and Scattering of Charged Particles, Cherenkov and Transition Radiation
\item Radiation by Moving Charges
\item Bremsstrahlung, Method of Virtual Quanta, Radiative Beta Processes
\item Radiation Damping, Classical Models of Charged Particles
\end{enumerate}

\item B. Felsager -- Geometry Particles and Fields
% Contacts with quantum theory of fields dynamics + differential geometry math
\begin{enumerate}
\setcounter{enumi}{2}
\item Dynamics of Classical Fields
\end{enumerate}

\begin{enumerate}
\setcounter{enumi}{5}
\item Differentiable Manifolds, Tensor analysis
\item Differential Forms, Exterior Calculus
\item Integral Calculus on Manifolds
\end{enumerate}

\item C.W. Misner, K.S. Thorne, J.A. Wheeler -- Gravitation
\begin{enumerate}
\setcounter{enumi}{1}
\item Foundations of Special Relativity
\item The Electromagnetic Field
\item Electromagnetism and Differential Forms
\end{enumerate}

\item L.D. Landau, E.M. Lifshitz -- Teoria dei Campi
\begin{enumerate}
\setcounter{enumi}{0}
\item Principio di Relatività
\item Meccanica Relativistica
\item Carica in un Campo Elettromagnetico
\item Equazioni del Campo Elettromagnetico
\item Campo Elettromagnetico Costante
\item Onde Elettromagnetiche
\item Propagazione della Luce
\item Campo di Cariche in Moto
\item Radiazione Elettromagnetica
\end{enumerate}

\item L.D. Landau, E.M. Lifshitz -- Elettrodinamica dei Mezzi Continui
\begin{enumerate}
\setcounter{enumi}{0}
\item Elettrostatica dei Conduttori
\item Elettrostatica nei Dielettrici
\item Corrente Continua
\item Campo Magnetico Costante
\item Ferromgnetismo e Antiferromagnetismo
\item Superconduttività
\item Campo Magnetico Quasi Stazionario
\item Idrodinamica Magnetica
\item Equazioni delle Onde Elettromagnetiche
\item Propagazione delle Onde Elettromagnetiche
\item Onde Elettromagnetiche in Mezzi Anisotropi
\item Dispersione Spaziale
\item Ottica non Lineare
\item Passaggio delle Particelle Veloci attraverso la Materia
\item Diffusione delle Onde Elettromagnetiche
\item Diffrazione dei Raggi X nei Cristalli
\end{enumerate}

\end{itemize}
	
\tableofcontents
%\addappheadtotoc

\mainmatter%%%%%%%%%%%%%%%%%%%%%%%%%%%%%%%%%%%%%%%%%%%%%%%%%%%%%%%
\chapter*{Evoluzione spazio-temporale del campo Elettromagnetico}\label{EM_P_01}

\section*{1.1 Premesse}\label{sec_1.1}
L'elettromagnetismo classico si basa sulle equazioni di Maxwell
\begin{align}
\div{\vb*{E}}  &= \frac{\rho}{\epsilon_0}\,, \label{eq:1.1} \\
\curl{\vb*{B}} - \frac{1}{c^2} \pdv{\vb*{E}}{t} &= \mu_0 \vb*{J}\,,\label{eq:1.2} \\
\div{\vb*{B}}  &= 0\,, \label{eq:1.3} \\
\curl{\vb*{E}} + \pdv{\vb*{B}}{t} &= 0\,.\label{eq:1.4}
\end{align}

ove le sorgenti $\rho$ e $\vb*{J}$ debbono soddisfare l'equazione di conservazione carica-corrente
\begin{equation}\label{eq:1.5}
\pdv{\rho}{t} + \div{\vb*{J}} = 0\,,
\end{equation}

Si assumano specificati nello spaziotempo $\rho(t, \vb*{x})$ e $\vb*{J}(t, \vb*{x})$, per i quali sia soddisfatta l'equazione di conservazione (\ref{eq:1.5}). 

Siano $\vb*{E}_0(\vb*{x})$ e $\vb*{B}_0(\vb*{x})$ campi vettoriali arbitrari nello spazio tali che 
\begin{align}
\div{\vb*{E}_0} &= \rho(t=0, \vb*{x})/\epsilon_0\,, \textrm{ e} \\ 
\div{\vb*{B}_0} &= 0\,. 
\end{align}

Allora esiste un'unica soluzione $\vb*{E}(t, \vb*{x}), \vb*{B}(t, \vb*{x})$ delle equazioni di \\Maxwell (\ref{eq:1.1})-(\ref{eq:1.4}) tale che
\begin{align}
\vb*{E}(t=0, \vb*{x}) &= \vb*{E}_0(\vb*{x})\,, \textrm{ e} \label{eq:1.6}\\ 
\vb*{B}(t=0, \vb*{x}) &= \vb*{B}_0(\vb*{x})\,. \label{eq:1.7}
\end{align}


Esistono tante soluzioni delle equazioni di Maxwell con 
$\rho$ e $\vb*{J}$ fissati, quante sono le possibili scelte dei campi $\vb*{E}_0(\vb*{x}), \vb*{B}_0(\vb*{x})$ la cui divergenza soddisfa le condizioni di cui sopra. Il fatto che questa informazione iniziale del campo elettromagnetico possa essere specificata liberamente dimostra che il campo elettromagnetico ha i suoi propri gradi di libertà dinamici indipendenti: la soluzione delle equazioni di Maxwell \tit{non è determinata} da $\rho$ e $\vb*{J}$.


\section*{1.2 Analisi del problema}\label{sec_1.2}

Le premesse esposte nella precedente sezione \ref{sec_1.1} indicano la possibilità di determinare l'evoluzione spazio-temporale del campo elettromagnetico nell'assunzione che siano in qualche modo note -- perché indipendentemente imposte dall'esterno -- le densità della carica $\rho$ e della corrente $\vb*{J}$ in una regione di spazio $\Omega$ per un certo intervallo di tempo $T$. Assumiamo che la regione $\Omega$ sia chiusa, connessa e limitata e che le densità di carica e corrente siano nulle all'esterno di $\Omega$: in pratica sarà sufficiente che siano nulle entro una distanza $c T$ dalla superficie $\partial \Omega$ di $\Omega$.

Per fissare le idee, consideriamo una regione di \tit{spazio vuoto} $\Omega$ racchiusa da una sfera di raggio $R$, assumendo che carica e corrente siano nulle ovunque, tranne che sulla superficie $\partial \Omega$ della sfera. 

Per quanto esposto nella precedente sezione 1.1, possiamo considerare assegnate a priori la densità di carica $\rho(t; \theta, \phi)$ e $\vb*{J}(t; \theta, \phi)$ per $t \in [0:T]$, ove $\theta$ 
e $\phi$ sono le coordinate sferiche di un punto sulla superficie, purché i valori assegnati siano consistenti con la legge di \tit{conservazione locale} della carica, espressa dall'equazione di continuità carica-corrente (\ref{eq:1.5}).

Nella sezione 1.3 analizziamo modi per garantire il requisito di continuità.

Una maggior varietà di problemi può essere generata considerando una regione di \tit{spazio vuoto} $\Omega$ compresa tra due superfici chiuse e connesse della quali una è interamente contenuta nell'altra. Il caso forse più semplice è quello in cui la frontiera $\partial \Omega$ di $\Omega$ è costituita dall'unione delle frontiere (disgiunte) $\partial \Omega_1$ ed $\partial \Omega_2$ di due sfere\footnote{Non necessariamente \tit{concentriche}.} di raggio $R_1$ ed $R_2$ ove $R_1 < R_2$. In tal caso si assumerà che carica e corrente siano nulle ovunque, tranne che sulle superfici $\partial \Omega_1$ e $\partial \Omega_2$. Si noti che in tal caso la carica totale su $\partial \Omega_1$ e la carica totale su $\partial \Omega_2$ \tit{potranno} essere separatamente assegnate e \tit{dovranno} essere separatamente conservate. 
\pagebreak
\section*{1.3 Evoluzione consistente del campo carica-corrente}\label{sec_1.3}
\begin{enumerate}[(I)]
\item \tbi{METODO I}  
%-----------------------------------------------------------------------------
	%-------------------------------------------------------------------------
	\begin{enumerate}[(a)]
	\item Definisci una distribuzione di carica totale $Q$ sulla superficie $S$ con un metodo qualsiasi. Esempio: 6 \tit{poli} di carica \tit{positiva} nelle direzioni corrispondenti alle 6 facce di un cubo inscritto nella sfera, e 8 \tit{poli} di carica \tit{negativa} nelle direzioni corrispondenti agli 8 spigoli del medesimo cubo.  
	\item Definisci il movimento di un vettore unitario $\vu{n}(t)$ che applicato al centro della sfera definisce implicitamente una \tit{curva} $(\theta(t), \phi(t))$ nello spazio tridimensionale; 
	\item Scegli la velocità angolare\footnote{Nel caso più semplice questa velocità angolare avrà un valore costante oppure, ad esempio, il suo valore potrebbe oscillare periodicamente tra due valori.} $\omega$ con cui ruotare la distribuzione di carica attorno all'asse che contiene il centro della sfera ed il punto di coordinate sferiche 
$(\theta(t), \phi(t))$;
	\item Ad ogni istante $t_i \geq 0$, ottieni la distribuzione di carica all'istante $t_{i+1} = t_i + \Delta t$ in base al cambio di direzione dell'asse di rotazione (da $\vu{n}(t_i)$ a $\vu{n}(t_{i+1})$) e alla velocità angolare $\omega$ con cui la distribuzione di carica ruota attorno all'asse;
	\item Calcola $\vb*{J}(t_{i+1}; \theta(t_{i+1}), \phi(t_{i+1}) )$  in base al movimento dell'asse di rotazione $\vu{n}(t)$ e alla velocità di rotazione $\omega$ attorno ad esso.
	\end{enumerate}
	%-------------------------------------------------------------------------
%-----------------------------------------------------------------------------
\item \tbi{METODO II} 
%-----------------------------------------------------------------------------
	%-------------------------------------------------------------------------
	\begin{enumerate}[(a)]
	\item Definisci una distribuzione di carica totale $Q$ sulla superficie $S$ in termini di un  suo sviluppo in armoniche sferiche ortogonali, che coinvolga $k$ armoniche, con $k \geq 2$;  
	\item Varia nel tempo in modo continuo i coefficienti $c_k(t)$ dello sviluppo partendo da un set \tit{iniziale} $c^{i}_1, \ldots, c^{i}_k$ per arrivare ad un set \tit{finale} $c^{f}_1, \ldots, c^{f}_k$ in modo che in ciascun punto la derivata temporale della densità sia contenuta entro un limite predefinito;
	\item Utilizza un modello discreto della superficie sferica per determinare\footnote{Questo potrebbe richiedere la soluzione di un sistema lineare sparso di grandi dimensioni.} numericamente $\vb*{J}(t_k)$.
	\end{enumerate}
	%-------------------------------------------------------------------------
%-----------------------------------------------------------------------------
\end{enumerate}
\pagebreak
\section*{1.4 Densità di carica $\rho$ e di corrente $\vb*{J}$}\label{sec_1.4}
Per quanto esposto nella precedente sezione 1.1, possiamo assegnare i valori iniziali del campo elettromagnetico, cioè dei campi vettoriali $\vb*{E}_0 = \vb*{E}(0; \vb*{x})$ e $\vb*{B}_0 = \vb*{B}(0; \vb*{x})$, con l'unico vincolo che valgano le equazioni (\ref{eq:1.6}) e (\ref{eq:1.7}), cioè che il campo magnetico iniziale $\vb*{B}_0$ abbia ovunque divergenza nulla e che la divergenza del campo elettrico $\vb*{E}_0 = \rho(0; \vb*{x})/\epsilon_0$ sia ovunque determinata dalla densità di carica. 

Nella sezione 1.5 introduciamo una possibile \tit{discretizzazione} della regione spazio-temporale $\Omega$ e della sua frontiera $\partial \Omega$ con particolare riferimento al caso in cui $\Omega$ è la porzione di spazio racchiusa tra 2 \tit{sfere concentriche}\footnote{La scelta è dettata dalla facilità con cui in questo caso si può discretizzare la regione $\Omega$: basta collegare con segmenti \tit{radiali} coppie di punti corrispondenti di cui il primo si trova nella superficie sferica più interna ed il secondo si trova su quella più esterna.}, la frontiera $\partial \Omega$ essendo costituita dall'unione delle 2 superfici sferiche. 

Nella sezione 1.6 analizziamo come si possano definire valori iniziali del campo elettromagnetico consistenti con i vincoli imposti alla divergenza di $\vb*{E}_0$ e $\vb*{B}_0$.

Nella sezione 1.7 discutiamo una versione \tit{discretizzata} del problema ai valori iniziali, ed analizziamo come si possano calcolare i valori del campo elettromagnetico su una griglia di punti ad un certo istante $t+\Delta t$ posto che siano noti i valori del campo e delle \tit{sorgenti} ($\rho$ e $\vb*{J}$) al tempo $t$. 
 

\section*{1.5 Discretizzazione di $\Omega$ e della sua frontiera}\label{sec_1.5}


\section*{1.6 Valori iniziali del campo elettromagnetico}\label{sec_1.6}


\section*{1.7 Evoluzione temporale del campo discretizzato}\label{sec_1.6}



\appendixpage
\appendix
\chapter{Metodi computazionali nel Metodo I  della sez. 1.3}\label{app:P_01_A}

Come punto di partenza nella implementazione del metodo è necessario suddividere la superficie $\vb*{S}$ di una sfera mediante un insieme di triangoli sferici che la ricoprano completamente senza sovrapporsi. I \tit{vertici} di tali triangoli definiscono una \tit{mesh} di punti sulla sfera. Una mesh con ottime caratteristiche di uniformità può essere costruita per suddivisione ricorsiva\footnote{La procedura ricorsiva consiste nel suddividere ciascun triangolo in $4$ sub-triangoli, introducenti come nuovi vertici i punti che dividono a metà ciascuno degli archi di cerchio comuni a due triangoli sferici adiacenti.} dei $20$ triangoli implicitamente definiti da un \tit{icosaedro regolare} inscritto nella sfera. 

L'obiettivo del Metodo I è definire l'evoluzione di un campo di \tit{densità} di carica la cui dinamica soddisfi non solo il vincolo della conservazione della carica totale $Q$ (assumeremo generalmente che la carica totale sia nulla) ma soprattutto il vincolo della \tit{conservazione locale} cioè la cosiddetta equazione di continuità: la differenza tra la carica totale di carica racchiusa da una qualsiasi porzione chiusa di superficie $\vb*{\omega}$ in un intervallo di tempo $\Delta t$ deve coincidere con il \tit{flusso} di corrente che attraversa la frontiera $\partial \vb*{\sigma}$ di  $\vb*{\omega}$.  

\section*{A.1 Titolo A1}\label{sec:A.1}

\section*{A.2 Titolo A2}\label{sec:A.2}

\section*{A.3 Titolo A3}\label{sec:A.3}




\backmatter%%%%%%%%%%%%%%%%%%%%%%%%%%%%%%%%%%%%%%%%%%%%%%%%%%%%%%%
%%%%%%%%%%%%%%%%%%%%%%%%% referenc.tex %%%%%%%%%%%%%%%%%%%%%%%%%%%%%%
% sample references
% 
% Use this file as a template for your own input.
%
%%%%%%%%%%%%%%%%%%%%%%%% Springer-Verlag %%%%%%%%%%%%%%%%%%%%%%%%%%

%
% BibTeX users please use
% \bibliographystyle{}
% \bibliography{}
%
% Non-BibTeX users please use
\begin{thebibliography}{99.}
%
% and use \bibitem to create references.
%
% Use the following syntax and markup for your references
%
% Monograph
\bibitem{Griffiths_4th} D.J. Griffiths (2017)
Introduction to Electrodynamics. Cambridge University Press, Cambridge

% Monograph
\bibitem{Felsager_1981} B. Felsager (1981)
Geometry, Particles and Fields. Odense University Press

% Monograph
\bibitem{BudakFomin_1973} B.M. Budak, S.V. Fomin (1973)
Multiple Integrals, Field Theory and Series. Mir Publishers, Moscow

% Monograph
\bibitem{Postnikov_II_1982} Mikhail Postnikov (1982)
Lectures in Geometry, Semester II. Linear Algebra and Differential Geometry. Mir Publishers, Moscow

\end{thebibliography}

%\printindex

%%%%%%%%%%%%%%%%%%%%%%%%%%%%%%%%%%%%%%%%%%%%%%%%%%%%%%%%%%%%%%%%%%%%%%

\end{document}





