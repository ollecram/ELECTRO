\setcounter{chapter}{0}
\renewcommand{\thechapter}{5}
\chapter{Electrodynamics}\label{ch:5}
\setcounter{equation}{0}	        % To start with Equation 1
\counterwithin{equation}{chapter}	% Equation numbering will be 2.1 2.2 2.3   ... 

In this chapter, we consider the general case where $\phi$, $\vb{A}$ and $\rho$, $\vb{J}$ are time dependent. The equations of electrodynamics are discussed in section \ref{sec:5.1}, and the Lorenz gauge is introduced. We solve for the retarded Green's function in section \ref{sec:5.2}, from which the solution to Maxwell's equations for general $\rho$, $\vb{J}$ with no incoming radiation can be obtained. Multipole expansions of the electromagnetic field of the retarded solution are obtained in section \ref{sec:5.3}. The retarded Green's function is then used in section \ref{sec:5.4} to obtain the solution to Maxwell's equations with prescribed values of the electromagnetic field at an initial time. We discuss plane wave solutions in section \ref{sec:5.5}. The chapter concludes with a discussion of electrodynamics in conducting cavities and waveguides in section \ref{sec:5.6}.

\section{The Equations of Electrodynamics}\label{sec:5.1}
The equations of electrodynamics have been presented in chapter 1. As discussed in section \ref{sec:1.1}, the fundamental dynamical variables are the potentials $\phi$ and $\vb{A}$, which are considered to be equivalent if and only if they differ by a gauge transformation:
\begin{equation}\label{eq:5.1}
\phi \rightarrow \phi' = \phi - \dv{\chi}{t}, \quad \quad \vb{A} \rightarrow \vb{A'} = \vb{A} + \grad \chi\:.
\end{equation}

% BOX 1 **START** -- << Maxwell equations in Gaussian units >>

\parindent=0pt  % Set the paragraph indentation to 0 (normal = 10pt) before the box 
\parbox{\textwidth}{\begin{mdframed}[style=MyFrame] %Added "\parbox{\textwidth}{"
%\lipsum[1]
\subsubsection*{BOX 5.1 -- Gauge invariance in Gaussian units}\label{box:8.1}
\begin{equation*}
\phi \rightarrow \phi' = \phi - \frac{1}{c}\dv{\chi}{t}, \quad \quad 
\vb{A} \rightarrow \vb{A'} = \vb{A} + \grad \chi\;.\quad \quad\quad \quad [G5.1]
\end{equation*}
\end{mdframed}} %Added a closing "}" here
\parindent=10pt % Set the paragraph indentation to normal (10pt) 
% BOX 1 ** END ** -- << Maxwell equations in Gaussian units >>


The equations of electrodynamics (i.e. Maxwell's equations) are
\begin{align}
\vb{E} &= - \grad \phi - \pdv{\vb{A}}{t}						\,,\label{eq:5.2}\\
\vb{B} &= \curl{\vb{A}}                  						\,,\label{eq:5.3}\\
\div{\vb{E}} &= \frac{\rho}{\epsilon_0} 						\,,\label{eq:5.4}\\
\curl{\vb{B}} &= \frac{1}{c^2} \pdv{\vb{E}}{t} + \mu_0 \vb{J}	\,.\label{eq:5.5}
\end{align}
The first two of these equations define $\vb{E}$ and $\vb{B}$ in terms of $\phi$ and $\vb{A}$. They imply that 
\begin{align}
\div{\vb{B}} = 0                                                \,,\label{eq:5.6}\\
\curl{\vb{E}} + \pdv{\vb{B}}{t}= 0                              \,,\label{eq:5.7}
\end{align}

% BOX 2 **START** -- << Maxwell equations in Gaussian units >>

\parindent=0pt  % Set the paragraph indentation to 0 (normal = 10pt) before the box 
\parbox{\textwidth}{\begin{mdframed}[style=MyFrame] %Added "\parbox{\textwidth}{"
%\lipsum[1]
\subsubsection*{BOX 5.2 -- Maxwell equations in Gaussian units}\label{box:8.2}
\begin{align*}
\vb{E} &= - \grad \phi - \frac{1}{c}\pdv{\vb{A}}{t}\,,\quad\quad\quad &[G5.2]\\
\vb{B} &= \curl{\vb{A}}\,,&[G5.3]\\
\div{\vb{E}} &= 4 \pi \rho\,,&[G5.4]\\
\curl{\vb{B}} &= \frac{1}{c} \pdv{\vb{E}}{t} + \frac{4\pi}{c} \vb{J}\,,&[G5.5]\\
\div{\vb{B}} &= 0\,,&[G5.6]\\
\curl{\vb{E}} + \frac{1}{c} \pdv{\vb{B}}{t }&= 0\,.&[G5.7]\\
\end{align*}
\end{mdframed}} %Added a closing "}" here
\parindent=10pt % Set the paragraph indentation to normal (10pt) 
% BOX 2 ** END ** -- << Maxwell equations in Gaussian units >>

Equations (\ref{eq:5.6}) and (\ref{eq:5.7}) are equivalent to eqs. (\ref{eq:5.2}) and (\ref{eq:5.3}) in a topologically trivial region\footnote{More precisely, by a \quotes{topologically trivial region} in the present context, I mean a region in which any closed 2-dimensional surface $S$ is the boundary od a 3-dimensional (compact) volume, and every closed loop $\mathscr{C}$ is the boundary of a 2-dimensional (compact) surface. The necessary and sufficient condition for the existence of a vector potential $\vb{A}$ such that $\vb{B} = \curl{\vb{A}}$ is that $\int_S \vb{B} \cdot \vu{n} = 0$ for any closed 2-dimensional surface $S$. The necessary and sufficient condition for the existence of a scalar potential $\phi$ satisfying eq. (\ref{eq:5.2}) is $\int_{\mathscr{C}} [\vb{E} + \pdv*{\vb{A}}{t}] \cdot \dd{\vb{l} = 0}$ for any closed loop $\mathscr{C}$. If eqs. (\ref{eq:5.6}) and (\ref{eq:5.7}) hold in a topologically trivial region, then the necessary and sufficient condition for the existence of $\vb{A}$ will automatically hold by Gauss's theorem, and the necessary and sufficient condition for the existence of $\phi$ will automatically hold by Stokes's theorem.} 
(i.e., if eqs. (\ref{eq:5.6}) and (\ref{eq:5.7}) hold in a topologically trivial region, they imply the existence of potentials $\phi$ and $\vb{A}$ in that region and satisfying eqs. (\ref{eq:5.2}) and (\ref{eq:5.3})). 
Taking the time derivative of eq. (\ref{eq:5.4}) and adding it to $c^2$ times the divergence of eq. (\ref{eq:5.4})---using the fact that $c^2 = \flatfrac{1}{\epsilon_0 \mu_0}$---we obtain the charge-current conservation law:
\begin{equation}\label{eq:5.8}
\pdv{\rho}{t} + \div{\vb{J}} = 0\,.
\end{equation}
It is worth noting that in the source-free case (i.e., when $\rho = \vb{J} = 0$), except for one sign difference and factors of $c$, eqs. (\ref{eq:5.6}) and (\ref{eq:5.7}) take the same form as eqs. (\ref{eq:5.4}) and (\ref{eq:5.5}) with $\vb{E}$ and $\vb{B}$ interchanged. It follows immediately that in any source-free, topologically trivial region, Maxwell's equations are invariant under a \tit{duality transformation}:
\begin{equation}\label{eq:5.9}
\vb{E} \rightarrow c \vb{B}\,, \quad\quad c \vb{B} \rightarrow \vb{E}\,,
\end{equation}
that is, if $\vb{E}, \vb{B}$ solve eqs. (\ref{eq:5.4})-(\ref{eq:5.7}) with  $\rho = 0$, $\vb{J} = 0$, then so do 
$\vb{E'}, \vb{B'}$ with  $\vb{E'},= c \vb{B}$, $\vb{B'} = - \flatfrac{\vb{E}}{c}$. More generally, for any real number $\alpha$, the source-free Maxwell equations (\ref{eq:5.4})-(\ref{eq:5.7}) are invariant under the \tit{duality rotation}:
\begin{equation}\label{eq:5.10}
\vb{E} \rightarrow \cos\alpha \vb{E} + \sin\alpha (c \vb{B})\,, \quad\quad \vb{B} \rightarrow \cos\alpha \vb{B} - \sin\alpha ( \vb{E}/c)\,,
\end{equation}
As already discussed in section (\ref{sec:1.2}), the energy density, momentum density, and stress tensor of the electromagnetic field are, respectively,
\begin{equation}\label{eq:5.11}
\mathscr{E} = \frac{1}{2} \left(\epsilon_0 \abs{\vb{E}}^2 + \frac{1}{\mu_0} \abs{\vb{B}}^2 \right)\,,  
\end{equation}
\begin{equation}\label{eq:5.12}
\pmb{\mathscr{P}} = \epsilon_0 \vb{E} \cross \vb{B}\,,
\end{equation}
\begin{equation}\label{eq:5.13}
\Theta_{ij} = \epsilon_{0} E_i E_j + \frac{1}{\mu_0} B_i B_j  -\frac{1}{2} \delta_{ij} \left(\epsilon_0 \abs{\vb{E}}^2 + \frac{1}{\mu_0} \abs{\vb{B}}^2 \right)   \,.
\end{equation}

The total energy and momentum of the electromagnetic field are obtained by integrating eq. (\ref{eq:5.11}) and eq. (\ref{eq:5.12}) over all space.

% BOX 3 **START** -- << Energy, Momentum and Stress Tensor in Gaussian units >>

\parindent=0pt  % Set the paragraph indentation to 0 (normal = 10pt) before the box 
\parbox{\textwidth}{\begin{mdframed}[style=MyFrame] %Added "\parbox{\textwidth}{"
%\lipsum[1]
\subsubsection*{BOX 5.3 -- Energy, Momentum and Stress in Gaussian units}\label{box:8.3}
\begin{align*}
\mathscr{E} &= \frac{1}{8 \pi} \left(\abs{\vb{E}}^2 + \abs{\vb{B}}^2 \right)\,,\quad\quad\quad &[G5.11]\\ 
\pmb{\mathscr{P}} &= \frac{c}{4 \pi} \vb{E} \cross \vb{B}\,, &[G5.12]\\
\Theta_{ij} &= \mqty( \mathscr{E} &\pmb{\mathscr{P}}_x/c &\pmb{\mathscr{P}}_y/c &\pmb{\mathscr{P}}_y/c\\ 
                      \pmb{\mathscr{P}}_x/c &\sigma_{xx} &\sigma_{xy} &\sigma_{xz}\\
                      \pmb{\mathscr{P}}_y/c &\sigma_{yx} &\sigma_{yy} &\sigma_{yz}\\
                      \pmb{\mathscr{P}}_z/c &\sigma_{zx} &\sigma_{zy} &\sigma_{zz})\,, &[G5.13]
\end{align*}
where $\sigma_{\alpha \beta}\:\: (\alpha, \beta = x, y, z)$  is the flux of momentum \tit{component} 
$\pmb{\mathscr{P}}_\alpha$ flowing in unit time across a surface element of unit area\\ perpendicular to the $\beta$  axis.
\end{mdframed}} %Added a closing "}" here
\parindent=10pt % Set the paragraph indentation to normal (10pt) 
% BOX 2 ** END ** -- << Energy, Momentum and Stress Tensor in Gaussian units >>




In special relativity, there is no distinction between a \quotes{flow of mass} (i.e., momentum) and a \quotes{flow of energy}---apart from a factor of $c^2$ needed to give these quantities conventional units. Thus, $c^2$ times the momentum density of the electromagnetic field gives the electromagnetic energy flux,  
\begin{equation}\label{eq:5.14}
\pmb{\mathscr{S}} = c^2 \pmb{\mathscr{P}} = c^2 \epsilon_0 \vb{E} \cross \vb{B} = \frac{1}{\mu_0} \vb{E} \cross \vb{B}\,,
\end{equation}
where, again, we use the fact that $c^2 = 1/(\epsilon_0 \mu_0)$. $\pmb{\mathscr{S}}$ is known as the \tit{Poynting vector}. For a surface $S$, the flux $\mathscr{F}$ of electromagnetic energy through $S$ (i.e., the electromagnetic energy that flows through $S$ per unit area per unit time) is given by 
\begin{equation}\label{eq:5.15}
\mathscr{F} =\pmb{\mathscr{S}} \cdot \vu{n}\,,
\end{equation}

where $\vu{n}$ is the unit normal to $S$.  The flow of electromagnetic energy out of a volume $\mathscr{V}$ bounded by surface $S$ is thus given by
\begin{equation}\label{eq:5.16}
\int_{S} \mathscr{F} \dd{S} = \int_{S} \pmb{\mathscr{S}} \cdot \vu{n}  \dd{S} = \int_{\mathscr{V}} \div{\pmb{\mathscr{S}}} \dd{V} \,.
\end{equation}

If electromagnetic energy were conserved by itself, we would have 
$\dv*{}{t} \int_{\mathscr{V}} \mathscr{E} \dd{V} = \int_{\mathscr{V}} \pdv*{\mathscr{E}}{t} \dd{V} = - \int_{\mathscr{V}} \div{\pmb{\mathscr{S}}} \dd{V}$, and therefore (since $\mathscr{V}$ is arbitrary), $\pdv*{\mathscr{E}}{t} = - \div{\pmb{\mathscr{S}}}$. However, a direct computation yields
\begin{equation}\label{eq:5.17}
\begin{aligned}
\pdv{\mathscr{E}}{t} &= 
\epsilon_0 \vb{E} \cdot \pdv{\vb{E}}{t} + \frac{1}{\mu_0} \vb{B} \cdot \pdv{\vb{B}}{t} + \frac{1}{\mu_0} \div{(\vb{E} \cross \vb{B})} \\
&= \epsilon_0 \vb{E} \cdot  ( c^2 \curl{\vb{B}} - c^2 \mu_0 \vb{J} ) + \frac{1}{\mu_0} \vb{B}  \cdot (- \curl{\vb{E}}) + \frac{1}{\mu_0} \div{(\vb{E} \cross \vb{B})} \\
&= - \vb{E} \cdot \vb{J}  + \frac{1}{\mu_0} [\vb{E} \cdot (\curl{\vb{B}})  - \vb{B} \cdot (\curl{\vb{E}}) + \div{(\vb{E} \cross \vb{B})}]    \\
&= - \vb{E} \cdot \vb{J}\,,
\end{aligned}
\end{equation}
where eqs. (\ref{eq:5.6}) and (\ref{eq:5.7}) were used in the second line, the relation $c^2 \epsilon_0 \mu_0 = 1$ was used in the third line, and the identity 
\begin{equation}\label{eq:5.18}
\begin{aligned}
\div{(\vb{E} \cross \vb{B})} &=  \sum_{i,j,k} \epsilon_{ijk}\partial_i (E_j B_k) = 
\sum_{i,j,k} \epsilon_{ijk} \left[(\partial_i E_j) B_k + E_j (\partial_i B_k)) \right]\\ 
                           &= \vb{B} \cdot (\curl{\vb{E}}) - \vb{E} \cdot (\curl{\vb{B}}) 
\end{aligned}
\end{equation}
was used to get the last line. This means that if total energy (i.e., the energy of the electromagnetic field plus the energy of matter) is to be conserved, then the electromagnetic field must be adding energy density to the matter at the rate
\begin{equation}\label{eq:5.19}
\pdv{\mathscr{E_{matter}}}{t} =  \vb{J} \cdot \vb{E}\,.
\end{equation}

Thus $\vb{J} \cdot \vb{E}$ \tit{is the rate at which energy per unit volume is transferred from the electromagnetic field to matter.} 

In a similar manner and by a similar calculation, the failure of momentum conservation to hold for the electromagnetic field momentum alone is given by
\begin{equation}\label{eq:5.20}
\pdv{\mathscr{P}_i}{t} - \sum_{j=1}^3 \partial_j \Theta_{ij} = - \left[ \rho E_i + (\vb{J} \cross \vb{B})_i   \right]\;.
\end{equation}

If total momentum is to be conserved, then the rate of change of the momentum density of matter must be given by minus the right side of eq. (\ref{eq:5.20}). In other words, the electromagnetic field must exert a force per unit volume, $\vb{f}$, on matter given by 
\begin{equation}\label{eq:5.21}
\vb{f} = \rho \vb{E} + \vb{J} \cross \vb{B}\;.
\end{equation}
which is referred to as the \tit{Lorentz force}.

It should be emphasized that the entire content of electromagnetic theory is expressed by Maxwell's equations (\ref{eq:5.2})--(\ref{eq:5.5}); equations (\ref{eq:5.11})--(\ref{eq:5.13}) for energy density, momentum density, and stress of the electromagnetic field; and equations (\ref{eq:5.19}) and (\ref{eq:5.21}), which express that energy and momentum are conserved for the total system composed of electromagnetic field and matter.

We now substitute eqs.  (\ref{eq:5.2}) and (\ref{eq:5.3}) into equations  (\ref{eq:5.4}) and (\ref{eq:5.5}) to write Maxwell's equation purely in terms of $\phi$ and $\vb{A}$:
\begin{equation}\label{eq:5.22}
- \laplacian \phi - \pdv{}{t} \div{\vb{A}} = \frac{\rho}{\epsilon_0}\,,
\end{equation}
\begin{equation}\label{eq:5.23}
- \laplacian{\vb{A}} - \grad{ (\div{\vb{A}}) }+ \frac{1}{c^2} \pdv{}{t} \grad{\phi} + \frac{1}{c^2}\pdv[2]{\vb{A}}{t} = \mu_0 \vb{J}\,,
\end{equation}
where the identity eq. (\ref{eq:4.16}) was used to get eq. (\ref{eq:5.23}). A tremendous simplification of these equations can be made by transforming the potentials to a new gauge $(\phi', \vb{A}')$ such that
\begin{equation}\label{eq:5.24}
\frac{1}{c^2} \pdv{\phi'}{t} + \div{\vb{A}'} = 0\,.
\end{equation}

This condition is known as the \tit{Lorenz}\footnote{Note that there is no \quotes{t} in Lorenz. The gauge condition eq. (\ref{eq:5.24}) is named after Ludvig Lorenz, a nineteenth century Danish physicist/mathematician, not Hendrik Lorentz, the Dutch physicist after whom the Lorentz transformation is named.} \tit{gauge condition}. By eq. (\ref{eq:5.1}), such a gauge can be chosen if we can find a function $\chi$ that satisfies
\begin{equation}\label{eq:5.25}
\frac{1}{c^2} \pdv{\phi}{t} - \frac{1}{c^2} \pdv[2]{\chi}{t} + \div{\vb{A}} + \laplacian{\chi} = 0\,.
\end{equation}
that is, if we can solve the equation
\begin{equation}\label{eq:5.26}
\Box \chi = -s\,,
\end{equation}
with $s= (1/c^2) \pdv{\phi}{t} + \div{\vb{A}}$, where  
\begin{equation}\label{eq:5.27}
\Box \equiv -\frac{1}{c^2} \pdv[2]{}{t} + \laplacian\;.
\end{equation}
The operator $\Box$ is known as the d'\tit{Alembertian} or \tit{wave operator}, and equation (\ref{eq:5.26}) is known as the 
\tit{wave equation with source} $s$. It is well known that this equation can be solved for any smooth $s$, and indeed, this will follow from results we obtain in section \ref{sec:5.4}. Thus, without loss of generality, we may always use the gauge freedom eq. (\ref{eq:5.1}) to put the potentials $\phi$ and $\vb{A}$ in the Lorenz gauge. It is important to note that the Lorenz gauge is \tit{not} unique. If $\chi$ satisfies 
eq. (\ref{eq:5.25}), then so does $\chi + \psi$, where $\psi$ is any solution to the homogeneous wave equation $\Box \psi = 0$. As we shall see in section (\ref{sec:5.4}), there are many solutions to this equation.

We now transform our potentials  $\phi$ and $\vb{A}$ to the Lorenz gauge potentials satisfying eq. (\ref{eq:5.24}). For notational simplicity, we drop the primes and denote the transformed potentials as  $\phi$ and $\vb{A}$ rather than  $\phi'$ and $\vb{A}'$. 
Maxwell'sequations (\ref{eq:5.22}) and (\ref{eq:5.23}) then become simply
  
\begin{align}
\Box \phi   &= - \frac{\rho}{\epsilon_0}\label{eq:5.28}\,,\\
\Box \vb{A} &= - \mu_0 \vb{J}\,.\label{eq:5.29}
\end{align}

Thus the full content of Maxwell's equations is expressed by the wave equations (\ref{eq:5.28}) and (\ref{eq:5.29}) together with the Lorenz gauge condition 
\begin{equation}\label{eq:5.30}
\frac{1}{c^2} \pdv{\phi}{t} + \div{\vb{A}} = 0\;.
\end{equation}

\section{Retarded Green's Function}\label{sec:5.2}
We see from eqs. (\ref{eq:5.28}) and (\ref{eq:5.29}) that the key to being able to solve Maxwell's equations is to be able to solve the wave equation with source
\begin{equation}\label{eq:5.31}
\Box \psi = -f\;.
\end{equation}

If we know how to obtain solutions $\psi$ to eq. (\ref{eq:5.31}) for a given $f$, then we can immediately solve eqs.  (\ref{eq:5.28}) and (\ref{eq:5.29}). Of course, we must still solve eq. (\ref{eq:5.30}) as well, but we will see that this equation is automatically satisfied for the retarded solution to eqs. (\ref{eq:5.28}) and (\ref{eq:5.29}) for sources with suitable fall-off.

As in electrostatics, we will be able to obtain a solution to eq. (\ref{eq:5.31}) if we can find a Green's function, that is, a solution to
\begin{equation}\label{eq:5.32}
\Box_{(t, \vb{x})} G(t, \vb{x};t', \vb{x}') = -\delta(\vb{x} - \vb{x}')(t - t')\,,
\end{equation}

where the subscript $(t, \vb{x})$ on $\Box$ indicates that the derivatives appearing in the d'Alembertian operator $\Box$ are taken with respect to the unprimed variables, not the primed variables. Note that, in contrast to eq. (2.67), G depends on $t$ and $t'$ as well as $\vb{x}$ and $\vb{x}'$, and the right side of eq. (\ref{eq:5.32}) has a delta function in $t - t'$ as well as in $\vb{x} - \vb{x}'$. As in electrostatics, given $G$, a solution $\psi$  to eq. (\ref{eq:5.31}) can then be obtained via  
\begin{equation}\label{eq:5.33}
\psi(t, \vb{x}) = \bigint G(t, \vb{x};t', \vb{x}') f(t', \vb{x}') \dd[3]x' \dd{t'}\,,
\end{equation}
provided that this integral converges. Note that I have used the expression \quotes{a Green's function} and \quotes{a solution} because neither the Green's function (\ref{eq:5.32}) nor solutions to eq. (\ref{eq:5.31}) are unique. Indeed, we will explicitly encounter the nonuniqueness below when attempting to solve for $G$. However, we will see that there is a unique \quotes{retarded} Green's function, for which $G$ vanishes for $t < t'$. In section \ref{sec:5.4}, we use the retarded Green's function to characterize the general solution to eq. (\ref{eq:5.31}).

We seek to solve eq.  (\ref{eq:5.32}) by using Fourier transforms. For an integrable function $F : \R \rightarrow \R$, we define its Fourier transform $\hat{F}$ by 
\begin{equation}\label{eq:5.34}
\hat{F}(k) = \frac{1}{\sqrt{2\pi}} \int_{- \infty}^{\infty} F(x) e^{-ikx}\dd{x}\,.
\end{equation}

The notion of a Fourier transform can be extended to distributions, in which case the Fourier transform yields a distribution. In particular, the Fourier transform of the delta function $\delta_{x_0} = \delta(x - x_0)$ is well defined by the function
\begin{equation}\label{eq:5.35}
\hat{\delta}_{x_0}(k) = \frac{1}{\sqrt{2\pi}} e^{-ik x_0}\;,
\end{equation}
as would be expected by formally replacing $F(x)$ by $\delta(x - x_0)$ in eq. (\ref{eq:5.34}). 

For a smooth function $F$, with suitably fast fall-off at infinity, it can be shown that the inverse of the Fourier transform is given by
\begin{equation}\label{eq:5.36}
F(x) = \frac{1}{\sqrt{2\pi}} \int_{- \infty}^{\infty} \hat{F}(k) e^{-ikx} e^{+ikx_0}\dd{k}\,.
\end{equation}

that is, one can recover $F$ from its Fourier transform $\hat{F}$ by the same formula as eq. (\ref{eq:5.34}) except for the sign change in the exponential. Equation (\ref{eq:5.36}) also applies to distributions when suitably interpreted. In particular, the delta function is given by 
\begin{equation}\label{eq:5.37}
\delta_{x_0}(x) = \frac{1}{\sqrt{2\pi}} \int_{-\infty}^{\infty} \hat{\delta}_{x_0}(k) e^{+ikx}\dd{k} =  
\frac{1}{2\pi} \int_{-\infty}^{\infty} e^{-ikx_0} e^{+ikx} \dd{k}\,.
\end{equation}

This equation---which appears very commonly in physics texts---may not look very sensible, since the integral on the right side of eq. (\ref{eq:5.37}) clearly does not converge. However, what this equation is really supposed to mean is that for any smooth function $f$ with suitably fall-off at infinity, we have
\begin{equation}\label{eq:5.38}
\delta_{x_0}(f) = \frac{1}{2 \pi} \int_{-\infty}^{\infty} \dd{k} e^{-ikx_0} \int_{-\infty}^{\infty} \dd{x} e^{+ikx} f(x)\,.
\end{equation}
This statement is correct, because 
\begin{equation}\label{eq:5.39}
\frac{1}{2 \pi} \int_{-\infty}^{\infty} \dd{k} e^{-ikx_0} \int_{-\infty}^{\infty} \dd{x} e^{+ikx} f(x) = 
\frac{1}{\sqrt{2\pi}} \int_{-\infty}^{\infty} \dd{k} e^{-ikx_0} \hat{f}(-k) = f(x_0)\,.
\end{equation}

A major reason Fourier transforms are so useful is that differentiation in physical space corresponds to multiplication by $ik$ in Fourier transform space. To see this, let $F$ be a smooth function that falls off sufficiently rapidly at infinity. Then the Fourier transform 
of $\dv*{F}{x}$ is given by
\begin{equation}\label{eq:5.40}
\begin{aligned}
\hat{\dv{F}{x}}(k) &= \frac{1}{\sqrt{2\pi}} \int_{-\infty}^{\infty} \dv{F}{x}(x) e^{-ikx} \dd{x}\\
                   &= \frac{1}{\sqrt{2\pi}} \int_{-\infty}^{\infty} (ik) F(x) e^{-ikx} \dd{x}\\
                   &= ik \hat{F}(k)\,,
\end{aligned}
\end{equation}
where we integrated by parts in the second line, disregarding the boundary term at infinity because of the fall-off of $F$. 
On account of eq. (\ref{eq:5.40}), any partial differential equation with constant coefficients can be converted to an algebraic equation in Fourier transform space.  
 
We now attemp to solve for $G$, eq. (\ref{eq:5.32}), by means of Fourier transforms. To simplify the notation we set $t' = \vb{x}' = 0$, and we also set $c = 1$. (We will restore  $t', \vb{x}', \text{ and } c$ at the end of the calculation.) Define the (4-dimensional) Fourier transform of $G$ by
\begin{equation}\label{eq:5.41}
\hat{G}(\omega, \vb{k}) = \frac{1}{(2 \pi)^2} \int_{-\infty}^{\infty} G(t, \vb{x}) e^{+i \omega t} e^{-i \vb{k}\cdot\vb{x}} \dd{t} \dd[3]x\,.
\end{equation}
Note that, by standard convention, the time Fourier transform is defined\footnote{This is done so that $(\omega/c, \vb{k})$ are the components of a 4-vector in special relativity (see chapter 8).} by integrating with $e^{+i \omega t}$  rather than $e^{-i \omega t}$. Taking the Fourier transform of eq. (\ref{eq:5.32}) with respect to $t$ and $\vb{x}$ and using eq. (\ref{eq:5.35}) (with $x_0 = 0$) and eq. (\ref{eq:5.40}), we obtain
\begin{equation}\label{eq:5.42}
(\omega^2 - k^2) \hat{G}(\omega, \vb{k}) = - {\left( \frac{1}{\sqrt{2 \pi}} \right)}^4 = - \frac{1}{4\pi^2}\,,
\end{equation} 

where we have written $k = \abs{\vb{k}}$. One may think that the solution to this equation is simply  
\begin{equation}\label{eq:5.43}
\hat{G}(\omega, \vb{k}) = - \frac{1}{4\pi^2} \frac{1}{(\omega^2 - k^2)} = - \frac{1}{4\pi^2} \, \frac{1}{(\omega - k)} \, \frac{1}{(\omega + k)}\,.
\end{equation}
However, dividing by $(\omega^2 - k^2)$ is actually an illegal step, because this quantity can be zero. The difficulty arising from this can be seen if we attempt to take the inverse Fourier transform of $\hat{G}$ with respect to $\omega$ (but not $\vb{k}$), so as to obtain the Fourier transform of $G$ with respect to space but not time, which we denote as $\tilde{G}$:
\begin{equation}\label{eq:5.44}
\tilde{G}(t, \vb{k}) \equiv \frac{1}{\sqrt{2 \pi}} \int_{-\infty}^{\infty} \hat{G}(\omega, \vb{k}) e^{-i \omega t} \dd\omega\,.
\end{equation}

Then, we have
\begin{equation}\label{eq:5.45}
\tilde{G}(t, \vb{k}) = - \frac{1}{4 \pi^2 \sqrt{2 \pi}} \int_{-\infty}^{\infty} \frac{e^{-i \omega t}}{(\omega - k) (\omega + k)} \dd\omega\,.
\end{equation}


However, there are logarithmic divergences in the integral on the right side of eq. (\ref{eq:5.45}) at 
$\omega = \pm k$, so the right side is ill defined (even as a distribution). This reflects the fact that many Green's functions satisfy eq. (\ref{eq:5.32}), so we cannot be expected to be able to solve for $G$ without providing additional input as to which Green's function we seek. 

This difficulty can be dealt with by suitably regularizing eq. (\ref{eq:5.45}) in such a way that the right side is well defined and eq. (\ref{eq:5.42}) continues to hold. A simple way of doing this is to infinitesimally displace the poles at $\omega = \pm k$ in eq. (\ref{eq:5.45}) into the complex $\omega$-plane. The case of most interest for us is to displace both poles infinitesimally into the lower half of the $\omega$-plane, thereby defining the \tit{retarded Green's function}
\footnote{As noted below, displacement of both poles into the upper half of the $\omega$-plane yields the advanced Green's function. Displacement of the pole at $\omega = k$ into the lower half-plane and the pole at $\omega = -k$ into the upper half-plane yields the Feynman propagator.} 
$\tilde{G}_{ret}$, given by
\begin{equation}\label{eq:5.46}
\tilde{G}(t, \vb{k}) = - \frac{1}{4 \pi^2 \sqrt{2 \pi}} \int_{-\infty}^{\infty} \frac{e^{-i \omega t}}{(\omega - k + i \epsilon) (\omega + k + i \epsilon)} \dd\omega\,,
\end{equation}
where $\epsilon > 0$ and the limit $\epsilon \rightarrow 0$ is to be taken after the integral is performed. We can view eq. (\ref{eq:5.46}) as a contour integral in the complex $\omega$-plane. For $t < 0$, the integral is exponentially damped in the upper half of the $\omega$-plane, and we may \quotes{close the contour} in that half-plane. The resulting closed contour encloses no singularities---since the poles have been pushed into the lower half-plane--so, by Cauchy's theorem, we obtain
\begin{equation}\label{eq:5.47}
\tilde{G}_{ret}(t, \vb{k}) = 0, \quad\quad\quad \text{for } t < 0\,.
\end{equation}
This condition uniquely characterizes the retarded Green's function. The fact that $\tilde{G}_{ret}$ vanishes prior to the \quotes{turn-on} of the delta function source at $t = 0$ can be interpreted as saying that it is providing the solution with \quotes{no incoming radiation.} This is the solution of physical relevancein problems where no radiation is present prior to the presence of the source.

For $t > 0$, we can similarly close the contour in the lower half of the $\omega$-plane. However, the resulting closed contour now contains poles at $\omega = \pm k -i\epsilon$. By Cauchy's theorem, we obtain for $t > 0$
\begin{equation}\label{eq:5.48}
\begin{aligned}
\tilde{G}(t, \vb{k}) &= + \frac{1}{4 \pi^2 \sqrt{2 \pi}} 2\pi i \left[ \frac{e^{-ikt}}{2k} - \frac{e^{+ikt}}{2k}  \right]\\
                     &= \frac{1}{2 \pi \sqrt{2 \pi}} \frac{\sin kt}{k}\,,
\end{aligned}
\end{equation}
where the sign change in the first line as compared with eq. (\ref{eq:5.46}) results from the contour running the \quotes{wrong way}.
  
We now take the inverse Fourier transform of eq. (\ref{eq:5.48}) with respect to $\vb{k}$ to obtain the retarded Green's function for $t > 0$ in position space:  

\section{Multipole Expansion}\label{sec:5.3}

\subsection{Cartesian Multipole Expansion of the Radiation Field for a Nonrelativistic Source}

\subsection{General Multipole Expansion for a Relativistic Source}

\section{The Initial Value Formulation for Maxwell's Equations}\label{sec:5.4}

\section{Plane Waves}\label{sec:5.5}

\section{Conducting Cavities and Waveguides}\label{sec:5.6}

\subsection{Conducting Cavities}

\subsection{Waveguides}

%===================================================================================

\section*{Problems}


