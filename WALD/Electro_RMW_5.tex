\setcounter{chapter}{0}
\renewcommand{\thechapter}{5}
\chapter{Electrodynamics}\label{ch:5}
\setcounter{equation}{0}	        % To start with Equation 1
\counterwithin{equation}{chapter}	% Equation numbering will be 2.1 2.2 2.3   ... 

In this chapter, we consider the general case where $\phi$, $\vb{A}$ and $\rho$, $\vb{J}$ are time dependent. The equations of electrodynamics are discussed in section \ref{sec:5.1}, and the Lorenz gauge is introduced. We solve for the retarded Green's function in section \ref{sec:5.2}, from which the solution to Maxwell's equations for general $\rho$, $\vb{J}$ with no incoming radiation can be obtained. Multipole expansions of the electromagnetic field of the retarded solution are obtained in section \ref{sec:5.3}. The retarded Green's function is then used in section \ref{sec:5.4} to obtain the solution to Maxwell's equations with prescribed values of the electromagnetic field at an initial time. We discuss plane wave solutions in section \ref{sec:5.5}. The chapter concludes with a discussion of electrodynamics in conducting cavities and waveguides in section \ref{sec:5.6}.

\section{The Equations of Electrodynamics}\label{sec:5.1}
The equations of electrodynamics have been presented in chapter 1. As discussed in section \ref{sec:1.1}, the fundamental dynamical variables are the potentials $\phi$ and $\vb{A}$, which are considered to be equivalent if and only if they differ by a gauge transformation:
\begin{equation}\label{eq:5.1}
\phi \rightarrow \phi' = \phi - \dv{\chi}{t}, \quad \quad \vb{A} \rightarrow \vb{A'} = \vb{A} + \grad \chi\:.
\end{equation}

% BOX 1 **START** -- << Maxwell equations in Gaussian units >>

\parindent=0pt  % Set the paragraph indentation to 0 (normal = 10pt) before the box 
\parbox{\textwidth}{\begin{mdframed}[style=MyFrame] %Added "\parbox{\textwidth}{"
%\lipsum[1]
\subsubsection*{BOX 5.1 -- Gauge invariance in Gaussian units}\label{box:8.1}
\begin{equation*}
\phi \rightarrow \phi' = \phi - \frac{1}{c}\dv{\chi}{t}, \quad \quad 
\vb{A} \rightarrow \vb{A'} = \vb{A} + \grad \chi\;.\quad \quad\quad \quad [G5.1]
\end{equation*}
\end{mdframed}} %Added a closing "}" here
\parindent=10pt % Set the paragraph indentation to normal (10pt) 
% BOX 1 ** END ** -- << Maxwell equations in Gaussian units >>


The equations of electrodynamics (i.e. Maxwell's equations) are
\begin{align}
\vb{E} &= - \grad \phi - \pdv{\vb{A}}{t}						\,,\label{eq:5.2}\\
\vb{B} &= \curl{\vb{A}}                  						\,,\label{eq:5.3}\\
\div{\vb{E}} &= \frac{\rho}{\epsilon_0} 						\,,\label{eq:5.4}\\
\curl{\vb{B}} &= \frac{1}{c^2} \pdv{\vb{E}}{t} + \mu_0 \vb{J}	\,.\label{eq:5.5}
\end{align}
The first two of these equations define $\vb{E}$ and $\vb{B}$ in terms of $\phi$ and $\vb{A}$. They imply that 
\begin{align}
\div{\vb{B}} = 0                                                \,,\label{eq:5.6}\\
\curl{\vb{E}} + \pdv{\vb{B}}{t}= 0                              \,,\label{eq:5.7}
\end{align}

% BOX 2 **START** -- << Maxwell equations in Gaussian units >>

\parindent=0pt  % Set the paragraph indentation to 0 (normal = 10pt) before the box 
\parbox{\textwidth}{\begin{mdframed}[style=MyFrame] %Added "\parbox{\textwidth}{"
%\lipsum[1]
\subsubsection*{BOX 5.2 -- Maxwell equations in Gaussian units}\label{box:8.2}
\begin{align*}
\vb{E} &= - \grad \phi - \frac{1}{c}\pdv{\vb{A}}{t}\,,\quad\quad\quad &[G5.2]\\
\vb{B} &= \curl{\vb{A}}\,,&[G5.3]\\
\div{\vb{E}} &= 4 \pi \rho\,,&[G5.4]\\
\curl{\vb{B}} &= \frac{1}{c} \pdv{\vb{E}}{t} + \frac{4\pi}{c} \vb{J}\,,&[G5.5]\\
\div{\vb{B}} &= 0\,,&[G5.6]\\
\curl{\vb{E}} + \frac{1}{c} \pdv{\vb{B}}{t }&= 0\,.&[G5.7]\\
\end{align*}
\end{mdframed}} %Added a closing "}" here
\parindent=10pt % Set the paragraph indentation to normal (10pt) 
% BOX 2 ** END ** -- << Maxwell equations in Gaussian units >>

Equations (\ref{eq:5.6}) and (\ref{eq:5.7}) are equivalent to eqs. (\ref{eq:5.2}) and (\ref{eq:5.3}) in a topologically trivial region\footnote{More precisely, by a \quotes{topologically trivial region} in the present context, I mean a region in which any closed 2-dimensional surface $S$ is the boundary od a 3-dimensional (compact) volume, and every closed loop $\mathscr{C}$ is the boundary of a 2-dimensional (compact) surface. The necessary and sufficient condition for the existence of a vector potential $\vb{A}$ such that $\vb{B} = \curl{\vb{A}}$ is that $\int_S \vb{B} \cdot \vu{n} = 0$ for any closed 2-dimensional surface $S$. The necessary and sufficient condition for the existence of a scalar potential $\phi$ satisfying eq. (\ref{eq:5.2}) is $\int_{\mathscr{C}} [\vb{E} + \pdv*{\vb{A}}{t}] \cdot \dd{\vb{l} = 0}$ for any closed loop $\mathscr{C}$. If eqs. (\ref{eq:5.6}) and (\ref{eq:5.7}) hold in a topologically trivial region, then the necessary and sufficient condition for the existence of $\vb{A}$ will automatically hold by Gauss's theorem, and the necessary and sufficient condition for the existence of $\phi$ will automatically hold by Stokes's theorem.} 
(i.e., if eqs. (\ref{eq:5.6}) and (\ref{eq:5.7}) hold in a topologically trivial region, they imply the existence of potentials $\phi$ and $\vb{A}$ in that region and satisfying eqs. (\ref{eq:5.2}) and (\ref{eq:5.3})). 
Taking the time derivative of eq. (\ref{eq:5.4}) and adding it to $c^2$ times the divergence of eq. (\ref{eq:5.4})---using the fact that $c^2 = \flatfrac{1}{\epsilon_0 \mu_0}$---we obtain the charge-current conservation law:
\begin{equation}\label{eq:5.8}
\pdv{\rho}{t} + \div{\vb{J}} = 0\,.
\end{equation}
It is worth noting that in the source-free case (i.e., when $\rho = \vb{J} = 0$), except for one sign difference and factors of $c$, eqs. (\ref{eq:5.6}) and (\ref{eq:5.7}) take the same form as eqs. (\ref{eq:5.4}) and (\ref{eq:5.5}) with $\vb{E}$ and $\vb{B}$ interchanged. It follows immediately that in any source-free, topologically trivial region, maxwell's equations are invariant under a \tit{duality transformation}:
\begin{equation}\label{eq:5.9}
\vb{E} \rightarrow c \vb{B}\,, \quad\quad c \vb{B} \rightarrow \vb{E}\,,
\end{equation}
that is, if $\vb{E}, \vb{B}$ solve eqs. (\ref{eq:5.4})-(\ref{eq:5.7}) with  $\rho = 0$, $\vb{J} = 0$, then so do 
$\vb{E'}, \vb{B'}$ with  $\vb{E'},= c \vb{B}$, $\vb{B'} = - \flatfrac{\vb{E}}{c}$. More generally, for any real number $\alpha$, the source-free Maxwell equations (\ref{eq:5.4})-(\ref{eq:5.7}) are invariant under the \tit{duality rotation}:
\begin{equation}\label{eq:5.9}
\vb{E} \rightarrow \cos\alpha \vb{E} + \sin\alpha (c \vb{B})\,, \quad\quad \vb{B} \rightarrow \cos\alpha \vb{B} - \sin\alpha ( \vb{E}/c)\,.
\end{equation}




\section{Retarded Green's Function}\label{sec:5.2}

\section{Multipole Expansion}\label{sec:5.3}

\subsection{Cartesian Multipole Expansion of the Radiation Field for a Nonrelativistic Source}

\subsection{General Multipole Expansion for a Relativistic Source}

\section{The Initial Value Formulation for Maxwell's Equations}\label{sec:5.4}

\section{Plane Waves}\label{sec:5.5}

\section{Conducting Cavities and Waveguides}\label{sec:5.6}

\subsection{Conducting Cavities}

\subsection{Waveguides}

%===================================================================================

\section*{Problems}


