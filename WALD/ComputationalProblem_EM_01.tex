\chapter*{Problema 1}\label{EM_P_01}

\section*{1.1 Premesse}\label{ref:sec_1.1}
L'elettromagnetismo classico si basa sulle equazioni di Maxwell
\begin{align}
\div{\vb*{E}}  &= \frac{\rho}{\epsilon_0}\,, \label{eq:1.1} \\
\curl{\vb*{B}} - \frac{1}{c^2} \pdv{\vb*{E}}{t} &= \mu_0 \vb*{J}\,,\label{eq:1.2} \\
\div{\vb*{B}}  &= 0\,, \label{eq:1.3} \\
\curl{\vb*{E}} + \pdv{\vb*{B}}{t} &= 0\,.\label{eq:1.4}
\end{align}

ove le sorgenti $\rho$ e $\vb*{J}$ debbono soddisfare l'equazione di conservazione carica-corrente
\begin{equation}\label{eq:1.5}
\pdv{\rho}{t} + \div{\vb*{J}} = 0\,,
\end{equation}

Si assumano specificati nello spaziotempo $\rho(t, \vb*{x})$ e $\vb*{J}(t, \vb*{x})$, per i quali sia soddisfatta l'equazione di conservazione (\ref{eq:1.5}). 

Siano $\vb*{E}_0(\vb*{x})$ e $\vb*{B}_0(\vb*{x})$ campi vettoriali arbitrari nello spazio tali che 
\begin{align}
\div{\vb*{E}_0} &= \rho(t=0, \vb*{x})/\epsilon_0\,, \textrm{ e} \\ 
\div{\vb*{B}_0} &= 0\,. 
\end{align}

Allora esiste un'unica soluzione $\vb*{E}(t, \vb*{x}), \vb*{B}(t, \vb*{x})$ delle equazioni di \\Maxwell (\ref{eq:1.1})-(\ref{eq:1.4}) tale che
\begin{align}
\vb*{E}(t=0, \vb*{x}) &= \vb*{E}_0(\vb*{x})\,, \textrm{ e} \\ 
\vb*{B}(t=0, \vb*{x}) &= \vb*{B}_0(\vb*{x})\,.
\end{align}


Esistono tante soluzioni delle equazioni di Maxwell con 
$\rho$ e $\vb*{J}$ fissati, quante sono le possibili scelte dei campi $\vb*{E}_0(\vb*{x}), \vb*{B}_0(\vb*{x})$ la cui divergenza soddisfa le condizioni di cui sopra. Il fatto che questa informazione iniziale del campo elettromagnetico possa essere specificata liberamente dimostra che il campo elettromagnetico ha i suoi propri gradi di libertà dinamici indipendenti. \\Le soluzioni alle equazioni di Maxwell \tit{non} sono determinate da $\rho$ e $\vb*{J}$.


\section{Problema}\label{ref:sec_2}



