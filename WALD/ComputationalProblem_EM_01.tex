\chapter*{Evoluzione spazio-temporale del campo Elettromagnetico}\label{EM_P_01}

\section*{1.1 Premesse}\label{sec_1.1}
L'elettromagnetismo classico si basa sulle equazioni di Maxwell
\begin{align}
\div{\vb*{E}}  &= \frac{\rho}{\epsilon_0}\,, \label{eq:1.1} \\
\curl{\vb*{B}} - \frac{1}{c^2} \pdv{\vb*{E}}{t} &= \mu_0 \vb*{J}\,,\label{eq:1.2} \\
\div{\vb*{B}}  &= 0\,, \label{eq:1.3} \\
\curl{\vb*{E}} + \pdv{\vb*{B}}{t} &= 0\,.\label{eq:1.4}
\end{align}

ove le sorgenti $\rho$ e $\vb*{J}$ debbono soddisfare l'equazione di conservazione carica-corrente
\begin{equation}\label{eq:1.5}
\pdv{\rho}{t} + \div{\vb*{J}} = 0\,,
\end{equation}

Si assumano specificati nello spaziotempo $\rho(t, \vb*{x})$ e $\vb*{J}(t, \vb*{x})$, per i quali sia soddisfatta l'equazione di conservazione (\ref{eq:1.5}). 

Siano $\vb*{E}_0(\vb*{x})$ e $\vb*{B}_0(\vb*{x})$ campi vettoriali arbitrari nello spazio tali che 
\begin{align}
\div{\vb*{E}_0} &= \rho(t=0, \vb*{x})/\epsilon_0\,, \textrm{ e} \\ 
\div{\vb*{B}_0} &= 0\,. 
\end{align}

Allora esiste un'unica soluzione $\vb*{E}(t, \vb*{x}), \vb*{B}(t, \vb*{x})$ delle equazioni di \\Maxwell (\ref{eq:1.1})-(\ref{eq:1.4}) tale che
\begin{align}
\vb*{E}(t=0, \vb*{x}) &= \vb*{E}_0(\vb*{x})\,, \textrm{ e} \label{eq:1.6}\\ 
\vb*{B}(t=0, \vb*{x}) &= \vb*{B}_0(\vb*{x})\,. \label{eq:1.7}
\end{align}


Esistono tante soluzioni delle equazioni di Maxwell con 
$\rho$ e $\vb*{J}$ fissati, quante sono le possibili scelte dei campi $\vb*{E}_0(\vb*{x}), \vb*{B}_0(\vb*{x})$ la cui divergenza soddisfa le condizioni di cui sopra. Il fatto che questa informazione iniziale del campo elettromagnetico possa essere specificata liberamente dimostra che il campo elettromagnetico ha i suoi propri gradi di libertà dinamici indipendenti: la soluzione delle equazioni di Maxwell \tit{non è determinata} da $\rho$ e $\vb*{J}$.


\section*{1.2 Analisi del problema}\label{sec_1.2}

Le premesse esposte nella precedente sezione \ref{sec_1.1} indicano la possibilità di determinare l'evoluzione spazio-temporale del campo elettromagnetico nell'assunzione che siano in qualche modo note -- perché indipendentemente imposte dall'esterno -- le densità della carica $\rho$ e della corrente $\vb*{J}$ in una regione di spazio $\Omega$ per un certo intervallo di tempo $T$. Assumiamo che la regione $\Omega$ sia chiusa, connessa e limitata e che le densità di carica e corrente siano nulle all'esterno di $\Omega$: in pratica sarà sufficiente che siano nulle entro una distanza $c T$ dalla superficie $\partial \Omega$ di $\Omega$.

Per fissare le idee, consideriamo una regione di \tit{spazio vuoto} $\Omega$ racchiusa da una sfera di raggio $R$, assumendo che carica e corrente siano nulle ovunque, tranne che sulla superficie $\partial \Omega$ della sfera. 

Per quanto esposto nella precedente sezione 1.1, possiamo considerare assegnate a priori la densità di carica $\rho(t; \theta, \phi)$ e $\vb*{J}(t; \theta, \phi)$ per $t \in [0:T]$, ove $\theta$ 
e $\phi$ sono le coordinate sferiche di un punto sulla superficie, purché i valori assegnati siano consistenti con la legge di \tit{conservazione locale} della carica, espressa dall'equazione di continuità carica-corrente (\ref{eq:1.5}).

Nella sezione 1.3 analizziamo modi per garantire il requisito di continuità.

Una maggior varietà di problemi può essere generata considerando una regione di \tit{spazio vuoto} $\Omega$ compresa tra due superfici chiuse e connesse della quali una è interamente contenuta nell'altra. Il caso forse più semplice è quello in cui la frontiera $\partial \Omega$ di $\Omega$ è costituita dall'unione delle frontiere (disgiunte) $\partial \Omega_1$ ed $\partial \Omega_2$ di due sfere\footnote{Non necessariamente \tit{concentriche}.} di raggio $R_1$ ed $R_2$ ove $R_1 < R_2$. In tal caso si assumerà che carica e corrente siano nulle ovunque, tranne che sulle superfici $\partial \Omega_1$ e $\partial \Omega_2$. Si noti che in tal caso la carica totale su $\partial \Omega_1$ e la carica totale su $\partial \Omega_2$ \tit{potranno} essere separatamente assegnate e \tit{dovranno} essere separatamente conservate. 
\pagebreak
\section*{1.3 Evoluzione consistente del campo carica-corrente}\label{sec_1.3}
\begin{enumerate}[(I)]
\item \tbi{METODO I}  
%-----------------------------------------------------------------------------
	%-------------------------------------------------------------------------
	\begin{enumerate}[(a)]
	\item Definisci una distribuzione di carica totale $Q$ sulla superficie $S$ con un metodo qualsiasi. Esempio: 6 \tit{poli} di carica \tit{positiva} nelle direzioni corrispondenti alle 6 facce di un cubo inscritto nella sfera, e 8 \tit{poli} di carica \tit{negativa} nelle direzioni corrispondenti agli 8 spigoli del medesimo cubo.  
	\item Definisci il movimento di un vettore unitario $\vu{n}(t)$ che applicato al centro della sfera definisce implicitamente una \tit{curva} $(\theta(t), \phi(t))$ nello spazio tridimensionale; 
	\item Scegli la velocità angolare\footnote{Nel caso più semplice questa velocità angolare avrà un valore costante oppure, ad esempio, il suo valore potrebbe oscillare periodicamente tra due valori.} $\omega$ con cui ruotare la distribuzione di carica attorno all'asse che contiene il centro della sfera ed il punto di coordinate sferiche 
$(\theta(t), \phi(t))$;
	\item Ad ogni istante $t_i \geq 0$, ottieni la distribuzione di carica all'istante $t_{i+1} = t_i + \Delta t$ in base al cambio di direzione dell'asse di rotazione (da $\vu{n}(t_i)$ a $\vu{n}(t_{i+1})$) e alla velocità angolare $\omega$ con cui la distribuzione di carica ruota attorno all'asse;
	\item Calcola $\vb*{J}(t_{i+1}; \theta(t_{i+1}), \phi(t_{i+1}) )$  in base al movimento dell'asse di rotazione $\vu{n}(t)$ e alla velocità di rotazione $\omega$ attorno ad esso.
	\end{enumerate}
	%-------------------------------------------------------------------------
%-----------------------------------------------------------------------------
\item \tbi{METODO II} 
%-----------------------------------------------------------------------------
	%-------------------------------------------------------------------------
	\begin{enumerate}[(a)]
	\item Definisci una distribuzione di carica totale $Q$ sulla superficie $S$ in termini di un  suo sviluppo in armoniche sferiche ortogonali, che coinvolga $k$ armoniche, con $k \geq 2$;  
	\item Varia nel tempo in modo continuo i coefficienti $c_k(t)$ dello sviluppo partendo da un set \tit{iniziale} $c^{i}_1, \ldots, c^{i}_k$ per arrivare ad un set \tit{finale} $c^{f}_1, \ldots, c^{f}_k$ in modo che in ciascun punto la derivata temporale della densità sia contenuta entro un limite predefinito;
	\item Utilizza un modello discreto della superficie sferica per determinare\footnote{Questo potrebbe richiedere la soluzione di un sistema lineare sparso di grandi dimensioni.} numericamente $\vb*{J}(t_k)$.
	\end{enumerate}
	%-------------------------------------------------------------------------
%-----------------------------------------------------------------------------
\end{enumerate}
\pagebreak
\section*{1.4 Densità di carica $\rho$ e di corrente $\vb*{J}$}\label{sec_1.4}
Per quanto esposto nella precedente sezione 1.1, possiamo assegnare i valori iniziali del campo elettromagnetico, cioè dei campi vettoriali $\vb*{E}_0 = \vb*{E}(0; \vb*{x})$ e $\vb*{B}_0 = \vb*{B}(0; \vb*{x})$, con l'unico vincolo che valgano le equazioni (\ref{eq:1.6}) e (\ref{eq:1.7}), cioè che il campo magnetico iniziale $\vb*{B}_0$ abbia ovunque divergenza nulla e che la divergenza del campo elettrico $\vb*{E}_0 = \rho(0; \vb*{x})/\epsilon_0$ sia ovunque determinata dalla densità di carica. 

Nella sezione 1.5 introduciamo una possibile \tit{discretizzazione} della regione spazio-temporale $\Omega$ e della sua frontiera $\partial \Omega$ con particolare riferimento al caso in cui $\Omega$ è la porzione di spazio racchiusa tra 2 \tit{sfere concentriche}\footnote{La scelta è dettata dalla facilità con cui in questo caso si può discretizzare la regione $\Omega$: basta collegare con segmenti \tit{radiali} coppie di punti corrispondenti di cui il primo si trova nella superficie sferica più interna ed il secondo si trova su quella più esterna.}, la frontiera $\partial \Omega$ essendo costituita dall'unione delle 2 superfici sferiche. 

Nella sezione 1.6 analizziamo come si possano definire valori iniziali del campo elettromagnetico consistenti con i vincoli imposti alla divergenza di $\vb*{E}_0$ e $\vb*{B}_0$.

Nella sezione 1.7 discutiamo una versione \tit{discretizzata} del problema ai valori iniziali, ed analizziamo come si possano calcolare i valori del campo elettromagnetico su una griglia di punti ad un certo istante $t+\Delta t$ posto che siano noti i valori del campo e delle \tit{sorgenti} ($\rho$ e $\vb*{J}$) al tempo $t$. 
 

\section*{1.5 Discretizzazione di $\Omega$ e della sua frontiera}\label{sec_1.5}


\section*{1.6 Valori iniziali del campo elettromagnetico}\label{sec_1.6}


\section*{1.7 Evoluzione temporale del campo discretizzato}\label{sec_1.6}

