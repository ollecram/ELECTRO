\chapter{Metodi computazionali nel Metodo I  della sez. 1.3}\label{app:P_01_A}

Come punto di partenza nella implementazione del metodo è necessario suddividere la superficie $\vb*{S}$ di una sfera mediante un insieme di triangoli sferici che la ricoprano completamente senza sovrapporsi. I \tit{vertici} di tali triangoli definiscono una \tit{mesh} di punti sulla sfera. Una mesh con ottime caratteristiche di uniformità può essere costruita per suddivisione ricorsiva\footnote{La procedura ricorsiva consiste nel suddividere ciascun triangolo in $4$ sub-triangoli, introducenti come nuovi vertici i punti che dividono a metà ciascuno degli archi di cerchio comuni a due triangoli sferici adiacenti.} dei $20$ triangoli implicitamente definiti da un \tit{icosaedro regolare} inscritto nella sfera. 

L'obiettivo del Metodo I è definire l'evoluzione di un campo di \tit{densità} di carica la cui dinamica soddisfi non solo il vincolo della conservazione della carica totale $Q$ (assumeremo generalmente che la carica totale sia nulla) ma soprattutto il vincolo della \tit{conservazione locale} cioè la cosiddetta equazione di continuità: la differenza tra la carica totale di carica racchiusa da una qualsiasi porzione chiusa di superficie $\vb*{\omega}$ in un intervallo di tempo $\Delta t$ deve coincidere con il \tit{flusso} di corrente che attraversa la frontiera $\partial \vb*{\sigma}$ di  $\vb*{\omega}$.  

\section*{A.1 Titolo A1}\label{sec:A.1}

\section*{A.2 Titolo A2}\label{sec:A.2}

\section*{A.3 Titolo A3}\label{sec:A.3}


