%%%%%%%%%%%%%%%%%%%% book.tex %%%%%%%%%%%%%%%%%%%%%%%%%%%%%

\documentclass[english, 11pt]{book}

\usepackage{babel}
% choose options for [] as required from the list
% in the Reference Guide, Sect. 2.2

\usepackage{makeidx}         % allows index generation
\usepackage{graphicx}        % standard LaTeX graphics tool
                             % when including figure files
\usepackage{multicol}        % used for the two-column index
\usepackage[bottom]{footmisc}% places footnotes at page bottom
% etc.
% see the list of further useful packages
% in the Reference Guide, Sects. 2.3, 3.1-3.3
\usepackage[normalem]{ulem}

\usepackage[shortlabels]{enumitem}	% to be able to resume enumerated lists

\usepackage{amsmath}	% To be able to slash
\usepackage{amsfonts}	% To be able to use \mathbb ... 
\usepackage{amssymb}	% To be able to use \nmid ... 
\usepackage{amsthm}		% \qed, \qedhere
\usepackage{slashed}	% any character (dirac)
% Physics package 
% https://tex.stackexchange.com/questions/38978/how-can-i-manually-install-a-latex-package-debian-ubuntu-linux
\usepackage{physics}	
\usepackage[title,toc,page]{appendix}

% To put accents below letters
\usepackage{accents}
% To write two equations side by side
\usepackage{multicol}

% To use PGF/TikZ https://tex.stackexchange.com/questions/3622/best-way-to-generate-nice-function-plots-in-latex
\usepackage{tikz}
\usetikzlibrary{datavisualization}
\usetikzlibrary{datavisualization.formats.functions}

% To create a placeholder paragraph with Latin text
\usepackage{lipsum}

% To create framed text boxes with custom defined styles 
%\usepackage[linewidth=1pt]{mdframed}
\usepackage[framemethod=TikZ]{mdframed}
\mdfdefinestyle{MyFrame}{%
    linecolor=brown,			% blue, orange, brown, ...
    outerlinewidth=1pt,
    roundcorner=10pt,
    innertopmargin=\baselineskip,
    innerbottommargin=\baselineskip,
    innerrightmargin=15pt,
    innerleftmargin=15pt,
    backgroundcolor=gray!5!white}

% Force chapter numbering to restart within each part
\makeatletter
%\@addtoreset{chapter}{part}
\makeatletter


\makeindex             % used for the subject index
                       % please use the style svind.ist with
                       % your makeindex program


%%%%%%%%%%%%%%%%%%%%%%%%%%%%%%%%%%%%%%%%%%%%%%%%%%%%%%%%%%%%%%%%%%%%%

\begin{document}

%========================================================================

\newcommand{\quotes}[1]{``#1''}
\newcommand{\sfT}{$\mathsf{T}$}
\newcommand{\udT}{\rotatebox[origin=c]{180}{$\mathsf{T}$}}


%Bold calligraphic letters 
\newcommand{\N}{\mathbb{N}}	% integers
\newcommand{\Z}{\mathbb{Z}}	% relative
\newcommand{\Q}{\mathbb{Q}}	% rationals
\newcommand{\R}{\mathbb{R}}	% reals
\newcommand{\C}{\mathbb{C}}	% complex
\newcommand{\F}{\mathbb{F}}	% generic field 1
\newcommand{\K}{\mathbb{K}}	% generic field 2

%Plain calligraphic letters 
\newcommand{\U}{\mathcal{U}}    % sets 1
\newcommand{\V}{\mathcal{V}}    % sets 2
\newcommand{\W}{\mathcal{W}}    % sets 3
%\newcommand{\S}{\mathcal{S}}    % sets 4  -- complaint about command \S already defined
%\newcommand{\T}{\mathcal{T}}    % sets 5

% To create boldface vectors with a hat or check accent 
\newcommand{\hatvb}[1]{\vb{\hat{#1}}}
\newcommand{\chkvb}[1]{\vb{\check{#1}}}

% To create not-bold vectors with a hat or check accent 
\newcommand{\hatv}[1]{\hat{#1}}
\newcommand{\chkv}[1]{\check{#1}}

% To create <bra| and |ket> with a hat to denote unitary and/or basis vectors
\newcommand{\hbra}[1]{\bra*{\hat{#1}}}
\newcommand{\hket}[1]{\ket*{\hat{#1}}}

% To put accents below letters. 
\newcommand{\ut}[1]{\underaccent{\tilde}{#1}}
\newcommand{\uh}[1]{\underaccent{\hat}{#1}}
\newcommand{\form}[1]{\uh{#1}}

% To create italic, bold, bolditalic text
\newcommand{\tit}[1]{\textit{#1}}
\newcommand{\tbf}[1]{\textbf{#1}}
\newcommand{\tbi}[1]{\textit{\textbf{#1}}}

\DeclareRobustCommand{\rchi}{{\mathpalette\irchi\relax}}
\newcommand{\irchi}[2]{\raisebox{\depth}{$#1\chi$}} % inner command, used by \rchi

%----------------------------------------------------------------------------------------
%   r Griffiths, curls and fonts from mt2pro[lite]->pro
%----------------------------------------------------------------------------------------
%\def\rcurs{s}
%\def\brcurs{\vb{\rcurs}}
%\def\hrcurs{\vu{\rcurs}}

%========================================================================


\author{Marcello Vitaletti}
\title{Special Relativity\\
{\small }}
\maketitle
\begin{mdframed}[leftmargin=-10pt,rightmargin=-10pt]
These notes are my annotated transcript of books on the subject referred to by the title. Ellipses in the text mean that part of the text from the original text has been omitted. Framed text boxes like this one, contain my original annotations.
\end{mdframed}


\frontmatter%%%%%%%%%%%%%%%%%%%%%%%%%%%%%%%%%%%%%%%%%%%%%%%%%%%%%%

%\include{dedic}

%\chapter*{Plan}
\label{plan} 

In this book I am keeping notes about the theory of classical electromagnetism, 
as exposed in various books. In particular, I intend to cover the following materials:

\begin{itemize}

\item B. Felsager -- Geometry Particles and Fields
\begin{enumerate}
\setcounter{enumi}{0}
\item Electromagnetism (1.1 to 1.4)
\end{enumerate}

\item C. Cattaneo -- Teoria Einsteniana della Gravitazione
\begin{enumerate}
\setcounter{enumi}{0}
\item Elementi di Algebra e Analisi Lineare
\end{enumerate}

\item D.J. Griffiths -- Introduction to Electrodynamics
\begin{enumerate}
\setcounter{enumi}{0}
\item Vector Analysis
\item Electrostatics
\item Potentials
\item Electric Fields in Matter
\item Magnetostatics
\item Magnetic Fields in Matter
\item Electrodynamics
\item Conservation Laws
\item Electromagnetic Waves
\item Radiation
\item Electrodynamics and Relativity
\item Potentials and Fields
\item Helmoltz Theorem
\end{enumerate}

\item J.D. Jackson -- Classical Electrodynamics, 2nd Edition
\begin{enumerate}
\setcounter{enumi}{0}
\item Introduction to Electrostatics
\item Boundary Value Problems in Electrostatics - I
\item Boundary Value Problems in Electrostatics - II
\item Multipoles, Electrostatics of Macroscopic Media, Dielectrics
%\item Magnetostatics
%\item Time Varying Fields, Maxwell Equations, Conservation Laws
%\item Plane Electromagnetic Waves and Wave Propagation
%\item Wave Guides and Resonant Cavities
%\item Simple Radiating Systems, Scattering and Diffraction
%\item Magnetohydrodynamics and Plasma Physics
\end{enumerate}

\item J.D. Jackson -- Classical Electrodynamics, 3rd Edition
\begin{enumerate}
\setcounter{enumi}{4}
\item Magnetostatics, Faraday's Law, Quasi-Static Fields
\item Maxwell Equations, Macroscopic Electromagnetism, Conservation Laws
\item Plane Electromagnetic Waves and Wave Propagation
\item Wave Guides, Resonant Cavities and Optical Fibers
\item Radiating Systems, Multipole Fields and Radiation
\item Scattering and Diffraction
\item Special Theory of Relativity
\item Dynamics of Relativistic Particles and Electromagnetic Fields
\end{enumerate}

\item B. Felsager -- Geometry Particles and Fields
% Contacts with quantum theory of particles dynamics in EM fields
\begin{enumerate}
\setcounter{enumi}{1}
\item Interaction of Fields and Particles
\end{enumerate}

\item J. Franklin -- Advanced Mechanics and General Relativity
\begin{enumerate}
\setcounter{enumi}{1}
\item Relativistic Mechanics
\item Tensors
\item Curved Space
\item Scalar Field Theory
\item Tensor Field Theory (6.1 to 6.5)
\end{enumerate}

\item J.D. Jackson -- Classical Electrodynamics, 3rd Edition
\begin{enumerate}
\setcounter{enumi}{12}
\item Collisions, Energy Loss and Scattering of Charged Particles, Cherenkov and Transition Radiation
\item Radiation by Moving Charges
\item Bremsstrahlung, Method of Virtual Quanta, Radiative Beta Processes
\item Radiation Damping, Classical Models of Charged Particles
\end{enumerate}

\item B. Felsager -- Geometry Particles and Fields
% Contacts with quantum theory of fields dynamics + differential geometry math
\begin{enumerate}
\setcounter{enumi}{2}
\item Dynamics of Classical Fields
\end{enumerate}

\begin{enumerate}
\setcounter{enumi}{5}
\item Differentiable Manifolds, Tensor analysis
\item Differential Forms, Exterior Calculus
\item Integral Calculus on Manifolds
\end{enumerate}

\item C.W. Misner, K.S. Thorne, J.A. Wheeler -- Gravitation
\begin{enumerate}
\setcounter{enumi}{1}
\item Foundations of Special Relativity
\item The Electromagnetic Field
\item Electromagnetism and Differential Forms
\end{enumerate}

\item L.D. Landau, E.M. Lifshitz -- Teoria dei Campi
\begin{enumerate}
\setcounter{enumi}{0}
\item Principio di Relatività
\item Meccanica Relativistica
\item Carica in un Campo Elettromagnetico
\item Equazioni del Campo Elettromagnetico
\item Campo Elettromagnetico Costante
\item Onde Elettromagnetiche
\item Propagazione della Luce
\item Campo di Cariche in Moto
\item Radiazione Elettromagnetica
\end{enumerate}

\item L.D. Landau, E.M. Lifshitz -- Elettrodinamica dei Mezzi Continui
\begin{enumerate}
\setcounter{enumi}{0}
\item Elettrostatica dei Conduttori
\item Elettrostatica nei Dielettrici
\item Corrente Continua
\item Campo Magnetico Costante
\item Ferromgnetismo e Antiferromagnetismo
\item Superconduttività
\item Campo Magnetico Quasi Stazionario
\item Idrodinamica Magnetica
\item Equazioni delle Onde Elettromagnetiche
\item Propagazione delle Onde Elettromagnetiche
\item Onde Elettromagnetiche in Mezzi Anisotropi
\item Dispersione Spaziale
\item Ottica non Lineare
\item Passaggio delle Particelle Veloci attraverso la Materia
\item Diffusione delle Onde Elettromagnetiche
\item Diffrazione dei Raggi X nei Cristalli
\end{enumerate}

\end{itemize}
	
\tableofcontents
%\addappheadtotoc

\mainmatter%%%%%%%%%%%%%%%%%%%%%%%%%%%%%%%%%%%%%%%%%%%%%%%%%%%%%%%

\part{N. David Mermin\\ {\small \quotes{It's About Time}. (Annotated transcript)}}
\chapter{Morin -- The Principle of Relativity}
\label{ch:Morin_01}
Einstein based the theory of relativity on two postulates. 
The first is known as the principle of relativity:\\
\tit{No phenomena have properties corresponding to the concept of absolute rest.}

The principle was first enunciated by Galileo, three centuries earlier, and was built into the classical mechanics of Newton. In this chapter we shall elaborate Einstein's concise statement of the principle of relativity, and then explore how the principle can be used to discover some elementary but not entirely obvious facts about how things behave. 

The principle of relativity fits the same pattern of \tit{invariance} principles. It would naturally appear as the last one in the following list of four, where each principle states that the content\footnote{By the \tit{content} of a physical law we mean the set of all assertions that are explicitly made (or are logically implied) by that law.} of some physical laws remains the same under a specified change in the \tit{description} of the physical world.  

\begin{itemize}[noitemsep,topsep=0pt]
\item \tbi{Translational invariance in space}\\
The change is a rigid \tbi{translation} of the origin of the spatial axes.
\item \tbi{Rotational invariance in space}\\
The change is a rigid \tbi{rotation} of the the three spatial axes.
\item \tbi{Translational invariance in time}\\
The change is a \tbi{translation} of the origin of time.
\item \tbi{Principle of Relativity}\\
The change is the \tit{uniform, rigid motion} of the spatial axes along a fixed direction.
\end{itemize}

In working with the principle of relativity, one uses the term \tit{frame of reference}. This is the system in terms of which you have chosen to describe things. For example, it is natural for flight attendants and passengers of an airplane to choose a reference frame that is fixed with respect to the airplane, but any motion occurring inside the airplane can be described equally well with a reference frame that is fixed with respect to the ground. 

Another important term is \tit{inertial frame of reference}.  In an inertial frame, objects on which no forces act remain stationary (if they were at rest) or keep moving with constant speed along a fixed direction (if they were in motion). Equivalently: if the reference frame is an inertial one, objects subject to no forces do not accelerate
\footnote{Acceleration is the \tbi{vector} $\dot{\vb{v}}$ representing the change of velocity $\vb{v}$ in the unit of time. At a given time $t$, $\dot{\vb{v}}$ can be decomposed as the sum of a vector parallel to $\vb{v}$ and a vector $\vb{\omega}$ orthogonal to $\vb{v}$. If $\vb{\omega} = \vb{0}$ there is no change in the \tit{direction} of $\vb{v}$.}.  

Any frame whose axes move with constant velocity with respect to an inertial frame is itself an inertial frame. Each frame thus defines a whole equivalence class of reference frames in uniform relative motion with respect to each other. 

People sometimes take the principle of relativity to mean, loosely speaking, that the behavior of a uniformly moving object should not depend on how fast it is moving, or that motion with uniform velocity cannot affect any properties of an object. This is simply wrong. The principle of relativity only requires that if an  object has certain properties in a frame of reference in which the object is stationary, then if the same object moves uniformly, it will have the same properties \tit{in a frame of reference that moves uniformly with it}. 

Take for example the \tit{Doppler effect}. If a yellow light moves away from you at an enormous speed, the color you see changes from yellow to red; it it moves toward you at an enormous speed, the color changes from yellow to blue. So the color of an object in a fixed frame of reference can depend on whether it is moving or at rest, and in what direction it is moving. What the relativity principle guarantees is that if a light is seen to be yellow when it is stationary, then when it moves with uniform velocity it will still be seen as yellow \tit{by someone who moves with that same velocity}.

We will be applying the principle of relativity to learn some quite extraordinary things by examining the same sets of events in different frames of reference. Some of the things we shall learn in this way are so surprising that they are hard to believe at first. The general procedure for doing this is always the same: \tit{Take a situation which you don't fully understand. Find a new frame of reference in which you do understand it. Examine it in that new frame of reference. Then translate your understanding in the new frame back into the language of the old one}.
\\\tbf{Example 1.}\\
Newton first law of motion states that in the absence of an external force a uniformly moving body continues to move uniformly. This law follows from the principle of relativity and a very much simpler law. The simpler law merely states that in the absence of an external force, a stationary body continues to remain stationary. 

Suppose we only know the simpler law. The principle of relativity tells us that it must be be true in all inertial frames of reference. If we want to learn about the subsequent behavior of a ball initially moving at 50 f/sec in the absence of an external force, all we have to do is find an inertial frame of reference in which we can apply the simpler law. The frame we need is clearly the one that moves at 50 f/sec in the same direction as the ball, since in that frame the ball is stationary. In that frame we can apply the law that in absence of an external force a stationary body remains stationary. Assuming that if an object is undisturbed in one inertial frame of reference, then it is undisturbed in any other inertial frame of reference (that the condition of no force acting on an object is an \tit{invariant} condition independent of the frame of reference in which the object is described), we conclude that in the original frame the ball must continue to move at 50 f/sec in the absence of an external force. 
\\\tbf{Example 2.}\\
Suppose we have two identical perfectly elastic balls. Identical elastic balls have the property that if you shoot them directly at each other with the same speed, then after they collide each bounces back in the direction it came from with the same speed that it had before the collision. Question: What happens if one of the balls is at rest and you shoot the other one directly at it?

There is a long tradition of answering such questions by invoking the conservation of energy and momentum.At this stage it is both entertaining and instructive to understand how this and many related questions can be answered using nothing but the principle of relativity.

To figure out what happens, using only the principle of relativity, first draw a picture illustrating the rule you know: when the balls move at each other with equal speeds, they simply rebound with the same speeds. 

Then draw a picture of the new situation. The white ball moves to the right along the tracks, in a railroad station, at 10 f/sec toward the stationary black ball. 

Now think about how this would look if we sere describing it from the frame of reference of a train moving through the station to the right at 5 f/sec. Since the white ball covers 10 feet of track per second, and the train covers 5 feet of track per second, every second the white ball gains 5 feet on the train. So in the frame of reference of a train moving to the right at 5 f/sec, the white ball moves at 5 f/sec. Since the black ball is stationary with respect to the tracks, in the train frame it moves to the \tit{left} at 5 f/sec, just as the tracks do. 

Therefore, in the frame of this particular train the unknown situation before the collision becomes an instance of the known situation, in which the balls approach each other with the same speed. By the principle of relativity, the equality of conditions \tit{before} the collision implies the equality of conditions \tit{after} the collision, namely that the balls will bounce back, away from each other, with equal speeds. 

Finding what happens in the frame of reference of the station is now a matter of translating the experiment outcome as described in the frame of the train to the frame of the station. After the collision the white ball moves to the left at 5 f/sec in the train frame, so it must be stationary in the station frame. After the collision, the black ball moves to the right at 5 f/sec in the train frame, so it must be moving to the right at 10 f/sec in the station frame. 

So we have used the principle of relativity to learn something new about identical elastic balls: if one is at rest and the other bumps it head-on, then the moving one comes to a complete stop and the stationary one moves off with the velocity the formerly moving one (the white ball) originally had. This is a fact familiar to all plyers of billiards, but not many of them realize that it is simply a consequence of the much more obvious fact (less frequently encountered in billiards) that when two balls collide head-on with equal and opposite speeds, each bounces back the way it came with its original speed. 
\\\tbf{Example 3.}\\
Two identical sticky balls have the property that if they are fired directly at each other with equal speeds, then they stick together upon collision and the resulting compound ball is stationary. If a sticky ball is fired at 10 f/sec directly at another identical sticky ball that is stationary and the two stick together, with what speed and in what direction will the compound ball move after the collision?

We can again answer the question using only the principle of relativity, by viewing the initial moving white ball and initially stationary black ball from the frame of reference of a train in which both are moving with the same speed but in opposite directions. Such a train moves along the direction of motion of the white ball but only at 5 f/sec. In the train frame the situation before the collision is the one we know about: the balls move at each other at the same speeds. Therefore we know that in the train frame the compound ball is stationary after the collision. But since the train moves down the tracks at 5 f/sec and the compount is stationary in the train frame, in the track frame it will move down the tracks at 5 f/sec -- the same speed as the train moves in the track frame. This solves the problem: when the moving ball strickes the stationary ball, the compound ball moves at half the original speed of the moving ball.    
\\\tbf{Example 4.}\\
This one has the virtue that it will not be obvious how to solve it by exploiting the conservation of momentum, but it is easily solved using the principle of relativity. 

Suppose we have two elastic balls, but one of them is very big and the other is very small. If the big ball is stationary and the small ball is fired directly at it, the small ball simply bounces back in the direction it came from with the same speed, and the big ball stays at rest.(Think of throwing a table-tennis ball directly at a bowling ball.) With what speed will each ball move after the collision, if the small ball is stationary and the big ball is fired directly at it with a speed of 10 f/sec?

We wish to examine the initial situation in a frame of reference in which the big ball is stationary, so we must now view the collision in the frame of a train moving, with the big ball, at 10 f/sec to the left. In that frame the small ball will move at 10 f/sec to the right, and the situation before the collision is the one we understand. So in the train frame we know that after the collision the big ball will remain stationary and the small ball will move at 10 f/sec to the left. Returning to the description in the station frame, we note that after the collision the big ball moves with the train, at 10 f/sec to the left. The little ball, however, moves at 20 f/sec to the left, since in each second it gains 10 feet on the train, which has itself moved 10 feet to the left. So if the little ball is initially stationary, then after a collision with the big ball it moves off at \tit{twice} the speed of the big one. 
\\\tbf{Example 5.}\\
What happens if the big and little ball of the previous example approach each other with the \tit{same} speed--say 5f/sec. In that case the train providing the frame of reference in which we know the answer moves, with the big ball, at 5 f/sec to the left, so the little ball moves to the right at 10 f/sec in the train frame. After the collision the big ball remains stationary in the train frame, while the little ball moves to the left at 10 f/sec. So back in the station frame the little ball moves to the left at 15 f/sec, with \tit{triple} its original speed.

You can see a spectacular demonstration of this by placing the little ball, for example a tennis ball, at the very top of the big ball, for example a basketball, and then dropping them on a hard surface very carefully, so that the little ball does not roll off the top of the big one. When the big ball hits the floor it reverses its direction of motion without a change in speed, so for a very brief moment the big ball is moving up and the little ball is moving down, both going at the same speed. Immediately after that, the little ball flies up at nearly three times its original speed. As it happens, the height reached by a ball moving up is proportional to the \tit{square} of its initial speed, so if losses due to various kinds of friction are unimportant, the little ball can shoot up to almost nine times the height from which it was originally dropped!     

I hope these examples will give you a feeling for how the principle of relativity is actually used, and for the power it can have to predict behavior under apparently unfamiliar conditions. Before starting to apply it under genuinely unfamiliar conditions, we must look a little more closely at some of the reasoning we used in these simpler examples.

      % Morin - It's About Time - The principle of Relativity
\chapter{Morin -- Combining (Small) Velocities}
\label{ch:Morin_02}
In chapter \ref{ch:Morin_01} we examined the power of the principle of relativity, deducing the non entirely obvious outcomes of certain collisions by considering other collisions whose outcomes were self evident. We must now emphasize that besides using the principle of relativity , we repeatedly made implicit use of another rule that enabled us to relate the velocity of a ball in the train frame to its velocity in the station frame. 

The rule we implicitly made use of in chapter \ref{ch:Morin_01} goes under the name of \tit{nonrelativistic velocity addition law}. While it may strike you as obvious it is not, in fact, exactly correct.  The rule is accurate to a phenomenally high degree of precision when all relevant velocities are no more than many thousands of feet per second, but when velocities become as large as many millions of feet per second, we shall find, quite surprisingly, that the rule has to be modified.

\quotes{Nonrelativistic} is an unfortunate term, but everybody uses it and so shall we. It does not mean, as you might think, \quotes{in contradiction to the principle of relativity.} It comes from the fact that the body of lore constructed by applying the principle of relativity to certain strange facts about motion at very high speeds has come to be known as the theory of relativity. As a result, the term \quotes{nonrelativistic} refers to how we thought the world behaved before we learned about the theory of relativity. Since at low speeds things actually do behave almost exactky the way we used to think they did before we learned about the theory of relativity, \quotes{nonrelativistic} means valid to a high degree of accuracy when all speeds are sufficiently small. How small will emerge in subsequente chapters, but note in this regard that even the speed of bullets from guns (many thousands of feet per second) sound as very small indeed. 

Before stating the nonrelativistic velocity addition law, we must establish a convention on the direction of motion along which velocities are taken to be positive. For almost all of the points we shall be making, it suffices to consider objects confined to move along a single direction, which we shall often take to be that of a long straight railroad track. There are only two possible directions of motion along such a track, and we need names for them. Let us therefore take the tracks to run east-west and agree that motion to the east is assigned a positive velocity, while motion to the west is given a negative velocity. Thus a ball going west at \tit{speed} of 5 f/sec has, according to our convention, a \tit{velocity} of -5 f/sec.

There is a certain amount of subtlety (but not very much) in these matters. A velocity is always defined with respect to a frame of reference. Let a train move east at 10 f/sec. Suppose a ball moves toward the rear of the train at 3 f/sec so that in the track (or station) frame it moves east at only 7 f/sec. In the track frame the velocity of the ball is +7 f/sec, since it is moving east. But in the train frame the ball is moving toward the rear of the train--i.e. toward   the west. So its train-frame velocity is -3 f/sec. It is important to keep in mind that in this context \quotes{toward west} means \quotes{toward the western end of the train}. West is a direction, not a place. If the train is heading from California to New York, the ball is getting farther from California (which many people think of as \quotes{the West}) even though it is moving toward the western end of the train. This, of course, is because the train is moving away from California at a higher speed than the ball is moving west in the train frame.

In pictures of events along the track in various frames of reference, we shall take the tracks to be more or less horizontal and shall follow the convention of mapmakers, taking east to be to  the right, and west to be to the left.  

Let $X$ be an object (a ball if you like) moving uniformly along the tracks. There is a frame of reference in which the velocity of $X$ is $0$--in which $X$ is stationary. This frame is, of course, the frame of reference of a train that is moving uniformly along the tracks with the same velocity as $X$. Such a frame is called the \tit{proper} frame of $X$. There is nothing particularly virtuous about this particular frame of reference, It's just that every uniformly moving object does have a unique frame associated with it in a natural way--namely, the frame in which it does not move. (Rest is always to be regarded as a special case of motion: motion with zero speed.) Often one leaves out the word \quotes{proper} and just refers, for example, to the frame of the ball or the ball frame. If an object is moving nonuniformly then there is no \tit{inertial} frame in which it remains stationary at all times; a frame of reference in which a nonuniformly moving object remains stationary must be a noninertial frame. 

If $Y$ is a second object movng uniformly along the tracks with a velocity different from the velocity of $X$, then we can, if we wish, describe the motion of $Y$ in the proper frame of $X$, calling $Y$'s velocity in $X$'s frame $v_{YX}$. The expression $v_{YX}$ is read as \quotes{the velocity of $Y$ with respect to $X$.} It doesn't matter whether we think of either $Y$ or $X$ as being an object or a frame of reference--the proper frame of the associated object--since both the object and the associated frame of reference move with the same velocity.

With this, we can now state in all its abstract glory the nonrelativistic velocity addition law. If $X$, $Y$, and $Z$ all move with uniform velocity along the same straight line, then 
\begin{equation}\label{eq:Morin_02.1}
v_{XZ} = v_{XY} + v_{YZ}\,. 
\end{equation}

In words, the velocity of $X$ with respect to $Z$ is the sum of the velocity of $X$ with respect to $Y$ and the velocity of $Y$ with respect to $Z$. Or, if you prefer, the velocity of $X$ in frame $Y$ and the velocity of frame $Y$ in frame $Z$.

Suppose, for example, $X$ is a ball, $Y$ is a train, and $Z$ is the station (tracks). Then \ref{eq:Morin_02.1} says that the velocity of the ball in the station frame is the velocity of the ball in the train frame plus the velocity of the train in the station frame. This should be evident when all the velocities are positive. But it also works when some of the velocities are negative. 

Usually, of course, it's simpler just to reason one's way to the answer without having to invoke the abstract form (\ref{eq:Morin_02.1}) of the addition law. Soon we shall learn that the addition law (\ref{eq:Morin_02.1}) is not exactly correct, being valid to an extremely high degree of accuracy when all speeds are not too large. When enormous velocities enter the story, or if we want the right answer to fantastically high precision, then we must use a modified form of (\ref{eq:Morin_02.1}), and it is then important to use the formula that expresses the modified addition law, since commonsense reasoning no longer gives the right answer. So you should be sure you understand how to use the nonrelativistic addition law (\ref{eq:Morin_02.1}), even when what it tells you is \quotes{obvious}.

One important consequence of (\ref{eq:Morin_02.1}) (which turns out to remain valid at any speed) is that 
\begin{equation}\label{eq:Morin_02.2}
v_{XZ} = - v_{YX}\,. 
\end{equation}

If $X$ moves with a certain speed with respect to $Y$, then $Y$ moves with that same speed with respect to $X$, but in the opposite direction. This (fairly obvious) relation follows directly from the general rule (\ref{eq:Morin_02.1}). For consider the special case in which $Z$ and $X$ are identical, so that $v_{XZ}$ becomes $v_{XX}$, the velocity of $X$ in the frame in which $X$ is stationary, i.e. in its proper frame. The velocity of $X$ in its proper frame is $0$, so we have
\begin{equation}\label{eq:Morin_02.3}
0 = v_{XX} = v_{XY} + v_{YX}\,, 
\end{equation}

and this immediately gives (\ref{eq:Morin_02.2}).

How would one go about justifying the rule (\ref{eq:Morin_02.1}) to a stubborn person who did not find it obvious? Consider this instance of it: Let a train move east in the track frame. If a ball moves east in the train frame at 5 f/sec, then in one second the ball gets 5 feet nearer the front of the train. And if the train moves at 10 f/sec in the track frame, then in one second the train gets 10 feet further east along the track. So in one second the ball gets 15 feet further east along the track--the 5 it gains on the train and the additional 10 the train gains on the track. But the ball getting 15 feet further east along the track in one second is precisely what we mean when we say the ball moves at 15 f/sec in the track frame. Who could doubt this? Indeed, I encourage you not to doubt it until you find boringly familiar the points made in this and the preceding chapter. 

But I do call your attention to an apparently innocent phrase that turns out, surprisingly, to be fraught with danger: \tit{in one second}. We have implicitly assumed that \quotes{in one second}
means the same thing in the train frame as it does in the track frame. But suppose that were not true. Suppose \quotes{in one second} in the train frame meant something different from \quotes{in one second} in the track frame. What would happen to the argument we just gave? We would have to replace \quotes{in one second} by something like \quotes{in one second according to train time} or \quotes{in one second according to track time.} The argument we just went through then starts off fine, but is a bit more cumbersome:

\hangindent=0.7cm If the ball moves east in the train at 5 f/sec then in one second according to train time it gets 5 feet further down the train. And if the train moves at 10 f/sec in the track frame, then in one second according to track time it gets 10 feet further east along the track.

\noindent But then we come to: 

\hangindent=0.7cm So \tit{in one second} the ball gets 15 feet further east along the track--the 5 it gains on the train and the additional 10 the train gains on the track.

\noindent What can the italicized \quotes{in one second} mean here? The first 5 feet are gained in one second of train time, the second 10 feet are gained in one second of track time. Collapsing both into a single, unqualified \quotes{in one second} makes no sense unless track time and train time are the same. 

For the moment, we will not pursue this any further. But be aware that the simple rule 
(\ref{eq:Morin_02.1}) telling us how velocities combine relies on the implicit assumption that there is nothing problematic about the idea of a single unique notion of time that can be used equally well in any frame of reference. It was Einstein's great insight in 1905 that this apparently obvious assumption is, in fact, false. \quotes{It came to me,} he said to a colleague many years later, \quotes{that time was suspect.} When the assumption of a unique frame-independent time fails, it takes other \quotes{obvious} assumptions down with it. 

That failure, however, is so slight as to be of no importance when all speeds of interest are small compared with that of light, as they were in the examples we examined in chapter \ref{ch:Morin_01}. So now we must turn to how the speed of light enters the story.        % Morin - It's About Time - Combining (Small) Velocities
\chapter{The Speed of Light}
\label{ch:Morin_03}

When you turn on a light, how long does it take the light to get from the bulb to the things it illuminates?
Galileo apparently tried to answer this by stationing two people on top of two mountains, a large distance $D$ apart. Alice opens her lantern, Bob opens his the instant he sees Alice's, and Alice notes the time $T$ that passes between the moment she open hers and the moment she sees the light returning from Bob's.

To get the speed $c$ with which the light moves from her mountaintop to Bob's and back again, Alice just divides twice the distance between the mountains by the delay time $T$ to get
\begin{equation}\label{eq:Morin_03.1}
c = 2 D / T\,. 
\end{equation}

$\cdots$

Three centuries later, Galileo's unsuccessful attempt was realized by replacing the two mountains by the Earth and the Moon. The Moon is so far away that it takes radar more than $2$ seconds to get there and bounce back. But by then the speed of light was known to high precision by other methods. Note, by the way, that the speed of radar is the same as the speed of light. Both are forms of electromagnetic radiation and all forms of electromagnetic radiation (light, radar, radio, x-rays, gamma rays, TV signals, for example) have the same speed in empty space. 

Light travels so fast that to measure its speed either you have to let it travel an enormous distance, or you have to make very accurate measurements of extremely tiny intervals of time. The very first successful estimate of the speed of light came from using astronomical distances. Galileo, who plays many roles in the story of relativity, discovered the four major moons of Jupiter earlier in the 17th century. In 1676 careful observations by Ole Romer, of the regularly occurring moments when a moon disappeared within Jupiter's shadow, revealed that sometimes these Jovian lunar eclipses lagged behind schedule by about 10 minutes, and sometimes they came in 10 minutes ahead. It was noted that they were ahead of schedule when the Earth was closest to Jupiter and behind when the Earth was furthest away. One concludes that the time it takes light to cross the orbit of the Earth must be something like 20 minutes. This gives an estimate of several hundred thousand kilometers per second for the speed of light. 

$\cdots$

Today we have highly sophisticated ways to measure the speed of light and know that it is $299,792,458$ meters per second (m/sec). Furthermore,, that is what it always shall be, because as of 1983 the meter has been \tit{defined} to be not the distance between two scratches on a platinum-iridium bar lovingly cared for in Paris, but as the distance light travels in $1/299,792,458$ of a second. Our unit of length (the meter) now is tied to our unit of time (the second). You might think that since the speed of light is now fixed forever by definition of the meter, this means that there is no longer any point in striving to measure it more and more accurately. But such improved experiments now provide more and more accurate measurements of the length of a meter--better and better standards of length. The experiments remain just as important as they used to be. What has changed is how we describe what we have learned from them. 

There are two useful numerical near coincidences associated with the speed of light being 299,792,458 m/sec:

First, the number is extremely close to 300 million m/sec or 300,000 kilometers per second (km/sec). Physicists are very used to taking it to be $3 \times 10^8$ m/sec. 

Second, the corresponding English unit is about 186,000 miles per second. Since there are 5,280 feet in a mile, this works out to about 982,000,000 feet per second. Thus, within 2 percent accuracy the speed of light is 1 billion feet per second or, in more practical units, 1 foot per nanosecond. Note that people working with the metric system can take \tit{1 light nanosecond} (which amounts to 30 cm.) as an equivalent and convenient unit of length. Nanoseconds and feet are also relevant to the accuracy of the global positioning system (GPS), which uses satellites broadcasting time signals every nanosecond. The signals are therefore spaced a foot (actually, 30 cm.) apart as they arrive at the surface of the earth, establishing the foot as a measure of the accuracy of the system.

In thinking about relativity it is very convenient to measure speeds in units that assign an especially simple value to the speed of light. In 1959, the foot was officially defined to be exactly 0.3048 of a meter. Since the speed of light is exactly 299,792,458 m/sec, if only people in 1959 had defined the foot to be 0.299792458 of a meter, a mere 1.64 percent shorter, then the speed of light would now be \tit{exactly} 1 f/ns. This unit of length will prove to be so useful, that for the purposes of this book \tit{I hereby redefine the foot:}

Hencefort, by 1 \tit{foot} we shall mean \tit{1 light nanosecond} (the distance light travels in a nanosecond) which--for people used to the metric system--is extremely close to being \tbi{30 cm.}

For comparing with lesser speeds, it can sometimes be convenient to think of 1 foot per nanosecond as 1,000 feet per microsecond. Since the speed of sound in ordinary air is about 300 m/sec, hence 1,000 f/sec, we get
\begin{equation*}
\begin{aligned}
\text{Speed of light:}\: &\approx	1 f/\text{nanosecond} = 1,000 f/\text{microsecond}\,,	\text{and}	\\
\text{Speed of sound:}\: &\approx 300 m/\text{second} = 1,000 f/\text{second}\,, \text{hence}
\end{aligned}
\end{equation*}
\tit{light is a million times faster than sound}.  

There is something peculiar and, when you think about it, quite extraordinary about the unqualified assertion that the speed of light in empty space is 299,792,458 m/sec. Ordinarily, when you specify a speed to such high precision and indeed when you mention any speed at all, the question \quotes{with respect to what} comes irresistibly to mind. After all, the speed of an object depends on the frame of reference in which that speed is measured. As we have repeatedly noted, a ball that Alice throws while riding on a uniformly moving train has one speed with respect to the train but quite another speed with respect to the tracks. In the case of light there are two obvious possible answere to the question \quotes{with respect to what?}:
\\\\\tbf{First Obvious Answer}\\

The speed of light is 299,792,458 m/sec with respect to the source of light.

$\cdots$

This reasonable answer is contradicted by our current understanding of the electromagnetic character of light. In the 19th century there was a great unification of the laws of electricity and magnetism, completed by the Scottish physicist James Clerk Maxwell. Maxwell's equations led to the prediction that when electrically charged particles jiggle back and forth (as they do, for example, in a hot wire) they must emit radiant energy that travels at a speed of about 300,000,000 m/sec. Since this speed was numerically indistinguishable from the speed of light, it was natural to identify light with a particular form of such radiation (associated with a very rapid jiggling--almost a million billion times a second). Maxwell's equations imply quite unambiguosly that this speed does not depend on the speed of the source of the radiation. According to the theory, the speed of the light is the same whether the chunk of matter in which the charged particles are jiggling is stationary or moving toward or away from the direction in which the light is emitted.

People had also noted that the regularity of certain astronomical motions as observed from Earth was quite unaffected by whether the source of the light that enabled us to observe them was moving toward or away from us. So there was both theoretical and astronomical evidence that the speed of light did not depend on the speed of its source. 
\\\\\tbf{Second Obvious Answer}\\

With respect to a light medium (historically called the ether), 299,792,458 m/sec is the speed of light \tit{in vacuum}. Light goes significantly slower in transparent media like water or glass, and a little bit slower in air. The ether, then, would be a sort of irreducible residue of otherwise empty space--what remains after you've removed everything it is possible to remove.

The analogy now is not to bullets from a gun, but to sound, which is a wave in the air. Like the speed of light, the speed of sound does not depend on the speed of the source of the sound. Sound moves at a definite speed with respect to the air, whose vibrations constitute and transmit that sound. If light is a vibration of something called the ether, then the speed of light should be with respect to that ether. 

Since the Earth moves about the sun at a brisk clip of 30 km/sec in different directions, depending on the time of year, and the sun moves briskly about the center of our galaxy, it would be a remarkable coincidence if the Earth just happened to be stationary in the rest frame of the ether. One would expect there to be a kind of \quotes{ether wind} blowing past the Earth, leading to a dependence of the speed of light on Earth on the direction of that wind. The speed of light on Earth into the direction from which the ether wind was blowing ought to be a bit less than its speed along the direction of the wind. Efforts to detect such a difference failed to yield a clear-cut result, most famously in the Michelson-Morley esperiment of 1887. The measurement demonstrated that if the speed of light was fixed with respect to an ether, then the Earth, in spite of its complicated motion with respect to the galaxy, was improbably close to being at rest in the rest frame of that ether at the time the experiment was performed. Stubborn people considered the possibility that the Earth dragged the ether in its neighborhood along with it. But if that were so, then the apparent positions of the stars in the sky should shift through the year depending on the way in which the ether was being dragged by the Earth. No such shift was observed. 

The importance of the Michelson-Morley experiment in the historical development of relativity has been debated. Einstein apparently alludes to it in his famous 1905 paper setting forth relativity, but only once and then only in passing: \quotes{Examples of this sort, together with \tit{unsuccessful attempts to determine any motion of the earth relative to the `light medium,'} lead to the conjecture that...} (my italics). The reference is little more than parenthetical. Such attempts had to be mentioned, because had they been successful and unambiguosly demonstrated a significant direction dependence to the velocity of light on Earth, the theory of relativity would have been dead on arrival.

The \quotes{examples of this sort} that Einstein offered as the real motivation for his reexamination of the nature of time were all examples of the fact that the electric and magnetic behavior of matter is consistent with the principle of relativity, in spite of the then widespread view that there actually was a preferred inertial frame of reference for electromagnetic phenomena--the frame in which the ether was stationary. The equations of electromagnetic theory were held by many to be valid in that frame of reference and no other. Einstein noted, in effect, that even if this were so, a broad range of electromagnetic phenomena seemed to play out in exactly the same way in frames of reference other than the frame in which the ether was stationary. This led him to postulate that the laws of electromagnetism were, in fact, rigorously valid in arbitrary inertial frames of reference. If this postulate was valid, then, Einstein noted, \quotes{the introduction of a `luminiferous ether' will prove to be superfluous} because there would be no way to determine the rest frame of the ether by any physical experiment involving electromagnetic phenomena. It is this specific postulate--that what we now call the principle of relativity applies to electromagnetism as well as to Newtonian mechanics (where everybody agreed that it was indeed valid)--that Einstein named the \quotes{principle of relativity} (\tit{Prinzip der Relativitat}). 

Now if Maxwell equations are valid in any inertial frame of reference, and if they predict that electromagnetic radiation and light in particular propagate at a fixed speed that is independent of the speed of the source of the light, then light must propagate at the same speed in any inertial frame of reference. The answer to the question \quotes{with respect to what?} is, as we now know, \quotes{with respect to any inertial frame of reference}. The speed of light in vacuum is simply 299,792,458 m/sec in any inertial frame of reference, regardless of how fast the source of the light is moving, and regardless of the choice of frame of reference in which the measurement of the speed of light is made. If, for example, you race after the light in a rocket at 10 km/sec, you do not reduce its speed away from you to 299,782 km/sec. It still recedes from you at 299,792 km/sec. I emphasize that it is only the speed of light in \tit{vacuum} that has this special property. The speed of light in water \tit{does} depend on how fast you are moving with respect to the water, though not in an obvious way, as we shall see. Indeed, what is special here is not light, but the speed $c = 299,792,458$ m/sec. When one says \quotes{speed of light} without any qualification, one almost always means the speed of light in vacuum, 299,792,458 m/sec.

How can this be? How can there be a speed $c$ with the property that if something moves at speed $c$ then it must have speed $c$ in any inertial frame of reference? This fact--known as the \tit{constancy of the speed of light} is highly counterintuitive. Indeed, \quotes{counterintuitive} is too weak a word. It seems downright impossible. One of the central aims of this book is to remove this sense of impossibility and to see how it can, in fact, make perfect sense.

$\cdots$

To make sense of the constancy of the speed of light we must look very closely and critically at what it actually means to \quotes{have a speed} with respect to a particular frame of reference. When we say that an object moves uniformly with a certain speed $s$, we mean that it goes a certain distance $D$ in a certain time $T$ and that the distance and time are related by $D/T = s$. We are thus led to examine carefully how one actually measures such distances and how one actually measures such times.

Let $P$ be a valid procedure for carrying out the time and distance measurements that allow one to determine the speed of an object in a given inertial frame. Let Bob, carrying out the procedure $P$ in the frame of reference of a space station, measure the speed of a pulse of light as it zooms off into space. He finds that it moves at about 299,792 km/sec. Suppose Alice flies swiftly after the light at a speed Bob determines to be 792 km/sec. Bob will then (correctly) note that in each second the light gets an additional 299,792 km away from him and Alice gets an additional 792 km away, so that the distance between Alice and the light is growing at only 299,000 km/sec. But if Alice carries out the same procedure $P$ in the frame of reference of her rocket ship, she will find that the speed of light is 299,792 km/sec, so that in her own frame of reference the distance between her and the light is still growing at the full 299,792 km/sec. 

How are we to account for this discrepancy? Obviously the methods Alice uses to measure distances and times must be different from those used by Bob. But don't they use exactly the same procedure $P$? Yes, but you have to think about what \quotes{exactly the same} means. If Bob, for example, uses clocks that are stationary in the frame of his space station to measure times, then if Alice uses exactly the same procedure in her frame of reference, she must use clocks that are stationary in the frame of \tit{her} rocket ship. Thus in Bob's frame of reference Alice's clocks are moving, while his are not, and, of course, vice versa: in Alice's frame Bob's clocks are moving and hers are not. Similar considerations apply to the meter sticks they might use to measure distances. The not terribly subtle but easily ovelooked point is that Bob's procedure \tit{as described in Bob's frame of reference} must be exactly the same as Alice's procedure \tit{as described in Alice's frame of reference}. But Alice's procedure as described in \tit{Bob's} frame of reference is not exactly the same as Bob's procedure as described in \tit{Bob's} frame of reference.

It is this difference that makes it possible for either Bob or Alice to account, in an entirely rational way, for the discrepancy in their conclusions. The fact that Alice and Bob, using different frames of reference, both find exactly the same speed for one and the same pulse of light appears paradoxical only if you make several assumptions about the relation between the clocks and meter sticks used by Alice and Bob. Before 1905, everybody implicitly made all of these assumptions:
\begin{enumerate}[1.]
\item The procedure Alice uses to synchronize all the clocks in her frame of reference gives a set of clocks that Bob agrees are synchronized when he tests them against a set of clocks that he has synchronized using the same procedure in his own frame of reference. (\quotes{Same} here, as earlier, is to be taken to mean that what Bob does has the same description in his own frame of reference as Alice's procedure has in hers.)
\item The rate of a clock, as determined in Bob's frame of reference, is independent of how fast that clock moves with respect to Bob.
\item The length of a meter stick, as determined in Bob's frame of reference, is independent of how fast that meter stick moves with respect to Bob.
\end{enumerate}

If any of these assumptions is false, that we must reexamine the nonrelativistic velocity addition law--the rule specifying how the speed of an object changes as one changes the frame of reference in which its speed is measured. Today we know that \tit{all three} of these assumptions are false. The special theory of relativity gives a quantitative specification of how they fail, and how, when they are suitably corrected, one emerges with a simple and coherent picture of space and time measurements that is entirely in accord with the existence of an invariant speed--a speed that is the same in all inertial frames of reference.

The traditional (and simplest) way to arrive at this picture--the way we shall be taking and the way Einstein used--is simply to accept as a working hypothesis that in any inertial frame of reference, any procedure that correctly measures the speed of light in vacuum \tit{must} give 299,792,458 m/sec. We shall accept the strange fact that if Alice and Bob both measure the speed of one and the same pulse of light, they will both find it to be 299,792,458 m/sec, even though Alice and her measuring instruments may be moving in the same direction as the light with respect to Bob and his. By tentatively accepting this peculiar fact, and insisting that the principle of relativity must remain valid, we will be able to \tit{deduce} the precise way in which each of the three assumptions about the behavior of moving clocks and meter sticks must be modified. Once this is done and the corrected version of these three assumptions are identified and understood, the strange fact will cease to appear strange. More importantly, we will have acquired a firm understanding of the new and wonderful subtleties Einstein first realized about the nature of time. 

This remarkable property of light--that its speed does not depend on the frame of reference in which it is measured--is today called the \tit{principle of the constancy of the velocity of light.} The special theory of relativity is said to rest on two principles: the principle of relativity and the principle of the constancy of the velocity of light. In Einstein's great 1905 paper, he did not use the word \tit{Prinzip} for this second principle (as he did for the first). He characterized each principle as a \quotes{postulate} (Voraussetzung). His second postulate was that light in empty space moves with a velocity that is independent of the velocity of the body that emitted the light. This is tantamount to the second principle when it is conjoined with the first, which Einstein stated as the postulate that the concept of absolute rest has no more meaning for electromagnetic phenomena than it does for phenomena in ordinary Newtonian mechanics.




  

       % Morin - It's About Time - The Speed of Light
\chapter{Combining (Any) Velocities}
\label{ch:Morin_04}

In chapter \ref{ch:Morin_02} we argued that if Alice, a passenger on a train moving at $v$ feet per second, can throw a ball at $u$ feet per second, then if she throws that ball toward the front of the train, its speed $w$ with respect to the tracks will be 
\begin{equation}\label{eq:Morin_04.1}
w = u + v
\end{equation}
in the same direction as the train.

This is known as the nonrelativistic velocity addition law. It is called \quotes{nonrelativistic} because it is only accurate when the speeds $u$ and $v$ are small compared to the speed of light. Evidently it fails to work when $u = c$ (i.e. if Alice turns on a flashlight instead of throwing a ball) for we know that the speed $w$ of the light in the track frame will not be $c + v$ but simply $c$ -- the same value it has in the train frame!

Suppose, however, that Alice fired a gun that expelled bullets whose muzzle velocity $u$ was 90 percent of the speed of light. The \quotes{bullets}, if you insist on getting practical about it, could be pulses of light, traveling down the train in a pipe containing a fluid in which the speed of light was only 0.9 feet per nanosecond. (It is only the speed of the light \tit{in vacuum} that is the same in all frames of reference.) If the addition law (\ref{eq:Morin_04.1}) fails when $u = c$, it would be surprising if the law worked very well when $u$ was $0.9c$ -- and in fact it does not. Both (\ref{eq:Morin_04.1}) and the frame independence of the special velocity $c$ turn out to be a special case of a very general rule for compounding velocities that works whether or not the speeds involved are small compared to the speed of light. This \tit{relativistic velocity addition law} states that
\begin{equation}\label{eq:Morin_04.2}
w = \frac{u + v}{1 + \left( \frac{u}{c}\right) \left( \frac{v}{c}\right)}\,.
\end{equation}

If $u$ and $v$ are both small compared with the speed of light, then $u/c$ and $v/c$ are both small numbers. Their product is then a small fraction of a small number -- i.e. a \tit{very} small number -- so the relativistic rule (\ref{eq:Morin_04.2}) differs from the more familiar nonrelativistic rule (\ref{eq:Morin_04.1}) only by a factor that differs insignificantly from $1$. If, on the other hand, $u = c$, then (\ref{eq:Morin_04.2}) requires $w$ also to be $c$, whatever the value of $v$ may be. This (\ref{eq:Morin_04.2}) is consistent with our nonrelativistic experience -- i.e. with situations in which all relevant speeds are small compared to the speed of light -- as well as with the speed of one and the same pulse of light being the same in all inertial frames of reference.

We shall now show that the more general relativistic rule (\ref{eq:Morin_04.2}) is a direct and immediate consequence of the constancy of the velocity of light, together with the principle of relativity. We shall find that if the speed of light is the same in all inertial frames of reference, then the addition law (\ref{eq:Morin_04.1}) must be replaced by (\ref{eq:Morin_04.2}) regardless of what kind of moving objects we are describing and regardless of how fast they are moving. That so much more general rule follows from the special case of the constancy of the velocity of light, together with the principle of relativity, is a remarkable demonstration of the power of that principle. In its scope, the argument that will lead us to (\ref{eq:Morin_04.2}) is analogous to our extraction of Newton first law of motion in chapter \ref{ch:Morin_01} by applying the principle of relativity to the fact that stationary bodies remain stationary in the absence of an applied force. But while it is obvious, once you have understood the principle of relativity, that the special case of stationary objects remaining stationary implies that uniformly moving objects continue in the same state of uniform motion, the connection between the special case of the constancy of the speed of light and the much more general rule (\ref{eq:Morin_04.2}) is far from obvious.

Before we embark on this important application of the principle of relativity, you might note that the explicit occurrence of the speed $c$ in (\ref{eq:Morin_04.2}), even when none of the objects or frames of reference associated with $u, v, \text{ or } w$ have anything to do with light, gives an early indication that the speed $c$ is built into the very nature of space and time. Things that move at that special speed move at that speed in all frames of reference, as a direct consequence of (\ref{eq:Morin_04.2}) itself. Pulses of light in vacuum happen to be examples of such things. But the speed $c$ has an importance that goes beyond the fact that light moves at that speed in empty space. 

To develop a strategy for deducing the relativistic addition rule (\ref{eq:Morin_04.2}), we must first ask what goes wrong when we try to justify the nonrelativistic rule (\ref{eq:Morin_04.1}). The obvious way to determine the speed of an object is to determine the time it takes it to traverse a racetrack of known length. Doing this requires two clocks, placed at the two ends of the racetrack, to determine the exact times at which the object starts and finishes the race. To arrive at the nonrelativistic velocity addition law (\ref{eq:Morin_04.1}), we implicitly assumed that people using the train frame and people using the track frame would agree on whether those two clocks were synchronized. Prior to Einstein this essential assumption was never explicitly noted. Although people realized that it could be difficult as a practical matter to arrange for two clocks in faraway places to be synchronized, they took for granted that there was nothing about it that was problematic in principle. We also implicitly assumed that the people using different frames of reference would agree on the length of the racetrack between the two clocks and on the rates at which the clocks were running. 
to hit 
The constancy of the velocity of light means that the nonrelativistic addition law (\ref{eq:Morin_04.1}) cannot be correct for an object moving at the speed of light, and therefore it means that at least some of the assumptions on which (\ref{eq:Morin_04.1}) rests must be wrong. This, in turn, casts doubt on the validity of the nonrelativistic addition law for any velocities at all. But if we are not allowed to make such assumptions about the basic instruments with which we measure velocities, how can we deduce the correct rule for compounding velocities? One way to arrive at it would be to figure out, and then take fully into account, a set of new \quotes{relativistic} rules about clock-synchronization disagreements, rates of moving clocks, and lengths of moving measurement sticks, but this takes a bit of doing. It is, in fact the usual way of arriving at the correct relativistic addition law (\ref{eq:Morin_04.2}) in most expositions of the subject. Although we will eventually construct the new set of rules about clocks and measuring sticks, at this stage we don't know any of them. Nevertheless, it is possible and useful to figure out the correct velocity addition law before learning anything about the behavior of moving clocks and measuring sticks, and this is the path we shall follow. 

The direct way to get at (\ref{eq:Morin_04.2}) is to take advantage of the fact that we know the speed of at least one thing: light. By being clever we can use light to help us measure the speed of anything else in a way that makes no use whatever of either clocks or measuring sticks. This enables us to  deduce the rule for how velocities change when the frame of reference changes, without assuming anything at all about their behavior. The idea is to let the moving object -- call it a ball -- run a  race with a pulse of light -- call it a photon. By comparing how far the ball goes with how far the photon goes, we can figure out the speed of the ball. If, for example, the photon, moving at speed $c$, covers twice as much ground as the ball, then the speed of the ball must be $c/2$. (We shall consider only the case in which the ball goes slower than the photon. Later we will see that there is something highly problematic with balls that move faster than light.)

This neat idea runs into an immediate difficulty. Although the photon and the ball  start their race in the same place, they will be in different places at the end of the race. But to compare how much ground they cover during the race, we must be able to determine exactky where the ball is at the precise moment the photon reaches the finish line. To do this we need to synchronize clocks, one at the finish line and one with the ball. We can then determine where the ball is at the moment the photon reaches the finish line, by noting where the ball is when its clock reads exactly the same time that the clock at the finish line reads at the moment the photon gets to the finish line. But this requires knowing whether two clocks in different places are synchronized -- precisely the issue we wished to avoid. 

There is an easy way around this problem. Rather than end the race when the photon reaches the finish line, we arrange for it to hit a mirror and bounce back the way it came. We end the race only when the photon reencounters the ball, which is still moving in its original direction. By ending the race when the photon and the ball arrive at the same place, we solve the problem of determining, without clocks, just where the ball is along the path when the race ends. At the moment the race ends, the ball is precisely where the photon is.

Suppose this is all done on a train. We first describe the race using the train frame, Let the race start at the rear of the train and let the photon be reflected back toward the rear when it reaches the front. Suppose the photon meets the ball a fraction $f$ of the way from the front of the train back to the rear. (If, for example, the train consists of 100 identical cars, numbered $1,2,3,\ldots$ starting from the front, and the photon meets the ball in the passageway between cars 34 and 35, then $f = 0.34$.) Between the beginning and the end of the race, the photon has gone the entire length of the train \tit{plus} an additional fraction $f$ of that length, but between the beginning and end of the race, the ball has gone the entire length of the train \tit{minus} that same fraction of the length. The ratio of the distance covered by the ball to the distance covered by the photon is thus $(1 - f)/(1 + f)$.

Since the photon and the ball were both racing for the same time, this ratio must also be the ratio of their speeds. The fact that they were both racing for the same time is now unproblematic, requiring no clocks to establish it, because we have organized the race so that the photon and the ball are in the same place when they start the race and also when they finish it. So if we call the velocity of the ball in the train frame $u$, then, since the speed of the photon in either direction is $c$, 
\begin{equation}\label{eq:Morin_04.3}
\frac{u}{c} = \frac{1 - f}{1 + f}\,.
\end{equation}

The people on the train have thus measured the speed of the ball without using clocks and without having to know le length of the cars in their train. They only have to be able to count cars\footnote{If the ball met the photon some fraction of the way along a car, they would have to be able to compare the lengths of the two parts of the car, but they could do this without knowing the absolute length of either part by just counting up the number of times some measuring stick of unknown length went into both parts.}. So (\ref{eq:Morin_04.3}) summarizes a simple way to compare the velocities of two objects, which avoids using any clocks and avoids having to know any absolute distances. It is useful to rewrite (\ref{eq:Morin_04.3}) as a relation that expresses the fraction $f$ in terms of the speed $u$ of the ball and the speed of light $c\,$:
\begin{equation}\label{eq:Morin_04.4}
f = \frac{c - u}{c + u}\,.
\end{equation}

Now let us start all over again and analyze a similar race on the train, but this time using the frame of reference of the track, in which the train has a velocity $v$ (which we take to be less than the speed of light) and the ball has a velocity $w$. We take $u, v, \text{ and } w$ all to be positive -- i.e. the ball moves to the right in the train frame, and the train and ball move to the right in the track frame -- so that velocities and speeds are the same; the result we shall arrive at, however, turns out to be valid for any combination of positive and negative velocities. As before, the photon and ball both start at the rear of the train, the photon reaches the front first and bounces back toward the rear, and the race ends when the photon reencounters the ball. We again want to know what fraction of the way back along the train the photon has to go before it meets the ball. We want to express this fraction entirely in terms of the various speeds. This time the analysis is a bit more complicated, since the train is moving during the race.

We continue to assume that the photon moves with speed $c$ in both directions in the track frame. In a little while we are going to appeal to the constancy of the velocity of light to interpret this as exactly the same race as the one we analyzed in the train frame. Meanwhile, however, it might be a good idea to put the first race out of your mind while analyzing this one.  

$\cdots$

To analyze the race in the track frame we shall have to talk about track-frame distances and times. Our goal is to end up with relations like (\ref{eq:Morin_04.3}) or (\ref{eq:Morin_04.4}) that involve no times and lengths. The relations we seek involves only velocities, together with the fraction $f$ of the way back along the train the photon has to go before it meets the ball. All of the unknown distances and times we introduce will drop out at the end. 

Suppose it takes a time $T_0$ for the photon to get from the back of the train to the mirror at the front and a time $T_1$ for the reflected photon to get from the front to the point a fraction $f$ of the way back along the train where it reencounters the ball. Let $L$ be the length of the train and let $D$ be the distance between the front of the train and the ball at the moment the photon reaches the front of the train. All these times and distances are unknown track-frame times and distances, but since the reasoning that follows is entirely track-frame reasoning, and since the problematic quantities 
$D, L, T_0 \text{ and } T_1$ all drop out of the final result, this causes no difficulty. 

$\cdots$

Since $T_0$ is the time it takes the photon to get a distance $D$ ahead of the ball and since both start in the same place at the same moment and move toward the front with speeds $c$ and $w$, we must have 
\begin{equation}\label{eq:Morin_04.5}
D = c T_0 - w T_0\,.
\end{equation}

On the other hand, $T_1$ is the time it takes the photon and ball, initially a distance $D$ apart, to get back together. Since the photon covers a distance $c T_1$ during this time and the ball, $w T_1$, we have 
\begin{equation}\label{eq:Morin_04.6}
D = c T_1 + w T_1\,.
\end{equation}

Since we don't know the value of $D$, we shall eliminate it from these two relations. 
This gives us $c T_0 - w T_0 = c T_1 + w T_1$, which it is convenient to write in the form
\begin{equation}\label{eq:Morin_04.7}
\frac{T_1}{T_0} = \frac{c - w}{c + w} \,.
\end{equation}

But unfortunately we don't know the times $T_1$ and $T_0$. There is, however, a second quite similar way to get at the same ratio of these two times, by comparing the progress of the photon not with that of the ball, as we have just done, but with that of the train. Note first that $T_0$ is the time it takes the photon to get ahead of the rear of the train by the length $L$ of the train. Since the photon has speed $c$ and the train, speed $v$, 
\begin{equation}\label{eq:Morin_04.8}
L = c T_0 - v T_0 \,.
\end{equation}

Note next that $T_1$ is the time it takes the photon, moving toward the rear at speed $c$, to meet a point on the train originally a distance $f L$ away from it that moves toward it at velocity $v$. Thus 
\begin{equation}\label{eq:Morin_04.9}
L = c T_1 + v T_1 \,.
\end{equation}

We don't know the actual value of $L$ any more than we knew the actual value of $D$, but we can also eliminate $L$ from the last two equations. This gives us $c T_1 + v T_1 = f (c T_0 - v T_0)$, which gives us a second expression for the ratio of $T_1$ to $T_0$:
\begin{equation}\label{eq:Morin_04.10}
\frac{T_1}{T_0} = f \left( \frac{c - v}{c + v}  \right) \,.
\end{equation}

Although we don't know either $T_1$ of $T_0$, this expression for their ratio must be the same as the other expression (\ref{eq:Morin_04.7}). We conclude that 
\begin{equation}\label{eq:Morin_04.11}
f \left( \frac{c - v}{c + v}  \right) = \frac{c - w}{c + w} \,.
\end{equation}

This is the relation we need. All unknown times and distances have dropped out, and we have a relation involving only the fraction $f$ and some velocities. It follows immediately from (\ref{eq:Morin_04.11})that the fraction $f$ can be expressed in terms of the velocities $v$ and $w$ by
\begin{equation}\label{eq:Morin_04.12}
f = \left( \frac{c + v}{c - v} \right)\left( \frac{c - w}{c + w} \right) \,.
\end{equation}

I stress that as a piece of track-frame analysis, applicable to a race between a ball with track-frame speed $w$ and a photon with track-frame $c$, both of them on a train with track-frame speed $v$, there is nothing at all peculiar about the analysis leading to (\ref{eq:Morin_04.12}). 

As a reassuring check that we haven't made some mistake, notice the following. Suppose the velocity $v$ of the train in the track frame were $0$, Then the track frame would be the same as the train frame. Consequently $w$, the velocity of the ball in the track frame, would be the same as $u$, the velocity of the ball in the train frame. And, indeed, if you set $v$ to zero and take $w$ to be $u$, you do get back our old train-frame result (\ref{eq:Morin_04.4}). Galileo would have been quite happy with the argument leading to (\ref{eq:Morin_04.12}) (though we would have to turn the train into a boat.) Indeed, the result (\ref{eq:Morin_04.12}) remains entirely correct if we replace the photon by anything at all that moves with the same speed in both directions, and allow that speed, $c$, to be any speed at all greater than $w$ and $v$.

We do something that would not have been of Galileo's liking only when we now declare that if the photon really \tit{is} a photon, and if the speed $c$ really is the speed of light in vacuum, then these two pieces of analysis we have now completed can be taken to be train-frame and track-frame analyses of \tit{one and the same race}. In this race $u$ is the train-frame velocity of the ball, $w$ is the track-frame velocity of that same ball, and $v$ is the track-frame velocity of the train. Peculiarly, however -- and this is the \tit{only} peculiarity in the entire argument -- we now insist that the track-frame speed $c$ of that one photon (in either direction) is exactly the same as the train-frame speed $c$ of that same photon (in either direction). In both frames (and in both directions) that speed is $c$ (1 foot per nanosecond). This is the only place in the entire argument where we invoke the counterintuitive principle of the constancy of the velocity of light. 

But if we have indeed been describing one and the same race in two different frames, then $f$, the fraction of the way back from the front of the train where the photon meets the ball, must have the same value in either frame. For although there might (and indeed, as we shall see, there will) be disagreement between the two frames of reference over the length of the cars of the train, there can be no disagreement about where on the train the photon meets the ball. Their reunion could trigger an explosion, for example, that would make a smudge on the floor, which all observers in all frames could inspect later on at their leisure to confirm in which part of the car the meeting tool place. 

So the track-frame expression (\ref{eq:Morin_04.12}) for the fraction $f$ must agree with the train-frame expression (\ref{eq:Morin_04.4}). Setting them equal gives us a relation between the three velocities $w$, $u$, and $v$:
\begin{equation}\label{eq:Morin_04.13}
\left( \frac{c + v}{c - v} \right)\left( \frac{c - w}{c + w} \right) = \left( \frac{c - u}{c + u} \right)\,.
\end{equation}

It is useful to rewrite this relativistic velocity addition law in a form, like the form of the nonrelativistic addition law (\ref{eq:Morin_04.1}), in which $w$ appears on the left side and $u$ and $v$ on the right:
\begin{equation}\label{eq:Morin_04.14}
\left( \frac{c - w}{c + w} \right) = \left( \frac{c - u}{c + u} \right)  \left( \frac{c - v}{c + v} \right)\,.
\end{equation}

This is the relativistic rule that replaces the nonrelativistic rule (\ref{eq:Morin_04.1}). Instead of \tit{adding} $u$ and $v$ to get $w$ we must \tit{multiply} an expression involving $u$ by an expression of the same form involving $v$ to get a third expression of the same form invoving $w$. 

The relation between the nonrelativistic rule (\ref{eq:Morin_04.1}) and the relativistic rule (\ref{eq:Morin_04.14}) is not at all clear. To see that they are, in fact, rather simply related, one must carry out the elementary algebraic exercise of solving (\ref{eq:Morin_04.14}) for the velocity $w$ of the ball in the track frame in terms of its speed $u$ in the train frame and the speed $v$ of the train. The result is the relativistic velocity addition law stated in (\ref{eq:Morin_04.2}):
\begin{equation}\label{eq:Morin_04.15}
w = \frac{u + v}{1 + \left( \frac{u}{c}\right) \left( \frac{v}{c}\right)}\,.
\end{equation}

Although the two forms (\ref{eq:Morin_04.14}) and (\ref{eq:Morin_04.15}) of the velocity addition law are entirely equivalent ways of expressing the same relation among the three velocities $w$, $u$, and $v$, it is helpful to keep them both in mind, since one form can be more useful than the other, depending on the question one is asking. Thus the form (\ref{eq:Morin_04.15}) makes immediately evident (as noted at the beginning of this chapter) why the nonrelativistic addition law $w = u + v$ becomes quite accurate when $u$ and $v$ are small compared to the speed of light. The form (\ref{eq:Morin_04.14}), on the other hand, directly reveals the following important fact:

If the speed $u$ of the ball in the train frame and the speed $v$ of the train in the track frame are both less than the speed of light, then both $(c - u)/(c + u)$ and $(c - v)/(c + v)$ will be numbers between $0$ and $1$, this means that $(c - w)/(c + w)$ is between $0$ and $1$, which implies in turn that the speed $w$ of the ball in the track frame is also less than the speed of light. This conclusion also follows from (\ref{eq:Morin_04.15}), of course, but it is more evident as a consequence of (\ref{eq:Morin_04.14}).

Thus the obvious stratagem for producing an object moving faster that light does not work: if you have a cannon that shoot balls at 90 percent of the speed of light, and you put it on a train moving at 90 percent of the speed of light, pointing toward the front of the train, then the speed of the ball in the track frame will still be less than the speed of light. Indeed, in this particular case (\ref{eq:Morin_04.15}) tells us that the speed $w$ of the ball in the track frame will be a fraction 
$$\frac{0.9 + 0.9} {1 + (0.9)^2} = \frac{1.80}{1.81}$$ of the speed of light -- about 99.45 percent. This is the first indication we have found -- there will be several others -- that no material object can travel faster than the speed of light.

For many purposes it is helpful to abstract the relativistic addition law from the context of balls, trains and tracks and state it in terms of the velocities of certain objects(or frames of reference) with respect to other objects (or frames of reference). Let us regard the track as an object called $A$, the train as an object called $B$, and the ball as an object called $C$. The velocity $v$ of the train in the track frame we now call $v_{BA}$ -- the velocity of $B$ with respect to $A$. In the same way we call the velocity $u$ of the ball in the train frame $v_{CB}$, the velocity of $C$ with respect to $B$, and we call the velocity $w$ of the ball in the track frame, $v_{CA}$. In this language the two forms for the addition law become 
\begin{equation}\label{eq:Morin_04.16}
\left( \frac{c - v_{CA}}{c + v_{CA}} \right) = \left( \frac{c - v_{CB}}{c + v_{CB}} \right)  \left( \frac{c - v_{BA}}{c + v_{BA}} \right)\,.
\end{equation}

and 

\begin{equation}\label{eq:Morin_04.17}
v_{CA} = \frac{v_{CB} + v_{BA}}{1 + \left(\frac{v_{CB}}{c}\right) \left(\frac{v_{BA}}{c}\right)}\,.
\end{equation}

Another advantage of (\ref{eq:Morin_04.16}) over (\ref{eq:Morin_04.17}) emerges when you consider the case in which object $C$ is a rocket that itself emits a fourth object $D$. If $D$ has speed $v_{DC}$ with respect to $C$ what is the speed $v_{DA}$ of $D$ with respect to $A$? In other words, what form does the addition law take when we compount three speeds instead of just two? This leads to a great mess if we try to answer the question using form (\ref{eq:Morin_04.17}), but if we use the addition law in the form (\ref{eq:Morin_04.16}), we merely note the following:

The speed of $D$ with respect to $A$ can be arrived at by compounding the speed of $D$ with respect to $C$ and the speed of $C$ with respect to $A$. Applying the general rule (\ref{eq:Morin_04.14}) to this case gives 
\begin{equation}\label{eq:Morin_04.18}
\left( \frac{c - v_{DA}}{c + v_{DA}} \right) = \left( \frac{c - v_{DC}}{c + v_{DC}} \right)  \left( \frac{c - v_{CA}}{c + v_{CA}} \right)\,.
\end{equation}

But now we can apply (\ref{eq:Morin_04.14} again to express the quantity containing $v_{CA}$ in terms of $v_{CB}$ and $v_{BA}$ to get
\begin{equation}\label{eq:Morin_04.19}
\left( \frac{c - v_{DA}}{c + v_{DA}} \right) = \left( \frac{c - v_{DC}}{c + v_{DC}} \right)  \left( \frac{c - v_{CB}}{c + v_{CB}} \right) \left( \frac{c - v_{BA}}{c + v_{BA}} \right) \,.
\end{equation}

So to compound three speeds rather than just two, we just put a third term into the product in (\ref{eq:Morin_04.16}) to get (\ref{eq:Morin_04.19}). Evidently if $D$ were a rocket that emitted a fifth object $E$ we could continue in this way, and so indefinitely. The rule in the form (\ref{eq:Morin_04.17}) would get more and more complicated, but in the form (\ref{eq:Morin_04.16}) it would retain the same simple form.

The additional law in either of its two forms (\ref{eq:Morin_04.17}) or (\ref{eq:Morin_04.16}) continues to hold even when not all the velocities have the same sign -- even, for example, when the ball moves toward the rear of the train rather than the front. If Alice throws a ball with speed $u$ toward the \tit{rear} of a train that moves with positive velocity $v$ along the track, then the velocity $w$ of the ball along the track is given by
\begin{equation}\label{eq:Morin_04.20}
w = \frac{-u + v}{1 - \left( \frac{u}{c}\right) \left( \frac{v}{c}\right)}\,,
\end{equation}

since this is what (\ref{eq:Morin_04.2}) reduces to when $u$ is replaced by $-u$. (It is a useful exercise to check this by repeating the analysis of this chapter for the case where the race starts at the front of the train rather than at the rear.)

An easier if more abstract way to see that (\ref{eq:Morin_04.16}) and (\ref{eq:Morin_04.19}) remain valid even when all velocities are not positive is to note that, although we have derived (\ref{eq:Morin_04.16}) in the case where all three of the velocities $v_{CA}$, $v_{CB}$, and $v_{BA}$ are positive, we can introduce negative velocities by exploiting the general fact that 
\begin{equation}\label{eq:Morin_04.21}
v_{YX} = -v_{XY}\,.
\end{equation}

We can, for example, express the positive $v_{BA}$ in (\ref{eq:Morin_04.16}) in terms of the negative velocity $v_{AB}$, writing $v_{BA} = - v_{AB}$. Having done this we can then algebraically transform (\ref{eq:Morin_04.16}) into 
\begin{equation}\label{eq:Morin_04.22}
\left( \frac{c - v_{CB}}{c + v_{CB}} \right) = \left( \frac{c - v_{CA}}{c + v_{CA}} \right)  \left( \frac{c - v_{AB}}{c + v_{AB}} \right)\,.
\end{equation}

Notice that this has exactly the same form as (\ref{eq:Morin_04.16}) -- only the labels $A$ and $B$ have been interchanged. But now one of the three velocities, $v_{AB}$, is negative.

Similar tricks using (\ref{eq:Morin_04.21}) enable one to reexpress (\ref{eq:Morin_04.16}) in other equivalent forms in which either or both of the two velocities on the right are negative.

Remarkably, an experiment confirming the validity of the relativistic velocity addition law (\ref{eq:Morin_04.2}) was performed in 1851 by Fizeau, a few years after he measured the speed of light and more than half a century before Einstein wrote his 1905 paper in which the relativistic addition law first appears. 

$\cdots$
 







 





      % Morin - It's About Time - Combining (Any) Velocities
\chapter{Simultaneous Events; Synchronized Clocks}
\label{ch:Morin_05}
The puzzlement we feel at the fact that a given pulse of light has the same speed in both the track frame and the train can be traced to a deeply ingrained misconception about the fundamental nature of time. Until we learn otherwise -- and prior to Einstein in 1905, nobody had learned otherwise -- we implicitly believe that there is an absolute meaning to the simultaneity of two events that happen in different places, independent of the frame of reference in which the events are described. This assumption is so pervasive in our view of the world that it is built into the very language we speak, making it difficult to reexamine the question of what it actually means to assert that two events in different places are simultaneous.

Before embarking on such a reexamination, it is necessary to take a careful look at the term \tit{event}, which plays a fundamental role in the relativistic description of the world. An event is something that happens at a definite place at a definite time. It is, if you like, the space-time generalization of the purely spatial geometric notion of point.

If you look at this a little more closely,you realize that, like the concept of a point, the concept of an event is an idealization. No object we can actually get our hands on has the property of zero spatial extension that characterizes a geometric point, and no process we will be talking about has zero extension in both space and time.

(Although zero extension in time is captured by the word \quotes{instantaneous,} I know of no English word -- other than \quotes{pointlike,} if that is a word -- that signifies zero extension in space.) Whether or not we wish to view something as an event depends on the spatial and temporal differences we want to discriminate between. If, for example, the relevant time scale is years and the relevant distance scale is hundreds of kilometers, then it makes sense to view the meeting of a class between 1:25 and 2:40 in room 115 of Rockefeller Hall on the Cornell University campus in Ithaca, New York, as an event (at least in frames of reference that are not moving too rapidly with respect to the Earth). But if the relevant scales are minutes and feet, it does not. So a phenomenon can be viewed as constituting a single event in a given frame of reference, it its temporal and spatial extension in that frame are both small compared with all other times and distances of interest. All of the events we will be examining will qualify as events in all the frames of reference we are interested in -- they can be viewed as space-time points.

How can we decide whether two different events, happening in different places, that are simultaneous in the train frame are also simultaneous in the track frame? To be concrete, suppose one event consists of quickly making a tiny mark on the tracks (as they speed past) from the rear of the train, and the other consists of doing the same from the front. The two events could be anything else you like $\cdots$ But since it will be useful to mark the spot along the tracks where each event occurs, it is simplest to take the two events to be nothing more that two acts of marking the tracks.

How does Alice, who used the train frame, persuade herself that the two marks on the tracks are made at the same time? Well, she could provide both ends of the train with accurate clocks and confirm that each mark was made when the clock at its end of the train read noon. But how can she be sure that the two clocks are properly \tit{synchronized}? How does she know that they both read noon \tit{at the same time}?

Trying to check the simultaneity of the two events with clocks gets Alice nowhere, since confirming that the clocks are properly synchronized requires her to have precisely what we're trying to construct: a way to confirm that two events in different places  -- in this case, each clock reading noon -- happen \tit{at the same time}.

This is a centrally important point. It is useful to have two clocks in different places only if they are properly synchronized. But \quotes{synchronized} means that the clocks have the same reading \tit{at the same time}. Therefore you need a way to check that two events in different places are simultaneous, if you want to check that the two clocks in different places are synchronized. The question of whether clocks in different places are synchronized, and the question of whether events in different places are simultaneous, are simply different aspects of the same fundamental puzzle. You can answer one question if and only if you can answer the other.

Let uis try again. Alice could bring the two clocks together, directly confirm that they read the same time when they're both in the same place, and then have them carried to the two ends of the train. But how does she know how fast each clock was running as it was carried to its end of the train?Faced with a phenomenon as peculiar as the constancy of the velocity of light, it would be rash to assume she knew anything about the rate at which a clock runs when it is moving. (We will learn how to deal with this in chapter \ref{ch:Morin_06}.) The straightforward way to check on whether the clocks have done anything peculiar while being carried to the two ends of the train would be to compare what each reads when it gets there with the reading of a stationary clock at that end of the train. But we can only do this if those two stationary clocks are properly synchronized, which brings us right back to the same problem.

Ah, but suppose, even though we don't know how it might affect their rates, the two clocks start at the exact middle of the train and are carried to the two ends in exactly the same way except, of course, that one of the two clock-transportation procedures is executed in the opposite direction from the other. Then, however erratically its motion caused one clock to behave during the journey, the other, having experienced just the same kind of trip, would have run erratically in exactly the same way. So even if they lost or gained time because of their motion, the two clocks would still agree when they arrived at the ends of the train. That method of providing both ends of the train with synchronized clocks ought to work. And it does! In the train frame.

But now we are faced with another problem Even if Alice did cleverly use two such centrally synchronized, symmetrically transported clocks to confirm that two events at the two ends of the train were simultaneous in the train frame, Bob, using the track frame, would not agree that the two clock-transportation procedures were identical, because in the track frame motion toward the front of the train is \tit{not} insignificantly different from motion toward the rear.

Bob \tit{would} agree with Alice that the reading of one of her clocks, the instant it arrived at its end of the train, was the same as the reading of the other clock, the instant it arrived at the other end, since Alice and Bob can't disagree about things that happen \tit{both} in the same place \tit{and} at the same time. But Bob need not agree with Alice that the identical readings of the clocks meant that it took an identical amount of time for each clock to get to its end, since the clocks were not moving symmetrically in the track frame and therefore might be running at different rates during their journeys from the center to the ends. So Bob will have to do a rather elaborate calculation to determine whether each clock reached its end of the train \tit{at the same time} as the other clock. That calculation would have to figure out how fast each of the clocks was moving in the track frame, and how far it had to go. It could get quite complicated. It can, howvere, be done, and it leads to a remarkable conclusion that we can extract by a much simpler stratagem.

Teh simpler stratagem, like our earlier method for finding the relativistic velocity addition law, avoids all worries about possibly misbehaving clocks by using in the train frame a nethod to check teh simultaneity of two events in different places that makes no use of clocks at all. This method can easily be analyzed in the track frame too. The analysis relies only on the fact that the speed of light is $c$ -- 1 foor per nanpsecond -- regardless of the direction the light is moving in and regardless of the frame of reference in which the speed is measured.

Why, you might ask, should we build such a strange fact into our procedure for determining whether two spatially separated events are simultaneous? If you do ask, then you have forgotten why we started worrying about whether simultaneity might depend on frame of reference. It was our hope that this might lead us to a clearer understanding of the constancy of the velocity of light. Whet we are doing is perfectly sensible. We \tit{start} from the strange fact of the constancy of the velocity of light, and see what it \tit{forces} us to conclude about the simultaneity of events. We shall find that it forces us to conclude that whether two events in different places are simultaneous does indeed depend on the frame of reference in a way that can be stated sim,ply and precisely.

Note first that Alice, on the train and using the train frame, can easily exploit the fact that light travels with a definite speed $c$ to arrange that the two marks on the tracks are made from the two ends of the train simultaneously. She places a lamp exactly halfway along the train, and then turns on the lamp. Light from the lamp races toward both ends of the train at the same speed $c$. Since the light has to travel the same distance (half the length of the train) in either direction, and moves at the same speed in either direction, it arrives at the two ends of the train \tit{at the same time}. So if the making of each mark on the tracks is triggered by the arrival of light, the marks will certainly be made at the same time. 

Alice has thus managed to produce a pair of events in different places that are simultaneous (in the train frame) without having to make any problematic use of clocks. This procedure, of course, works for any two signals that move from the center of the train to the two ends, as long as they move at the same speed. If the common speed of the signals is not the speed of light, however, the track frame analysis cannot be as concise as it is with light signals, because the two signals have different speeds in the track frame. The two speeds can be found with the help of the relativistic velocity addition law, and with considerably more effort one can generalize to arbitrary signals the analysis based on light signals. One then finds (see pp. 54-56) that this more general way to produce two simultaneous events in the train frame leads to a relation between the track-frame time and track frame distance between the two events that is exactly the same as the relation we now extract with much less effort using light signals. 

The question facing us is how this procedure, which convinced Alice that the two events are simultaneous in the train frame, will be interpreted by Bob, who uses the track frame. Bob will certainly agree that the lamp is indeed in the center of the train, for if the train is 100 cars long and the lamp is bolted down between cars 50 and 51, then there is no denying that it is indeed in the center\footnote{This is true even if the length of the train in the track frame is altered by its motion -- as we shall see in chapter \ref{ch:Morin_06} it is -- because whatever that alteration may be, it would be the same for both the front half and the rear half of the train.}. But in the track frame when the lamp is turned on and the light the light starts to move toward the two ends, the rear of the train moves toward the place where the light originated and the front moves away from it. Since the speed of the light in either direction is $c$ in the track frame -- remember we are using the strange fact that the speed of the light is 1f/sec in the track frame even though it is also 1 f/sec in the train frame -- in the track frame it will clearly take light less time to reach the rear of the train, which is heading toward the light to meet it, than it will take the light to reach the front of the train, which is running away as the light pursues it.

So Bob must conclude thatthe light reaches the rear of the train before it reaches the front, and therefore that the mark in the rear is made before the mark in the front. The very same evidence that convinces Alice, using the train frame, that the marks are made simultaneously, convinces Bob, using the track frame, that they are not made simultaneously. \tit{Whether or not two events at different places happen at the same time has no absolute meaning -- it depends on the frame of reference in which the events are described.} \footnote{Notice, in passing, that if you interchange the two words \quotes{place} and \quotes{time} in this sentence, the shocking assertion becomes quite humdrum: \tit{Whether or not two events at different times happen at the same place has no absolute meaning -- it depends on the frame of reference in which the events are described.}}

People using the train frame, for whom the marks are made simultaneously, cau use the arrival of the light signals to synchronize clocks at the front and rear of the train. Since people in the track frame maintain that the mark in the rear is made \tit{before} the mark in the front, the track people would also maintain that the synchronization procedure used by the train people had actually resulted in the clock in the front of the train being behind the clock in the rear. 

It is easy to find a precise quantitative measure of these disagreements, Let's analyze Alice's procedure for making marks simultaneously at both ends of the train, from the point of view of Bob in the track frame, where the train moves with speed $v$. It is convenient to call the length of the train $L$. I emphasize that by $L$ we mean the length of the train \tit{in the track frame}. Although we are used to assuming that the length of an object is independent of the frame in which it is measured, we can no longer take this for granted and, as noted earlier, we will indeed find it to be a false assumption. 

In part 1 of figure 5.1, the light is turned on in the middle of the train, and the two pulses of light -- which we shall again call photons -- start moving from the center toward the fromnt and the rear. 

Part 2 of figure 5.1 shows things a time $T_r$ later, just as the rearward moving photon meets the rear of the train, which has been moving toward it. At the instant of encounter a mark is made at the place on the tracks where the meeting takes place. During the time $T_r$ the photon (which moves with speed $c$) has covered a distance $cT_r$. That distance is just half the length of the train, reduced by the distance the rear of the train (which is moving with speed $v$) has moved toward the photon in the time $T_r$. So
\begin{equation}\label{eq:Morin_05.1}
c T_r = \frac{1}{2} L - v T_r \,.
\end{equation}

 

      % Morin - It's About Time - Simultaneous Events; Synchronized Clocks


\part{Robert M. Wald\\ {\small \quotes{Advanced Classical Electromagnetism}. (Annotated transcript)}}
\chapter{Special Relativity}
\label{ch:Wald_08}

Special relativity is the theory of spacetime structure formulated by Einstein in 1905. Properties of the electromagnetic field played a central role in motivating special relativity. Specifically, electromagnetism is not compatible with pre-relativity notions of spacetime structure unless there is a preferred rest frame (the \quotes{aether}), since, as we have seen, Maxwell's equations predict that electromagnetic waves propagate with a particular velocity $c$, which can only be true in some preferred rest frame if pre-relativity notions of spacetime structure are valid. The Michelson-Morley experiment failed to find such a preferred rest frame. Furthermore, as Einstein realized, some physical phenomena in electromagnetism appear to obey an invariance with respect to moving observers even if the description of these phenomena in terms of a preferred rest frame does not have such an invariance.

The theory of electromagnetism is far more elegant and simple when formulated in the framework of special relativity. It therefore is somewhat of a travesty that, well into the twenty-first century, special relativity is discussed here--as in other texts on electromagnetism--as a separate chapter toward the end of the book. The reason, of course, is that even though special relativity has been a well-established theory for much more than a century, its basic concepts are still so unfamiliar to most physicists that it is not feasible to begin the treatment of electromagnetism by giving its formulation in the framework of special relativity. It is my hope that this situation will be rectified by the twenty-second century. 

Einstein original formulation of special relativity relied heavily on the transformations between the labeling of events by different inertial observers and the invariance of the laws of physics under such transformations. The theory was reformulated in a much more geometrical manner by Minkowski in 1908, wherein it was recognized that the underlying structure of spacetime in special relativity is that of a spacetime metric\footnote{Minkowski introdued an imaginary time coordinate so as to obtain a Euclidean sacetime metric. However, although this approach remains in use in many treatments of special relativity, it does not generalize to curved spacetime and cannot be used in general relativity. We shall use a real time coordinate in our treatment, and our spacetime metric will therefore be of Lorentian signature.}.

Our tretament of special relativity will emphasize the role of the spacetime metric. Although Einstein was initially unimpressed by Minkowski's reformulation, he soon incorporated it into his thinking about gravitation. This led him to the theory of general relativity, wherein the spacetime metric becomes a dynamical variable that describes not only spacetime structure but also all the effects of gravity. However, we shall not discuss general relativity here. 

The framework of special relativity is presented in section \ref{sec:Wald_08.1}. The formulation of electromagnetism in the framework of special relativity is then given in section \ref{sec:Wald_08.2}. In section \ref{subsec:Wald_08.3.1}, we analyze the motion of a (relativistic) charged particle in an external electromagnetic field, including the solution for motion in a uniform electric field and in a uniform magnetic field. The Lienard-Wiechert solution describing the retarded field of a point charge in arbitrary motion is given in section \ref{subsec:Wald_08.3.2}, and properties of the radiated power for this solution are analyzed there as well.

\section{The Framework of Special Relativity}
\label{sec:Wald_08.1}
It is useful to think of space and time as composed of \quotes{events}--where each event corresponds to a point of space at an instant of time. The collection of all events comprises a 4-dimensional continuum, which I refer to as \quotes{spacetime}. 

I take as a starting point that there exist global families of inertial observers who \quotes{fill} all of spacetime (i.e., within each family, one and only one of these observers passes through each event in spacetime). I further assume that the observers in each family are all \quotes{at rest} with respect to one another, that they can consistently synchronize their clocks by some physical procedure, and that the spatial relationships between these observers are described by Euclidean geometry. Finally, I also assume that different families of such inertial observers all move at uniform velocity with respect to one another, so that the different families may be labeled by their valocity $v$ with respect to some reference family. These assumptions are true in both pre-relativity physics and in special relativity, so they would make a very poor starting point from a fundamental viewpoint. 

By the above assumptions, the inertial observers in a given family can uniquely label events by $(t, \vb{r})$, where $t$ denotes the time of the event and $\vb{r} = (x, y, z)$ are the spatial Cartesian coordinates of that event for that observer. I refer to the labeling of events in this way as \tit{inertial coordinates}. The assumption that events can be labeled in this way is implicit in every physics text or other reference where a \quotes{$t$} or \quotes{$\vb{r}$} appears in an equation. However, this labeling depends on (i) a choice of origin of time (i.e., what time is labeled as $t=0$); (ii) a choive of origin of space (i.e., what observer in the family is at $\vb{r}= \vb{0}$); (iii) a choice of orientation of axes (to define the $x-$, $y-$, and $z-$directions); and, most importantly for our present purposes, (iv) a choice of which family of inertial observers to use (i.e., a choice of the velocity $\vb{v}$ of the family). Different choices of origin of $t$ and $\vb{r}$, orientation of axes, and $\vb{v}$ will give rise to different labelings $(t', \vb{r}')$ of events. 

Usually, treatments of special relativity focus entirely on the difference in labeling of events between families of inertial observers who are moving with velocity $\vb{v}$ relative to one another. This is given by a Galilean transformation in pre-relativity physics and by a Lorentz transformation in special relativity. Although it certainly is useful to know the explicit form of this transformation, a nearly exclusive focus on this obscures the geometrical content of the theory. It is analogous to studying ordinary Euclidean geometry by focusing on how Cartesia coordinates transform under rotations. 

I thefore focus on the \quotes{invariant structure} of spacetime. The four numbers $(t, \vb{r})$ associated with an event do not, by themselves, convey meaningful information about the event, since they depend as much, for example, on the choice of origin in $t$ and $\vb{r}$ as they do on the event itself. Even if we consider the differences $(\Delta t, \Delta \vb{r})$ in the labeling of two events by a given family of inertial observers so as to eliminate the origin dependence, the values of $(\Delta t, \Delta \vb{r})$ will depend on the choice of orientation of axes as well as on the choice of family of inertial observers. It is of great interest to determine what quantities constructed out of $(\Delta t, \Delta \vb{r})$ are invariant (i.e., independent of these choices). Such quantities are well defined without making any arbitrary choices and hence can be considered as attributable to the structure of spacetime itself. 

In pre-relativity physics, there are two such invariant quantities: (1) the time interval $\Delta t$ between events, and (ii) the space interval $\abs{\Delta{\vb{r}}}^2$ between simultaneous events (i.e., events with $\Delta t = 0$). The space interval between nonsimultaneous events is not invariant, because if the family $O'$ of inertial observers moves with velocity $\vb{v}$ with respect to the family $O$, then we have 
\begin{equation}\label{eq:Wald_08.1}
\Delta \vb{r}' = \Delta \vb{r} - \vb{v} \Delta t \,,
\end{equation}

so $\abs{\Delta{\vb{r}'}}^2 \neq \abs{\Delta{\vb{r}}}^2$ if $\Delta t \neq 0$. In addition, the collection of \quotes{worldlines} of inertial observers (i.e., the possible paths in spacetime of inertial observers) also can be viewed as an additional aspect of spacetime structure. The worldlines of inertial observers contain additional information independent of (i) and (ii) in that they cannot be constructed from knowing only $\Delta t$ for all pairs of events, and knowing $\abs{\Delta{\vb{r}}}^2$ for simultaneous events. 

The situation in special relativity is simpler. In special relativity, there is a single invariant quantity, the specetime interval $I$ between any pair of events, given by
\begin{equation}\label{eq:Wald_08.2}
I = - c^2 (\Delta t)^2 + \abs{\Delta{\vb{r}}}^ 2\,.
\end{equation}

Furthermore, it can be shown that the worldlines of inertial observers can be constructed from a knowledge of $I$ between all pairs of events. Thus, $I$ provides the complete description of spacetime structure in special relativity.

To tie the previous paragraph to what people are usually taught in special relativity, note that, in special relativity, the labeling $(t, \vb{r})$ of events by a family $O$ of observers is related to the labeling $(t', \vb{r}')$ by a family $O'$ of observers moving with velocity $v$ in the $x$-direction relative to $O$ (and with the same origin event and the same orientation of axes as $O$) by a \tit{Lorentz transformation}:
\begin{equation}\label{eq:Wald_08.3}
\begin{aligned}
t' &= \gamma (t - vx/c^2)\,,\\
x' &= \gamma (x - v t)\,,\\
y' &= y\,,\\
z' &= z\,,
\end{aligned}
\end{equation}
where 
\begin{equation}\label{eq:Wald_08.4}
\gamma \equiv \frac{1}{\sqrt{1 - v^2/c^2}}\,.
\end{equation}

It is easily checked that $I$ is invariant under Lorentz transformations. Furthermore, it can be shown that the most general transformation that preserves $I$ is a \tit{Poincaré transformation} (i.e., a composition of Lorentz transformations, rotations, and translations, as well as parity and time reversal transformations). Thus, Lorentz transformations naturally arise as (part of) the symmetry group of $I$.

The spacetime interval $I$ has the same mathematical form as the squared distance in Euclidean geometry except for the minus sign in front of the contribution coming from  $(\Delta t)^2$. To pursue this further, we switch notation from $(t, \vb{r})$ to $x^\mu$ with $\mu=0,1,2,3\,$, where 
\begin{equation}\label{eq:Wald_08.5}
x^0 = ct\,,\:x^1 = x\,,\:x^2 = y\,,\:x^3 = z\,.
\end{equation}

Note the superscript position of $\mu$ in $x^\mu$, which will be important in order to align with notational conventions explained further below. We view $x^\mu$ as representing a spacetime displacement vector (relative to some origin in spacetime) in much the same way as we normally view $\vb{r}$ as representing a spatial displacement vector (relative to some origin in space). We view $I$, eq. (\ref{eq:Wald_08.2}), as arising from an \quotes{inner product} on spacetime displacement vectors, where the inner product of $x_1^\mu$ and $x_2^\mu$ is given in any inertial coordinates by 
\begin{equation}\label{eq:Wald_08.6}
I(x_1^\mu, x_2^\mu) = \sum_{\mu, \nu=0}^3 \eta_{\mu\nu} x_1^\mu x_2^\nu\,,
\end{equation}
where\footnote{Many authors  define $\eta_{\mu\nu}$ with an opposite sign convention, which results in sign changes in some formulas. The reader is advised to check the sign convention used for $\eta_{\mu\nu}$ when comparing formulas in different references. As mentioned in footnote 1 in this chapter, some authors use an imaginary time coordinate, in which case $\eta_{\mu\nu}$ would take a Euclidean form and normally would not be written down explicitly at all.}
\begin{equation}\label{eq:Wald_08.7}
\eta_{\mu,\nu} \equiv \mqty( -1 & 0 & 0 & 0 \\ 
                              0 & 1 & 0 & 0 \\
                              0 & 0 & 1 & 0 \\
                              0 & 0 & 0 & 1 )\:.
\end{equation}

I put \quotes{inner product} in quotes, because although $I$ is linear in each variable, symmetric and nondegenerate (i.e., $I(x_1^\mu, x_2^\mu) = 0$ for all $x_2^\mu$ if and only if $x_1^\mu = 0$), it fails to be positive definite. Nevertheless, it is closely analogous to the inner product on vectors in ordinary Euclidean geometry, 
\begin{equation}\label{eq:Wald_08.8}
(\vb{x}_1^\mu, \vb{x}_2^\mu) = \vb{x}_1 \cdot \vb{x}_2 = \sum_{i,j=1}^3 e_{i,j} \vb{x}^i_1 \vb{x}^j_2\,,
\end{equation}
where 
\begin{equation}\label{eq:Wald_08.9}
e_{i,j} \equiv \mqty( 1 & 0 & 0 \\
                      0 & 1 & 0 \\
                      0 & 0 & 1 )\:.
\end{equation}

We refer to $e_{i,j}$ as the \tit{metric of space} in Euclidean geometry. Similarly, we refer to $\eta_{\mu,\nu}$ as the \tit{metric of spacetime} in special relativity. 

It should be mentioned that the ability to give finite spatial or spacetime displacements a vector space structure, as implicitly assumed in the discussion above, is very special to \tit{flat} geometry. In a curved geometry--such as the 2-dimensional surface of a potato--there is no natural notion of adding two finite displacements about a point. Nevertheless, in a curved geometry, a notion of \quotes{infinitesimal displacements} about any point $p$ can be defined, as these infinitesimal displacements have a vector space structure. The vector space of infinitesimal displacements about $p$ is referred to as the \tit{tangent space} at $p$. In differential geometry, a metric would be defined as a (not necessarily positive-definite) inner product defined on the tangent space at $p$. However, the spacetime geometry of special relativity is flat. So we may treat \tit{finite} spacetime displacements $\Delta x^\mu \equiv x^\mu - x^\mu(p)$ about a point $p$ as \quotes{vectors}--and we can treat the metric as an inner product on these finite displacement vectors--as we have done above.

A striking feature of the spacetime metric eq.(\ref{eq:Wald_08.7}) is that there are nonzero spacetime displacement vectors $\Delta x^\mu$ about an event $p$ that are \tit{null}, in other words, such that 
\begin{equation}\label{eq:Wald_08.10}
\sum_{\mu, \nu} \eta_{\mu\nu} \Delta x^\mu \Delta x^\nu = 0\,.
\end{equation}

The collection of all null spacetime displacement vectors arount $p$ comprise a cone with vertex at $p$. The portion of the cone with $\Delta x^0 \geq 0$ is referred to as the \tit{future light cone} of $p$, whereas the portion with $\Delta x^0 \leq 0$ is referred to as its \tit{past light cone}. In the precise sense discussed in [ACE] chapter 5, electromagnetic radiation emitted at $p$ propagates along the future light cone of $p$ (see [ACE] section 5.2), whereas radiation observed at $p$ propagated to $p$ along its past light cone (see [ACE] section 5.4). The interior of the future light cone of $p$ is referred to as the \tit{future} of $p$. It is composed of events that, in principle, can be reached by an observer initially present at $p$. Similarly, the interior of the past light cone of $p$ is referred to as the \tit{past} of $p$. It is composed of events with the property that an observer starting at that event can, in principle, arrive at $p$. The events lying outside the light cone of $p$ are said to be \tit{spacelike related} to $p$. No observer can be present both at $p$ and at an event specelike related to $p$. In other words, in special relativity, nothing can travel faster than light.

If $q$ is an event that lies in the future of $p$, then there is a unique inertial observer who is present at both $p$ and $q$. The proper time $\Delta \tau$ that elapses on a clock carried by this observer between $p$ and $q$ is given in any inertial coordinates by\footnote{This can be seen by noting that the right side of eq.(\ref{eq:Wald_08.11}) is the spacetime interval between $p$ and $q$ and does not depend on the choice of inertial coordinates. In the frame of the inertial observer who goes from $p$ to $q$, we have $\Delta \vb{r} = 0$, so $\sum \eta_{\mu \nu} \Delta x^\mu \Delta x^\nu = -c^2 (\Delta t)^2\,.$}
\begin{equation}\label{eq:Wald_08.11}
\Delta \tau = \frac{1}{c} \sqrt{- \sum_{\mu, \nu} \Delta x^\mu \Delta x^\nu }\,,
\end{equation}
where  $\Delta x^\mu = x^{\mu}(q) - x^{\mu}(p)$. A general noninertial observer will trace out a curve in spacetime, called the \tit{worldline} of the observer. Any curve in spacetime may be specified by giving $x^{\mu}(\lambda)$, where $\lambda$ is an arbitrary parametrization of the curve. The tangent $T^\mu$ to the curve in this parametrization is defined by 
\begin{equation}\label{eq:Wald_08.12}
T^\mu =  \dv{x^{\mu}}{\lambda} \,.
\end{equation}
The tangent $T^\mu$ to the worldline of any observer must be timelike 
(i.e., $\sum_{\mu,\nu} \eta_{\mu\nu} T^\mu T^\nu < 0$), since the observer must stay within the light cone of any event that he/she passes through. The proper time elapsed on the clock of an arbitrary noninertial observer going between events $p$ and $q$ is given by
\begin{equation}\label{eq:Wald_08.13}
\Delta \tau = \frac{1}{c} \bigints_{\lambda(p)}^{\lambda(q)} {d\lambda \: \sqrt{- \sum_{\mu, \nu} T^\mu T^\nu} }  \,.
\end{equation}

It is not difficult to show that the inertial observer who passes through events $p$ and $q$ maximizes the elapsed proper time relative to all observers who pass between $p$ and $q$. This fact is often referred to as the \tit{twin paradox} (see problem 1 at the end of the chapter).

It is convenient to parametrize the worldline of an arbitrary observer by the proper time $\tau$ elapsed along the worldline. The tangent to the worldline of an arbitrary observer in this parametrization is called the 4-\tit{velocity} of the observer:
\begin{equation}\label{eq:Wald_08.14}
u^\mu = \dv{x^\mu}{\tau} \,.
\end{equation}
It follows from eq. (\ref{eq:Wald_08.13}) with $\lambda = \tau$ that $\sum_{\mu, \nu} \eta_{\mu\nu} u^\mu u^\nu = - c^2$. In any inertial coordinates, the velocity $\vb{v}$ of this observer is given by
\begin{equation}\label{eq:Wald_08.15}
\vb{v} = \dv{\vb{r}}{t} = c \dv{\vb{r}}{x^0} =  c \frac{d\vb{r}/d\tau}{dx^0/d\tau} = c \frac{\vb{u}}{u^0} \,.
\end{equation}
It follows from this relation and the normalization condition on $u^\mu$ that, in any inertial coordinates, the components of $u^\mu$ are given in terms of $\vb{v}$ by 
\begin{equation}\label{eq:Wald_08.16}
u^\mu = (c \gamma, \gamma \vb{v})\,,
\end{equation}
where $\gamma$ is given by eq.(\ref{eq:Wald_08.4}).

As discussed above, because of the flat spacetime geometry, the finite displacements $\Delta x^\mu = x^\mu - x^\mu(p)$ about $p$ comprise a 4-dimensional vector space, which we denote by $V_p$. The tangent $T^\mu$ to a curve passing through $p$ can naturally be viewed as a vector in $V_p$. For an arbitrary spacetime vector, we will strictly adhere to the notational convention that an \tit{upper} Greek letter index will be attached to the letter representing the vector (e.g., when a symbol such as $W^\mu$ appears below, one can immediately tell from the notation that $W^\mu$ is a spacetime vector).

Spacetime vectors in special relativity have a geometrical status in exactly the same sense that ordinary vectors have a geometrical status in ordinary, 3-dimensional space. Although the representation of a spacetime vector in terms of its components depends on a choice of inertial coordinate system, one may view the spacetime vector has having a geometrical meaning that does not depend on a choice of coordinates---in exactly the same manner as an ordinary vector can be viewed as having a geometrical meaning, independent of the representation of its components in a particular Cartesian coordinate system. 

Another class of objects that have a similar geometrical status are linear maps taking vectors to numbers. A linear map taking $V_p$ into $\R$ is called a \tit{dual vector}. The collection of all dual vectors at $p$ comprises a vector space of the same dimension as $V_p$, known as the \tit{dual space} to $V_p$ and denoted as $V^*_p$. Given any vector $W^\mu$ in $V_p$, we can use the spacetime metric $\eta_{\mu\nu}$ to construct a dual vector $L_W$ taking $V_p$ into $\R$ by
\begin{equation}\label{eq:Wald_08.17}
S^\mu \longrightarrow \sum_{\mu,\nu}\eta_{\mu\nu} S^\mu W^\nu\,,
\end{equation}
that is, $L_W$ maps the arbitrary vector $S^\mu$ into the number given by the right side of this equation. It is not difficult to show that, in the presence of a metric, all dual vectors arise in this manner. It is natural to denote the dual vector $L_W$ by  
\begin{equation}\label{eq:Wald_08.18}
W_\mu = \sum_{\nu=0}^3 \eta_{\mu\nu} W^\nu\,,
\end{equation}
so that the linear map $L_W$ is given by $$S^\mu \longrightarrow \sum_\nu S^\nu W_\nu\,.$$
In this notation, the lowered index on $W_\mu$ on the left side of eq. (\ref{eq:Wald_08.18}) indicates that this quantity is the dual vector obtained from $W^\mu$ rather than $W^\mu$ itself. For an arbitrary dual vector, we will strictly adhere to the notational convention that a \tit{lower} Greek letter index will be attached to the letter representing the dual vector (e.g., when a symbol such as $U_\mu$ appears below, one can immediately tell from the notation that $U_\mu$ is a spacetime dual vector).

Since the spacetime metric is nondegenerate, eq. (\ref{eq:Wald_08.18}) can be inverted to give 
\begin{equation}\label{eq:Wald_08.19}
W^\mu = \sum_{\nu=0}^3 \eta^{\mu\nu} W_\nu\,,
\end{equation}
where $\eta^{\mu\nu}$ denotes the inverse spacetime metric, whose matrix of components is given by the inverse matrix\footnote{In an inertial coordinate system, the components of $\eta_{\mu\nu}$ are given by eq. (\ref{eq:Wald_08.7}) (see [ACE] eq. (5.121)),and the inverse of eq.(\ref{eq:Wald_08.7}) then has exactly the same components as eq. (\ref{eq:Wald_08.7}) (see [ACE] eq. (5.121)). However, $\eta^{\mu\nu}$ and $\eta_{\mu\nu}$ are fundamentally very different objects, and this equality of their components would not hold in a general, noninertial coordinate system.} of $\eta_{\mu\nu}$. Following standard conventions, if $U_\mu$ is a dual vector, we denote the corresponding vector given by eq. (\ref{eq:Wald_08.19}) as $U^\mu$. Thus, we use $\eta^{\mu\nu}$ and $\eta_{\mu\nu}$ to raise and lower indices in the manner given by eq. (\ref{eq:Wald_08.18}) and eq. (\ref{eq:Wald_08.19}).

Vectors and dual vectors are fundamentally very different objects. Nevertheless, when a metric is present, vectors and dual vectors can be identified via the correspondence eq. (\ref{eq:Wald_08.18}) or equivalently, eq. (\ref{eq:Wald_08.19}). When dealing with a vector 
$\vb{v}$ in ordinary Euclidean space, the components $(v^1, v^2, v^3)$ of the vector in Cartesian coordinates are equal to the components  $(v_1, v_2, v_3)$ of the corresponding dual vector on account of the trivial form, eq. (\ref{eq:Wald_08.9}) of the Euclidean metric. Thus, in this case, if one treats a vector as a 3-tuple of numbers, one can get away with not making a distinction between a vector and its corresponding dual vector.\footnote{In this regard, it should be noted that in the previous chapters of this book, we denoted the Cartesian components of vectors such as the electric field $\vb{E}$ as $E_i$ rather than $E^i$. This is not because we were working with the corresponding dual vector but instead because there was no need to make a distinction between vectors and dual vectors and, correspondingly, no need to adhere to the conventions on the index positions that we have just introduced. We wrote \quotes{$E_i$} simply because it was typographically more convenient to write $E_i$ rather than $E^i$. However, from this point forward in this book, we will strictly adhere to our conventions on index positions.} This accounts for why many students of physics have never explicitly encountered the notion of a dual vector. However, in special relativity, in any inertial coordinates, the dual vector $W_\mu$ corresponding to $W^\mu$ has $W_0 = - W^0$, and we cannot get away with ignoring the distinction between vectors and dual vectors.

% BOX 1 **START** -- << Summary of notation conventions >>
\parindent=0pt  % Set the paragraph indentation to 0 (normal = 10pt) before the box 
\parbox{\textwidth}{\begin{mdframed}[style=MyFrame] %Added "\parbox{\textwidth}{"
%\lipsum[1]
Let us summarize the different notation conventions in use for vectors in 3-dimensional Euclidean space, versus those adopted for vectors and dual vectors in 4-dimensional \tit{spacetime}:
\begin{enumerate}[i)]
\item A \tbf{boldface} \tit{letter} is normally used to indicate the \tit{vector} character of an object (e.g.: $\vb{r}$) in 3-dimensional space. The object components with respect to a coordinate frame are denoted by plain font letters, like in 
\quotes{$\vb{r} = (x, y, z) $}.  
\item Objects in 4-dimensional spacetime are identified by a letter in plain font (e.g.: $x$) and one or more lower and upper indices. 
\item A \tit{subscript} or \tit{superscript} might be useful to denote distinct objects identified by the same \tit{letter} (e.g.: $\vb{r}_1, \ldots , \vb{r}_k$). This practice is \tbi{discouraged} in 4-dimensional \tit{spacetime}, where \tit{subscripts} or \tit{superscripts} generally need to appear beside that same \tit{letter} to specify \tbi{either} the \tit{tensorial character} of the object \tbi{or} to denote a specific \tit{component} of the latter.
\item A plain font letter with a Greek \tit{superscript} (e.g. $x^\mu$) denotes either a 4-dimensional \tit{vector} (i.e. a (1,0) type, rank 1 tensor) or the $\mu$-th component of the latter (e.g., when the \tit{letter} appears in the expression of a sum, its \tit{superscript} being one of the \tit{indices} being summed over).     
\item A plain font letter with a Greek \tit{subscript} (e.g. $x_\nu$) denotes either a 4-dimensional \tit{dual vector} (i.e. a (0,1) type, rank 1 tensor) or the $\nu$-th component of the latter (e.g., when the \tit{letter} appears in the expression of a sum, its \tit{subscript} being one of the \tit{indices} being summed over).  
\end{enumerate} 
%\lipsum[2]
\end{mdframed}} %Added a closing "}" here
\parindent=10pt % Set the paragraph indentation to normal (10pt) 
% BOX 1 ** END ** -- << Summary of notation conventions >>

A prime example of a dual vector at $p$ is the spacetime gradient $\delta_\mu f = \delta f / \delta x^\mu$ of a function evaluated at $p$. This is seen to be a dual vector as follows. For any curve with tangent $T^\mu$ at $p$, the quantity $\sum_\mu T^\mu \delta_\mu f$ represents the derivative $df / d \lambda$ of $f$ along the curve and thus is well defined, independently of any choice of coordinates. Thus $\delta_\mu f$ at $p$ naturally can be viewed as a linear map taking tangent vectors $T^\mu$ at $p$ into numbers (i.e., a dual vector). Note that in ordinary Euclidean space, the gradient of $f$ would normally be denoted by $\grad f$ in vector calculus notation (i.e., it would be treated as a vector rather than a dual vector). This is because the correspondence between vectors and dual vectors that is provided by the Euclidean metric is being used. Fundamentally, however, the gradient of $f$ is a dual vector.

Another important example of a dual vector arises when we consider plane wave solutions or, more generally, solutions in the geometric optics approximation. Let $\psi$ be a plane wave solution of the scalar wave equation
\begin{equation}\label{eq:Wald_08.20}
\psi(t, \vb{r}) = C \vexp{-i \omega t + i \vb{k}\cdot\vb{r} }\,.
\end{equation}

The phase of $\psi$, 
\begin{equation}\label{eq:Wald_08.21}
\Phi \equiv -\omega t +  \vb{k}\cdot\vb{r} \,,
\end{equation}
may be viewed as a linear map taking spacetime displacement vectors $x^\mu = (ct, \vb{r})$ into numbers. Thus 
\begin{equation}\label{eq:Wald_08.22}
k_\mu = (-\omega/c, k_1, k_2, k_3)
\end{equation}
defines a spacetime dual vector. More generally, for any $\psi$ of the spacetime geometric optics form (see [ACE] problem 1 of chapter 7)
\begin{equation}\label{eq:Wald_08.23}
\psi(t, \vb{r}) = \A(t, \vb{r}) \vexp{\Phi(t, \vb{r})}\,,
\end{equation}
the quantity 
\begin{equation}\label{eq:Wald_08.24}
k_\mu = \delta_\mu \Phi
\end{equation}
defines a dual vector field on spacetime. These results also apply to plane wave and geometric optics solutions to Maxwell's equations. It is important to note that, as shown in [ACE] problem 1 of chapter 7, the corresponding vector field 
\begin{equation}\label{eq:Wald_08.25}
k^\mu = \sum_\nu \eta^{\mu\nu} k_\nu
\end{equation}
is null, $$ \sum_{\mu\nu} \eta^{\mu\nu} k^\mu k^\nu = 0\,,$$
and it is tangent to straight lines (\quotes{light rays}) in spacetime. The upshot is that for plane waves--or, more generally, geometric optics solutions--the \tit{wave} 4-\tit{vector}
\begin{equation}\label{eq:Wald_08.26}
k^\mu = (\omega/c, \vb{k})
\end{equation}
plays a role for light rays that is very similar to the role played by the 4-velocity $u^\mu$ for particle motion. In particular, the change in frequency and direction of a light ray as seen by a moving observer follows from the fact that $k^\mu$ is a spacetime vector (see problem 4).

It should be noted that vectors and dual vectors transform differently under Lorentz transformations. Under the Lorentz transformation eq. (\ref{eq:Wald_08.3}) with origin at $p$, a vector $W^\mu$ at $p$ transforms as 
\begin{equation}\label{eq:Wald_08.27}
W^\mu \longrightarrow {W'}^\mu = \sum_\nu {\Lambda^\mu}_\nu \, W^\nu \,,
\end{equation}
where the matrix of components of the Lorentz transformation ${\Lambda^\mu}_\nu$ is given by\footnote{The \quotes{1 up} and \quotes{1 down} index positions in $\Lambda^\mu_\nu$ correspond to the fact that $\Lambda^\mu_\nu$ is a linear map on vectors and hence is a tensor of type (1,1) (see the discussion later in this section).} 
\begin{equation}\label{eq:Wald_08.28}
{\Lambda^\mu}_\nu = \mqty(  \gamma     & -\gamma v/c & 0 & 0 \\
                         -\gamma v/c &  \gamma     & 0 & 0 \\
                             0       &       0     & 1 & 0 \\
                             0       &       0     & 0 & 1 ) \;.
\end{equation}

In contrast, a dual vector $U_\mu$ transforms as\footnote{Note the index position of the summed index $\nu$ in eq. (\ref{eq:Wald_08.29}), which would correspond to taking the transpose of the matrix $\Lambda^{-1}$ in usual matrix multiplication rules. Note also that the Cartesian coordinate components of a rotation matrix $R^\mu_\nu$ in Euclidean geometry satisfy 
$(R^{-1})^\nu_\mu = R^\mu_\nu$, so the transformations eq.(\ref{eq:Wald_08.27}) and eq.(\ref{eq:Wald_08.29}) take the same form for rotations in Euclidean geometry, consistent with being able to treat vectors and dual vectors as \quotes{the same} in this case.}
\begin{equation}\label{eq:Wald_08.29}
U_\mu \longrightarrow {U'}_\mu = \sum_\nu {(\Lambda^{-1})^\nu}_\mu \, U_\nu \,,
\end{equation}
where ${(\Lambda^{-1})^\nu}_\mu$ is the inverse of ${\Lambda^\mu}_\nu$. It is easily checked that ${(\Lambda^{-1})^\mu}_\nu$ is given by the same formula as eq. (\ref{eq:Wald_08.28}) with $v \rightarrow - v$. The relations eq. (\ref{eq:Wald_08.18}) and eq. (\ref{eq:Wald_08.19}) are preserved under Lorentz transformations on account of the invariance of the metric under Lorentz transformations in the sense that (see problem 2)
\begin{equation}\label{eq:Wald_08.30}
\eta_{\alpha\beta} = \sum_{\mu,\nu} {(\Lambda^{-1})^\mu}_\alpha {(\Lambda^{-1})^\nu}_\beta \, \eta_{\mu\nu} \,.
\end{equation}

The important thing about vectors and dual vectors is not the specific formulas describing how they transform under Lorentz transformations but rather that they represent geometrical quantities that are well defined, independently of any choice of inertial coordinates. The worldline of a particle in spacetime is a well-defined curve that has physical meaning independently of how the events in spacetime are labeled with inertial or other coordinates. The 4-velocity $u^\mu$ of the particle (i.e., the normalized tangent to its worldline) similarly has a well-defined physical meaning. It is therefore reasonable that it could enter the physical laws of motion of the particle. The fact that $u^\mu$ transform according to eq. (\ref{eq:Wald_08.27}) under a change of inertial coordinates merely reflects that it has a well-defined geometrical meaning as a vector. Similarly, the gradient $\delta_\mu f$ of a function $f$ has a well-defined meaning as a dual vector, and the fact that it transforms as eq. (\ref{eq:Wald_08.29}) is merely a reflection of this. If $f$ is a physical quantity, it is reasonable that $\delta_\mu f$ could enter laws of physics involving $f$.

A tensor is a more general geometrical quantity than vectors and dual vectors that can be defined in a mathematical precise manner as follows: A \tit{tensor of type (k,l)} at $p$ is a multilinear (i.e., linear in each variable separately) map taking $k$ dual vectors and $l$ vectors at $p$ into $\R$. Thus, a tensor of type (0,1) is a linear map from vectors into numbers, i.e., it is a dual vector. A tensor of type (1,0) is a linear map from dual vectors into numbers. This might sound like yet another new object, but it is not difficult to see that such a \quotes{double dual vector} can be naturally identified with an ordinary vector,\footnote{This can be seen as follows. The action of a dual vector $U_\mu$ on a vector $S^\mu$ is given by $S^\mu \rightarrow \sum_\mu U_\mu S^\mu$. However, $\sum_\mu U_\mu S^\mu$ could equally well be viewed as defining an action of $S^\mu$ on dual vectors given by $U_\mu \rightarrow \sum_\mu U_\mu S^\mu$. In this manner, we may view any vector $S^\mu$ as corresponding to a double dual vector.} 
so a tensor of type (1,0) is just a vector. A tensor of type (0,2) is a map that takes a pair of vectors into $\R$ and is linear in each variable. The spacetime metric $\eta_{\mu\nu}$ is an example of a tensor of type (0,2). A tensor $T$ of type (1,1) is a map that takes a dual vector and a vector into $\R$ and is linear in each variable. A tensor $T$ of type (1,1) can be naturally identified\footnote{
To see this, note that for any $v \in V_p$, the quantity $T(\cdot, v)$ is a linear map from $V^*_p$ into $\R$ and thus is a double dual vector, which can be identified with a vector. Thus, $T$ can be uniquely associated with a linear map from $V_p$ into $V_p$.} 
with a linear map taking $V_p$ into $V_p$ (and also can be identified with a linear map taking $V^*_p$ into $V^*_p$). 

Following the index conventions given above for vectors and dual vectors, we shall denote a tensor $T$ of type $(k,l)$ with 
$k$ upper and $l$ lower Greek indices, 
that is as ${T^{\mu_1\ldots\mu_k}}_{\nu_1\ldots\nu_l}$, 
where all the indices range from 0 to 3. 
We can use $\eta^{\mu\nu}$ and $\eta_{\mu\nu}$ to raise and lower any of the indices in the manner given by 
eq. (\ref{eq:Wald_08.18}) and eq. (\ref{eq:Wald_08.19}), thereby enabling us to convert a tensor of type $(k,l)$ into a tensor 
of any other type with the same rank, where the rank $r$ of a tensor is defined by $r = k + l$. 
Note that a general tensor of rank $r$ on a 4-dimensional space has $4^r$ independent components. 

Under a Lorentz transformation, ${T^{\mu_1\ldots\mu_k}}_{\nu_1\ldots\nu_l}$ transforms as follows:
\begin{align}\label{eq:Wald_08.31}
\begin{split}
& {T^{\mu_1\ldots\mu_k}}_{\nu_1\ldots\nu_l} \longrightarrow {{T'}^{\mu_1\ldots\mu_k}}_{\nu_1\ldots\nu_l} \\
& \:\:\:=\: \sum_{\alpha_1\ldots\beta_l} 
{\Lambda^{\mu_1}}_{\alpha_1} \cdots {\Lambda^{\mu_k}}_{\alpha_k} \cdots {(\Lambda^{-1})^{\beta_1}}_{\nu_1} \cdots {(\Lambda^{-1})^{\beta_l}}_{\nu_l}\; {T^{\alpha_1\ldots\alpha_k}}_{\beta_1\ldots\beta_l} \,.
\end{split}
\end{align}

In other words, if a tensor of type $(k, l)$ has components ${T^{\mu_1\ldots\mu_k}}_{\nu_1\ldots\nu_l}$ in some inertial coordinates $x^\mu$, then the components of the tensor in the inertial coordinate system ${x'}^\mu = \sum_\nu {\Lambda^\mu}_\nu x^\nu$ are given by eq. (\ref{eq:Wald_08.31}).\footnote{Here we are taking a \quotes{passive view} of Lorentz transformations, wherein the tensor does not change, but its description in terms of coordinate components changes. Alternatively, we could take an \quotes{active view} of Lorentz transformations, wherein the inertial coordinate system is fixed but the Lorentz transformation is viewed as acting on the tensor, mapping the tensor $T$ at $x^\mu$ to a new tensor $T'$ at ${x'}^\mu$ via the formula (\ref{eq:Wald_08.31}). These views are equivalent (i.e., we obtain the same results by making a passive Lorentz transformation as we would by making an \tit{active} transformation by the \tit{inverse} of this Lorentz transformation).} 
Again, the important thing about a tensor is not the explicit formula eq. (\ref{eq:Wald_08.31}) for how its components transform under a change of inertial coordinates but the fact that it represents a well-defined object, independent of any choice of coordinates. 

A \tit{tensor field} of type $(k, l)$ is the specification of a type $(k, l)$ tensor at $p$ for app $p$. It is convenient to define a tensor of type $(0, 0)$ to be a number, in which case a function may be viewed as a tensor field of type $(0, 0)$.
We define the derivative of a tensor field ${T^{\mu_1\ldots\mu_k}}_{\nu_1\ldots\nu_l}$ of type $(k, l)$ to be the tensor field $\partial_\alpha {T^{\mu_1\ldots\mu_k}}_{\nu_1\ldots\nu_l}$  of type $(k, l+1)$ given in any inertial coordinates by 
\begin{equation}\label{eq:Wald_08.32}
\partial_\alpha {T^{\mu_1\ldots\mu_k}}_{\nu_1\ldots\nu_l} = \pdv{{T^{\mu_1\ldots\mu_k}}_{\nu_1\ldots\nu_l}}{x^\alpha} \;.
\end{equation}
 
Although this formula may look obvious/trivial, this simple notion of differentiation of a tensor field relies heavily on the flatness of the geometry of spacetime.\footnote{The notion of the gradient of a function (i.e., the derivative of type $(0, 0)$) does not require a flat geometry for the definition of eq. (\ref{eq:Wald_08.32}) to yield a dual vector, but a flat geometry is needed to use eq. (\ref{eq:Wald_08.32}) to define the derivative of a higher rank tensor in a coordinate invariant way.} In a curved geometry, there is still a unique, well-defined notion of differentiation of tensor fields that is determined by the metric, but this notion (usually referred to as \quotes{covariant differentiation}) is not obtained by merely taking partial derivatives in a coordinate system. However, we are not concerned with curved geometries here, and the more general notion of differentiation in a curved geometry reduces to eq. (\ref{eq:Wald_08.32}) in inertial coordinates in a flat geometry. Thus, eq. (\ref{eq:Wald_08.32}) is entirely satisfactory for our purposes.

We now turn to the implications of what we have said about the laws of physics in special relativity. We want the mathematical objects representing physical quantities to be well defined, using only the structure of spacetime in special relativity. Similarly, we want the equations satisfied by these objects to be well defined, using only the special relativistic structure of spacetime. This suggests the following two criteria for the formulation of the laws of physics in special relativity: (i) Physical quantities should be represented by spacetime tensors/tensor fields. (ii) The equations satisfied by these tensors/tensor fields should involve only these tensor fields, the spacetime metric, and operations that map spacetime tensors into spacetime tensors, \footnote{An important example of such an operation is the trace ${T^{\mu_1\ldots\mu_k}}_{\nu_1\ldots\nu_l} \rightarrow \sum_\alpha {T^{\mu_1\ldots\alpha\ldots\mu_k}}_{\nu_1\ldots\alpha\ldots\nu_l} $ over one upper index and one lower index, which is a well-defined operation taking a tensor of type $(k, l)$ into a tensor of type $(k-1, l-1)$. The fact that this operation is independent of the choice of inertial coordinates can be seen from eq. (\ref{eq:Wald_08.31}).}
such as the notion of differentiation given by eq. (\ref{eq:Wald_08.32}). These criteria correspond to the more common statement that in special relativity, the laws of physics should be invariant under Lorentz transformations. Any relation equating two tensors will be well defined independently of any choice of inertial coordinates and thus automatically will remain valid under a change of inertial coordinates. We refer to a theory satisfying  properties (i) and (ii) as being \tit{special relativistic covariant}.

The implications of the previous paragraph are well illustrated by considering the nature of the laws of particle mechanics in special relativity. In pre-relativity physics, we define the momentum $\vb{p}$ of a particle by 
\begin{equation}\label{eq:Wald_08.33}
\vb{p} = m \vb{v} = m \dv{\vb{r}}{t}\,,
\end{equation}

where $m$ denotes the mass of the particle, which is usually assumed to be constant. Newton's second law then takes the form 
\begin{equation}\label{eq:Wald_08.34}
\dv{\vb{p}}{t} = \vb{f}\,,
\end{equation}

where the form of the force $\vb{f}$ depends on the particular situation considered. However, spatial vectors have no well-defined status in special relativity, nor does $t$, so eq. (\ref{eq:Wald_08.34}) as it stands is not an acceptable law of physics in special relativity. We have two possibilities for formulating an acceptable version of the laws of particle mechanics: (a) It could be that eq. (\ref{eq:Wald_08.34}) is actually already special relativistically covariant, and all that needs to be done is formulate/rewrite this equation in such a way that it involves only spacetime tensors. (b) Equation (\ref{eq:Wald_08.34}) does not correspond to a special relativistically covariant equation, in which case it must be discarded and replaced by a new law that involves only spacetime tensors. In the present case, it is possibility (b) that holds. Thus, we must find a new law of particle mechanics---reducing to eq. (\ref{eq:Wald_08.34}) for nonrelativistic motion---that involves only spacetime tensors. 

The only reasonable candidate for a spacetime tensor that could represent the velocity of a particle is its 4-veocity, $u^\mu$. The only reasonable candidate for a spacetime tensor that could represent the momentum of the particle is 
\begin{equation}\label{eq:Wald_08.35}
p^\mu = m u^\mu\,.
\end{equation}
We refer to $p^\mu$ as the 4-momentum of the particle and, in special relativity, the parameter $m$ is usually referred to as the \quotes{rest mass} of the particle (to distinguish it from what some authors call the \quotes{relativistic mass}, $p^0/c = \gamma m$). The only reasonable candidate for a modified version of Newton's second law that involves only tensor quantities is 
\begin{equation}\label{eq:Wald_08.36}
\sum_\nu u^\nu \partial_\nu p^\mu = f^\mu\,.
\end{equation}
where the 4-force $f^\mu$ depends on the particular situation considered. Thus, the principles of the previous paragraph lead, in an essentially unique way, to the modification eq. (\ref{eq:Wald_08.36}) of Newton's second law that is special relativistically covariant. 

Note that $u^\mu \partial_\mu = d/d\tau$ (i.e., $u^\mu \partial_\mu$ is the derivative along the worldline of the particle using the proper time parametrization). Thus we may rewrite eq. (\ref{eq:Wald_08.36}) in the form
\begin{equation}\label{eq:Wald_08.37}
\frac{d}{d\tau} \left(m \frac{d x^\mu}{d \tau} \right) = f^\mu\,.
\end{equation}

It is also worth noting that if we assume that the 4-force $f^\mu$ is orthogonal to the 4-velocity $u^\mu$, then
\begin{equation}\label{eq:Wald_08.38}
\begin{aligned}
0 &= \sum_{\mu, \nu} \eta_{\mu, \nu} u^\mu f^\nu = \sum_{\mu, \nu} \eta_{\mu, \nu} u^\mu \dv{p^\nu}{\tau} \\
  &= \frac{1}{2} m \dv{\tau} \left(  \sum_{\mu, \nu} \eta_{\mu, \nu} u^\mu u^\nu \right) + \dv{m}{\tau} \eta_{\mu, \nu} u^\mu u^\nu \\
  &= -c^2 \dv{m}{\tau}\,. 
\end{aligned}
\end{equation}

Thus, if $\sum_{\mu, \nu} \eta_{\mu, \nu} u^\mu f^\nu = 0\,,$ then the rest mass of the particle does not change as the particle moves along its worldline.

The example of special relativistic particle mechanics also provides a good illustration of the manner in which spacetime tensors combine together---into a single quantity---quantities that would have been viewed as entirely distinct in pre-relativity physics. (This occurred whether or not the dynamical law themselves need to be modified to be manifestly special relativistically covariant.) In pre-relativity physics, the energy of a particle, $E = \frac{1}{2} m v^2\,,$ is a scalar quantity, and one has the freedom to add an arbitrary constant to it. It is entirely distinct from the momentum of the particle, $\vb{p} = m\vb{v}\,.$ However, in special relativity, these quantities are identified as the components of the 4-momentum $p^\mu$ of the particle:
\begin{equation}\label{eq:Wald_08.39}
(E/c, \vb{p}) = p^\mu = (m u^0, m \vb{u}) = (\gamma m c, \gamma m \vb{v})
\end{equation}

This gives rise to new formulas for momentum and energy, namely $\vb{p} = \gamma m \vb{v}\,,$ and $E = \gamma m c^2\,$. Most importantly, since $E$ is now the component of a spacetime vector rather than a scalar quantity, there is no longer any freedom to modify its definition by the addition of a constant. Thus, a particle at rest must be assigned the energy $E = m c^2$.

We conclude this section by stating two standard notational conventions that will be used throughout the rest of this book: (i)Up until this point, we have explicitly written the metric $\eta_{\mu \nu}$ wherever it has been used to convert vectors to dual vectors via the correspondence eq. (\ref{eq:Wald_08.18}). We will no longer continue to do so (e.g., having introduced the 4-velocity $u^\mu$, we will simply write $u_\mu$ for $\sum_\nu \eta_{\mu \nu} u^\nu$). Similarly, we will not write the inverse matrix explicitly when converting a dual vector to a vector via the eq. (\ref{eq:Wald_08.19}) (e.g., we will write $\delta^\mu f$ for $\sum_\nu \eta^{\mu \nu} \delta_\nu f\,$). (ii) We will omit the summation sign $\sum$ when summing over a repeated upper and lower index (e.g., we will simply write $u^\mu \delta_\mu f$ for $\sum_m u^\mu \delta_\mu f\,$). The omission of the summation sign in such sums was introcuced by Einstein and is referred to as the \tit{Einstein summation convention}. As a simple example of the notational conventions (i) and (ii) that will be used freely below, the first equality in (\ref{eq:Wald_08.38}) will now be written as $0 = u^\mu f_\mu$ (or, equivalently, as $0 = u_\mu f^\mu$). 

\section[Covariant Electromagnetic Theory]{The Formulation of Electromagnetic Theory in the Framework of Special Relativity}
\label{sec:Wald_08.2}

We wish to formulate the theory of electromagnetism in a manifestly special relativistically covariant manner. The discussion of electromagnetism (in the [ACE] chapters 1-7) thus far has involved spatial vector fields, such as $\vb{A}$, $\vb{E}$, and $\vb{B}$. Such spatial vectors have no well-defined status in special relativity. In accord with the discussion at the end of the previous section, we must either rewrite electromagnetic theory so that it involves only spacetime tensor fields, or we must discard the theory and replace it with a new one that does. In the case of Newtonian particle mechanics, 
we had to do the latter (i.e., we discarded the pre-relativity version, eq(\ref{eq:Wald_08.34}), of Newton's second law and replaced it with the modified version, eq. (\ref{eq:Wald_08.36}), that is special relativistically covariant). Fortunately, in the case of electromagnetism, we need only do the former; we may simply rewrite the theory in a manner wherein it can be explicitly seen to involve spacetime tensors. Thus, in particular, all the equations and results regarding electromagnetism obtained in the chapters 1-7 of this book remain valid in special relativity.

Our task is to reformulate all the equations of section 5.1 of the [ACE] book in spacetime tensor form.  The first step is to make sense of $\vb{A}$. The most straightforward way of incorporating a spatial vector, such as $\vb{A}$, into the framework of special relativity would be to make it part of a spacetime vector $A^\mu$. In fact, this can be done quite easily by combining the scalar potential $\phi$ with $\vb{A}$ to define a 4-vector:
\begin{equation}\label{eq:Wald_08.40}
A^\mu \equiv (\phi/c, \vb{A})\,.
\end{equation}
In other words, instead of viewing $\phi$ as a scalar function on spacetime--as would have been natural in pre-relativity physics--we now interpret $\phi/c$ as being the time component of a 4-vector field whose spatial componens are $\vb{A}$. 

In fact, it is more natural and convenient to work with the corresponding dual vector field (see eq. (\ref{eq:Wald_08.18})):
\begin{equation}\label{eq:Wald_08.41}
A_\mu \equiv (- \phi/c, A_1, A_2, A_3)\,,
\end{equation}
where the spatial components $(A_1, A_2, A_3)$ of $A_\mu$ are numerically equal to $(A_1, A_2, A_3) = \vb{A}$. A general gauge transformation eq. [ACE] book 5.1 then takes the form\footnote{
Note that the zeroth component of this equation is $-\phi/c \rightarrow -\phi/c + \pdv{\chi}{x^0} =  -\phi/c + (1/c) \pdv{\chi}{t}$, that is, $\phi \rightarrow \phi - \pdv{}{\chi}{t}$, that is, $\phi \rightarrow \phi - \pdv{\chi}{t}$ }
\begin{equation}\label{eq:Wald_08.42}
A_\mu \rightarrow {A'}_\mu = A_\mu + \delta_\mu \chi \,.
\end{equation}
Thus, we have written the potentials and their gauge transformations in a special relativistically covariant form. 

Next, we define the \tit{electromagnetic field-strength tensor} by
\begin{equation}\label{eq:Wald_08.43}
F_{\mu \nu} = \partial_\mu A_\nu - \partial_\nu A_\mu\,,
\end{equation}

so that $F_{\mu \nu}$ is a spacetime tensor field of type (0,2). Since $F_{\mu \nu}$ is antisymmetric, $F_{\mu \nu} = -F_{\nu \mu}$, it is easily seen that $F_{\mu \nu}$ has only $6$ independent components. These independent components of $F_{\mu \nu}$ can be taken to be
\begin{equation}\label{eq:Wald_08.44}
E_i/c = F_{i 0} = - \frac{1}{c} \pdv{\phi}{x^i} - \pdv{A_i}{x^0}\,, \quad 
B_i = \frac{1}{2} \sum_{j,k=1}^3 \epsilon_{ijk}F_{jk} = \sum_{j,k=1}^3 \epsilon_{ijk}\pdv{A_k}{x^j}\,,
\end{equation}
where $\epsilon_{ijk}$, defined by [ACE] eq. (4.6), is the fully antisymmetric \tit{Levi Civita} tensor. In more explicit terms, the components of the tensor $F_{\mu \nu}$ defined by eq. (\ref{eq:Wald_08.43}) are  
\begin{equation}\label{eq:Wald_08.45}
F_{\mu \nu} = \mqty(
0      &  -E_1/c   &   -E_2/c    &   -E_3/c  \\
E_1/c  &    0      &      B_3    &   -B_2    \\          
E_2/c  &  -B_3     &       0     &    B_1    \\
E_3/c  &   B_2     &     -B_1    &     0     )\;.
\end{equation}

Thus, $F_{\mu \nu}$ combines $\vb{E}$ and $\vb{B}$ into a single spacetime tensor field. In this way, we have reformulated the definitions ([ACE] (5.2) and (5.3)) of $\vb{E}$ and $\vb{B}$ in a special relativistically covariant manner. 

Our next task is to write Maxwell's equations in a manifestly special relativistically covariant form. The definition of $F_{\mu \nu}$ together with the equality of mixed partials implies that 
\begin{equation}\label{eq:Wald_08.46}
\partial_\alpha F_{\mu \nu} + \partial_\mu F_{\nu \alpha} + + \partial_\nu F_{\alpha \mu} = 0\,.
\end{equation}
This equation is equivalent to [ACE] eqs. (5.6) and (5.7), and thus reformulates these equations in a special relativistically covariant manner. To express the remaining Maxwell [ACE] equations (5.4) and (5.5) in a manifestly covariant form, we first combine the charge density $\rho$ and the current density $\vb{J}$  into a single spacetime vector field $J^\mu$, known as the \tit{charge-current 4-vector}:
\begin{equation}\label{eq:Wald_08.47}
J^\mu = (c \rho, \vb{J})  \,.
\end{equation}

This makes the source terms in Maxwell's equations into a special relativistically covariant quantity. 
Charge-current conservation [ACE] eq. (5.8) then takes the manifestly covariant form 
\begin{equation}\label{eq:Wald_08.48}
\partial_\mu J^\mu = 0 \,.
\end{equation}

It is then straightforward to check that the remaining Maxwell [ACE] equations (5.4) and (5.5) can be written as the single spacetime tensor equation\footnote{}
\begin{equation}\label{eq:Wald_08.49}
\partial^\alpha F_{\alpha \mu} = - \mu_0 J^\mu \,,
\end{equation}
where we are using the notational conventions described at the end of section \ref{sec:Wald_08.1}.

Our remaining task is to write the energy density, momentum density, and stresses of the electromagnetic field in a special relativistically covariant form. This can be done by defining the following tensor field of type (0,2):

\begin{equation}\label{eq:Wald_08.50}
T^{EM}_{\mu \nu} = \frac{1}{\mu_0} \left[{F_m}^\alpha F_{\nu \alpha} - \frac{1}{4} \eta_{\mu \nu} F_{\alpha \beta} F^{\alpha \beta} \right] \,,
\end{equation}
which is called the \tit{stress-energy-momentum tensor} (or stress-energy tensor, for short) of the electromagnetic field. It is straightforard to check that
\begin{align}
T^{EM}_{0 0} &= \frac{1}{2 \mu_0} \sum_i \left(\frac{1}{c^2} E_i^2 + B_i^2 \right) 
= \frac{1}{2} \left(\epsilon_0 \abs{\vb{E}}^2 + \frac{1}{\mu_0} \abs{\vb{B}}^2 \right) \,,\label{eq:Wald_08.51}\\
T^{EM}_{0 i} &= T^{EM}_{i 0} = \frac{1}{\mu_0} \sum_j F_{0 j} F_{i j} = - \frac{1}{\mu_0 c} (\vb{E} \times \vb{B})_i \,,\label{eq:Wald_08.52}\\
T^{EM}_{i j} &= - \frac{1}{\mu_0} \left[ \frac{1}{c^2} E_i  E_j + B_i B_j - \frac{1}{2} \delta_{i j} \left( \frac{1}{c^2}\abs{\vb{E}}^2 + \abs{\vb{B}}^2 \right)\right] \,.\label{eq:Wald_08.53}
\end{align}

Thus, $T^{EM}_{00}$ is the energy density of the electromagnetic field, eq. [ACE] (5.11). It can be seen that $-T^{EM}_{0i} / c$ is just the momentum density, eq. [ACE] (5.12), or equivalently,  $-c T^{EM}_{0i}$ is the Poynting flux,  eq. [ACE] (5.14). Finally, $- T^{EM}_{ij}$ is just the stress tensor $\Theta_{ij}$ of the electromagnetic field, eq. [ACE] (5.13). Thus, $T^{EM}_{\mu \nu}$ combines into a single spacetime tensor the stress, energy, and momentum properties of the electromagnetic field--quantities that would be viewed as distinct in pre-relativity physics. The energy and momentum conservation relations eq. [ACE] (5.17) and (5.20) then can be rewritten as the single spacetime tensor equation:
\begin{equation}\label{eq:Wald_08.54}
\partial^\alpha T^{EM}_{\alpha \mu} = F_{\mu \nu} J^\nu \;.
\end{equation}

Note that $T^{EM}_{\mu \nu}$ has vanishing trace: $\eta^{\mu \nu}T^{EM}_{\mu \nu} = 0$.  

It is a fundamental requirement of general relativity that all other forms of matter also can be assigned a stress-energy-momentum tensor $T^M_{\mu \nu}$, whose components have the same interpretation as given above for the electromagnetic stress-energy-momentum tensor. It is also a fundamental requirement of general relativity that the total stress-energy-momentum tensor $T^{EM}_{\mu \nu} + T^M_{\mu \nu}$ must satisfy an equation that expresses local conservation of energy and momentum. In the flat spacetime of special relativity, this conservation equation takes the form
\begin{equation}\label{eq:Wald_08.55}
\partial^\alpha \left[ T^{EM}_{\alpha \mu} + T^M_{\alpha \mu} \right] = 0\,,
\end{equation}
and thus
\begin{equation}\label{eq:Wald_08.56}
\partial^\alpha T^M_{\alpha \mu} = - \partial^\alpha T^{EM}_{\alpha \mu} = - F_{\nu \mu} J^\nu \,.
\end{equation}

In any inertial coordinates, the time component of eq. (\ref{eq:Wald_08.56}) is equivalent to the [ACE] formula (5.19) for the rate at which energy per unit volume is transferred from the electromagnetic field to matter. The space components of eq. (\ref{eq:Wald_08.56}) are equivalent to the [ACE] formula (5.21) for the Lorentz force density.

Finally, it is worth noting that---when expressed in terms of $A_\mu$ rather than $\phi$ and $\vb{A}$---the Lorenz gauge condition [ACE] eq. (5.24) takes the simpler form  
\begin{equation}\label{eq:Wald_08.57}
\partial^\mu A_\mu =0 \,.
\end{equation}
Thus, the Lorenz gauge condition is a special relativistically covariant condition. Maxwell's equations in Lorenz gauge take the form
\begin{equation}\label{eq:Wald_08.58}
\Box A_\mu \equiv \partial^\alpha \partial_\alpha A_\mu = - \mu_0 J_\mu\,,
\end{equation}
which is manifestly special relativistically covariant. The retarded solution to eq. (\ref{eq:Wald_08.58}) is
\begin{equation}\label{eq:Wald_08.59}
A_\mu = \frac{\mu_0}{4 \pi} \int{\frac{[J_\mu(t', \vb{x}')]_{\text{ret}}}{\abs{\vb{x} - \vb{x}'}} d^3 x'}   \,,
\end{equation}
which is equivalent to [ACE] eqs. (5.56) and (5.57).

Thus we have succeeded in writing all the equations and results of [ACE] section 5.1 in special relativistically covariant form. What has been gained by doing this? After all, as we have seen, the equations of electrodynamics as reformulated above are identical to the equations of electrodynamics as given in [ACE] section 5.1. However, this fact alone is very significant, because it shows that---unlike Newton's second law of particle mechanics---the theory of electromagnetism does not have to be modified to be compatible with the spacetime structure of special relativity. Furthermore, as I shall now explain, the formulation of the theory of electromagnetism in special relativistically covariant form gives rise to a major change in viewpoint on electromagnetic phenomena. 

From the perspective of pre-relativity physics, the equations of [ACE] section 5.1 make sense only in a preferred rest frame, presumably provided by an \quotes{aether.} Electromagnetic waves would then naturally be thought of as propagating disturbances of the aether (much as sound waves are propagating disturbances of the air), and it would be naturally to try to give a purely mechanical explanation of all electromagnetic phenomena--as Maxwell himself attempted to do. However, with the need for a preferred rest frame eliminated, the notion of an aether becomes entirely unnatural, and it becomes natural to view the electromagnetic field as a physical entity in its own right.

In the pre-relativity viewpoint, it would not be any more useful to ask how an observer moving with respect to the aether would interpret electromagnetic phenomena than it would be to ask how an observer moving through air would interpret sound waves. A simple description would be available only in the rest frame of the medium, and it would be most sensible to do all calculations in this rest frame. If one wanted to give a description of electromagnetic phenomena from the viewpoint of an observer moving through the aether, it would not be obvious how to define the electric and magnetic fields \quotes{seen} by such a moving observer. Whatever definition of electric and magnetic fields that one gave, with pre-relativity notions of spacetime structure, they could not satisfy an unmodified form of Maxwell's equations in the frame of the moving observer.

However, we have now seen that in special relativity, electromagnetic theory in the form given in [ACE] section 5.1 holds in all inertial coordinates. All inertial observers would give exactly the same description of electromagnetic phenomena in terms of spacetime tensors. But if they define electromagnetic quantities in terms of particular components of these tensors, different observers  will assign different values to these electromagnetic quantities. This difference in values is given by the tensor transformation law eq. (\ref{eq:Wald_08.31}). It is of interest to see more explicitly how this works in the case of charge and current densities $\rho$ and $\vb{J}$, and in the case of the electromagnetic fields $\vb{E}$ and $\vb{B}$.

First consider the charge current $J^\mu$, eq. (\ref{eq:Wald_08.47}). Since this quantity is a spacetime vector, by eq. (\ref{eq:Wald_08.27}), an observer $O'$ at event $p$ who is moving in the $x$-direction with velocity $v$ with respect to observer $O$ would assign to the 4-current at $p$ the components 
\begin{equation}\label{eq:Wald_08.60}
{J'}^\mu = \Lambda^\mu_\nu J^\nu \,.
\end{equation}

Writing this out, we see that the charge density $\rho$ and current density $\vb{J}$ transform as
\begin{align}
\rho' &= \gamma \rho - \gamma \frac{v}{c^2} J^1\,, \label{eq:Wald_08.61}\\
{J'}^1 &= \gamma J^1 - \gamma v \rho \,. \label{eq:Wald_08.62}
\end{align}

It should not be surprising that a charge density $\rho$ would contribute to the current density seen by a moving observer, as found in eq. (\ref{eq:Wald_08.62}). However, is perhaps surprising that a current density $\vb{J}$ contributes to the charge density seen by a moving observer. 

As we have seen, the electromagnetic field strengths $\vb{E}$ and $\vb{B}$ are, in fact components of the spacetime tensor $F_{\mu \nu}$, eq. (\ref{eq:Wald_08.43}). Since $F_{\mu \nu}$ is a tensor of type (0,2), by eq. (\ref{eq:Wald_08.31}), its components transform as 
\begin{equation}\label{eq:Wald_08.63}
F_{\mu \nu} \rightarrow F'_{\mu \nu} = \sum_{\alpha, \beta} {(\Lambda^{-1})^\alpha}_\mu {(\Lambda^{-1})^\beta}_\nu F_{\alpha \beta} \,.
\end{equation}

Thus, if observer $O$ sees fields $\vb{E}$ and $\vb{B}$ present at event $p$, we find from this equation that an observer $O'$ moving in the $x$-direction with velocity $v$ with respect to observer $O$ would determine that the fields at $p$ are 
\begin{align}
E'_1 = E_1     &,\, B'_1 = B_1\,,\label{eq:Wald_08.64}\\
E'_2 = \gamma (E_2 - v B_3)&,\, B'_2 = \gamma \left(B_2 + \frac{v}{c^2} E_3\right)\,,\label{eq:Wald_08.65}\\
E'_3 = \gamma (E_3 + v B_2) &,\, B'_3 = \gamma \left(B_3 - \frac{v}{c^2} E_2\right)\,.\label{eq:Wald_08.66}
\end{align}
In other words, we have 
\begin{align}
\vb{E}'_\parallel = \vb{E}_\parallel &,\, \vb{B}'_\parallel = \vb{B}_\parallel\,,\label{eq:Wald_08.67}\\
\vb{E}'_\perp = \gamma (\vb{E}_\perp + \vb{v} \times \vb{B})  &,\, \vb{B}'_\perp = \gamma (\vb{B}_\perp - \frac{1}{c^2} \vb{v} \times \vb{E})\,,\label{eq:Wald_08.68}
\end{align}
where \quotes{$\parallel$} and \quotes{$\perp$} respectively denote components parallel and perpendicular to $\vb{v}$. That $\vb{E}$ and $\vb{B}$ transform into each other under Lorentz transformations is a reflection of their being different components of a single spacetime tensor $F_{\mu \nu}$.

\section{Charged Particle Motion and Radiation}
\label{sec:Wald_08.3}
Thus far in this book, apart from our treatment of electrostatics, we have largely avoided discussing point charges. The main reason is that, as discussed in [ACE] section 1.4, at a fundamental level, charged matter must be viewed as continuously distributed rather than point-like. Nevertheless, as already stated in [ACE] section 1.4, the idealization of a point charge is very useful for considering (i) the motion of a small charged body whose self-fields can be neglected\footnote{Self-field effects will be considered in depth in [ACE] chapter 10.} and (ii) the radiation emitted by a small charged body undergoing arbitrary motion. We have delayed discussing these topics until now, so that we can give a fully relativistic treatment of both charged particle motion and radiation from a charged particle. 

\subsection{Charged particle motion}
\label{subsec:Wald_08.3.1}
As discussed at the end of section \ref{sec:Wald_08.1}, Newton's second law in special relativity takes the form
\begin{equation}\label{eq:Wald_08.69}
u^\nu \partial_\nu p^\mu = \dv{p^\mu}{\tau} = f^\mu\,,
\end{equation}
where $p^\mu = m u^\mu$. The electromagnetic 4-force $f^\mu$ exerted on a particle of charge $q$ is obtained by taking the limit of eq. (\ref{eq:Wald_08.56}) as the matter and charge distribution shrinks down to a worldline. This is a highly nontrivial limit in that, if taken at fixed charge, the electromagnetic self-energy of the body would diverge. We will deal with the issue of how properly take the point charge limit in [ACE] chapter 10. For now we merely ignore the effects of the \quotes{self-field} of the charged body and replace $F_{\mu \nu}$ on the right side of eq. (\ref{eq:Wald_08.56}) by the external field $F^\text{ext}_{\mu \nu}$, into which the body is placed.   The integral of the left side of eq. (\ref{eq:Wald_08.56}) over space at time $t$ yields
\begin{equation}\label{eq:Wald_08.70}
\begin{aligned}
\int{ \partial^\alpha T^M_{\alpha \mu}d^3x} &= \bigintsss{\left[-\frac{1}{c} \pdv{T^M_{0 \mu}}{t} + \sum_{i=1}^3 \partial_i T^M_{i \mu} \right] d^3x} \\
&= -\frac{1}{c}\dv{}{t} \int{T^M_{0 \mu} d^3x} = \dv{p_\mu}{t}\,,
\end{aligned}
\end{equation}
where $p^\mu$ is the total matter 4-momentum of the body,\footnote{We will see in [ACE] chapter 10 that the electromagnetic self-energy of the body--which is being neglected here--will contribute to its rest mass and 4-momentum.} and we used Gauss's law and the boundedness of the matter distribution in the second equality to set the integral of the spatial divergence of the matter stress-energy tensor to zero.

To evaluate the spatial integral of the right side of eq. (\ref{eq:Wald_08.56}), we need an expression for the charge-current 4-vector $J^\mu$ of the body in the limit that it has shrunk down to a worldline. Since $J^\mu$ is a spacetime vector, it is clear that it must point along the direction of the 4-velocity $u^\mu$ of the worldline in this limit--there simply is no other candidate for the direction of $J^\mu$. Clearly, $J^\mu$ must be localized on the worldline of the particle. Thus, if the particle motion is described in some inertial coordinate system as $\vb{x} = \vb{X}(t)$ (where $\vb{X}(t)$ is a vector function of $t$ with $\abs{d{\vb{X}}/dt} < c$), we must have
\begin{equation}\label{eq:Wald_08.71}
J^\mu (t, \vb{x}) \propto   \delta(\vb{x} - \vb{X}(t))\, u^\mu\,.
\end{equation}
Since $J^0 = c \rho$ (see eq. (\ref{eq:Wald_08.47})), the constant of proportionality in eq. (\ref{eq:Wald_08.71}) can be fixed by requiring that the spatial integral of $J^0$ be equal to $cq$. Since $u^0 = c\gamma$, we thereby obtain 
\begin{equation}\label{eq:Wald_08.72}
J^\mu (t, \vb{x}) = \frac{q}{\gamma} \delta(\vb{x} - \vb{X}(t))\, u^\mu\,.
\end{equation}

In terms of $\rho$ and $\vb{J}$, we obtain
\begin{equation}\label{eq:Wald_08.73}
 \rho(t, \vb{x}) = q \delta(\vb{x} - \vb{X}(t))\,, \quad 
\vb{J}(t, \vb{x}) = q \dv{\vb{X}}{t} \delta(\vb{x} - \vb{X}(t))\,.
\end{equation}

The spatial integral of the right side of eq. (\ref{eq:Wald_08.56}) with $F_{\nu \mu}$ replaced by $F^\text{ext}_{\nu \mu}$ then yields
\begin{equation}\label{eq:Wald_08.74}
- \int{ F^\text{ext}_{\nu \mu} J^\nu d^3x} = - \frac{q}{\gamma} F^\text{ext}_{\nu \mu} u^\nu\,.
\end{equation}

Thus, equating the spatial integrals of the left and right sides of eq. (\ref{eq:Wald_08.56}) in the point charge limit, we obtain
\begin{equation}\label{eq:Wald_08.75}
\dv{p^\mu}{t} = - \frac{q}{\gamma} F^\text{ext}_{\nu \mu} u^\nu\,.
\end{equation}

Using the fact that along the worldline of the particle, we have 
\begin{equation}\label{eq:Wald_08.76}
\dv{t}{\tau} = \frac{1}{c} \dv{X^0}{\tau} = \frac{u^0}{c} = \gamma\,,
\end{equation}
we see that eq. (\ref{eq:Wald_08.75}) can be put in the manifestly covariant form (\ref{eq:Wald_08.69}) with
\begin{equation}\label{eq:Wald_08.77}
 f_\mu = -q F^\text{ext}_{\nu \mu} u^\nu\,.
\end{equation}

Note that $u^\mu f_\mu = -q F^\text{ext}_{\nu \mu} u^\mu u^\nu = 0$, since $F^\text{ext}_{\mu \nu} = - F^\text{ext}_{\nu \mu}$. Thus, by eq. (\ref{eq:Wald_08.38}), the rest mass $m$ of a point charge does not change when it moves under the influence of an external electromagnetic field. 

In any inertial coordinates, the components of $u^\mu$ are $(c \gamma, \gamma \vb{v})$, and the components of $F^\text{ext}_{\mu \nu}$ are given by eq. (\ref{eq:Wald_08.45}). Thus the spatial components of eq. (\ref{eq:Wald_08.69}) with $f^\mu$ given by eq. (\ref{eq:Wald_08.77}) take the form
\begin{equation}\label{eq:Wald_08.78}
m \dv{(\gamma \vb{v})}{t} = q \gamma (\vb{E} + \vb{v} \times \vb{B})\,,
\end{equation}
where here and in the following, it is understood that $\vb{E}$ and $\vb{B}$ are the external fields. 
Using eq. (\ref{eq:Wald_08.76}), we see that eq. (\ref{eq:Wald_08.78}) is equivalent to   
\begin{equation}\label{eq:Wald_08.79}
m \dv{(\gamma \vb{v})}{t} = m \dv{(\gamma \vb{v})}{\tau} \dv{\tau}{t} = q(\vb{E} + \vb{v} \times \vb{B})\,.
\end{equation}
This differs from the usual nonrelativistic equation of motion for a charged particle found in elementary texts only in that on the left side of this equation, $\vb{v}$ is replaced by $\gamma \vb{v}$. 

We now consider two important, simple examples of charged particle motion. The first example is motion in a uniform electric field $\vb{E}$, that is, in some inertial coordinates $(t, \vb{x})$, we have that $\vb{E}$ is independent of $(t, \vb{x})$ and $\vb{B} = 0$. Without loss of generality, we may assume that $\vb{E}$ points in the $x$-direction: $\vb{E} = E \vb{\hat{x}}$. We may choose the origin of coordinates so that $\vb{X}(0) = \vb{0}$ (i.e., the particle is at $\vb{x} = 0$ at $t = 0$). For simplicity, consider the case where the particle is initially at rest: $d\vb{X}/dt = 0$ at $t=0$. It is clear from  eq. (\ref{eq:Wald_08.79}) that we will have $Y(t) = Z(t) = 0$ for all $t$, so we need only consider the $x$-motion. We have 
\begin{equation}\label{eq:Wald_08.80}
\dv{(\gamma v)}{t} = \frac{q}{m} E\,,
\end{equation}
where $v = dX/dt$. Integration of this equation using $v = 0$ at $t = 0$ yields
\begin{equation}\label{eq:Wald_08.81}
\gamma v = \frac{q}{m} E t\,,
\end{equation}
and hence, 
\begin{equation}\label{eq:Wald_08.82}
v = \dv{X}{t} = \frac{q E t /m}{[q + (qEt/(mc))^2]^{1/2}}\,.
\end{equation}
Integration of this equation with the initial condition $X = 0$ at $t = 0$ yields the solution
\begin{equation}\label{eq:Wald_08.83}
X(t) = \frac{mc^2}{qE} \left\{ [ 1 + (qEt/(mc))^2 ]^{1/2} - 1 \right\}\,.
\end{equation}
For $t \ll mc/(qE)$, we have $v \ll c$, and the solution reduces to 
\begin{equation}\label{eq:Wald_08.84}
X(t) \approx \frac{1}{2} \frac{qE}{m} t^2 \,.
\end{equation}

The second example is motion in a uniform magnetic field $\vb{B}$; that is, in some inertial coordinates 
$(t, \vb{x})$, we have that $\vb{B}$ is independent of $(t, \vb{x})$ and $\vb{E}=0$. 
The equation of motion eq. (\ref{eq:Wald_08.79}) becomes
\begin{equation}\label{eq:Wald_08.85}
\dv{\gamma \vb{v}}{t} = \frac{q}{m} \vb{v} \times \vb{B}\,.
\end{equation}
Dotting this equation with $\vb{v}$, we find that $v = \abs{\vb{v}}$ is constant. 
Since $\gamma$ is constant, we may rewrite this equation as  
\begin{equation}\label{eq:Wald_08.86}
\dv{\vb{v}}{t} = \frac{q}{\gamma m} \vb{v} \times \vb{B}\,.
\end{equation}

The general solution to this equation is helical motion (i.e., linear motion parallel to $\vb{B}$ and circular motion perpendicular to $\vb{B}$). More explicitly, taking $\vb{B}$ to point along the $z$-axis so that $\vb{B} = B \hat{\vb{z}}$, we find that the general solution to eq. (\ref{eq:Wald_08.86}) is
\begin{equation}\label{eq:Wald_08.87}
\begin{aligned}
X(t) &= X_0 + R_g cos(\omega_g t + \phi_0)\,,\\
Y(t) &= Y_0 - R_g sin(\omega_g t + \phi_0)\,,\\
Z(t) &= v_{z_0} t + z_0\,.
\end{aligned}
\end{equation}
Here $X_0$, $Y_0$, $v_{z_0}$ and $z_0$ are constants, and 
\begin{equation}\label{eq:Wald_08.88}
\omega_g \equiv \frac{qB}{\gamma m}\,, \quad R_g \equiv \frac{v_\perp}{\abs{\omega_g}}\,,
\end{equation}
where $v_\perp$ is the (constant) magnitude of the velocity\footnote{We must, of course, have $v^2_{z_0} + v^2_\perp < c^2$.} in the $x\text{-}y$ plane. 
We refer to $\omega_g$ as the \tit{gyrofrequency} (or \tit{cyclotron frequency}), and refer to $R_g$ as the \tit{gyroradius} (or \tit{Larmor radius}). For nonrelativistic motion, $\gamma \approx 1$, the gyrofrequency is approximately
\begin{equation}\label{eq:Wald_08.89}
\omega_g \approx \frac{qB}{m}\,,
\end{equation}
which is independent of the velocity.


\subsection{Radiation from a point charge in arbitrary motion}\label{subsec:Wald_08.3.2}
As discussed in section \ref{sec:Wald_08.1}, the worldline of a particle in special relativity must be a timelike curve, that is, a curve $\vb{x}=\vb{X}(t)$ with $\abs{d\vb{X}/dt} < c$. In principle, any timelike curve represents a possible particle motion. We wish to find the retarded solution for the electromagnetic field associated with a particle of charge $q$ whose worldline is an arbitrary timelike curve.\footnote{As we saw in [ACE] section 6.1, Maxwell's equations in a medium are the same as Maxwell's equations in vacuum with $\epsilon_0 \rightarrow \epsilon$, and $\mu_0 \rightarrow \mu$, and $c \rightarrow c/n$. If $n > 1$, it is possible to have a charged particle in the medium move with velocity $v > c/n$, so that effectively, the charge has a spacelike worldline. This gives rise to the phenomenon of Cherenkov radiation (see problem 10).}

The retarded solution with a point charge source is obtained by plugging eq. (\ref{eq:Wald_08.72}) into 
\begin{equation}\label{eq:Wald_08.90}
A^\mu(t,\vb{x}) = \frac{\mu_0}{4 \pi}\bigintsss{\frac{[J^\mu(t', \vb{x'})]_\text{ret}}{\abs{\vb{x} - \vb{x'}}} d^3x'}\,,
\end{equation}
or equivalently, by plugging eq. (\ref{eq:Wald_08.73}) into [ACE] eqs. (5.56) and (5.57). We obtain
\begin{align}
\phi(t, \vb{x}) &= q \frac{\mu_0 c^2}{4 \pi} \bigintsss{ \frac{\delta(\vb{x'} - \vb{X}(t_\text{ret}))}{\abs{\vb{x} - \vb{x'}}} d^3x'}\,,\label{eq:Wald_08.91} \\
\vb{A}(t, \vb{x}) &= q \frac{\mu_0}{4 \pi}\bigintsss{ \dv{\vb{X}}{t}(t_\text{ret})\frac{\delta(\vb{x'} - \vb{X}(t_\text{ret}))}{\abs{\vb{x} - \vb{x'}}} d^3x'}\,,\label{eq:Wald_08.92}
\end{align}
where 
\begin{equation}\label{eq:Wald_08.93}
t_\text{ret} = t - \frac{1}{c} \abs{\vb{x} - \vb{x'}}\,,
\end{equation}
and we used $\epsilon_0 \mu_0 c^2 = 1$ to eliminate $\epsilon_0$ in favor of $\mu_0$ and $c$ in all expressions here and below. It is important to notice that the argument of the $\delta$-function in the above integrals has a nontrivial dependence on $\vb{x'}$ because of the dependence of $t_\text{ret}$ on $\vb{x'}$. Therefore, to evaluate the integrals in eqs.  (\ref{eq:Wald_08.91}) and (\ref{eq:Wald_08.92}), we must use the fact that if $\vb{f}(\vb{x})$ is any vector function of $\vb{x}$ and $g(\vb{x})$ is any function of $\vb{x}$, then\footnote{Equation (\ref{eq:Wald_08.94}) follows from the fact that we can change coordinates to $\vb{y}=\vb{f}(\vb{x})$. The volume element $d^3y$ in the new coordinates is related to $d^3x$ by $d^3y=\abs{\mathcal{J}}d^3x$. The integral on the left side of eq. (\ref{eq:Wald_08.94}) thus becomes $\int g(\vb{y})\delta(\vb{y})(1/\abs{\mathcal{J}(\vb{y})}) d^3y$.}
\begin{equation}\label{eq:Wald_08.94}
\int{g(\vb{x})\delta(\vb{f}(\vb{x}))  d^3x} = \frac{g}{\abs{\mathcal{J}}} \bigg\rvert_{\vb{f}(\vb{x}) = 0}\,,
\end{equation}
where $\mathcal{J}$ denotes the determinant of the Jacobian matrix:
\begin{equation}\label{eq:Wald_08.95}
{\mathcal{J}^i}_j = \pdv{f^i}{x^j} \,.
\end{equation}
The Jacobian matrix relevant to the evaluation of the integrals in eqs. (\ref{eq:Wald_08.91}) and (\ref{eq:Wald_08.92}) is 
\begin{equation}\label{eq:Wald_08.96}
{\mathcal{J}^i}_j = \pdv{\left[{x'}^i - X^i(t - \abs{\vb{x} - \vb{x'}}/c) \right]}{{x'}^j} = {\delta^i}_j - \frac{x^j - {x'}^j}{c \abs{\vb{x} - \vb{x'}}} \dv{X^i}{t} \bigg\rvert_{t = t_\text{ret}}\,.
\end{equation}

The determinant of ${\mathcal{J}^i}_j$ is 
\begin{equation}\label{eq:Wald_08.97}
\mathcal{J} = 1 - \left( \frac{\vb{x} - \vb{x'}}{c \abs{\vb{x} - \vb{x'}}}\right) \cdot \dv{\vb{X}}{t}\bigg\rvert_{t = t_\text{ret}}\,.
\end{equation}

We now may perform the integrals in eqs. (\ref{eq:Wald_08.91}) and (\ref{eq:Wald_08.92}) to obtain our final result for the potentials of the retarded solution for a point charge in arbitrary motion:
\begin{equation}\label{eq:Wald_08.98}
\phi(t, \vb{x}) = q \frac{\mu_0 c^2}{4 \pi} \frac{1}{\alpha} \frac{q}{\abs{\vb{x} - \vb{X}(t_\text{ret})}} \,,
\end{equation}
\begin{equation}\label{eq:Wald_08.99}
\vb{A}(t, \vb{x}) = \frac{\mu_0}{4 \pi} \frac{1}{\alpha} \frac{q}{\abs{\vb{x} - \vb{X}(t_\text{ret})}} \dv{\vb{X}}{t}\bigg\rvert_{t = t_\text{ret}}\,,
\end{equation}
where 
\begin{equation}\label{eq:Wald_08.100}
\alpha = 1 - \frac{1}{c}\, \vb{\hat{n}} \cdot \dv{\vb{X}}{t}\bigg\rvert_{t = t_\text{ret}}\,,
\end{equation}
with
\begin{equation}\label{eq:Wald_08.101}
\vb{\hat{n}} = \frac{\vb{x} - \vb{X}(t_\text{ret})}{\abs{\vb{x} - \vb{X}(t_\text{ret})}}
\end{equation}
(i.e., $\vb{\hat{n}}$ is the unit vector that points from the position of the particle at $t_\text{ret}$ to the observation point $\vb{x}$). Equations (\ref{eq:Wald_08.98}) and (\ref{eq:Wald_08.99}) are called the \tit{Lienard-Wiechert potentials}.

The electric and magnetic fields can be computed from the Lienard-Wiechert potentials in the usual manner via [ACE] eqs. (5.2) and (5.3). However, when taking derivatives of $\phi$ and $\vb{A}$ with respect to $t$ and $\vb{x}$, it is important to recognize that in eqs. (\ref{eq:Wald_08.98}) and (\ref{eq:Wald_08.99}), the quantity $t_\text{ret}$ denotes the time of intersection of the past light cone of $(t, \vb{x})$ with the worldline of the particle. 
Thus, for the given particle worldline $\vb{X}(t_\text{ret})$ is the function of $(t, \vb{x})$ that is implicitly defined by 
\begin{equation}\label{eq:Wald_08.102}
t_\text{ret} = t - \frac{1}{c}{\abs{\vb{x} - \vb{X}(t_\text{ret})}}\,.
\end{equation}
Taking the partial derivative of this equation with respect to $t$, we obtain
\begin{equation}\label{eq:Wald_08.103}
\pdv{t_\text{ret}}{t} = 1 + \frac{1}{c}\vb{\hat{n}} \cdot \dv{\vb{X}}{t}\bigg\rvert_{t = t_\text{ret}} \pdv{t_\text{ret}}{t}\,,
\end{equation}
and hence
\begin{equation}\label{eq:Wald_08.104}
\pdv{t_\text{ret}}{t} = \frac{1}{1 - \frac{1}{c}\vb{\hat{n}} \cdot \dv{\vb{X}}{t}\big\rvert_{t = t_\text{ret}}} = \frac{1}{\alpha}\,.
\end{equation}
Similarly we find
\begin{equation}\label{eq:Wald_08.105}
\pdv{t_\text{ret}}{x^i} = - \frac{1}{c} \sum_j \vb{\hat{n}}_j \left(\delta_{ij} - \dv{X^j}{t}\pdv{t_\text{ret}}{x^i}\right)\,, 
\end{equation}
which yields
\begin{equation}\label{eq:Wald_08.106}
\grad t_\text{ret} = - \frac{1}{a} \vb{\hat{n}}\,. 
\end{equation}

Using these relations, we can take the derivatives of the potentials (\ref{eq:Wald_08.98}) and (\ref{eq:Wald_08.99}) with respect to $t$ and $\vb{x}$ to obtain the following formulas for the electric and magnetic fields:

\begin{align}
 \vb{E}(t, \vb{x}) & =  q \frac{\mu_0 c^2}{4 \pi} \frac{\left( \vb{\hat{n}} - \frac{1}{c} \dv{\vb{X}}{t} \right) \left( 1 - \frac{1}{c^2} \abs{\dv{\vb{X}}{t}}^2 \right)}{\alpha^3 \abs{\vb{x} - \vb{X}(t_\text{ret})}^2} + q \frac{\mu_0}{4 \pi} \frac{\vb{\hat{n}} \times \left[\left( \vb{\hat{n}} - \frac{1}{c} \dv{\vb{X}}{t} \right) \times \dv[2]{\vb{X}}{t} \right]}{\alpha^3 \abs{\vb{x} - \vb{X}(t_\text{ret})}^2}\,,\label{eq:Wald_08.107} \\
c\vb{B}(t, \vb{x}) & =  \vb{\hat{n}} \times \vb{E}(t, \vb{x})\,,\label{eq:Wald_08.108} 
\end{align}
where it is understood that $\dv*{\vb{X}}{t}$ and $\dv*[2]{\vb{X}}{t}$ are evaluated at $t_\text{ret}$. It should be emphasized that eqs. (\ref{eq:Wald_08.107}) and (\ref{eq:Wald_08.108}) are exact solutions and hold at all distances from the worldline $\vb{X}(t)$ of the point charge.

It is of interest to calculate the electromagnetic energy radiated to infinity, as given by the Poynting flux as $\abs{\vb{x}} \rightarrow \infty$. 
The first term in eq. (\ref{eq:Wald_08.107}) falls off as $1/\abs{\vb{x}}^2$, so only the second term contributes to the energy flux at infinity. In addition, the difference between $\vb{\hat{n}}$ and $\vb{\hat{x}}$ can be neglected at this order, where  $\vb{\hat{x}}$ is the unit outward radial vector. We find that the radiated power per unit solid angle at a given retarded time $t_\text{ret}$ is given by
\begin{equation}\label{eq:Wald_08.109}
\begin{aligned}
\dv{P}{\Omega} &= \lim_{\abs{\vb{x}} \rightarrow \infty} \abs{\vb{x}}^2 \mathcal{S} \cdot \vb{\hat{x}} \\
               &= \frac{1}{\mu_0} \lim_{\abs{\vb{x}} \rightarrow \infty} \abs{\vb{x}}^2 (\vb{E} \times \vb{B}) \cdot \vb{\hat{x}} \\
               &= \frac{q^2 \mu_0}{16 \pi^2 c \alpha^6} \bigg\lvert  \vb{\hat{x}} \times \left[\left(\vb{\hat{x}} - \frac{1}{c}\dv{\vb{X}}{t}\right) \times \dv[2]{\vb{X}}{t}\right]\bigg\rvert^2 \,,
\end{aligned}
\end{equation}
where $\dv*{\vb{X}}{t}$ and $\dv*[2]{\vb{X}}{t}$ are evaluated at $t_\text{ret}$. Note that the limit $\abs{\vb{x}} \rightarrow \infty$ at fixed retarded time means that the limit is being taken on spheres of radius $\abs{\vb{x}}$ lying on the future light cone of the event on the worldline of the particle at $t = t_\text{ret}$. 

Since the right side of eq. (\ref{eq:Wald_08.109}) is most naturally evaluated on spheres of constant retarded time at infinity it is natural to define the notion of energy flux per unit solid angle per unit \tit{retarded} time by 
\begin{equation}\label{eq:Wald_08.110}
\dv{P'}{\Omega} \equiv \dv{P}{\Omega} \dv{t}{t_\text{ret}} = \dv{P}{\Omega} \alpha = = \frac{q^2 \mu_0}{16 \pi^2 c \alpha^5} \bigg\lvert  \vb{\hat{x}} \times \left[\left(\vb{\hat{x}} - \frac{1}{c}\dv{\vb{X}}{t}\right) \times \dv[2]{\vb{X}}{t}\right]\bigg\rvert^2 \,.
\end{equation}

The total flux of energy between two retarded times would then be obtained by integrating the right side of eq. (\ref{eq:Wald_08.110}) over each sphere of constant retarded time and then integrating with respect to $t_\text{ret}$. As we shall see below, a simple formula can be obtained for the integral of $\dv*{P'}{\Omega}$ over a sphere of constant retarded time.

If the particle is at rest at some event on its worldline, then at the retarded time corresponding to that event, eq. (\ref{eq:Wald_08.109}) with $\dv*{\vb{X}}{t} = 0$ and $\alpha = 1$ reduces to
\begin{equation}\label{eq:Wald_08.111}
\begin{aligned}
\left(\dv{P}{\Omega}\right)_{\dv{\vb{X}}{t} = 0} &= \frac{q^2 \mu_0}{16 \pi^2 c} \bigg\lvert  \vb{\hat{x}} \times \left[\vb{\hat{x}} \times \dv[2]{\vb{X}}{t}\right]\bigg\rvert^2 \\
&=  \frac{q^2 \mu_0}{16 \pi^2 c} \left[                        
\abs{\dv[2]{\vb{X}}{t}}^2 - \left(\vb{\hat{x}} \cdot \dv[2]{\vb{X}}{t}\right)^2 \right]\,.
\end{aligned}
\end{equation}
This agrees with [ACE] eq. (5.79), since the dipole moment of the charged particle is $\vb{p}=q\vb{X}$.
In particular, the angular distribution of the energy flux has a purely dipole pattern (i.e., it varies with angle as $\sin^2 \theta$, where $\theta$ is the angle between $\dv*[2]{\vb{X}}{t}$ and the observation point).The effects of nonzero velocity, $\dv*{\vb{X}}{t} \neq 0$, on the nature of the angular distribution of the radiated power at the corresponding retarded time can be illustrated by considering the following two special cases. We orient our spatial axes so that
$\dv*{\vb{X}}{t}$ at the given retarded time instantaneously points in the $z$-direction:
$\dv*{\vb{X}}{t}= v \vb{\hat{z}}$. Then we have 
\begin{equation}\label{eq:Wald_08.112}
\alpha = 1 - \frac{v}{c} \cos \theta .
\end{equation}
First consider the case where $\dv*[2]{\vb{X}}{t}$ points in the same direction as $\dv*{\vb{X}}{t}$---as occurs for linear acceleration. Then eq. (\ref{eq:Wald_08.109} becomes
\begin{equation}\label{eq:Wald_08.113}
\left(\dv{P}{\Omega}\right)_\parallel = \frac{q^2 \mu_0}{16 \pi^2 c}
\frac{\sin^2 \theta}{\left( 1 - \frac{v}{c} \cos \theta\right)^6} \abs{\dv[2]{\vb{X}}{t}}^2\,,
\end{equation}
whereas $(\dv*{P'}{\Omega})_\parallel$ would be given by the same formula with one fewer power of $(1 - \frac{v}{c} \cos \theta)$ in the denominator. As the second example, consider the case where, at a given retarded time, $\dv*[2]{\vb{X}}{t}$ points orthogonally to $\dv*{\vb{X}}{t}$---as occurs for circular motion. For relativistic velocities, the radiation in this case is known as \tit{synchrotron radiation}. We further orient our axes so that $\dv*[2]{\vb{X}}{t}$ instantaneously points in the $x$-direction. Equation (\ref{eq:Wald_08.109}) then yields
\begin{equation}\label{eq:Wald_08.114}
\left(\dv{P}{\Omega}\right)_\perp = \frac{q^2 \mu_0}{16 \pi^2 c}
\frac{1}{\left( 1 - \frac{v}{c} \cos \theta\right)^6}
\left[\left( 1 - \frac{v}{c} \cos \theta\right)^2 - \left(1 -\frac{v^2}{c^2}     \right)\sin^2 \theta \cos^2 \phi \right] \abs{\dv[2]{\vb{X}}{t}}^2,
\end{equation}
where $(\theta, \phi)$ are the spherical coordinates of the observation point, with the axes chosen as above. Again $\dv*{P'}{\Omega})_\perp$ would be given by the same formula with one fewer power of $(1 - \frac{v}{c} \cos \theta)$ in the denominator. 
As eq. (\ref{eq:Wald_08.113}) and eq. (\ref{eq:Wald_08.114}) illustrate, if $v \rightarrow c$, the radiation becomes highly beamed in the forward direction. This beaming of the radiation as compared with that seen by an observer who is at rest relative to the particle can be understood as resulting from the combined effects of the Lorentz boosting of the field strengths, eq. (\ref{eq:Wald_08.68}), and aberration (see problem 4).

The total power radiated at a fixed retarded time is obtained by integrating $\dv*{P}{\Omega}$ over angles:
\begin{equation}\label{eq:Wald_08.115}
P = \int \dv{P}{\Omega} d\Omega = \int \dv{P}{\Omega} \sin \theta\, d\theta d\phi\,. 
\end{equation}

For the case of a particle instantaneously at rest at a given retarded time, the angular integration of eq.  (\ref{eq:Wald_08.111}) was already carried out in [ACE] eq. (5.81). We obtain Larmor's formula
\begin{equation}\label{eq:Wald_08.116}
P_0 =  \frac{q^2 \mu_0}{6 \pi c} \abs{\dv[2]{\vb{X}}{t}}^2 \,,
\end{equation}
where the subscript $0$ on $P$ indicates that this formula holds only when the particle is at rest at the given retarded time. For a particle at rest, we have 
\begin{equation}\label{eq:Wald_08.117}
a^\mu \equiv \dv{u^\mu}{\tau} = (0, \dv[2]{\vb{X}}{t})\,.
\end{equation}
Therefore, we may rewrite this formula as
\begin{equation}\label{eq:Wald_08.118}
P_0 =  \frac{q^2 \mu_0}{6 \pi c} a^\mu a_\mu \,.
\end{equation}
If the particle is not at rest, the angular integrals needed to calculate the radiated power are quite complicated. However, the total power per unit retarded time, 
\begin{equation}\label{eq:Wald_08.119}
P' = \int \dv{P'}{\Omega} d\Omega\,, 
\end{equation}
can be obtained quite simply by the following argument. In the instantaneous frame of the particle, the energy radiated to infinity over an infinitesimally small retarded time interval $\Delta t_0$ as measured in the rest frame is given by 
\begin{equation}\label{eq:Wald_08.120}
\Delta \mathcal{E}_0 = P_0 \Delta t_0\,. 
\end{equation}
In the instantaneous rest frame of the particle, the electromagnetic field carries no net momentum to infinity during this infinitesimal time interval, as can be seen from the fact that radiation flux eq. (\ref{eq:Wald_08.111}) is parity invariant. Now, by arguments similar to that of part (d) of problem 5, the 4-momentum radiated to infinity between the given retarded times transforms as a 4-vector under Lorentz boosts. Thus, in an arbitrary Lorentz frame, we have
\begin{equation}\label{eq:Wald_08.121}
\Delta \mathcal{E} = \gamma \Delta \mathcal{E}_0 = \gamma P_0 \Delta t_0\,, 
\end{equation}
with $\gamma = (1 - v^2/c^2)^{-1/2}$. By definition of $P'$, we have 
\begin{equation}\label{eq:Wald_08.122}
\Delta \mathcal{E} = P' \Delta t_\text{ret}\,. 
\end{equation}
Finally, in the arbitrary Lorentz frame, we have $\Delta t_\text{ret} = \gamma \Delta t_0$, since the elapsed coordinate time along the worldline of the particle is time dilated relative to the rest frame. \\
Putting all this together, we see that in an arbitrary frame, we have
\begin{equation}\label{eq:Wald_08.123}
P' = P_0  = frac{q^2 \mu_0}{6 \pi c} a^\mu a_\mu \,.            \,. 
\end{equation}
This, in any inertial frame, the total energy radiated to infinity in that frame between retarded times $t_1$ and $t_2$ is given by 
\begin{equation}\label{eq:Wald_08.124}
\mathcal{E} = \frac{q^2 \mu_0}{6 \pi c} \int_{t_1}^{t_2} a^\mu(t_\text{ret}) a_\mu(t_\text{ret}) \,, 
\end{equation}
where, in the integral, $t_\text{ret}$ is the retarded time coordinate in that inertial frame.




 





      % Wald - Advanced Classical Electromagnetism - Special Relativity

%\appendixpage
%\appendix
\chapter{Tensors}
\label{ch:Tensors} 

\section{Vector Algebra}
For the sake of fixing notation, let $\{\vu{i}, \vu{j}, \vu{k}\}$ be an \textit{orthonormal} basis of unit vectors in three-dimensional space, and the following a list of all possible scalar products of basis vectors. 
  
\begin{equation}
\begin{aligned} 
\vu{i}\vdot \vu{i} &= \vu{j} \vdot \vu{j} = \vu{k} \vdot \vu{k} = 1 \\ 
\vu{i} \vdot \vu{j} &= \vu{j} \vdot \vu{k} = \vu{k} \vdot \vu{i} = 0 
\label{eq:basis_dot_products}
\end{aligned}
\end{equation}

For a \textit{right-handed} basis, the following is a list of all possible \textit{vector} products of basis vectors.
  
\begin{equation}
\begin{aligned} 
\vu{i} \cross \vu{i} &= \vu{j} \cross \vu{j} = \vu{k} \cross \vu{k} = 0 \\ 
\vu{i} \cross \vu{j} &= \vu{k} = - \vu{j} \cross \vu{i}\\
\vu{j} \cross \vu{k} &= \vu{i} = - \vu{k} \cross \vu{j}\\
\vu{k} \cross \vu{i} &= \vu{j} = - \vu{i} \cross \vu{k}
\label{eq:basis_vector_products}
\end{aligned}
\end{equation}





\backmatter%%%%%%%%%%%%%%%%%%%%%%%%%%%%%%%%%%%%%%%%%%%%%%%%%%%%%%%
%%%%%%%%%%%%%%%%%%%%%%%%% referenc.tex %%%%%%%%%%%%%%%%%%%%%%%%%%%%%%
% sample references
% 
% Use this file as a template for your own input.
%
%%%%%%%%%%%%%%%%%%%%%%%% Springer-Verlag %%%%%%%%%%%%%%%%%%%%%%%%%%

%
% BibTeX users please use
% \bibliographystyle{}
% \bibliography{}
%
% Non-BibTeX users please use
\begin{thebibliography}{99.}
%
% and use \bibitem to create references.
%
% Use the following syntax and markup for your references
%
% Monograph
\bibitem{Griffiths_4th} D.J. Griffiths (2017)
Introduction to Electrodynamics. Cambridge University Press, Cambridge

% Monograph
\bibitem{Felsager_1981} B. Felsager (1981)
Geometry, Particles and Fields. Odense University Press

% Monograph
\bibitem{BudakFomin_1973} B.M. Budak, S.V. Fomin (1973)
Multiple Integrals, Field Theory and Series. Mir Publishers, Moscow

% Monograph
\bibitem{Postnikov_II_1982} Mikhail Postnikov (1982)
Lectures in Geometry, Semester II. Linear Algebra and Differential Geometry. Mir Publishers, Moscow

\end{thebibliography}

\printindex

%%%%%%%%%%%%%%%%%%%%%%%%%%%%%%%%%%%%%%%%%%%%%%%%%%%%%%%%%%%%%%%%%%%%%%

\end{document}





