\chapter{Combining (Any) Velocities}
\label{ch:Morin_04}

In chapter \ref{ch:Morin_02} we argued that if Alice, a passenger on a train moving at $v$ feet per second, can throw a ball at $u$ feet per second, then if she throws that ball toward the front of the train, its speed $w$ with respect to the tracks will be 
\begin{equation}\label{eq:Morin_04.1}
w = u + v
\end{equation}
in the same direction as the train.

This is known as the nonrelativistic velocity addition law. It is called \quotes{nonrelativistic} because it is only accurate when the speeds $u$ and $v$ are small compared to the speed of light. Evidently it fails to work when $u = c$ (i.e. if Alice turns on a flashlight instead of throwing a ball) for we know that the speed $w$ of the light in the track frame will not be $c + v$ but simply $c$ -- the same value it has in the train frame!

Suppose, however, that Alice fired a gun that expelled bullets whose muzzle velocity $u$ was 90 percent of the speed of light. The \quotes{bullets}, if you insist on getting practical about it, could be pulses of light, traveling down the train in a pipe containing a fluid in which the speed of light was only 0.9 feet per nanosecond. (It is only the speed of the light \tit{in vacuum} that is the same in all frames of reference.) If the addition law (\ref{eq:Morin_04.1}) fails when $u = c$, it would be surprising if the law worked very well when $u$ was $0.9c$ -- and in fact it does not. Both (\ref{eq:Morin_04.1}) and the frame independence of the special velocity $c$ turn out to be a special case of a very general rule for compounding velocities that works whether or not the speeds involved are small compared to the speed of light. This \tit{relativistic velocity addition law} states that
\begin{equation}\label{eq:Morin_04.2}
w = \frac{u + v}{1 + \left( \frac{u}{c}\right) \left( \frac{v}{c}\right)}\,.
\end{equation}

If $u$ and $v$ are both small compared with the speed of light, then $u/c$ and $v/c$ are both small numbers. Their product is then a small fraction of a small number -- i.e. a \tit{very} small number -- so the relativistic rule (\ref{eq:Morin_04.2}) differs from the more familiar nonrelativistic rule (\ref{eq:Morin_04.1}) only by a factor that differs insignificantly from $1$. If, on the other hand, $u = c$, then (\ref{eq:Morin_04.2}) requires $w$ also to be $c$, whatever the value of $v$ may be. This (\ref{eq:Morin_04.2}) is consistent with our nonrelativistic experience -- i.e. with situations in which all relevant speeds are small compared to the speed of light -- as well as with the speed of one and the same pulse of light being the same in all inertial frames of reference.

We shall now show that the more general relativistic rule (\ref{eq:Morin_04.2}) is a direct and immediate consequence of the constancy of the velocity of light, together with the principle of relativity. We shall find that if the speed of light is the same in all inertial frames of reference, then the addition law (\ref{eq:Morin_04.1}) must be replaced by (\ref{eq:Morin_04.2}) regardless of what kind of moving objects we are describing and regardless of how fast they are moving. That so much more general rule follows from the special case of the constancy of the velocity of light, together with the principle of relativity, is a remarkable demonstration of the power of that principle. In its scope, the argument that will lead us to (\ref{eq:Morin_04.2}) is analogous to our extraction of Newton first law of motion in chapter \ref{ch:Morin_01} by applying the principle of relativity to the fact that stationary bodies remain stationary in the absence of an applied force. But while it is obvious, once you have understood the principle of relativity, that the special case of stationary objects remaining stationary implies that uniformly moving objects continue in the same state of uniform motion, the connection between the special case of the constancy of the speed of light and the much more general rule (\ref{eq:Morin_04.2}) is far from obvious.

Before we embark on this important application of the principle of relativity, you might note that the explicit occurrence of the speed $c$ in (\ref{eq:Morin_04.2}), even when none of the objects or frames of reference associated with $u, v, \text{ or } w$ have anything to do with light, gives an early indication that the speed $c$ is built into the very nature of space and time. Things that move at that special speed move at that speed in all frames of reference, as a direct consequence of (\ref{eq:Morin_04.2}) itself. Pulses of light in vacuum happen to be examples of such things. But the speed $c$ has an importance that goes beyond the fact that light moves at that speed in empty space. 

To develop a strategy for deducing the relativistic addition rule (\ref{eq:Morin_04.2}), we must first ask what goes wrong when we try to justify the nonrelativistic rule (\ref{eq:Morin_04.1}). The obvious way to determine the speed of an object is to determine the time it takes it to traverse a racetrack of known length. Doing this requires two clocks, placed at the two ends of the racetrack, to determine the exact times at which the object starts and finishes the race. To arrive at the nonrelativistic velocity addition law (\ref{eq:Morin_04.1}), we implicitly assumed that people using the train frame and people using the track frame would agree on whether those two clocks were synchronized. Prior to Einstein this essential assumption was never explicitly noted. Although people realized that it could be difficult as a practical matter to arrange for two clocks in faraway places to be synchronized, they took for granted that there was nothing about it that was problematic in principle. We also implicitly assumed that the people using different frames of reference would agree on the length of the racetrack between the two clocks and on the rates at which the clocks were running. 
to hit 
The constancy of the velocity of light means that the nonrelativistic addition law (\ref{eq:Morin_04.1}) cannot be correct for an object moving at the speed of light, and therefore it means that at least some of the assumptions on which (\ref{eq:Morin_04.1}) rests must be wrong. This, in turn, casts doubt on the validity of the nonrelativistic addition law for any velocities at all. But if we are not allowed to make such assumptions about the basic instruments with which we measure velocities, how can we deduce the correct rule for compounding velocities? One way to arrive at it would be to figure out, and then take fully into account, a set of new \quotes{relativistic} rules about clock-synchronization disagreements, rates of moving clocks, and lengths of moving measurement sticks, but this takes a bit of doing. It is, in fact the usual way of arriving at the correct relativistic addition law (\ref{eq:Morin_04.2}) in most expositions of the subject. Although we will eventually construct the new set of rules about clocks and measuring sticks, at this stage we don't know any of them. Nevertheless, it is possible and useful to figure out the correct velocity addition law before learning anything about the behavior of moving clocks and measuring sticks, and this is the path we shall follow. 

The direct way to get at (\ref{eq:Morin_04.2}) is to take advantage of the fact that we know the speed of at least one thing: light. By being clever we can use light to help us measure the speed of anything else in a way that makes no use whatever of either clocks or measuring sticks. This enables us to  deduce the rule for how velocities change when the frame of reference changes, without assuming anything at all about their behavior. The idea is to let the moving object -- call it a ball -- run a  race with a pulse of light -- call it a photon. By comparing how far the ball goes with how far the photon goes, we can figure out the speed of the ball. If, for example, the photon, moving at speed $c$, covers twice as much ground as the ball, then the speed of the ball must be $c/2$. (We shall consider only the case in which the ball goes slower than the photon. Later we will see that there is something highly problematic with balls that move faster than light.)

This neat idea runs into an immediate difficulty. Although the photon and the ball  start their race in the same place, they will be in different places at the end of the race. But to compare how much ground they cover during the race, we must be able to determine exactky where the ball is at the precise moment the photon reaches the finish line. To do this we need to synchronize clocks, one at the finish line and one with the ball. We can then determine where the ball is at the moment the photon reaches the finish line, by noting where the ball is when its clock reads exactly the same time that the clock at the finish line reads at the moment the photon gets to the finish line. But this requires knowing whether two clocks in different places are synchronized -- precisely the issue we wished to avoid. 

There is an easy way around this problem. Rather than end the race when the photon reaches the finish line, we arrange for it to hit a mirror and bounce back the way it came. We end the race only when the photon reencounters the ball, which is still moving in its original direction. By ending the race when the photon and the ball arrive at the same place, we solve the problem of determining, without clocks, just where the ball is along the path when the race ends. At the moment the race ends, the ball is precisely where the photon is.

Suppose this is all done on a train. We first describe the race using the train frame, Let the race start at the rear of the train and let the photon be reflected back toward the rear when it reaches the front. Suppose the photon meets the ball a fraction $f$ of the way from the front of the train back to the rear. (If, for example, the train consists of 100 identical cars, numbered $1,2,3,\ldots$ starting from the front, and the photon meets the ball in the passageway between cars 34 and 35, then $f = 0.34$.) Between the beginning and the end of the race, the photon has gone the entire length of the train \tit{plus} an additional fraction $f$ of that length, but between the beginning and end of the race, the ball has gone the entire length of the train \tit{minus} that same fraction of the length. The ratio of the distance covered by the ball to the distance covered by the photon is thus $(1 - f)/(1 + f)$.

Since the photon and the ball were both racing for the same time, this ratio must also be the ratio of their speeds. The fact that they were both racing for the same time is now unproblematic, requiring no clocks to establish it, because we have organized the race so that the photon and the ball are in the same place when they start the race and also when they finish it. So if we call the velocity of the ball in the train frame $u$, then, since the speed of the photon in either direction is $c$, 
\begin{equation}\label{eq:Morin_04.3}
\frac{u}{c} = \frac{1 - f}{1 + f}\,.
\end{equation}

The people on the train have thus measured the speed of the ball without using clocks and without having to know le length of the cars in their train. They only have to be able to count cars\footnote{If the ball met the photon some fraction of the way along a car, they would have to be able to compare the lengths of the two parts of the car, but they could do this without knowing the absolute length of either part by just counting up the number of times some measuring stick of unknown length went into both parts.}. So (\ref{eq:Morin_04.3}) summarizes a simple way to compare the velocities of two objects, which avoids using any clocks and avoids having to know any absolute distances. It is useful to rewrite (\ref{eq:Morin_04.3}) as a relation that expresses the fraction $f$ in terms of the speed $u$ of the ball and the speed of light $c\,$:
\begin{equation}\label{eq:Morin_04.4}
f = \frac{c - u}{c + u}\,.
\end{equation}

Now let us start all over again and analyze a similar race on the train, but this time using the frame of reference of the track, in which the train has a velocity $v$ (which we take to be less than the speed of light) and the ball has a velocity $w$. We take $u, v, \text{ and } w$ all to be positive -- i.e. the ball moves to the right in the train frame, and the train and ball move to the right in the track frame -- so that velocities and speeds are the same; the result we shall arrive at, however, turns out to be valid for any combination of positive and negative velocities. As before, the photon and ball both start at the rear of the train, the photon reaches the front first and bounces back toward the rear, and the race ends when the photon reencounters the ball. We again want to know what fraction of the way back along the train the photon has to go before it meets the ball. We want to express this fraction entirely in terms of the various speeds. This time the analysis is a bit more complicated, since the train is moving during the race.

We continue to assume that the photon moves with speed $c$ in both directions in the track frame. In a little while we are going to appeal to the constancy of the velocity of light to interpret this as exactly the same race as the one we analyzed in the train frame. Meanwhile, however, it might be a good idea to put the first race out of your mind while analyzing this one.  

$\cdots$

To analyze the race in the track frame we shall have to talk about track-frame distances and times. Our goal is to end up with relations like (\ref{eq:Morin_04.3}) or (\ref{eq:Morin_04.4}) that involve no times and lengths. The relations we seek involves only velocities, together with the fraction $f$ of the way back along the train the photon has to go before it meets the ball. All of the unknown distances and times we introduce will drop out at the end. 

Suppose it takes a time $T_0$ for the photon to get from the back of the train to the mirror at the front and a time $T_1$ for the reflected photon to get from the front to the point a fraction $f$ of the way back along the train where it reencounters the ball. Let $L$ be the length of the train and let $D$ be the distance between the front of the train and the ball at the moment the photon reaches the front of the train. All these times and distances are unknown track-frame times and distances, but since the reasoning that follows is entirely track-frame reasoning, and since the problematic quantities 
$D, L, T_0 \text{ and } T_1$ all drop out of the final result, this causes no difficulty. 

$\cdots$

Since $T_0$ is the time it takes the photon to get a distance $D$ ahead of the ball and since both start in the same place at the same moment and move toward the front with speeds $c$ and $w$, we must have 
\begin{equation}\label{eq:Morin_04.5}
D = c T_0 - w T_0\,.
\end{equation}

On the other hand, $T_1$ is the time it takes the photon and ball, initially a distance $D$ apart, to get back together. Since the photon covers a distance $c T_1$ during this time and the ball, $w T_1$, we have 
\begin{equation}\label{eq:Morin_04.6}
D = c T_1 + w T_1\,.
\end{equation}

Since we don't know the value of $D$, we shall eliminate it from these two relations. 
This gives us $c T_0 - w T_0 = c T_1 + w T_1$, which it is convenient to write in the form
\begin{equation}\label{eq:Morin_04.7}
\frac{T_1}{T_0} = \frac{c - w}{c + w} \,.
\end{equation}

But unfortunately we don't know the times $T_1$ and $T_0$. There is, however, a second quite similar way to get at the same ratio of these two times, by comparing the progress of the photon not with that of the ball, as we have just done, but with that of the train. Note first that $T_0$ is the time it takes the photon to get ahead of the rear of the train by the length $L$ of the train. Since the photon has speed $c$ and the train, speed $v$, 
\begin{equation}\label{eq:Morin_04.8}
L = c T_0 - v T_0 \,.
\end{equation}

Note next that $T_1$ is the time it takes the photon, moving toward the rear at speed $c$, to meet a point on the train originally a distance $f L$ away from it that moves toward it at velocity $v$. Thus 
\begin{equation}\label{eq:Morin_04.9}
L = c T_1 + v T_1 \,.
\end{equation}

We don't know the actual value of $L$ any more than we knew the actual value of $D$, but we can also eliminate $L$ from the last two equations. This gives us $c T_1 + v T_1 = f (c T_0 - v T_0)$, which gives us a second expression for the ratio of $T_1$ to $T_0$:
\begin{equation}\label{eq:Morin_04.10}
\frac{T_1}{T_0} = f \left( \frac{c - v}{c + v}  \right) \,.
\end{equation}

Although we don't know either $T_1$ of $T_0$, this expression for their ratio must be the same as the other expression (\ref{eq:Morin_04.7}). We conclude that 
\begin{equation}\label{eq:Morin_04.11}
f \left( \frac{c - v}{c + v}  \right) = \frac{c - w}{c + w} \,.
\end{equation}

This is the relation we need. All unknown times and distances have dropped out, and we have a relation involving only the fraction $f$ and some velocities. It follows immediately from (\ref{eq:Morin_04.11})that the fraction $f$ can be expressed in terms of the velocities $v$ and $w$ by
\begin{equation}\label{eq:Morin_04.12}
f = \left( \frac{c + v}{c - v} \right)\left( \frac{c - w}{c + w} \right) \,.
\end{equation}

I stress that as a piece of track-frame analysis, applicable to a race between a ball with track-frame speed $w$ and a photon with track-frame $c$, both of them on a train with track-frame speed $v$, there is nothing at all peculiar about the analysis leading to (\ref{eq:Morin_04.12}). 

As a reassuring check that we haven't made some mistake, notice the following. Suppose the velocity $v$ of the train in the track frame were $0$, Then the track frame would be the same as the train frame. Consequently $w$, the velocity of the ball in the track frame, would be the same as $u$, the velocity of the ball in the train frame. And, indeed, if you set $v$ to zero and take $w$ to be $u$, you do get back our old train-frame result (\ref{eq:Morin_04.4}). Galileo would have been quite happy with the argument leading to (\ref{eq:Morin_04.12}) (though we would have to turn the train into a boat.) Indeed, the result (\ref{eq:Morin_04.12}) remains entirely correct if we replace the photon by anything at all that moves with the same speed in both directions, and allow that speed, $c$, to be any speed at all greater than $w$ and $v$.

We do something that would not have been of Galileo's liking only when we now declare that if the photon really \tit{is} a photon, and if the speed $c$ really is the speed of light in vacuum, then these two pieces of analysis we have now completed can be taken to be train-frame and track-frame analyses of \tit{one and the same race}. In this race $u$ is the train-frame velocity of the ball, $w$ is the track-frame velocity of that same ball, and $v$ is the track-frame velocity of the train. Peculiarly, however -- and this is the \tit{only} peculiarity in the entire argument -- we now insist that the track-frame speed $c$ of that one photon (in either direction) is exactly the same as the train-frame speed $c$ of that same photon (in either direction). In both frames (and in both directions) that speed is $c$ (1 foot per nanosecond). This is the only place in the entire argument where we invoke the counterintuitive principle of the constancy of the velocity of light. 

But if we have indeed been describing one and the same race in two different frames, then $f$, the fraction of the way back from the front of the train where the photon meets the ball, must have the same value in either frame. For although there might (and indeed, as we shall see, there will) be disagreement between the two frames of reference over the length of the cars of the train, there can be no disagreement about where on the train the photon meets the ball. Their reunion could trigger an explosion, for example, that would make a smudge on the floor, which all observers in all frames could inspect later on at their leisure to confirm in which part of the car the meeting tool place. 

So the track-frame expression (\ref{eq:Morin_04.12}) for the fraction $f$ must agree with the train-frame expression (\ref{eq:Morin_04.4}). Setting them equal gives us a relation between the three velocities $w$, $u$, and $v$:
\begin{equation}\label{eq:Morin_04.13}
\left( \frac{c + v}{c - v} \right)\left( \frac{c - w}{c + w} \right) = \left( \frac{c - u}{c + u} \right)\,.
\end{equation}

It is useful to rewrite this relativistic velocity addition law in a form, like the form of the nonrelativistic addition law (\ref{eq:Morin_04.1}), in which $w$ appears on the left side and $u$ and $v$ on the right:
\begin{equation}\label{eq:Morin_04.14}
\left( \frac{c - w}{c + w} \right) = \left( \frac{c - u}{c + u} \right)  \left( \frac{c - v}{c + v} \right)\,.
\end{equation}

This is the relativistic rule that replaces the nonrelativistic rule (\ref{eq:Morin_04.1}). Instead of \tit{adding} $u$ and $v$ to get $w$ we must \tit{multiply} an expression involving $u$ by an expression of the same form involving $v$ to get a third expression of the same form invoving $w$. 

The relation between the nonrelativistic rule (\ref{eq:Morin_04.1}) and the relativistic rule (\ref{eq:Morin_04.14}) is not at all clear. To see that they are, in fact, rather simply related, one must carry out the elementary algebraic exercise of solving (\ref{eq:Morin_04.14}) for the velocity $w$ of the ball in the track frame in terms of its speed $u$ in the train frame and the speed $v$ of the train. The result is the relativistic velocity addition law stated in (\ref{eq:Morin_04.2}):
\begin{equation}\label{eq:Morin_04.15}
w = \frac{u + v}{1 + \left( \frac{u}{c}\right) \left( \frac{v}{c}\right)}\,.
\end{equation}

Although the two forms (\ref{eq:Morin_04.14}) and (\ref{eq:Morin_04.15}) of the velocity addition law are entirely equivalent ways of expressing the same relation among the three velocities $w$, $u$, and $v$, it is helpful to keep them both in mind, since one form can be more useful than the other, depending on the question one is asking. Thus the form (\ref{eq:Morin_04.15}) makes immediately evident (as noted at the beginning of this chapter) why the nonrelativistic addition law $w = u + v$ becomes quite accurate when $u$ and $v$ are small compared to the speed of light. The form (\ref{eq:Morin_04.14}), on the other hand, directly reveals the following important fact:

If the speed $u$ of the ball in the train frame and the speed $v$ of the train in the track frame are both less than the speed of light, then both $(c - u)/(c + u)$ and $(c - v)/(c + v)$ will be numbers between $0$ and $1$, this means that $(c - w)/(c + w)$ is between $0$ and $1$, which implies in turn that the speed $w$ of the ball in the track frame is also less than the speed of light. This conclusion also follows from (\ref{eq:Morin_04.15}), of course, but it is more evident as a consequence of (\ref{eq:Morin_04.14}).

Thus the obvious stratagem for producing an object moving faster that light does not work: if you have a cannon that shoot balls at 90 percent of the speed of light, and you put it on a train moving at 90 percent of the speed of light, pointing toward the front of the train, then the speed of the ball in the track frame will still be less than the speed of light. Indeed, in this particular case (\ref{eq:Morin_04.15}) tells us that the speed $w$ of the ball in the track frame will be a fraction 
$$\frac{0.9 + 0.9} {1 + (0.9)^2} = \frac{1.80}{1.81}$$ of the speed of light -- about 99.45 percent. This is the first indication we have found -- there will be several others -- that no material object can travel faster than the speed of light.

For many purposes it is helpful to abstract the relativistic addition law from the context of balls, trains and tracks and state it in terms of the velocities of certain objects(or frames of reference) with respect to other objects (or frames of reference). Let us regard the track as an object called $A$, the train as an object called $B$, and the ball as an object called $C$. The velocity $v$ of the train in the track frame we now call $v_{BA}$ -- the velocity of $B$ with respect to $A$. In the same way we call the velocity $u$ of the ball in the train frame $v_{CB}$, the velocity of $C$ with respect to $B$, and we call the velocity $w$ of the ball in the track frame, $v_{CA}$. In this language the two forms for the addition law become 
\begin{equation}\label{eq:Morin_04.16}
\left( \frac{c - v_{CA}}{c + v_{CA}} \right) = \left( \frac{c - v_{CB}}{c + v_{CB}} \right)  \left( \frac{c - v_{BA}}{c + v_{BA}} \right)\,.
\end{equation}

and 

\begin{equation}\label{eq:Morin_04.17}
v_{CA} = \frac{v_{CB} + v_{BA}}{1 + \left(\frac{v_{CB}}{c}\right) \left(\frac{v_{BA}}{c}\right)}\,.
\end{equation}

Another advantage of (\ref{eq:Morin_04.16}) over (\ref{eq:Morin_04.17}) emerges when you consider the case in which object $C$ is a rocket that itself emits a fourth object $D$. If $D$ has speed $v_{DC}$ with respect to $C$ what is the speed $v_{DA}$ of $D$ with respect to $A$? In other words, what form does the addition law take when we compount three speeds instead of just two? This leads to a great mess if we try to answer the question using form (\ref{eq:Morin_04.17}), but if we use the addition law in the form (\ref{eq:Morin_04.16}), we merely note the following:

The speed of $D$ with respect to $A$ can be arrived at by compounding the speed of $D$ with respect to $C$ and the speed of $C$ with respect to $A$. Applying the general rule (\ref{eq:Morin_04.14}) to this case gives 
\begin{equation}\label{eq:Morin_04.18}
\left( \frac{c - v_{DA}}{c + v_{DA}} \right) = \left( \frac{c - v_{DC}}{c + v_{DC}} \right)  \left( \frac{c - v_{CA}}{c + v_{CA}} \right)\,.
\end{equation}

But now we can apply (\ref{eq:Morin_04.14} again to express the quantity containing $v_{CA}$ in terms of $v_{CB}$ and $v_{BA}$ to get
\begin{equation}\label{eq:Morin_04.19}
\left( \frac{c - v_{DA}}{c + v_{DA}} \right) = \left( \frac{c - v_{DC}}{c + v_{DC}} \right)  \left( \frac{c - v_{CB}}{c + v_{CB}} \right) \left( \frac{c - v_{BA}}{c + v_{BA}} \right) \,.
\end{equation}

So to compound three speeds rather than just two, we just put a third term into the product in (\ref{eq:Morin_04.16}) to get (\ref{eq:Morin_04.19}). Evidently if $D$ were a rocket that emitted a fifth object $E$ we could continue in this way, and so indefinitely. The rule in the form (\ref{eq:Morin_04.17}) would get more and more complicated, but in the form (\ref{eq:Morin_04.16}) it would retain the same simple form.

The additional law in either of its two forms (\ref{eq:Morin_04.17}) or (\ref{eq:Morin_04.16}) continues to hold even when not all the velocities have the same sign -- even, for example, when the ball moves toward the rear of the train rather than the front. If Alice throws a ball with speed $u$ toward the \tit{rear} of a train that moves with positive velocity $v$ along the track, then the velocity $w$ of the ball along the track is given by
\begin{equation}\label{eq:Morin_04.20}
w = \frac{-u + v}{1 - \left( \frac{u}{c}\right) \left( \frac{v}{c}\right)}\,,
\end{equation}

since this is what (\ref{eq:Morin_04.2}) reduces to when $u$ is replaced by $-u$. (It is a useful exercise to check this by repeating the analysis of this chapter for the case where the race starts at the front of the train rather than at the rear.)

An easier if more abstract way to see that (\ref{eq:Morin_04.16}) and (\ref{eq:Morin_04.19}) remain valid even when all velocities are not positive is to note that, although we have derived (\ref{eq:Morin_04.16}) in the case where all three of the velocities $v_{CA}$, $v_{CB}$, and $v_{BA}$ are positive, we can introduce negative velocities by exploiting the general fact that 
\begin{equation}\label{eq:Morin_04.21}
v_{YX} = -v_{XY}\,.
\end{equation}

We can, for example, express the positive $v_{BA}$ in (\ref{eq:Morin_04.16}) in terms of the negative velocity $v_{AB}$, writing $v_{BA} = - v_{AB}$. Having done this we can then algebraically transform (\ref{eq:Morin_04.16}) into 
\begin{equation}\label{eq:Morin_04.22}
\left( \frac{c - v_{CB}}{c + v_{CB}} \right) = \left( \frac{c - v_{CA}}{c + v_{CA}} \right)  \left( \frac{c - v_{AB}}{c + v_{AB}} \right)\,.
\end{equation}

Notice that this has exactly the same form as (\ref{eq:Morin_04.16}) -- only the labels $A$ and $B$ have been interchanged. But now one of the three velocities, $v_{AB}$, is negative.

Similar tricks using (\ref{eq:Morin_04.21}) enable one to reexpress (\ref{eq:Morin_04.16}) in other equivalent forms in which either or both of the two velocities on the right are negative.

Remarkably, an experiment confirming the validity of the relativistic velocity addition law (\ref{eq:Morin_04.2}) was performed in 1851 by Fizeau, a few years after he measured the speed of light and more than half a century before Einstein wrote his 1905 paper in which the relativistic addition law first appears. 

$\cdots$
 







 





