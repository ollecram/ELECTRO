\chapter{Field Lagrangians}
\label{ch:JFrank_CFT_3}
\section{Review of Lagrangian Mechanics}\label{sec:JFrank_CFT_3.1}
In classical mechanics, Newton's second law is the primary quantitative tool. Given a potential $U(\vb{r})$, we have to solve
\begin{equation}\label{eq:JFrank_CFT_3.1}
m \ddot{\vb{r}} = - \grad U(\vb{r})
\end{equation}
subject to some initial or boundary conditions.
Then $\dot{\vb{r}}(t) = x(t) \hat{\vb{x}} + y(t) \hat{\vb{y}} + z(t) \hat{\vb{z}}$ is a vector that tells us the location of a particle of mass $m$ at time $t$ moving under the influence of $U(\vb{r})$.

As written, Newton's second law refers to Cartesian coordinates, so the above is really three equations
\begin{equation}\label{eq:JFrank_CFT_3.2}
\begin{aligned}
m \ddot{x}(t) &= - \pdv{U}{x}\,, \\
m \ddot{y}(t) &= - \pdv{U}{y}\,, \\
m \ddot{z}(t) &= - \pdv{U}{z}\,. 
\end{aligned}
\end{equation}

If we want to change coordinates, we have to manipulate these equations appropriately, and that can get unwieldy. For example, if we switch to cylindrical coordinates (a mild transformation, comparatively), in which\\ $x = s \cos \phi$, $y = s \sin \phi$, and $z$ remains the same, the equations in (\ref{eq:JFrank_CFT_3.2}) become
\begin{equation}\label{eq:JFrank_CFT_3.3}
\begin{aligned}
m \left( \ddot{s} \cos \phi - 2 \dot{s} \dot{\phi} \sin \phi - s \ddot{\phi} \sin \phi - s \dot{\phi}^2 \cos \phi \right) &= - \pdv{U}{s} \pdv{s}{x} - \pdv{U}{\phi} \pdv{\phi}{x}\,, \\
m \left( \ddot{s} \sin \phi + 2 \dot{s} \dot{\phi} \cos \phi + s \ddot{\phi} \cos \phi - s \dot{\phi}^2 \sin \phi \right) &= - \pdv{U}{s} \pdv{s}{y} - \pdv{U}{\phi} \pdv{\phi}{y}\,, \\
m \ddot{z}(t) &= - \pdv{U}{z}\,. 
\end{aligned}
\end{equation}

While Newton's second law can rarely be solved analytically, it is even harder to glean information when the physics is obscured by a different choice of coordinate system. Even the free particle solution is complicated in cylindrical coordinates. You would have to take linear combinations of the equations above to isolate, for example, equations for $\ddot{s}$ and $\ddot{\phi}$. 

Enter the Lagrangian, and the Euler-Lagrange equations of motion are coordinate invariant, meaning that they are structurally identical in all coordinate systems. We define the Lagrangian, $L = T - U$, the kinetic energy minus the potential energy for a particular physical configuration. Then we can integrate to get a functional (a function that takes a function and returns a number), the action:
\begin{equation}\label{eq:JFrank_CFT_3.4}
S[\vb{r}(t)] = \bigintsss_{t_0}^{t_f} L(\vb{r}(t), \dot{\vb{r}}(t)) dt =
\bigintsss_{t_0}^{t_f} \left(\frac{1}{2} m \, \dot{\vb{r}} \cdot \dot{\vb{r}} - U(\vb{r})\right) dt\,.
\end{equation}

The action assigns a number to a given trajectory, described by the vector pointing from the origin to the location of the particle at time $t$: $\vb{r}(t)$ (defined for all $t \in [t_0, t_f]$). The \quotes{dynamical trajectory,} the one taken by the particle according to Newton's second law, is the one that minimizes $S$.\footnote{Technically, we'll be \tit{extremizing} $S$, by setting its derivative to zero, but that extremum is, typically, a minimum.} That minimization occurs over the space of all possible paths connecting the initial and final locations of the particle.

When thinking about the minimum value for $S$, we have to compare apples with apples. Physical observation provides the relevant physical boundary conditions: we see the particle at location $\vb{r}_0$ at time $t_0$, and then we observe it later at location $\vb{r}_f$, time $t_f$.  What happens in between is the question. In [CFT] Figure 3.1, we see two different paths connecting the same endpoints; which one minimizes the action? In order to carry out the minimization, we perturb about the dynamical trajectory (which is at this point unknown). Let $\vb{r}(t)$ be the true trajectory, the solution to Newton's second law with appropriate boundary values. Then an arbitrary perturbation looks like: $\vb{r}(t) + \vb{u}(t)$ where $\vb{u}(t_0) = \vb{u}(t_f) = 0$ to respect the boundary conditions. How does the action respond to this arbitrary perturbation? If we take $\vb{u}(t)$ small, then we can expand the action, 
\begin{equation}\label{eq:JFrank_CFT_3.5}
\begin{aligned}
S[\vb{r}(t) + \vb{u}(t)] &= \bigintsss_{t_0}^{t_f} L(\vb{r}+\vb{u}, \dot{\vb{r}}+\dot{\vb{u}}) dt \\
&\approx \bigintsss_{t_0}^{t_f} L(\vb{r}, \dot{\vb{r}}) dt  
+  \bigintsss_{t_0}^{t_f} \left[\pdv{L}{\vb{r}} \cdot \vb{u} + \pdv{L}{\dot{\vb{r}}} \cdot \dot{\vb{u}} \right] dt \\
& = S[\vb{r}(t)] + \delta S\,,
\end{aligned}
\end{equation}
where we obtain the second line by Taylor expansion in $\vb{u}(t)$ and $\dot{\vb{u}}(t)$ (assuming both are small). The notation here is convenient, but warrants definition. The derivative $\pdv*{L}{\vb{r}}$ is really just the gradient of $L$, what we would normally call $\grad L$. Similarly, $\pdv*{R}{\dot{\vb{r}}}$ is a vector whose components are the derivative of $L$ with respect to $\dot{x}$, $\dot{y}$, and $\dot{z}$ (in cartesian coordinates):
\begin{equation}\label{eq:JFrank_CFT_3.6}
\pdv{L}{\dot{\vb{r}}} \equiv \pdv{L}{\dot{x}} \vb{\hat{x}} + \pdv{L}{\dot{y}} \vb{\hat{y}} + \pdv{L}{\dot{z}} \vb{\hat{z}}\,.
\end{equation}
Now for the trajectory described by $\vb{r}(t)$ to be a minimum, we must have $\delta S = 0$ ($\delta S$ is as close as we can get to a \quotes{derivative} for $S$, and we demand that it vanish) for all $\vb{u}$. To see what this implies about $\vb{r}(t)$ itself, note that $\vb{u}$ is arbitrary, but once it has been chosen, $\dot{\vb{u}}$ is determined. We'd like to write all of $\delta S$ in terms of the arbitrary $\vb{u}$. We can do that by integrating the second term in $\delta S$ by parts, noting that $\vb{u}$ vanishes at the end points of the integration (by construction):
\begin{equation}\label{eq:JFrank_CFT_3.7}
\begin{aligned}
\delta S &= \bigintsss_{t_0}^{t_f} \left[\pdv{L}{\vb{r}} \cdot \vb{u} + \pdv{L}{\dot{\vb{r}}} \cdot \dot{\vb{u}} \right] dt \\
         &= \bigintsss_{t_0}^{t_f} \left[\pdv{L}{\vb{r}} - \dv{}{t}\pdv{L}{\dot{\vb{r}}} \right] \cdot \vb{u} \, dt = 0
\end{aligned}
\end{equation}
and for this to hold for all excursions $\vb{u}$, we must have
\begin{equation}\label{eq:JFrank_CFT_3.8}
\pdv{L}{\vb{r}} - \dv{}{t} \pdv{L}{\dot{\vb{r}}} = 0\,,
\end{equation}
the Euler-Lagrange equations of motion. 
It is important to remember that we have not assumed any particular coordinates for $\vb{r}(t)$. We can write $L$ in terms of any coordinates we like; the equations of motions do not change. Sometimes we highlight that by writing:
\begin{equation}\label{eq:JFrank_CFT_3.9}
\pdv{L}{q^i} - \dv{}{t} \pdv{L}{\dot{q}^i} = 0\,,\quad \text{for}\: i= 1, 2,3, 
\end{equation}
where $\{ q^i\}_{i=1}^3$ are the \quotes{generalized coordinates.}

This procedure, of extremizing an action, may seem obscure. Why should a peculiar combination of kinetic and potential energy lead to equations of motion like (\ref{eq:JFrank_CFT_3.9}) that reproduce Newton's second law? And what, if any, physical insight does the story here contain? It is interesting that for $U = 0$, there is a natural geometric interpretation for both the action and the process of minimizing. The action is proportional to the length squared along the curve defined by $\vb{r}(t)$, and its extremization is really length minimization. For $U \neq 0$, we lose the length interpretation and rely on the correctness of the equations of motion (meaning the reproduction of Newton's second law) in Cartesian coordinates to justify the procedure. 

\subsection*{Example}  
Of the two pieces that make up $L$, the kinetic energy is always the same. For a particle of mass $m$ moving along the path $\vb{r}(t)$, $T=\frac{1}{2} m \,\dot{\vb{r}}(t) \cdot \dot{\vb{r}}(t)$. The potential energy depends on the physical system (the mass is attached to a spring, or it's charged and in the presence of an electric field, etc.); we'll leave it as $U(\vb{r})$ for now. In Cartesian coordinates,  $T=\frac{1}{2} m (\dot{x}^2 + \dot{y}^2 + \dot{z}^2)$, and 
$U(x, y, z)$ is the potential energy. Then using (\ref{eq:JFrank_CFT_3.9}), we recover precisely (\ref{eq:JFrank_CFT_3.2}). If we instead use cylindrical coordinates, 
$T=\frac{1}{2} m \, \left(\dot{s}^2 + s^2 \dot{\phi}^2 + \dot{z}^2 \right)$ with $U(s, \phi, z)$ as the potential energy. Now our three equations of motion still come from (\ref{eq:JFrank_CFT_3.9}) using 
$q^1=s,\, q^2=\phi,\text{ and } q^3=z$; we get
\begin{equation}\label{eq:JFrank_CFT_3.10}
\begin{aligned}
m \ddot{s} - m s \dot{\phi}^2 &= - \pdv{U}{s}\\
m \dv{\left( m s^2 \dot{\phi} \right)}{t} &= - \pdv{U}{\phi}\\
m \ddot{z} &= - \pdv{U}{z}\,.
\end{aligned}
\end{equation}

These are much better than (\ref{eq:JFrank_CFT_3.3}), although they are still a coupled nonlinear set in general. \\

\tbf{Problem 3.1}

\tbf{Problem 3.2}

\tbf{Problem 3.3}
\begin{equation}\label{eq:JFrank_CFT_3.11}
S = \int{L dt}\,,
\end{equation}
\begin{equation}\label{eq:JFrank_CFT_3.12}
S = \alpha \int{\bar{L} d \bar{t}}\,,
\end{equation}


\subsection{Length}\label{eq:JFrank_CFT_3.1.1}  
For a particle moving along a trajectory given by $\vb{r}(t)$, as in [CFT] Figure 3.2, we know that the  distance traveled in a time interval $dt$ is just

\begin{equation}\label{eq:JFrank_CFT_3.13}
d l = \sqrt{\dv{\vb{r}}{t} \cdot \dv{\vb{r}}{t}}
\end{equation}

so that the length of the trajectory, as the particle goes from $\vb{a}$ at time $t_0$ to $\vb{b}$ at time $t_f$, is 
\begin{equation}\label{eq:JFrank_CFT_3.14}
l = \int_{\vb{a}}^{\vb{b}} dl = \int_{t_0}^{t_f} \sqrt{\dv{\vb{r}}{t} \cdot \dv{\vb{r}}{t}} dt\,.
\end{equation}

Thie expression is \quotes{reparametrization invariant,} meaning that the parameter $t$ can be changed to anything else without changing the form of $l$ (you just rename $t$). That parameter freedom makes good geometrical sense: the length of a curve is independent of how you parametrize the curve (as long as you go from the correct starting point to the correct stopping point, of course). The integrand of the length is related to the free-particle Lagrangian; it looks like the square root of the kinetic energy (up to constants). The extremizing trajectories of the action are length-minimizing curves, i.e., straight lines. 

In special relativity, we learn to measure lengths differently. For a particle moving in space-time, with $x^\mu(t)$ a vector pointing from the origin to its space-time location at time $t$ (we could use a different parameter of course), we have:
\begin{equation}\label{eq:JFrank_CFT_3.15}
d l = \sqrt{- \dv{x^\mu}{t} \dv{x_\mu}{t}}\,,
\end{equation}
(the minus sign is just to make the length real, pure convention based on our choice of signs in the Minkowski metric), and again, we could use any parameter we like in $l = \int dl $. 

% BOX 1 **START** -- << Link eq. 3.15 to the spacetime interval ds of special relativity >>
\parindent=0pt  % Set the paragraph indentation to 0 (normal = 10pt) before the box 
\parbox{\textwidth}{\begin{mdframed}[style=MyFrame] %Added "\parbox{\textwidth}{"
%\lipsum[1]
That $dl$ in eq. (\ref{eq:JFrank_CFT_3.15}) is the infinitesimal \tit{spacetime interval}--hence the fundamental invariant quantity in SR--follows from a few logical steps that we briefly recall in this box (where I denote a spacetime interval as $ds$, rather than $dl$):
\begin{enumerate}[i)]
\item It is a most fundamental assumption in SR that $-c^2 dt^2 + d{\vb{r} \cdot d\vb{r}} = \eta_{\mu \nu} dx^\mu dx^\nu = dx_\nu dx^\nu$--the \tit{spacetime interval} associated to a pair of events--is a scalar quantity whose numerical value is the same in every inertial frame. Moreover, if the two events belong to the spacetime trajectory of a material particle, any such interval must be \tit{time-like}, hence this scalar quantity is never greater than $0$ .    
\item In a reference frame where the particle is istantaneously at rest, the spatial part $d{\vb{r}}$ of the interval is clearly $0$, hence for two events along the worldline of the particle the scalar quantity $d\tau^2 \equiv ds^2 = c^2 dt^2 - d{\vb{r} \cdot d\vb{r}} = - dx_\nu dx^\nu$ takes a positive value. $d\tau$ is also called the particle's \tit{proper time}. 
\item Because of i), computing the integral of $d\tau$ over a finite time interval $t_f - t_0$ yields, in all inertial frames,  the same numerical value. 
\item Different choices can be made to parametrize the particle trajectory. In eq. (\ref{eq:JFrank_CFT_3.15}) and the equivalent eq. (\ref{eq:JFrank_CFT_3.16}) the parameter chosen is $t$, i.e. the \tit{time} coordinate of events along the particle trajectory, as measured in a given inertial frame. 
\end{enumerate}  
%\lipsum[2]
\end{mdframed}} %Added a closing "}" here
\parindent=10pt % Set the paragraph indentation to normal (10pt) 
% BOX 1 ** END ** -- << Link eq. 3.15 to the spacetime interval ds of special relativity >>

Sticking with time, we can compute the length of a curve in special relativity:
\begin{equation}\label{eq:JFrank_CFT_3.16}
l = \int_{t_0}^{t_f} \sqrt{c^2 - \dv{\vb{r}}{t} \cdot \dv{\vb{r}}{t}} dt\,.
\end{equation}

The action for special relativity is more closely related to length that the one in non-relativistic mechanics. We start with an integrand proportional to $dl$ itself, not the length-squared-like starting point of the non-relativistic action (the kinetic energy term is missing the square root that would make it proportional to length). For an arbitrary curve parameter, $\lambda$, we have relativistic action 
\begin{equation}\label{eq:JFrank_CFT_3.17}
S[x(\lambda)] = \alpha \int_{\lambda_0}^{\lambda_f} \sqrt{- \dv{x^\mu}{\lambda} \dv{x_\mu}{\lambda}} d\lambda\,,
\end{equation}
and reparametrization invariance tells us we can make $\lambda$ anything, and the coordinate time is a good choice. Proper time is another natural candidate for curve parameter. In that case, you have the side constraint defining proper time, namely, 
\begin{equation}\label{eq:JFrank_CFT_3.18}
\dv{t}{\tau} = \frac{1}{\sqrt{1 - \frac{\vb{v} \cdot \vb{v}}{c^2}}} \,.
\end{equation}


\tbf{Problem 3.4}

\tbf{Problem 3.5}

\tbf{Problem 3.6}

\tbf{Problem 3.7}

\begin{equation}\label{eq:JFrank_CFT_3.19}
S = - mc \int{\sqrt{- \dv{x^\mu}{t} \dv{x_\mu}{t}}\, dt} + \alpha \int{\dv{x^\mu}{t} A_\mu \, dt} \,,
\end{equation}


\subsection{Noether's Theorem}\label{eq:JFrank_CFT_3.1.2}  
Aside from the uniformity of the equations of motion, the Lagrange approach makes it easy to see/prove Noether's theorem connecting symmetries to conservation laws. Suppose we have a free particle, so that there is no potential energy, and we work in Cartesian coordinates. 
Then $L = \frac{1}{2} m \left(\dot{x}^2 + \dot{y}^2 + \dot{z}^2 \right)$, and the Lagrangian (and hence the action) is insensitive to a shift in coordinates; i.e., introducing 
$\bar{x} = x + x_0$, $\bar{y} = y + y_0$ and $\bar{z} = z + z_0$, for arbitrary constants 
$\{x_0, y_0, z_0\}$ doesn't change the value or form of $L$, which is just    
$\bar{L} = \frac{1}{2} m \left(\dot{\bar{x}}^2 + \dot{\bar{y}}^2 + \dot{\bar{z}}^2 \right)$. 
Noether's theorem says that this \quotes{symmetry} (or isometry) is associated with conserved quantities. Those quantities are easy to isolate and interpret given the Euler-Lagrange equations of motion, which looks like: 
\begin{equation}\label{eq:JFrank_CFT_3.20}
\dv{(m \dot{x})}{t} = 0, \:\:\:\dv{(m \dot{y})}{t} = 0, \:\:\:\dv{(m \dot{z})}{t} = 0\,.
\end{equation}

Since there is no coordinate dependence in $L$, the quantities $m \dot{x}$, $m \dot{y}$ and
$m \dot{z}$ all take on constant values, and we recognize these as the three components of momentum for the particle. 

As another example, suppose we are working in spherical coordinates, for which 
\begin{equation}\label{eq:JFrank_CFT_3.21}
T = \frac{1}{2} m \left(\dot{r}^2 + r^2 \dot{\theta}^2 + r^2 \sin^2 \theta \dot{\phi}^2 \right)
\end{equation}

and we have a potential that is spherically symmetric, depending only on $r$: $U(r)$. Then think of the equation of motion for $\phi$, a variable that does not appear in the Lagrangian at all: 
\begin{equation}\label{eq:JFrank_CFT_3.22}
\dv{}{t} \left( m r^2 \sin^2 \theta \dot{\phi} \right) = 0
\end{equation}
which tells us that $m r^2 \sin^2 \theta \dot{\phi}$ is a constant of the motion (an angular momentum, by the looks of it). 

These are cases of \quotes{ignorable} coordinates -- if a Lagrangian does not depend explicitly on one of the coordinates, call it $q_c$, then the associated \quotes{momentum} $\pdv*{L}{q_c}$ is conserved (meaning constant along the trajectory). That's Noether's theorem in this context, and the symmetry is $q_c \rightarrow q_c + A$ for constant $A$ (a coordinate shift).  

\section{Fields}\label{sec:JFrank_CFT_3.2}
Can we formulate an action for fields , as we did for particles? The advantages to such a formulation are the same for fields as for particles: compactness, structure-revelation, and access to conservation laws via Noether's theorem. We'll start with a field equation, the field version of an equation of motion, and develop an action that returns that field equation when extremized.

We know several field equations by now. The simplest one governs a scalar field $\phi$ in vacuum:
\begin{equation}\label{eq:JFrank_CFT_3.23}
\Box \phi \equiv - \frac{1}{c^2} \pdv[2]{\phi}{t} + \laplacian{\phi} = 0\,,
\end{equation}

just the wave equation with fundamental speed $c$. This field equation is analogous to Newton's second law -- it tells us how to find $\phi$ for a source-free region of space-time (with boundary conditions specified, ensuring unique solution). The details of the coordinate system are hidden inside $\laplacian{\phi}$. 

Is there an underlying action minimization that gives the field equation? Define the following action:
\begin{equation}\label{eq:JFrank_CFT_3.24}
S[\phi] = \frac{1}{2} \bigintsss{\left[-\left(\pdv{\phi}{x^0} \right)^2 + \grad{\phi} \vdot \grad{\phi} \right] d^4x }\,,
\end{equation}
with $d^4x \equiv dx^0 d\tau$, where $x^0\equiv ct$ is the temporal coordinate, with dimension of length, and $d\tau$ is the spatial volume element (not proper time!). Thus, we integrate in all four dimensions over some domain with a boundary (could be at infinity). If we write
\begin{equation}\label{eq:JFrank_CFT_3.25}
\pdv{\phi}{x^\mu} \dot{\equiv} \mqty( \pdv{\phi}{x^0} \\ \\
                                     \pdv{\phi}{x} \\ \\
                                     \pdv{\phi}{y} \\ \\
                                     \pdv{\phi}{z}) \equiv \partial_\mu \phi \equiv \phi_{, \mu}
\end{equation}
working in Cartesian coordinates, then we can express the action neatly as 
\begin{equation}\label{eq:JFrank_CFT_3.26}
S[\phi] = \frac{1}{2} \int{\partial_\mu \phi\,\eta^{\mu \nu} \partial_\nu \phi } \,d^4x\,,
\end{equation}

where $\eta^{\mu \nu}$ is the Minkowski metric, 
\begin{equation}\label{eq:JFrank_CFT_3.27}
 \eta^{\mu \nu} \dot{=} \mqty( -1 & 0 & 0 & 0 \\ 
                                0 & 1 & 0 & 0 \\ 
                                0 & 0 & 1 & 0 \\ 
                                0 & 0 & 0 & 1)
\end{equation}
in Cartesian coordinates. Notice that (\ref{eq:JFrank_CFT_3.26}) is just the analogue of the free $(U = 0)$ particle action (\ref{eq:JFrank_CFT_3.4}) with $t$-derivatives generalized to coordinate derivatives, and the dot product generalized to the space-time inner product. 

Now suppose we do the same thing as before: introduce a perturbation to the field that vanishes on the boundaries of the domain (again, the observables live on the boundary, and we don't want our additional arbitrary function to spoil those). Take arbitrary $\chi$ (a function of position and time) and expand the action
\begin{equation}\label{eq:JFrank_CFT_3.28}
\begin{aligned}
S[\phi + \chi]  & = \frac{1}{2} \int{\partial_\mu (\phi + \chi) \,\eta^{\mu \nu} \partial_\nu (\phi + \chi) } \,d^4x\,, \\ 
                & \approx \frac{1}{2} \int{\partial_\mu \phi\,\eta^{\mu \nu} \partial_\nu \phi } \,d^4x\, +  \int{\partial_\mu \phi\,\eta^{\mu \nu} \partial_\nu \chi } \,d^4x\,.
\end{aligned}
\end{equation}
The first term is just the original action value, and it is $\delta S$ -- the second term -- that we will set equal to zero for all $\chi$ to establish a minimum. 

% BOX 2 **START** -- << Link eq. 3.15 to the spacetime interval ds of special relativity >>
\parindent=0pt  % Set the paragraph indentation to 0 (normal = 10pt) before the box 
\parbox{\textwidth}{\begin{mdframed}[style=MyFrame] %Added "\parbox{\textwidth}{"
Note that the last integral in the second line of eq. (\ref{eq:JFrank_CFT_3.28}) accounts for two equal terms which contribute to $\delta S$ -- hence the lack of the factor $1/2$ in front of that integral. 
An extra term being omitted -- hence the approximation symbol '$\approx$' --  is the integral of $\partial_\mu \chi\,\eta^{\mu \nu} \partial_\nu \chi$. Neglecting this term seems to be justified by the fact that it does not depend on $\phi$, hence no condition could arise from this term that could be translated into a condition on $\phi$ for the action to reach a minimum. However, this argument is weak/vague, hence this point needs to be clarified. 
\end{mdframed}} %Added a closing "}" here
\parindent=10pt % Set the paragraph indentation to normal (10pt) 
% BOX 2 ** END ** -- << Link eq. 3.15 to the spacetime interval ds of special relativity >>

As in the particle case, the issue is that while $\chi$ is arbitrary, $\partial_\nu \chi$ is not. We want to write $\delta S$ in terms of $\chi$, and that means we have to \quotes{flip} a derivative from $\chi$ back onto $\phi$. 

The divergence theorem holds for our four-dimensional integrals just as it does for three-dimensional ones:
\begin{equation}\label{eq:JFrank_CFT_3.29}
\int_{\Omega} \partial_\mu A^\mu d^4x = \oint_{\delta \Omega} A^\mu d^3x
\end{equation}
where $\Omega$ is the domain of the integration with boundary $\delta \Omega$ and $d^3x$ is the surface \quotes{area} element. 

Keeping in mind that $\chi$ vanishes on the boundary of the domain, 
\begin{equation}\label{eq:JFrank_CFT_3.30}
\int \partial_\nu (\partial_\mu \phi \eta^{\mu\nu} \chi) d^4x = \oint (\partial_\mu \phi \eta^{\mu\nu} \chi) d^3x = 0\,,
\end{equation}
but using the product rule, we also have
\begin{equation}\label{eq:JFrank_CFT_3.31}
\int \partial_\nu (\partial_\mu \phi \eta^{\mu \nu} \chi) d^4x = \int \chi (\partial_\nu \partial_\mu \phi \eta^{\mu\nu}) d^4x + \int{\partial_\mu \phi\,(\eta^{\mu \nu} \partial_\nu \chi)} \,d^4x\,.
\end{equation}
Since the whole thing must be zero by (\ref{eq:JFrank_CFT_3.30}), the two terms on the right are equal and opposite, and the second term is the one appearing in $\delta S$ from (\ref{eq:JFrank_CFT_3.28}). Substituting the (negative of the) first term there, we have the desired dependence on $\chi$ alone:
\begin{equation}\label{eq:JFrank_CFT_3.32}
\delta S = - \int{(\eta^{\mu\nu} \partial_\mu \partial_\nu \phi)} \chi  \,d^4x\,,
\end{equation}
and for this to be zero for all $\chi$, so that $\phi$ is really minimizing, we must have:
\begin{equation}\label{eq:JFrank_CFT_3.33}
- \eta^{\mu\nu} \partial_\mu \partial_\nu \phi = 0
\end{equation}
which is precisely $\Box \phi = 0$, the field equation that was our original target. 

The more general setting (although still in Cartesian coordinates) is:
\begin{equation}\label{eq:JFrank_CFT_3.34}
S[\phi] = \int \Lg(\phi, \partial_\mu \phi) d^4x\,,
\end{equation}
where $\Lg$ is a \quotes{Lagrangian} that takes, in theory, $\phi$ and $\partial_\mu \phi$ as arguments. The variation in this case yields the analogue of the Euler-Lagrange equations from above:
\begin{equation}\label{eq:JFrank_CFT_3.35}
\pdv{\Lg}{\phi} - \partial_\mu \left(\pdv{\Lg}{(\partial_\mu \phi)} \right) = 0\,.
\end{equation}
which can be compared with (\ref{eq:JFrank_CFT_3.8}). 

\tbf{Problem 3.8}

\tbf{Problem 3.9}

\tbf{Problem 3.10}

\tbf{Problem 3.11}

\tbf{Problem 3.12}
\begin{equation}\label{eq:JFrank_CFT_3.36}
S[\psi, \chi] = \int{\left( \frac{1}{2} \psi_{,\mu} \eta^{\mu\nu} \psi_{,\nu} +  \frac{1}{2} \chi_{,\mu} \eta^{\mu\nu} \chi_{,\nu}   \right)}  \,d^4x\,,
\end{equation}

\tbf{Problem 3.13}

\tbf{Problem 3.14}
\begin{equation}\label{eq:JFrank_CFT_3.37}
S[\phi] = - K^2 \int{ \sqrt{1 - \frac{\psi_{,\mu} \psi_{,\nu} \eta^{\mu\nu} }{K^2}} }  \,d^4x\,,
\end{equation}


\tbf{Problem 3.15}
\begin{equation}\label{eq:JFrank_CFT_3.38}
S[\Psi] = \int{\Lg(\Psi, \Psi^*, \dot{\Psi}, \dot{\Psi}^*, \Psi_{,j}, \Psi^*_{,j})}  \,d^4x\,,
\end{equation}
where dots denote time derivatives, and $\Psi_{,j} \equiv \pdv*{\Psi}{x^j}$ for $j=1,2,3$, the spatial derivatives. 

\begin{equation}\label{eq:JFrank_CFT_3.39}
- \pdv{}{t} \left(\pdv{\Lg}{\dot{\Psi}^*}\right)
- \pdv{}{x^j} \left(\pdv{\Lg}{\Psi^*_{,j}}\right) + \pdv{\Lg}{\Psi^*} = 0
\end{equation}
and the complex conjugate of this equation (varying with respect to $\Psi$ instead of $\Psi^*$).

\begin{equation}\label{eq:JFrank_CFT_3.40}
\Lg = \frac{\hbar^2}{2m} \grad \Psi^* \cdot \grad \Psi + V \Psi^* \Psi - \frac{i \hbar}{2} 
(\Psi^* \dot{\Psi} - \Psi \dot{\Psi}^*) \,, 
\end{equation}
with constants $\hbar$ and $m$ and real function $V$. Run it through your field equation (either one) and see what you get. 
 








  

 