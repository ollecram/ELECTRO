\chapter{Morin -- Combining (Small) Velocities}
\label{ch:Morin_02}
In chapter \ref{ch:Morin_01} we examined the power of the principle of relativity, deducing the non entirely obvious outcomes of certain collisions by considering other collisions whose outcomes were self evident. We must now emphasize that besides using the principle of relativity , we repeatedly made implicit use of another rule that enabled us to relate the velocity of a ball in the train frame to its velocity in the station frame. 

The rule we implicitly made use of in chapter \ref{ch:Morin_01} goes under the name of \tit{nonrelativistic velocity addition law}. While it may strike you as obvious it is not, in fact, exactly correct.  The rule is accurate to a phenomenally high degree of precision when all relevant velocities are no more than many thousands of feet per second, but when velocities become as large as many millions of feet per second, we shall find, quite surprisingly, that the rule has to be modified.

\quotes{Nonrelativistic} is an unfortunate term, but everybody uses it and so shall we. It does not mean, as you might think, \quotes{in contradiction to the principle of relativity.} It comes from the fact that the body of lore constructed by applying the principle of relativity to certain strange facts about motion at very high speeds has come to be known as the theory of relativity. As a result, the term \quotes{nonrelativistic} refers to how we thought the world behaved before we learned about the theory of relativity. Since at low speeds things actually do behave almost exactky the way we used to think they did before we learned about the theory of relativity, \quotes{nonrelativistic} means valid to a high degree of accuracy when all speeds are sufficiently small. How small will emerge in subsequente chapters, but note in this regard that even the speed of bullets from guns (many thousands of feet per second) sound as very small indeed. 

Before stating the nonrelativistic velocity addition law, we must establish a convention on the direction of motion along which velocities are taken to be positive. For almost all of the points we shall be making, it suffices to consider objects confined to move along a single direction, which we shall often take to be that of a long straight railroad track. There are only two possible directions of motion along such a track, and we need names for them. Let us therefore take the tracks to run east-west and agree that motion to the east is assigned a positive velocity, while motion to the west is given a negative velocity. Thus a ball going west at \tit{speed} of 5 f/sec has, according to our convention, a \tit{velocity} of -5 f/sec.

There is a certain amount of subtlety (but not very much) in these matters. A velocity is always defined with respect to a frame of reference. Let a train move east at 10 f/sec. Suppose a ball moves toward the rear of the train at 3 f/sec so that in the track (or station) frame it moves east at only 7 f/sec. In the track frame the velocity of the ball is +7 f/sec, since it is moving east. But in the train frame the ball is moving toward the rear of the train--i.e. toward   the west. So its train-frame velocity is -3 f/sec. It is important to keep in mind that in this context \quotes{toward west} means \quotes{toward the western end of the train}. West is a direction, not a place. If the train is heading from California to New York, the ball is getting farther from California (which many people think of as \quotes{the West}) even though it is moving toward the western end of the train. This, of course, is because the train is moving away from California at a higher speed than the ball is moving west in the train frame.

In pictures of events along the track in various frames of reference, we shall take the tracks to be more or less horizontal and shall follow the convention of mapmakers, taking east to be to  the right, and west to be to the left.  

Let $X$ be an object (a ball if you like) moving uniformly along the tracks. There is a frame of reference in which the velocity of $X$ is $0$--in which $X$ is stationary. This frame is, of course, the frame of reference of a train that is moving uniformly along the tracks with the same velocity as $X$. Such a frame is called the \tit{proper} frame of $X$. There is nothing particularly virtuous about this particular frame of reference, It's just that every uniformly moving object does have a unique frame associated with it in a natural way--namely, the frame in which it does not move. (Rest is always to be regarded as a special case of motion: motion with zero speed.) Often one leaves out the word \quotes{proper} and just refers, for example, to the frame of the ball or the ball frame. If an object is moving nonuniformly then there is no \tit{inertial} frame in which it remains stationary at all times; a frame of reference in which a nonuniformly moving object remains stationary must be a noninertial frame. 

If $Y$ is a second object movng uniformly along the tracks with a velocity different from the velocity of $X$, then we can, if we wish, describe the motion of $Y$ in the proper frame of $X$, calling $Y$'s velocity in $X$'s frame $v_{YX}$. The expression $v_{YX}$ is read as \quotes{the velocity of $Y$ with respect to $X$.} It doesn't matter whether we think of either $Y$ or $X$ as being an object or a frame of reference--the proper frame of the associated object--since both the object and the associated frame of reference move with the same velocity.

With this, we can now state in all its abstract glory the nonrelativistic velocity addition law. If $X$, $Y$, and $Z$ all move with uniform velocity along the same straight line, then 
\begin{equation}\label{eq:Morin_02.1}
v_{XZ} = v_{XY} + v_{YZ}\,. 
\end{equation}

In words, the velocity of $X$ with respect to $Z$ is the sum of the velocity of $X$ with respect to $Y$ and the velocity of $Y$ with respect to $Z$. Or, if you prefer, the velocity of $X$ in frame $Y$ and the velocity of frame $Y$ in frame $Z$.

Suppose, for example, $X$ is a ball, $Y$ is a train, and $Z$ is the station (tracks). Then \ref{eq:Morin_02.1} says that the velocity of the ball in the station frame is the velocity of the ball in the train frame plus the velocity of the train in the station frame. This should be evident when all the velocities are positive. But it also works when some of the velocities are negative. 

Usually, of course, it's simpler just to reason one's way to the answer without having to invoke the abstract form (\ref{eq:Morin_02.1}) of the addition law. Soon we shall learn that the addition law (\ref{eq:Morin_02.1}) is not exactly correct, being valid to an extremely high degree of accuracy when all speeds are not too large. When enormous velocities enter the story, or if we want the right answer to fantastically high precision, then we must use a modified form of (\ref{eq:Morin_02.1}), and it is then important to use the formula that expresses the modified addition law, since commonsense reasoning no longer gives the right answer. So you should be sure you understand how to use the nonrelativistic addition law (\ref{eq:Morin_02.1}), even when what it tells you is \quotes{obvious}.

One important consequence of (\ref{eq:Morin_02.1}) (which turns out to remain valid at any speed) is that 
\begin{equation}\label{eq:Morin_02.2}
v_{XZ} = - v_{YX}\,. 
\end{equation}

If $X$ moves with a certain speed with respect to $Y$, then $Y$ moves with that same speed with respect to $X$, but in the opposite direction. This (fairly obvious) relation follows directly from the general rule (\ref{eq:Morin_02.1}). For consider the special case in which $Z$ and $X$ are identical, so that $v_{XZ}$ becomes $v_{XX}$, the velocity of $X$ in the frame in which $X$ is stationary, i.e. in its proper frame. The velocity of $X$ in its proper frame is $0$, so we have
\begin{equation}\label{eq:Morin_02.3}
0 = v_{XX} = v_{XY} + v_{YX}\,, 
\end{equation}

and this immediately gives (\ref{eq:Morin_02.2}).

How would one go about justifying the rule (\ref{eq:Morin_02.1}) to a stubborn person who did not find it obvious? Consider this instance of it: Let a train move east in the track frame. If a ball moves east in the train frame at 5 f/sec, then in one second the ball gets 5 feet nearer the front of the train. And if the train moves at 10 f/sec in the track frame, then in one second the train gets 10 feet further east along the track. So in one second the ball gets 15 feet further east along the track--the 5 it gains on the train and the additional 10 the train gains on the track. But the ball getting 15 feet further east along the track in one second is precisely what we mean when we say the ball moves at 15 f/sec in the track frame. Who could doubt this? Indeed, I encourage you not to doubt it until you find boringly familiar the points made in this and the preceding chapter. 

But I do call your attention to an apparently innocent phrase that turns out, surprisingly, to be fraught with danger: \tit{in one second}. We have implicitly assumed that \quotes{in one second}
means the same thing in the train frame as it does in the track frame. But suppose that were not true. Suppose \quotes{in one second} in the train frame meant something different from \quotes{in one second} in the track frame. What would happen to the argument we just gave? We would have to replace \quotes{in one second} by something like \quotes{in one second according to train time} or \quotes{in one second according to track time.} The argument we just went through then starts off fine, but is a bit more cumbersome:

\hangindent=0.7cm If the ball moves east in the train at 5 f/sec then in one second according to train time it gets 5 feet further down the train. And if the train moves at 10 f/sec in the track frame, then in one second according to track time it gets 10 feet further east along the track.

\noindent But then we come to: 

\hangindent=0.7cm So \tit{in one second} the ball gets 15 feet further east along the track--the 5 it gains on the train and the additional 10 the train gains on the track.

\noindent What can the italicized \quotes{in one second} mean here? The first 5 feet are gained in one second of train time, the second 10 feet are gained in one second of track time. Collapsing both into a single, unqualified \quotes{in one second} makes no sense unless track time and train time are the same. 

For the moment, we will not pursue this any further. But be aware that the simple rule 
(\ref{eq:Morin_02.1}) telling us how velocities combine relies on the implicit assumption that there is nothing problematic about the idea of a single unique notion of time that can be used equally well in any frame of reference. It was Einstein's great insight in 1905 that this apparently obvious assumption is, in fact, false. \quotes{It came to me,} he said to a colleague many years later, \quotes{that time was suspect.} When the assumption of a unique frame-independent time fails, it takes other \quotes{obvious} assumptions down with it. 

That failure, however, is so slight as to be of no importance when all speeds of interest are small compared with that of light, as they were in the examples we examined in chapter \ref{ch:Morin_01}. So now we must turn to how the speed of light enters the story.  