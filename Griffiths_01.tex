\chapter{Griffiths -- Electrostatics}
\label{ch:Griffiths_01} 

\section{THE ELECTRIC FIELD}
\subsection{Coulomb's Law}
The \tit{electric force} exerted by a \tit{source charge} $q$ located \tit{at rest} in $\vb{r}'$ on a \tit{test charge} $Q$ located at $\vb{r}$  is (\tbf{Coulomb's law})
\begin{equation}
\label{eq:G01_coulomb}
\vb{F} = \frac{qQ}{s^2} \vu{s}
\end{equation}

where 

\begin{equation}
\label{eq:G01_separation}
\vb{s} = \vb{r} - \vb{r}'
\end{equation}

is the \tbi{separation vector} between the the two charges, directed from the source point to the test point. 

Based on the \tbi{superposition principle}, the effect of multiple source charges (at rest) $q_1, q_2, \ldots, q_n$ on the test charge $Q$ is simply the sum over all sources of the force generated by each one of them

\begin{equation}
\label{eq:G01_superposition}
\vb{F} =  \vb{F}_1 + \vb{F}_2 + \cdots + \vb{F}_n = Q \left( \frac{q_1}{s_1^2} \vu{s}_1 + \frac{q_2}{s_2^2} \vu{s}_2 + \cdots + \frac{q_n}{s_n^2} \vu{s}_n \right)
\end{equation}

\subsection{The Electric Field}
The \tbi{electrostatic field} $\vb{E}(\vb{r})$ is, by definition, the force exerted on a \tbi{unit test charge} located at the point $\vb{r}$ by all other charges (assumed at rest). With $\vb{s} = \vb{r} - \vb{r}'$, we therefore write namely

\begin{equation}
\label{eq:G01_electrostatic_field}
\vb{E}(\vb{r}) = \frac{\vb{F}}{Q} =  \sum_{i=1}^n \frac{q_i}{s_i^2} \vu{s}_i
\end{equation}

In terms of $\vb{E}(\vb{r})$ the \tbf{Coulomb's law} \ref{eq:G01_superposition} can be rewritten as 

\begin{equation}
\label{eq:G01_electrostatic_force}
\vb{F}(\vb{r}) = Q \vb{E}(\vb{r})
\end{equation}

\subsubsection*{Problem 1}
\begin{enumerate}[a)]
\item Twelve equal charges, $q$ are situated at the corners of a regular 12-sided polygon (one on each numeral of a clock face). What is the net force on a test charge $Q$ at the center?
\item Suppose \tit{one} of the 12 $q$'s is removed (the one at \quotes{6 o'clock}). What is the force on $Q$? Explain your reasoning carefully.
\item Now $13$ equal charges, $q$ are placed at the corners of a regular 13-sided polygon. What is the net force on a test charge $Q$ at the center?
\item If one of the 13 $q$'s is removed what is the force on $Q$? Explain your reasoning carefully.
\end{enumerate}

\subsubsection*{Solution}
a) The net force is $\vb{0}$ because the field generated by each charge is \tit{balanced} by the equal and opposite field generated by the charge that sits in front of it across the center.\\
b) This is the force exerted by the \tit{single unbalanced charge}, the one at \quotes{12 o'clock}: 
$$ \vb{F} = - \frac{qQ}{R^2}\, \vu{k}$$ 
c) The net force is still $\vb{0}$ because there is no direction the electric field $\vb{E}$ can take without breaking the symmetry of the arrangement. This means that the field generated by each charge is balanced, at the center, by the sum of the fields generated by the remaining charges.\\
d) Because of the previous point, the net force at the center is the \tit{opposite} of the force that would be added if the removed charge would be put back in. Assuming \quotes{6 o'clock} to be the place of the latter, the force has the same value computed for the case b). 

\subsubsection*{Example 1}
Find the electric field a distance $z$ above the midpoint between two equal charges ($q$) a distance $d$ apart. 

\subsubsection*{Solution}
Let choose a Cartesian reference frame with \tbi{origin} $O$ in the midpoint between the two charges, the $\vu{i}$ axis parallel to the line along which they are located, and the $\vu{k}$ axis orthogonal to that line. In this frame the position vectors of the two charges are $\vb{r}_1 = -d/2\vu{i}$ and $\vb{r}_2 = +d/2\vu{i}$, while the position at which the electric field must be computed is $\vb{r} =  z \vu{k}$. 

The separation vectors of the sources from the test point $\vb{r}$ are 
\begin{align*}
\vb{s}_1 &= \vb{r} - \vb{r}_1 = z \vu{k} + \frac{d}{2} \vu{i} \\
\vb{s}_2 &= \vb{r} - \vb{r}_2 = z \vu{k} - \frac{d}{2} \vu{i} 
\end{align*}

The next step is to take the sum \ref{eq:G01_electrostatic_field} and we observe that 

\begin{align*}
& s_1^2 = z^2 +(d/2)^2 = s_2^2 \\
& \vb{s}_1 + \vb{s}_2 = 2z \vu{k} = \sqrt{z^2 +(d/2)^2}  (\vu{s}_1 + \vu{s}_2)  \\
& \vu{s}_1 + \vu{s}_2 = \frac{\vb{s}_1 + \vb{s}_2}{\sqrt{z^2 +(d/2)^2}} = \frac{2z \vu{k}}{\sqrt{z^2 +(d/2)^2}} = 2 \cos \theta \vu{k}
\end{align*}

where $\theta$ is the angle between a separation vector with the $\vu{k}$ axis. Finally

\begin{equation*}
\vb{E}(\vb{r}) = \sum_{i=1}^2 \frac{q_i}{s_i^2} \vu{s}_i = \frac{q \,(\vu{s}_1 + \vu{s}_2)}{z^2 +(d/2)^2}  = \frac{2q \cos \theta \, \vu{k}}{z^2 + (d/2)^2} = \frac{2qz \, \vu{k}}{  [z^2 +(d/2)^2]^{3/2}  }
\end{equation*}

When $z \gg d$ any dependency on $d$ can be eliminated from the denominator giving, as expected 

$$\vb{E}(z) \approx \lim_{d \rightarrow 0} \vb{E}(\vb{r}) = \frac{2q \, \vu{k}}{z^2}$$ 

namely the same field as of a single particle with twice as much charge. 


\subsubsection*{Problem 2}
Find the electric field (magnitude and direction) a distance $z$ above the midpoint between equal and opposite charges ($\pm q$) a distance $d$ apart (same as Example 1, except that the charge at $x = +d/2$ is $-q$).

\subsubsection*{Solution}

The separation vectors of the sources from the test point $\vb{r}$ are 
\begin{align*}
\vb{s}_1 &= \vb{r} - \vb{r}_1 = z \vu{k} + \frac{d}{2} \vu{i} \\
\vb{s}_2 &= \vb{r} - \vb{r}_2 = z \vu{k} - \frac{d}{2} \vu{i} 
\end{align*}

The next step is to take the sum \ref{eq:G01_electrostatic_field} and we observe that 

\begin{align*}
& s_1^2 = z^2 +(d/2)^2 = s_2^2 \\
& q\,(\vb{s}_1 - \vb{s}_2) = q d\, \vu{i} = q\, \sqrt{z^2 +(d/2)^2}  (\vu{s}_1 - \vu{s}_2)  \\
& q\,(\vu{s}_1 - \vu{s}_2) =\frac{ q \, (\vb{s}_1 - \vb{s}_2)}{\sqrt{z^2 +(d/2)^2}} = \frac{qd\, \vu{i}}{\sqrt{z^2 +(d/2)^2}} = 2q \sin \theta \, \vu{i}
\end{align*}

where $\theta$ is the angle between a separation vector with the $\vu{k}$ axis. Finally

\begin{equation*}
\vb{E}(\vb{r}) = \sum_{i=1}^2 \frac{q_i}{s_i^2} \vu{s}_i = \frac{q \,(\vu{s}_1 - \vu{s}_2)}{z^2 +(d/2)^2} = \frac{qd \, \vu{i}}{  [z^2 +(d/2)^2]^{3/2}  }   = \frac{2q \sin \theta \, \vu{i}}{z^2 + (d/2)^2}
\end{equation*}

We observe that, as expected $$\lim_{d \rightarrow 0} \vb{E}(\vb{r}) = 0$$ while for $z \gg d$ (with $d$ fixed) we obtain a \tit{dipole} field: 
$$\vb{E}(z) \approx \frac{qd}{z^3}  \, \vu{i}$$ 


\subsection{Continuous Charge Distributions}

Our definition of the electric field (Eq. \ref{eq:G01_electrostatic_field}) assumes that the source of the field is a collection of discrete point charges $q_i$. If instead the charge is distributed continuosly over some region, the sum becomes an integral:

\begin{equation}
\label{eq:G01_electrostatic_field_continuous}
\vb{E}(\vb{r}) = \int \frac{\vu{s}}{s^2} \,dq
\end{equation}

If the charge is spread along a \tit{line} then $dq = \lambda dl'$, where $\lambda$ is the charge per unit length and $dl'$ is the element of length along the line; if it is smeared out over a \tit{surface} then $dq = \sigma da'$, where $\sigma$ is the charge per unit surface and $da'$ is an element of area over the surface; and if the charge fills a \tit{volume} with charge per unit volume $\rho$, then $dq = \rho d\tau'$, where $d\tau'$ is an element of volume:

\begin{equation}
\label{eq:G01_electrostatic_field_continuous_line}
\vb{E}(\vb{r}) = \int \lambda(\vb{r}') \frac{\vu{s}}{s^2} \,dl' =  \int \lambda(\vb{r}') \frac{\vb{r} - \vb{r}'}{\abs{\vb{r} - \vb{r}'}^3} \,dl'
\end{equation}

\begin{equation}
\label{eq:G01_electrostatic_field_continuous_surf}
\vb{E}(\vb{r}) = \int \sigma(\vb{r}') \frac{\vu{s}}{s^2} \,da'  =   \int \sigma(\vb{r}') \frac{\vb{r} - \vb{r}'}{\abs{\vb{r} - \vb{r}'}^3} \,da'
\end{equation}

\begin{equation}
\label{eq:G01_electrostatic_field_continuous_vol}
\vb{E}(\vb{r}) = \int \rho(\vb{r}') \frac{\vu{s}}{s^2} \,d\tau' = \int \rho(\vb{r}') \frac{\vb{r} - \vb{r}'}{\abs{\vb{r} - \vb{r}'}^3} \,d\tau'
\end{equation}
 
\subsubsection*{Example 2}
Find the electric field a distance $z$ above the midpoint of a straight line segment of length $2L$ that carries a uniform line charge \tit{density} $\lambda$. 

\subsubsection*{Solution}
Adopting a reference frame whose origin is the midpoint of the line distribution, we exploit the result of Example 1
whereby the contribution of two line elements $dl_1$ and $dl_2$ located symmetrically with respect to the origin is 
$$d \vb{E}(z) = \vu{k} \frac{2\,q\,z}{[x^2 + z^2]^{3/2}} = \vu{k} \frac{\lambda (dx_1 + dx_2) \,z}{[x^2 + z^2]^{3/2}}$$
\begin{align*}
\vb{E}(z) &= \int_{-L}^{+L} \lambda(\vb{r}') \frac{\vb{r} - \vb{r}'}{\abs{\vb{r} - \vb{r}'}^3} \,dl'\\
          &= \vu{k}  \int_{0}^{L} \frac{ 2 \, \lambda \,z \, dx}{[x^2 + z^2]^{3/2}} \\
          &=  2 \lambda z\, \vu{k} \int_{0}^{L} \frac{dx}{[x^2 + z^2]^{3/2}}  
\end{align*}

The last integral has the form 
$$\int \frac{dx}{r^3}$$
with $r = [x^2 + z^2]^{1/2}$. The \tit{primitive} of this integral (Dwight 200.3) is $$\frac{1}{z^2}\,\frac{x}{r}$$ Therefore
\begin{align*}
\vb{E}(z) &=  \frac{2 \lambda}{z} \, \left[ \frac{x}{\sqrt{x^2 + z^2}} \right]_0^L  \, \vu{k} \\
          &=  \frac{2 \lambda L }{z\sqrt{L^2 + z^2}} \, \vu{k} 
\end{align*}

When $z \gg L$ any dependency on $L$ can be eliminated from the denominator giving, as expected 

$$\vb{E}(z) \approx \lim_{L \rightarrow 0} \vb{E}(z) = \frac{2 \lambda L }{z^2} \, \vu{k} $$ 

namely the field of a point particle carrying the total charge $2 \lambda L$. 

\subsubsection*{Problem 3}
Find the electric field a distance $z$ above one end of a straight line segment of length $L$ that carries a uniform line charge $\lambda$. Check that your formula is consistent with what you would expect for the case $z \gg L$.

\subsubsection*{Solution}

\begin{align*}
\vb{r} &= z \vu{k}, \:\: \vb{r}' = x \vu{i} \\
\vb{s} &= \vb{r} - \vb{r}' = z \vu{k} - x \vu{i}, \:\: s = \sqrt{x^2 + z^2}, \:\: \vu{s} = \frac{z \vu{k} - x \vu{i}}{\sqrt{x^2 + z^2}} \\
\vb{E}(z) &=  \int_0^L \frac{\lambda \, \vu{s} \, dx}{s^2} \\
          &=  \lambda \, z \, \vu{k} \int_0^L \frac{dx}{[x^2 + z^2]^{3/2}} \, - \lambda \, \vu{i} \int_0^L \frac{x \, dx}{[x^2 + z^2]^{3/2}}
\end{align*}

Primitives of the last two integrals are (Dwight 200.03 and 201.03)
$$ \frac{1}{z^2} \frac{x}{\sqrt{x^2 + z^2}}, \:\:\: - \frac{1}{\sqrt{x^2 + z^2}}$$

Therefore

\begin{align*}
\vb{E}(z) &=  \frac{\lambda}{z} \, \left[ \frac{x}{\sqrt{x^2 + z^2}} \right]_0^L  \, \vu{k} +   \lambda \, \left[\frac{1}{\sqrt{x^2 + z^2}}\right]_0^L \, \vu{i} \\
          &= \frac{\lambda\,L}{z\,\sqrt{L^2 + z^2}}\, \vu{k} + \lambda \left(\frac{1}{\sqrt{L^2 + z^2}} - \frac{1}{z}        \right) \, \vu{i} \\
          &= \frac{\lambda}{z} \left[ \left(\frac{z}{\sqrt{L^2 + z^2}} - 1 \right) \, \vu{i} + \frac{L}{\sqrt{L^2 + z^2}}\, \vu{k}  \right] 
\end{align*}

When $z \gg L$ 

$$\vb{E}(z) \approx \lim_{L \rightarrow 0} \vb{E}(z) = \frac{\lambda L}{z^2} \, \vu{k} $$ 

\subsubsection*{Problem 4}
Find the electric field a distance $z$ above the center of a square loop (side $a$) carrying uniform line charge $\lambda$. [\tit{Hint}: Use the result of Ex. 2]

\subsubsection*{Solution}
Let $d = \sqrt{a^2/4 + z^2}$ be the distance between the point $P$, located a distance $z$ above the center of the square, and the \tit{midpoint} of each side. The magnitude of the field $\vb{E}_i^P$ generated at $P$ by the $i$-th side of the loop can be obtained from the solution of Ex. 2 with the following correspondence between variables:

$$ a \leftrightarrow 2L, \:\: d = \sqrt{(a/2)^2 + z^2} \leftrightarrow z$$  

We also observe that we must refer the solution of Ex. 2 (separately for each side) to a frame that contains both the line element and the point $P$. The angle $\phi$ formed by the $\vu{k}$ axis of such frame and the vertical $\vu{z}$ through the center of the loop is determined by  

$$\cos(\phi) = \vu{k} \cdot \vu{z} = \frac{z}{d} = \frac{z}{\sqrt{a^2/4 + z^2}}$$

Clearly, the component of $E$ orthogonal to $\vu{z}$ takes opposite values on opposite sides of the loop, so it can be ignored altogether: we must only ensure that a factor $\cos(\phi)$ be applied to the field magnitude $\abs{\vb{E}}$ at $P$ to obtain the projection of $\vb{E}$  along $\vu{z}$. 

After this preamble, the magnitude of the field at $P$ generated by a \tit{single side} is 
\begin{align*}
\abs{\vb{E}} &= \frac{2 \lambda L}{z\, \sqrt{L^2 + z^2}} \rightarrow \frac{a \lambda}{d\, \sqrt{a^2/4 + d^2}}\\
             &= \frac{a\,\lambda}  {d\, \sqrt{a^2/4 + d^2}} = \frac{a\,\lambda}    {\sqrt{a^2/4 + z^2}\sqrt{a^2/4 + a^2/4 + z^2)}} 
\end{align*}

The component of $\vb{E}$ at the point $P$ is obtained by multiplying the above result by $4 \cos(\phi) = 4z/d$:

\begin{align*}
\vb{E}^P(z) &= \frac{4 a \lambda z}  {d^2\, \sqrt{a^2/4 + d^2}} \, \vu{z} \\
            &= \frac{4 a \lambda z}  {(a^2/4 + z^2)\sqrt{a^2/2 + z^2)}}\, \vu{z}
\end{align*}
   
When $z \gg a$ 

$$\vb{E}^P(z) \approx \lim_{a \rightarrow 0} \vb{E}^P(z) =  \frac{4\,q\,\vu{z}}{z^2}$$    


\subsubsection*{Problem 5}
Find the electric field a distance $z$ above the center of a circular loop of radius $r$ carrying a uniform line charge $\lambda$.

\subsubsection*{Solution}
Let $d = \sqrt{r^2 + z^2}$ be the distance between the point $P$, located a distance $z$ above the center of the circular loop and the charge associated to an infinitesimal line element $dl'$ along the loop. Introducing polar coordinates $r, \theta$ in the plane of the loop (with the origin $O$ in the center of the latter) we observe that the vector $\vb{E}(\theta)$ generated by a line element $dl'$ located at an angle $\theta$ along the loop makes an angle $\phi$ with the axis $\vu{k}$. This angle is determined by 
$$\cos(\phi) = \vu{k} \cdot \frac{\vb{E}(\theta)}{\abs{\vb{E}(\theta)}} = \frac{z}{d} = \frac{z}{\sqrt{r^2 + z^2}}$$
  
The \tit{sum of the electric fields} at $P$ generated by two line elements separated by an angle $\pi$ yield a vector that is parallel to $\vu{k}$ (that is, orthogonal to the loop) because the projections onto the plane of the loop of the electric fields generated by the two line elements are \tit{equal in magnitude and opposite in direction}. 

In conclusion, we can obtain the total field $\vb{E}$ at the point $P$ by integrating the magnitude of the field generated by line elements multiplied by the factor $\cos(\phi) = z/d$, therefore

\begin{align*}
\vb{E}^P(z) &= \int_0^{2\pi}  \frac{\cos(\phi) \lambda  r}{d^2} d\theta  \, \vu{k} = \frac{\cos(\phi) \lambda  r}{d^2} \int_0^{2\pi} d\theta \, \vu{k}\\
            &= \frac{(2 \pi r \lambda) \cos(\phi)}{d^2} \, \vu{k} \\
            &= \frac{q\,z}{d^{3}}  \, \vu{k} \\
            &=  \frac{q \, z}{(\sqrt{r^2 + z^2})^{3}} \, \vu{k}
\end{align*}

where $q = 2 \pi r \lambda$ is the total charge on the loop. When $z \gg r$ 

$$\vb{E}^P(z) \approx \lim_{r \rightarrow 0} \vb{E}^P(z) =  \frac{q\,\vu{k}}{z^2}$$      


\subsubsection*{Problem 6}
Find the electric field a distance $z$ above the center of a flat circular disk of radius $R$ that carry a uniform surface charge density $\sigma$. What does your formula give in the limit $R \rightarrow \infty$? Also check the case $z \gg R$.

\subsubsection*{Solution}
We can think of the disk as the union of circular loops of radius $r$ and thickness $dr$, with the reasoning for each loop going just like in the previous problem. As a result the solution is 
\begin{align*}
\vb{E}^P(z) &= \int_0^R \frac{q(r)\,z}{d^{3}} dr \, \vu{k} \\
            &= \int_0^R \frac{2\pi r \sigma z}{(\sqrt{r^2 + z^2})^{3}} dr \, \vu{k} = 2\pi \sigma z \, \int_0^R \frac{r}{(\sqrt{r^2 + z^2})^{3}} dr \, \vu{k} 
\end{align*}

The primitive of the integrand is (Dwight 201.03) $-1/\sqrt{r^2 + z^2}$, therefore
\begin{align*}
\vb{E}^P(z) &=  2\pi \sigma z \, \left| \frac{-1}{\sqrt{r^2 + z^2}} \right|_0^R  \, \vu{k} \\
			&=  2\pi \sigma z \, \left[ \frac{1}{z} - \frac{1}{\sqrt{R^2 + z^2}} \right] \, \vu{k} 
\end{align*}

\subsubsection*{Problem 7}
Find the electric field a distance $z$ from the center of a spherical surface of radius $R$ that carries a uniform charge density $\sigma$. Treat the case $z < R$ (inside) as well as $z > R$ (outside). Express your answer in terms of the total charge $q$ on the sphere. [\tit{Hint:} Use the law of cosines to write $s$ in terms of $R$ and $\theta$. Be sure to take the \tit{positive} square root: $\sqrt{R^2 + z^2 -2Rz} = (R - z)$ if $R > z$, but it's $z-R$ if $R < z$.]

\subsubsection*{Solution}
We choose \tit{spherical coordinates} in a reference frame whose origin is in the center of the sphere and the field is measured at coordinate $z$ along the $\vu{k}$ axis. In spherical coordinates the area element at the surface of the sphere of radius $R$ is $$da = R d\theta (R \sin \theta) d\phi = R^2 \sin \theta d\theta d\phi$$ and the associated charge is $\sigma da$. 

If $\vu{i}$ and $\vu{j}$ are orthogonal unit vectors in the plane containing the origin and orthogonal to $\vu{k}$, the cartesian coordinates of the field \tit{test point} $\vb{r}$ and of the charge \tit{source point} $\vb{r}'$ are, respectively 
\begin{align*}
\vb{r}  &= z \vu{k} \\
\vb{r}' &= R \sin \theta \left( \cos \phi \vu{i} + \sin \phi \vu{j} \right) + R \cos \theta \vu{k}
\end{align*}
Therefore the \tit{separation vector} $\vb{s}$ and its magnitude are  
\begin{align*}
\vb{s} &= \vb{} \vb{r} - \vb{r}' = -  R \sin \theta \left( \cos \phi \vu{i} + \sin \phi \vu{j} \right) + (z - R \cos \theta) \vu{k} \\
\abs{\vb{s}}^2 &= \abs{\vb{r} - \vb{r}'}^2 = R^2 \sin^2 \theta + \left(z - R \cos \theta \right)^2 \\
              &= R^2 + z^2 -2zR \cos \theta
\end{align*}

The field at $\vb{r}$ generated by an area element $da$ is 
$$\vb{E}(\vb{r}) = \frac{\sigma da \vu{s}}{\abs{\vb{s}}^2} = \frac{\sigma da \vb{s}}{\abs{\vb{s}}^{3/2}}$$

The total field at at $z\vu{k}$ is obtained by a double integral over the sphere surface. We observe that $\vb{s}$ components in the $\vu{i}-\vu{j}$ plane whould produce a vanishing sum when integrated over the spherical coordinate $\phi$, so we can safely ignore them altogether. The field at the \tit{test point} $z\vu{k}$ is thus given by the integral
\begin{align*}
\vb{E}(z) &= \int_0^{2\pi} d\phi \int_0^{\pi} \frac{(z - R \cos \theta) \sigma R^2 \sin \theta \; d\theta }{\left(R^2 + z^2 -2zR \cos \theta \right)^{3/2}}  \:  \vu{k}\\
          &= 2 \pi \sigma R^2 \int_0^{\pi} \frac{(z - R \cos \theta) \sin \theta \; d\theta }{\left(R^2 + z^2 -2zR \cos \theta \right)^{3/2}}  \:  \vu{k}
\end{align*}

Setting $u = \cos \theta; \, du = -\sin \theta d\theta$ and considering that $\theta = 0 \rightarrow u = +1$ and $\theta = \pi \rightarrow u = -1$ we can rewrite the above expression for $\vb{E}$ as follows

$$\vb{E}(z) = 2 \pi R^2 \sigma \int_{-1}^{+1} \frac{z - Ru}{\left(R^2 + z^2 -2Rzu \right)^{3/2}}  \, du \:  \vu{k}$$

The \tit{primitive} of the function under the integral is\footnote{While this can be easily verified, no attempt was made at deriving this formula.} 
$$F(u) = \frac{1}{z^2} \frac{zu - R}{\sqrt{R^2 + z^2 - 2Rzu}}$$
whence the integral value is 
\begin{align*}
F(1) - F(-1) &=  \frac{1}{z^2} \left[ \frac{(z-R)}{\sqrt{R^2+z^2 - 2Rz}} - \frac{(-z-R)}{\sqrt{R^2+z^2 + 2Rz}}  \right] \\
             &=  \frac{1}{z^2} \left[ \frac{(z-R)}{\sqrt{(z - R)^2}} - \frac{(-z-R)}{\sqrt{(z + R)^2}}  \right] \\
             &=  \frac{1}{z^2} \left[ \frac{(z-R)}{\abs{z - R}} + \frac{(z+R)}{\abs{z + R}}  \right]
\end{align*}

We note that the espression between square brackets is $0$ \tbi{inside} the sphere ($z < R$) while its value is $2$ \tbi{outside} the sphere ($z > R$), therefore  
\begin{align*}
\vb{E}(z) &= \frac{4 \pi R^2 \sigma}{z^2} = \frac{q}{z^2}\,; \:\:\:\:\:  (z > R)\\
\vb{E}(z) &= 0\,; \:\:\:\:\:\:\:\:\:\:\:\:\:\:\:\:\:\:\:\:\:\:\:\:\:\:\:\:     (z < R)
\end{align*}
Null \tbi{inside}, the field \tbi{outside} the sphere is the same that would be generated by a \tbi{point charge} $q$ located at the center.
  
\subsubsection*{Problem 8}
Use your result in Prob. 7 to find the field inside and outside a solid sphere of radius $R$ that carries a uniform volume charge density $\rho$. Express your answer in terms of the total charge of the sphere, $q$. Draw a graph of $\abs{\vb{E}}$ as a function of the distance from the center.

\subsubsection*{Solution}
The total charge is $q = (4/3) \pi R^3 \rho$ and the field must of course be directed \tit{radially}. Using the conclusions of Prob. 7, its intensity at distance $z > R$ (outside the sphere) is 

$$\abs{\vb{E}}(z) = \frac{q}{z^2}\,; \:\:\:\:\:\:\:\:\:\: (z \ge R)$$

The field at a distance $z$ inside the sphere ($z < R$)  is the same that would be generated by a \tbi{point charge} $q(z)$ located at the center, where 
$$\abs{\vb{E}} = \frac{q(z)}{z^2} = \frac{q z^3}{R^3} \cdot \frac{1}{z^2} = \frac{qz}{R^3}\,; \:\:\: (z \le R)$$ 

\begin{tikzpicture}
\datavisualization [scientific axes=clean,
                    y axis=grid,
                    visualize as smooth line/.list={inside,outside},
                    style sheet=strong colors,
                    style sheet=vary dashing,
                    inside={label in legend={text=$qz/R^3$}},
                    outside={label in legend={text=$q/z^2$}},
                    data/format=function
                    ]
data [set=inside] {
  var x : interval [0.0:1.0];
  func y = \value x;
}
data [set=outside] {
  var x : interval [1.0:5.0];
  func y = 1.0/(\value x*\value x);
};
\end{tikzpicture}

\section{DIVERGENCE AND CURL OF ELECTROSTATIC FIELDS}
\subsection{Field Lines, Flux and Gauss's Law}

Equation \ref{eq:G01_electrostatic_field_continuous_vol} is the recipe for computing the electric \tit{field} of a charge distribution, then equation \ref{eq:G01_electrostatic_force} tells us what the \tit{force} on a charge $Q$ placed in this field will be. The integrals involved in computing $\vb{E}$ can be very difficult to calculate, even for relatively simple charge distributions. One then often exploits certain properties of the electric field by which the latter can be determined without an explicit calculation of the integral in \ref{eq:G01_electrostatic_field_continuous_vol}. 

Much can be understood about a vector field $\vb{E}$ by the values of its \tit{divergence} and \tit{curl}. There is indeed a general result of vector calculus -- Helmoltz's Theorem -- by which any vector field $\vb{A}$ that is sufficiently smooth and satisfying appropriate boundary conditions can be decomposed into the sum of a \tit{curl-free} ($\curl{\vb{A}}=\vb{0}$) field plus a \tit{divergence-free} ($\div{\vb{A}}=0$) field.     

$\vb{E}$ is a  \tit{curl-free} vector field and this is a consequence of 
\begin{itemize}
\item it being \tit{linearly} dependent on the source charges, so that a \tit{superposition principle} holds; 
\item it being directed \tit{radially} from any infinitesimal charge-carrying volume $d\tau$, so that integrating over a path that lies on a spherical surface centered on the source charge yields \tit{zero}; 
\item its \tit{magnitude} at any point $\vb{r}$ being a function of the distance of that point from any infinitesimal charge-carrying volume $d\tau$ located at $\vb{r}'$ ($\abs{\vb{E}(\vb{r})} = f(\abs{\vb{r} - \vb{r}'})$)\footnote{Note that the following demonstration does not require the field to decay by an inverse-square law as the electric field does.}. 
\end{itemize}

The reasoning to demonstrate that $\vb{E}$ is \tit{curl-free} goes as follows:
\begin{itemize}[-]
\item By Stokes theorem of vector calculus, given $\Gamma$ any closed curve and $\Sigma$ any closed surface bounded by  $\Gamma$, the \tit{line-integral} of $\vb{E}$ along $\Gamma$ (also called the \tit{circulation} of $\vb{E}$) equals the \tit{surface-integral} of $\curl{\vb{E}}$ (also called the \tit{flux of rotation}). 
\item Any closed path $\Gamma$ can be approximated to any precision degree by a curve made of an infinite sequence of steps where each step consists of an infinitesimal \tit{radial} displacement followed by a displacement along a spherical surface centered on the source charge $d\tau$ located at $\vb{r}'$. The latter type of displacement in each step \tbf{contributes nothing} to the line integral of $\vb{E}$, because the field is at right-angle with the path.
\item Given that only \tit{radial displacements} contribute to the line integral of $\vb{E}$ the latter must vanish on any closed path, since all outwardly directed displacements must be compensated by an equivalent amount of inwardly directed ones in order for the path to close.    
\item Due to the superposition principle, the vanishing of the \tit{circulation} thus holds for any field $\vb{E}$ and this in turn implies the vanishing of the \tit{flux of rotation}. The vanishing of the latter over the infinitely many surfaces bounded by an arbitrarily chosen curve $\Gamma$ clearly requires that $\curl{\vb{E}}=\vb{0}$.     
\end{itemize}
%\footnote{This is true also of fields generated by charges that are not static.}

Let now turn to the \tit{divergence} of the electric field. Firts of all we note that the \tit{flux} of $\vb{E}$ over a spherical surface $\Sigma$ enclosing a point charge $q$ located at the sphere's center is 
\begin{equation}
\int_{\Sigma} \vb{E} \cdot d\vb{a} = \int \int  \left( \frac{q \vu{r}}{r^2} \right) \cdot \left( r^2 \sin \theta  d\theta d\phi \vu{r} \right) \, = \, 4\pi q
\end{equation}

The above equation tells us that the flux equals $4\pi$ times the \tit{enclosed charge} regardless of the sphere's radius. This is due to the \tit{squared-inverse} dependency of the field strenght on distance, which causes the cancellation of $r^2$ terms appearing in the numerator and denumerator of the above integral. 

As we shall see, this result must actually hold for any closed surface of whatever shape. Therefore, 
\begin{equation}
\label{eq:Gauss_integral_form}
\int_{\Sigma} \vb{E} \cdot d\vb{a} = \, 4\pi Q_{enc}
\end{equation}

where $$Q_{enc} = \sum_{i=1}^n q_i$$ is the sum of all charges enclosed by the surface $\Sigma$. 

Equation \ref{eq:Gauss_integral_form} is a statement of the Gauss law in \tit{integral form}. An equivalent statement in \tit{differential} form can be obtained by applying the \tit{divergence theorem} by which 

$$ \int_{\Sigma} \vb{E} \cdot d\vb{a} =  \int_{V} \left( \div{\vb{E}} \right) d\tau$$

As we know from \ref{eq:Gauss_integral_form} the surface integral on the left of the above equation is equal to $4\pi$ times the enclosed charge $Q_{enc}$ and the latter can be expressed as the volume integral of the charge density $\rho$
 
$$ \int_{\Sigma} \vb{E} \cdot d\vb{a} =  4\pi \int_{V} \rho d\tau$$

Therefore, the volume integrals at the right hand side of these two expressions for the \tit{flux} of $\vb{E}$ must be equal, which implies the equality of the expressions being integrated, namely

\begin{equation}
\label{eq:Gauss_differential_form}
\div{\vb{E}} = 4 \pi \rho
\end{equation}

\subsection{The Divergence of $\vb{E}$}
The divergence of $\vb{E}$ can be calculated directly from equation \ref{eq:G01_electrostatic_field_continuous_vol}. The integral in that equation must be extended to a volume enclosing all charges or equivalently to the whole space, as we do here:

\begin{equation}
\vb{E}(\vb{r}) = \int_{all space} \rho(\vb{r}') \frac{\vu{s}}{s^2} \,d\tau' = \int \rho(\vb{r}') \frac{\vb{r} - \vb{r}'}{\abs{\vb{r} - \vb{r}'}^3} \,d\tau'
\end{equation}

We observe that the $\vb{r}$ dependency is contained in $\vb{s} = \vb{r} - \vb{r}'$, therefore 

$$\div{\vb{E}} = \int \div{ \left( \frac{\vb{r} - \vb{r}'}{\abs{\vb{r} - \vb{r}'}^3} \right)}\, \rho(\vb{r}') \,d\tau'$$   

In section \ref{subsec:divergence_of_r_over_r2} we calculated the divergence of $\vu{r}/r^2$, whence

$$\div{\left({\frac{\vu{r}}{r^2}}\right)} = 4\pi \delta^3(\vb{r}) \:\: \longrightarrow \:\: \div{ \left( \frac{\vb{r} - \vb{r}'}{\abs{\vb{r} - \vb{r}'}^3} \right)} = 4\pi \delta^3(\vb{r} - \vb{r}')$$

therefore 

\begin{equation}
\label{eq:Gauss_differential_form}
\div{\vb{E}(\vb{r})} = \int 4\pi \delta^3(\vb{r} - \vb{r}') \, \rho(\vb{r}') \,d\tau' = 4 \pi \rho(\vb{r})   
\end{equation}


The Gauss law in \tit{integral form} (equation \ref{eq:Gauss_integral_form}) is recovered from \ref{eq:Gauss_differential_form} by applying the divergence theorem:

\begin{equation*}
\int_{\Sigma} \vb{E} \cdot d\vb{a} = \int_V  \div{\vb{E}} \, d\tau = 4 \pi \int_V \rho d\tau = 4\pi Q_{enc} 
\end{equation*}

\subsection{Applications of Gauss's Law}
When the direction of the electric field is dictated a-priori by \tit{symmetry considerations} the Gauss law in integral form can be the easiest wy to compute electric fields. 

\subsubsection*{Example 3}
Find the field outside a uniformly charged solid sphere of radius $R$ and total charge $q$.  

\subsubsection*{Solution}

For any spherical surface with the same center and radius $r > R$, Gauss law states that

\begin{equation*}
\int_{\Sigma} \vb{E} \cdot d\vb{a} = 4\pi Q_{enc} 
\end{equation*}

By symmetry, the field can only be \tit{radial} therefore $\vb{E}(\vb{r}) = \abs{\vb{E}} \vu{r}$, where $\abs{\vb{E}}$ is \tit{constant} over the spherical surface.The area element in the surface integral is also directed radially ($d\vb{a} = da  \vu{r} $) so that 

$$\vb{E} \cdot d\vb{a} = \abs{\vb{E}} da $$

and

\begin{align*}
\int_{\Sigma} \vb{E} \cdot d\vb{a} &=  \abs{\vb{E}}\, \int _{\Sigma} da \\
4\pi q &=  \abs{\vb{E}} \, \int \int r^2 \sin \theta d\theta d\phi = \abs{\vb{E}} 4\pi r^2
\end{align*}

From the last equality it follows that $\abs{\vb{E}} = q/r^2$ and, finally 

$$\vb{E} =  \frac{q \vu{r}}{r^2} $$

the same field produced by a point charge $q$ concentrated at the center. 

Symmetry is crucial to the application of Gauss's law. The three kinds of symmetry where the law works are 

\subsubsection*{Example 4}
A long cylinder carries a charge density that is proportional to the distance from the axis: $\rho = ks$, for some constant $k$. Find the electric field inside the cylinder.

\subsubsection*{Solution}
The Gaussian surface of choice in this case is a cylindel of length $l$ and radius $s$.

$$\int_{\Sigma} \vb{E} \cdot d\vb{a} = 4\pi Q_{enc} $$

where

\begin{equation*}
 Q_{enc} = \int_V \rho d\tau = \int (k s') (ds'\,d\phi \, dx = 2\pi k l \int_0^s {s'}^2 ds' = \frac{2}{3}\pi k l s^3 
\end{equation*}

By Gauss's law $4\pi Q_{enc}$ is the \tit{flux} of $\vb{E}$ which in this case is $\abs{\vb{E}}$ times the area of the Gaussian cylinder, namely

$$4 \pi Q_{enc} =  \frac{8}{3}\pi k l s^3 = \abs{\vb{E}} 2 \pi s l$$

whereby 

$$\vb{E} = \abs{\vb{E}} \vu{s} = \frac{4}{3}k s^2 \vu{s}$$


\subsubsection*{Example 5}
An infinite plane carries a uniform surface charge $\sigma$. Find its electric field.

\subsubsection*{Solution}
The Gaussian surface of choice is a \quotes{pillbox} extending equal distances above and below the plane, with the two faces parallel to the plane having area $A$. The electric field $\vb{E}$ must be normal to the plane and constant in magnitude on both the upper and lower sides. 

$$\int_{\Sigma} \vb{E} \cdot d\vb{a} = 2 \abs{\vb{E}} A $$
$$4 \pi Q_{enc} =  \frac{8}{3}\pi k l s^3 = \abs{\vb{E}} 2 \pi s l$$

\begin{align*}
\int_{\Sigma} \vb{E} \cdot d\vb{a} &=  2 \abs{\vb{E}} A \\
4\pi q &=  4 \pi \sigma A \\
4 \pi \sigma A = 2 \abs{\vb{E}} A
\end{align*}

whereby 

$$\vb{E} = \abs{\vb{E}} \vu{n} = 2 \pi \sigma \vu{n}$$