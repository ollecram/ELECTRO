\chapter{Griffiths -- Electrostatics}
\label{ch:Griffiths_01} 

\section{THE ELECTRIC FIELD}
\subsection{Coulomb's Law}
The \tit{electric force} exerted by a \tit{source charge} $q$ located \tit{at rest} in $\vb{r}'$ on a \tit{test charge} $Q$ located at $\vb{r}$  is (\tbf{Coulomb's law})
\begin{equation}
\label{eq:G01_coulomb}
\vb{F} = \frac{qQ}{s^2} \vu{s}
\end{equation}

where 

\begin{equation}
\label{eq:G01_separation}
\vb{s} = \vb{r} - \vb{r}'
\end{equation}

is the \tbi{separation vector} between the the two charges, directed from the source point to the test point. 

Based on the \tbi{superposition principle}, the effect of multiple source charges (at rest) $q_1, q_2, \ldots, q_n$ on the test charge $Q$ is simply the sum over all sources of the force generated by each one of them

\begin{equation}
\label{eq:G01_superposition}
\vb{F} =  \vb{F}_1 + \vb{F}_2 + \cdots + \vb{F}_n = Q \left( \frac{q_1}{s_1^2} \vu{s}_1 + \frac{q_2}{s_2^2} \vu{s}_2 + \cdots + \frac{q_n}{s_n^2} \vu{s}_n \right)
\end{equation}

\subsection{The Electric Field}
The \tbi{electrostatic field} $\vb{E}(\vb{r})$ is, by definition, the force exerted on a \tbi{unit test charge} located at the point $\vb{r}$ by all other charges (assumed at rest). With $\vb{s} = \vb{r} - \vb{r}'$, we therefore write namely

\begin{equation}
\label{eq:G01_electrostatic_field}
\vb{E}(\vb{r}) = \frac{\vb{F}}{Q} =  \sum_{i=1}^n \frac{q_i}{s_i^2} \vu{s}_i
\end{equation}

In terms of $\vb{E}(\vb{r})$ the \tbf{Coulomb's law} \ref{eq:G01_superposition} can be rewritten as 

\begin{equation}
\label{eq:G01_electrostatic_force}
\vb{F}(\vb{r}) = Q \vb{E}(\vb{r})
\end{equation}

\subsubsection*{Problem 1}
\begin{enumerate}[a)]
\item Twelve equal charges, $q$ are situated at the corners of a regular 12-sided polygon (one on each numeral of a clock face). What is the net force on a test charge $Q$ at the center?
\item Suppose \tit{one} of the 12 $q$'s is removed (the one at \quotes{6 o'clock}). What is the force on $Q$? Explain your reasoning carefully.
\item Now $13$ equal charges, $q$ are placed at the corners of a regular 13-sided polygon. What is the net force on a test charge $Q$ at the center?
\item If one of the 13 $q$'s is removed what is the force on $Q$? Explain your reasoning carefully.
\end{enumerate}

\subsubsection*{Solution}
a) The net force is $\vb{0}$ because the field generated by each charge is \tit{balanced} by the equal and opposite field generated by the charge that sits in front of it across the center.\\
b) This is the force exerted by the \tit{single unbalanced charge}, the one at \quotes{12 o'clock}: 
$$ \vb{F} = - \frac{qQ}{R^2}\, \vu{k}$$ 
c) The net force is still $\vb{0}$ because there is no direction the electric field $\vb{E}$ can take without breaking the symmetry of the arrangement. This means that the field generated by each charge is balanced, at the center, by the sum of the fields generated by the remaining charges.\\
d) Because of the previous point, the net force at the center is the \tit{opposite} of the force that would be added if the removed charge would be put back in. Assuming \quotes{6 o'clock} to be the place of the latter, the force has the same value computed for the case b). 

\subsubsection*{Example 1}
Find the electric field a distance $z$ above the midpoint between two equal charges ($q$) a distance $d$ apart. 

\subsubsection*{Solution}
Let choose a Cartesian reference frame with \tbi{origin} $O$ in the midpoint between the two charges, the $\vu{i}$ axis parallel to the line along which they are located, and the $\vu{k}$ axis orthogonal to that line. In this frame the position vectors of the two charges are $\vb{r}_1 = -d/2\vu{i}$ and $\vb{r}_2 = +d/2\vu{i}$, while the position at which the electric field must be computed is $\vb{r} =  z \vu{k}$. 

The separation vectors of the sources from the test point $\vb{r}$ are 
\begin{align*}
\vb{s}_1 &= \vb{r} - \vb{r}_1 = z \vu{k} + \frac{d}{2} \vu{i} \\
\vb{s}_2 &= \vb{r} - \vb{r}_2 = z \vu{k} - \frac{d}{2} \vu{i} 
\end{align*}

The next step is to take the sum \ref{eq:G01_electrostatic_field} and we observe that 

\begin{align*}
& s_1^2 = z^2 +(d/2)^2 = s_2^2 \\
& \vb{s}_1 + \vb{s}_2 = 2z \vu{k} = \sqrt{z^2 +(d/2)^2}  (\vu{s}_1 + \vu{s}_2)  \\
& \vu{s}_1 + \vu{s}_2 = \frac{\vb{s}_1 + \vb{s}_2}{\sqrt{z^2 +(d/2)^2}} = \frac{2z \vu{k}}{\sqrt{z^2 +(d/2)^2}} = 2 \cos \theta \vu{k}
\end{align*}

where $\theta$ is the angle between a separation vector with the $\vu{k}$ axis. Finally

\begin{equation*}
\vb{E}(\vb{r}) = \sum_{i=1}^2 \frac{q_i}{s_i^2} \vu{s}_i = \frac{q \,(\vu{s}_1 + \vu{s}_2)}{z^2 +(d/2)^2}  = \frac{2q \cos \theta \, \vu{k}}{z^2 + (d/2)^2} = \frac{2qz \, \vu{k}}{  [z^2 +(d/2)^2]^{3/2}  }
\end{equation*}

When $z \gg d$ any dependency on $d$ can be eliminated from the denominator giving, as expected 

$$\vb{E}(z) \approx \lim_{d \rightarrow 0} \vb{E}(\vb{r}) = \frac{2q \, \vu{k}}{z^2}$$ 

namely the same field as of a single particle with twice as much charge. 


\subsubsection*{Problem 2}
Find the electric field (magnitude and direction) a distance $z$ above the midpoint between equal and opposite charges ($\pm q$) a distance $d$ apart (same as Example 1, except that the charge at $x = +d/2$ is $-q$).

\subsubsection*{Solution}

The separation vectors of the sources from the test point $\vb{r}$ are 
\begin{align*}
\vb{s}_1 &= \vb{r} - \vb{r}_1 = z \vu{k} + \frac{d}{2} \vu{i} \\
\vb{s}_2 &= \vb{r} - \vb{r}_2 = z \vu{k} - \frac{d}{2} \vu{i} 
\end{align*}

The next step is to take the sum \ref{eq:G01_electrostatic_field} and we observe that 

\begin{align*}
& s_1^2 = z^2 +(d/2)^2 = s_2^2 \\
& q\,(\vb{s}_1 - \vb{s}_2) = q d\, \vu{i} = q\, \sqrt{z^2 +(d/2)^2}  (\vu{s}_1 - \vu{s}_2)  \\
& q\,(\vu{s}_1 - \vu{s}_2) =\frac{ q \, (\vb{s}_1 - \vb{s}_2)}{\sqrt{z^2 +(d/2)^2}} = \frac{qd\, \vu{i}}{\sqrt{z^2 +(d/2)^2}} = 2q \sin \theta \, \vu{i}
\end{align*}

where $\theta$ is the angle between a separation vector with the $\vu{k}$ axis. Finally

\begin{equation*}
\vb{E}(\vb{r}) = \sum_{i=1}^2 \frac{q_i}{s_i^2} \vu{s}_i = \frac{q \,(\vu{s}_1 - \vu{s}_2)}{z^2 +(d/2)^2}  = \frac{2q \sin \theta \, \vu{i}}{z^2 + (d/2)^2} = \frac{2qd \, \vu{i}}{  [z^2 +(d/2)^2]^{3/2}  }
\end{equation*}

\subsection{Continuous Charge Distributions}

Our definition of the electric field (Eq. \ref{eq:G01_electrostatic_field}) assumes that the source of the field is a collection of discrete point charges $q_i$. If instead the charge is distributed continuosly over some region, the sum becomes an integral:

\begin{equation}
\label{eq:G01_electrostatic_field_continuous}
\vb{E}(\vb{r}) = \int \frac{\vu{s}}{s^2} \,dq
\end{equation}

If the charge is spread along a \tit{line} then $dq = \lambda dl'$, where $\lambda$ is the charge per unit length and $dl'$ is the element of length along the line; if it is smeared out over a \tit{surface} then $dq = \sigma da'$, where $\sigma$ is the charge per unit surface and $da'$ is an element of area over the surface; and if the charge fills a \tit{volume} with charge per unit volume $\rho$, then $dq = \rho d\tau'$, where $d\tau'$ is an element of volume:

\begin{equation}
\label{eq:G01_electrostatic_field_continuous_line}
\vb{E}(\vb{r}) = \int \lambda(\vb{r}') \frac{\vu{s}}{s^2} \,dl' =  \int \lambda(\vb{r}') \frac{\vb{r} - \vb{r}'}{\abs{\vb{r} - \vb{r}'}^3} \,dl'
\end{equation}

\begin{equation}
\label{eq:G01_electrostatic_field_continuous_surf}
\vb{E}(\vb{r}) = \int \sigma(\vb{r}') \frac{\vu{s}}{s^2} \,da'  =   \int \sigma(\vb{r}') \frac{\vb{r} - \vb{r}'}{\abs{\vb{r} - \vb{r}'}^3} \,da'
\end{equation}

\begin{equation}
\label{eq:G01_electrostatic_field_continuous_vol}
\vb{E}(\vb{r}) = \int \rho(\vb{r}') \frac{\vu{s}}{s^2} \,d\tau' = \int \rho(\vb{r}') \frac{\vb{r} - \vb{r}'}{\abs{\vb{r} - \vb{r}'}^3} \,d\tau'
\end{equation}
 
\subsubsection*{Example 2}
Find the electric field a distance $z$ above the midpoint of a straight line segment of length $2L$ that carries a uniform line charge \tit{density} $\lambda$. 

\subsubsection*{Solution}
Adopting a reference frame whose origin is the midpoint of the line distribution, we exploit the result of Example 1
whereby the contribution of two line elements $dl_1$ and $dl_2$ located symmetrically with respect to the origin is 
$$d \vb{E}(z) = \vu{k} \frac{2\,q\,z}{[x^2 + z^2]^{3/2}} = \vu{k} \frac{\lambda (dx_1 + dx_2) \,z}{[x^2 + z^2]^{3/2}}$$
\begin{align*}
\vb{E}(z) &= \int_{-L}^{+L} \lambda(\vb{r}') \frac{\vb{r} - \vb{r}'}{\abs{\vb{r} - \vb{r}'}^3} \,dl'\\
          &= \vu{k}  \int_{0}^{L} \frac{ 2 \, \lambda \,z \, dx}{[x^2 + z^2]^{3/2}} \\
          &=  2 \lambda z\, \vu{k} \int_{0}^{L} \frac{dx}{[x^2 + z^2]^{3/2}}  
\end{align*}

The last integral has the form 
$$\int \frac{dx}{r^3}$$
with $r = [x^2 + z^2]^{1/2}$. The \tit{primitive} of this integral (Dwight 200.3) is $$\frac{1}{z^2}\,\frac{x}{r}$$ Therefore
\begin{align*}
\vb{E}(z) &=  \frac{2 \lambda}{z} \, \left[ \frac{x}{\sqrt{x^2 + z^2}} \right]_0^L  \, \vu{k} \\
          &=  \frac{2 \lambda L }{z\sqrt{L^2 + z^2}} \, \vu{k} 
\end{align*}

When $z \gg L$ any dependency on $L$ can be eliminated from the denominator giving, as expected 

$$\vb{E}(z) \approx \lim_{L \rightarrow 0} \vb{E}(z) = \frac{2 \lambda L }{z^2} \, \vu{k} $$ 

namely the field of a point particle carrying the total charge $2 \lambda L$. 

\subsubsection*{Problem 3}
Find the electric field a distance $z$ above one end of a straight line segment of length $L$ that carries a uniform line charge $\lambda$. Check that your formula is consistent with what you would expect for the case $z \gg L$.

\subsubsection*{Solution}

\begin{align*}
\vb{r} &= z \vu{k}, \:\: \vb{r}' = x \vu{i} \\
\vb{s} &= \vb{r} - \vb{r}' = z \vu{k} - x \vu{i}, \:\: s = \sqrt{x^2 + z^2}, \:\: \vu{s} = \frac{z \vu{k} - x \vu{i}}{\sqrt{x^2 + z^2}} \\
\vb{E}(z) &=  \int_0^L \frac{\lambda \, \vu{s} \, dx}{s^2} \\
          &=  \lambda \, z \, \vu{k} \int_0^L \frac{dx}{[x^2 + z^2]^{3/2}} \, - \lambda \, \vu{i} \int_0^L \frac{x \, dx}{[x^2 + z^2]^{3/2}}
\end{align*}

Primitives of the last two integrals are (Dwight 200.03 and 201.03)
$$ \frac{1}{z^2} \frac{x}{\sqrt{x^2 + z^2}}, \:\:\: - \frac{1}{\sqrt{x^2 + z^2}}$$

Therefore

\begin{align*}
\vb{E}(z) &=  \frac{\lambda}{z} \, \left[ \frac{x}{\sqrt{x^2 + z^2}} \right]_0^L  \, \vu{k} +   \lambda \, \left[\frac{1}{\sqrt{x^2 + z^2}}\right]_0^L \, \vu{i} \\
          &= \frac{\lambda\,L\,z}{\sqrt{L^2 + z^2}}\, \vu{k} + \lambda \left(\frac{1}{\sqrt{L^2 + z^2}} - \frac{1}{z}        \right) \, \vu{i} \\
          &= \frac{\lambda\,L}{\sqrt{1 + (L/z)^2}}\, \vu{k} + \lambda \left(\frac{1}{\sqrt{L^2 + z^2}} - \frac{1}{z}        \right) \, \vu{i}
\end{align*}

When $z \gg L$ 

$$\vb{E}(z) \approx \lim_{L \rightarrow 0} \vb{E}(z) = \lambda L \, \vu{k} $$ 

\subsubsection*{Problem 4}
Find the electric field a distance $z$ above the center of a square loop (side $a$) carrying uniform line charge $\lambda$. [\tit{Hint}: Use the result of Ex. 2]

\subsubsection*{Solution}
Let $d = \sqrt{a^2/4 + z^2}$ be the distance between the point $P$, located a distance $z$ above the center of the square, and the \tit{midpoint} of each side. The magnitude of the field $\vb{E}_i^P$ generated at $P$ by the $i$-th side of the loop can be obtained from the solution of Ex. 2 with the following correspondence between variables:

$$ a \leftrightarrow 2L, \:\: d = \sqrt{a^2/4 + z^2} \leftrightarrow z$$  

We also observe that we must refer the solution of Ex. 2 (separately for each side) to a frame that contains both the line element and the point $P$. The angle $\phi$ formed by the $\vu{k}$ axis of such frame and the vertical $\vu{z}$ through the center of the loop is determined by  

$$\cos(\phi) = \vu{k} \cdot \vu{z} = \frac{z}{d} = \frac{z}{\sqrt{a^2/4 + z^2}}$$

Clearly, the component of $E$ orthogonal to $\vu{z}$ takes opposite values on opposite sides of the loop, so it can be ignored altogether: we must only ensure that a factor $\cos(\phi)$ be applied to the field magnitude $\abs{\vb{E}}$ at $P$ to obtain the projection of $\vb{E}$  along $\vu{z}$. 

After this preamble, the magnitude of the field at $P$ generated by a \tit{single side} is 
\begin{align*}
\abs{\vb{E}} &= \frac{2 \lambda L}{z\, \sqrt{L^2 + z^2}} \rightarrow \frac{a \lambda}{d\, \sqrt{a^2/4 + d^2}}\\
             &= \frac{a\,\lambda}  {d\, \sqrt{a^2/4 + d^2}} = \frac{a\,\lambda}    {\sqrt{(a^2/4 + z^2)(a^2/4 + a^2/4 + z^2)}} \\
             &= \frac{a\,\lambda} {\sqrt{z^4 + (3/4) a^2 z^2 + a^4/8}}
\end{align*}

The component of $\vb{E}$ at the point $P$ is obtained by multiplying the above result by $4 \cos(\phi) = 4z/d$:

\begin{align*}
\vb{E}^P(z) &= \frac{4\, a\, \lambda\, z\, \vu{z}} {d\, \sqrt{z^4 + (3/4) a^2 z^2 + a^4/8}} \\
            &= \frac{4\, a\, \lambda\, z\, \vu{z}} {\sqrt{(a^2/4 + z^2)[z^4 + (3/4) a^2 z^2 + a^4/8]}} \\
            &= \frac{4\, q\, z\, \vu{z}} {\sqrt{z^6 + a^2 z^4 + (5/16) a^4 z^2 + a^6/16}}      
\end{align*}
   
When $z \gg a$ 

$$\vb{E}^P(z) \approx \lim_{a \rightarrow 0} \vb{E}^P(z) =  \frac{4\,q\,\vu{z}}{z^2}$$    


\subsubsection*{Problem 5}
Find the electric field a distance $z$ above the center of a circular loop of radius $r$ carrying a uniform line charge $\lambda$.

\subsubsection*{Solution}
Let $d = \sqrt{r^2 + z^2}$ be the distance between the point $P$, located a distance $z$ above the center of the circular loop and the charge associated to an infinitesimal line element $dl'$ along the loop. Introducing polar coordinates $r, \theta$ in the plane of the loop (with the origin $O$ in the center of the latter) we observe that the vector $\vb{E}(\theta)$ generated by a line element $dl'$ located at an angle $\theta$ along the loop makes an angle $\phi$ with the axis $\vu{k}$. This angle is determined by 
$$\cos(\phi) = \vu{k} \cdot \frac{\vb{E}(\theta)}{\abs{\vb{E}(\theta)}} = \frac{z}{d} = \frac{z}{\sqrt{r^2 + z^2}}$$
  
The \tit{sum of the electric fields} at $P$ generated by two line elements separated by an angle $\pi$ yield a vector that is parallel to $\vu{k}$ (that is, orthogonal to the loop) because the projections onto the plane of the loop of the electric fields generated by the two line elements are \tit{equal in magnitude and opposite in direction}. 

In conclusion, we can obtain the total field $\vb{E}$ at the point $P$ by integrating the magnitude of the field generated by line elements multiplied by the factor $\cos(\phi) = z/d$, therefore

\begin{align*}
\vb{E}^P(z) &= \int_0^{2\pi}  \frac{\cos(\phi) \lambda  r}{d^2} d\theta  \, \vu{k} = \frac{\cos(\phi) \lambda  r}{d^2} \int_0^{2\pi} d\theta \, \vu{k}\\
            &= \frac{(2 \pi r \lambda) \cos(\phi)}{d^2} \, \vu{k} \\
            &= \frac{q\,z}{d^{3/2}}  \, \vu{k} \\
            &=  \frac{q \, z}{(\sqrt{r^2 + z^2})^{3/2}} \, \vu{k}
\end{align*}

where $q = 2 \pi r \lambda$ is the total charge on the loop. When $z \gg r$ 

$$\vb{E}^P(z) \approx \lim_{r \rightarrow 0} \vb{E}^P(z) =  \frac{q\,\vu{k}}{z^2}$$      