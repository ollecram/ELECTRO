\chapter{Griffiths -- Electrostatics}
\label{ch:Griffiths_01} 

The \tit{electric force} exerted by a \tit{source charge} $q$ located \tit{at rest} in $\vb{r}'$ on a \tit{test charge} $Q$ located at $\vb{r}$  is (\tbf{Coulomb's law})
\begin{equation}
\label{eq:G01_coulomb}
\vb{F} = \frac{qQ}{s^2} \vu{s}
\end{equation}

where 

\begin{equation}
\label{eq:G01_separation}
\vb{s} = \vb{r} - \vb{r}'
\end{equation}

is the \tbi{separation vector} between the the two charges, directed from the source point to the test point. 

Based on the \tbi{superposition principle}, the effect of multiple source charges (at rest) $q_1, q_2, \ldots, q_n$ on the test charge $Q$ is simply the sum over all sources of the force generated by each one of them

\begin{equation}
\label{eq:G01_superposition}
\vb{F} =  \vb{F}_1 + \vb{F}_2 + \cdots + \vb{F}_n = Q \left( \frac{q_1}{s_1^2} \vu{s}_1 + \frac{q_2}{s_2^2} \vu{s}_2 + \cdots + \frac{q_n}{s_n^2} \vu{s}_n \right)
\end{equation}

The \tbi{electrostatic field} $\vb{E}(\vb{r})$ is, by definition, the force exerted on a \tbi{unit test charge} located at the point $\vb{r}$ by all other charges (assumed at rest), namely

\begin{equation}
\label{eq:G01_electrostatic_field}
\vb{E}(\vb{r}) = \frac{\vb{F}}{Q} =  \sum_{i=1}^n \frac{q_i}{s_i^2} \vu{s}_i
\end{equation}

In terms of $\vb{E}(\vb{r})$ the \tbf{Coulomb's law} \ref{eq:G01_superposition} can be rewritten as 

\begin{equation}
\label{eq:G01_electrostatic_force}
\vb{F}(\vb{r}) = Q \vb{E}(\vb{r})
\end{equation}

\subsubsection*{Example 1}
Find the electric field a distance $z$ above the midpoint between two equal charges ($q$) a distance $d$ apart. 

\subsubsection*{Solution}
Let choose a Cartesian reference frame with \tbi{origin} $O$ in the midpoint between the two charges, the $\vu{i}$ axis parallel to the line along which they are located, and the $\vu{k}$ axis orthogonal to that line. In this frame the position vectors of the two charges are $\vb{r}_1 = -d/2\vu{i}$ and $\vb{r}_2 = +d/2\vu{i}$, while the position at which the electric field must be computed is $\vb{r} =  z \vu{k}$. 

The separation vectors of the sources from the test point $\vb{r}$ are 
\begin{align*}
\vb{s}_1 &= \vb{r} - \vb{r}_1 = z \vu{k} + \frac{d}{2} \vu{i} \\
\vb{s}_2 &= \vb{r} - \vb{r}_2 = z \vu{k} - \frac{d}{2} \vu{i} 
\end{align*}

The next step is to take the sum \ref{eq:G01_electrostatic_field} and we observe that 

\begin{align*}
& s_1^2 = z^2 +(d/2)^2 = s_2^2 \\
& \vb{s}_1 + \vb{s}_2 = 2z \vu{k} = \sqrt{z^2 +(d/2)^2}  (\vu{s}_1 + \vu{s}_2)  \\
& \vu{s}_1 + \vu{s}_2 = \frac{\vb{s}_1 + \vb{s}_2}{\sqrt{z^2 +(d/2)^2}} = \frac{2z \vu{k}}{\sqrt{z^2 +(d/2)^2}} = 2 \cos \theta \vu{k}
\end{align*}

where $\theta$ is the angle between a separation vector with the $\vu{k}$ axis. Finally

\begin{equation*}
\vb{E}(\vb{r}) = \sum_{i=1}^2 \frac{q_i}{s_i^2} \vu{s}_i = \frac{q \,(\vu{s}_1 + \vu{s}_2)}{z^2 +(d/2)^2}  = \frac{2q \cos \theta \, \vu{k}}{z^2 + (d/2)^2} = \frac{2qz \, \vu{k}}{  [z^2 +(d/2)^2]^{3/2}  }
\end{equation*}

When $z \gg d$ any dependency on $d$ can be eliminated from the denominator giving 

$$\lim_{d \rightarrow 0} \vb{E}(\vb{r}) = \frac{2q \, \vu{k}}{z^2}$$ 

the same field as of a single particle with twice as much charge. 