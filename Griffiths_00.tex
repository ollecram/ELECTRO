\chapter{Griffiths -- Vector Analysis}
\label{Griffiths_00} 

\section{Vector Algebra}
For the sake of fixing notation, let $\{\vu{i}, \vu{j}, \vu{k}\}$ be an \textit{orthonormal} basis of unit vectors in three-dimensional space, and the following a list of all possible scalar products of basis vectors. 
  
\begin{equation}
\begin{aligned} 
\vu{i}\vdot \vu{i} &= \vu{j} \vdot \vu{j} = \vu{k} \vdot \vu{k} = 1 \\ 
\vu{i} \vdot \vu{j} &= \vu{j} \vdot \vu{k} = \vu{k} \vdot \vu{i} = 0 
\label{eq:basis_dot_products}
\end{aligned}
\end{equation}

For a \textit{right-handed} basis, the following is a list of all possible \textit{vector} products of basis vectors.
  
\begin{equation}
\begin{aligned} 
\vu{i} \cross \vu{i} &= \vu{j} \cross \vu{j} = \vu{k} \cross \vu{k} = 0 \\ 
\vu{i} \cross \vu{j} &= \vu{k} = - \vu{j} \cross \vu{i}\\
\vu{j} \cross \vu{k} &= \vu{i} = - \vu{k} \cross \vu{j}\\
\vu{k} \cross \vu{i} &= \vu{j} = - \vu{i} \cross \vu{k}
\label{eq:basis_vector_products}
\end{aligned}
\end{equation}

Any vector $\vb{A}$ can be resolved into a linear combination of the basis vectors
\begin{equation*}
\vb{A} = A_x \, \vu{i} + A_y \, \vu{j} + A_z \, \vu{k} 
\end{equation*}

where $A_x, A_y, A_z$ are the vector \textit{components} in that basis. 

When the basis is assumed, a list of comma separated numbers within parentheses indicate the components of a vector in that basis: 

\begin{equation*}
(A_x, A_y, A_z) \equiv \vb{A} = A_x  \vu{i} + A_y  \vu{j} + A_z  \vu{k} 
\end{equation*}

The \textit{dot} product of two vectors is the \textit{scalar}  
\begin{equation}
\vb{A} \vdot \vb{B} = A_x B_x + A_y B_y + A_z B_z 
\label{eq:dot_product}
\end{equation}

The \textit{cross} product of two vectors is the \textit{vector}  
\begin{equation}
\begin{aligned}
\vb{A} \cross \vb{B} &= (A_y B_z - A_z B_y)\, \vu{i} \\
                     &+ (A_z B_x - A_x B_z)\, \vu{j} \\
                     &+ (A_x B_y - A_y B_x)\, \vu{k}
\end{aligned}
\label{eq:cross_product}
\end{equation}

A useful mnemonic expression equivalent to \ref{eq:cross_product} is the \textit{determinant} 
\begin{equation*}
\mqty| \vu{i} &  \vu{j} &  \vu{k} \\ 
A_x & A_y & A_z \\
B_x & B_y & B_z|
\end{equation*}


It follows from \ref{eq:basis_vector_products} that for any two vectors $\vb{A}$ and $\vb{B}$ 
\begin{equation}
\vb{A} \cross \vb{B} = - \vb{B} \cross \vb{A}
\label{eq:cross_product_reflection}
\end{equation}

\subsection{Triple products}
Since the cross product of two vector is itself a vector, it can be dotted or crossed with a third vector to form a \textit{triple} product. 

\subsubsection{Scalar triple product}
The norm of $\vb{A} \vdot (\vb{B} \cross \vb{C})$ is the volume of a parallelepiped generated by $\vb{A}$, $\vb{B}$ and $\vb{C}$, since $\abs{ \vb{B} \cross \vb{C} }$ is the area of the base, while the dot product multiplies that number by the length of the projection of $\vb{A}$ orthogonal to the area. The choice of which vectors make the base of the parallelepiped is arbitrary, while the volume is independent of that choice, therefore: 

\begin{equation}
\vb{A} \vdot (\vb{B} \cross \vb{C}) = \vb{B} \vdot (\vb{C} \cross \vb{A}) = \vb{C} \vdot (\vb{A} \cross \vb{B})
\label{eq:scalar_triple_product}
\end{equation}

When $\vb{A}$, $\vb{B}$ and $\vb{C}$ form a \textit{right-handed} [\textit{left-handed}] basis --not necessarily orthonormal-- the above forms evaluate to the same \textit{positive} [\textit{negative}] value.  

Because of \ref{eq:cross_product_reflection} the following three extra forms are also equivalent among each other but they evaluate to the \textit{opposite} value, thus yielding a \textit{negative} [\textit{positive}] value when $\vb{A}$, $\vb{B}$ and $\vb{C}$ form a \textit{right-handed} [\textit{left-handed}] basis.

\begin{equation}
\vb{A} \vdot (\vb{C} \cross \vb{B}) = \vb{B} \vdot (\vb{A} \cross \vb{C}) = \vb{C} \vdot (\vb{B} \cross \vb{A})
\end{equation}

\subsubsection{Vector triple product}

The vector triple product can be simplified by the $\vb{BAC}\mathbf{-}\vb{CAB}$ rule:
\begin{equation}
\vb{A} \cross (\vb{B} \cross \vb{C}) = \vb{B} (\vb{A} \vdot \vb{C}) - \vb{C} (\vb{A} \vdot \vb{B})
\label{eq:vector_triple_product}
\end{equation}

Note that 

\begin{equation*}
(\vb{A} \cross \vb{B}) \cross \vb{C} = - \vb{C} \cross (\vb{A} \cross \vb{B}) = \vb{B} (\vb{A} \vdot \vb{C}) - \vb{A} (\vb{B} \vdot \vb{C})
\end{equation*}

so changing the grouping yields a different vector, namely

\begin{equation*}
\vb{A} \cross (\vb{B} \cross \vb{C}) - (\vb{A} \cross \vb{B}) \cross \vb{C} = \vb{A} (\vb{B} \vdot \vb{C})  - \vb{C} (\vb{A} \vdot \vb{B})
\end{equation*}

All \textit{higher} products can be similarly reduced, so that it is never necessary for an expression to contain more than one cross product in any term. For instance
\begin{equation}
\begin{aligned} 
(\vb{A} \cross \vb{B}) \vdot (\vb{C} \cross \vb{D}) &= (\vb{A} \vdot \vb{C}) (\vb{B} \vdot \vb{D}) - (\vb{A} \vdot \vb{D}) (\vb{B} \vdot \vb{C})\, ;  \\
\vb{A} \cross [\vb{B} \cross (\vb{C} \cross \vb{D})] &= \vb{B} [\vb{A} \vdot (\vb{C} \cross \vb{D})] -  (\vb{A} \vdot \vb{B})(\vb{C} \cross \vb{D})
\end{aligned}
\end{equation}

\subsubsection{Problem 1}
Prove that 
\begin{equation*}
[\vb{A} \cross (\vb{B} \cross \vb{C})] + [\vb{B} \cross (\vb{C} \cross \vb{A})] +  [\vb{C} \cross (\vb{A} \cross \vb{B})] = 0  
\end{equation*}

By applying the BAC-CAB rule to each one of the above three terms one obtains the following six   terms\footnote{Each line is obtained by the previous one by putting in each place the vector which follows -- in the $ABC$ cycle-- the one in the same place in the line above.}
\begin{equation*}
\begin{aligned} 
\vb{A} (\vb{B} \vdot \vb{C}) &- \vb{C} (\vb{A} \vdot \vb{B})\, +\\
\vb{B} (\vb{C} \vdot \vb{A}) &- \vb{A} (\vb{B} \vdot \vb{C})\, +\\
\vb{C} (\vb{A} \vdot \vb{B}) &- \vb{B} (\vb{C} \vdot \vb{A})
\end{aligned}
\end{equation*}
Each scalar product appears twice, each time it is multiplied by the remaining vector but with opposite signs.\qed. 


\section{Position, Displacement, Separation Vectors}

The vector connecting the origin $\mathcal{O}$ to the location of a point in three dimensions is called the \textbf{position vector}
\begin{equation}
\vb{r} \equiv x \, \vu{i} + y \, \vu{j} + z \, \vu{k} 
\end{equation}

Following Griffiths, we reserve the letter $\vb{r}$ for that purpose. Its magnitude 

\begin{equation}
r = \sqrt{x^2 + y^2 + z^2} 
\end{equation}

is the distance from the origin, and 

\begin{equation}
\vu{r} = \frac{\vb{r}}{r} = \frac{x \, \vu{i} + y \, \vu{j} + z \, \vu{k}}{\sqrt{x^2 + y^2 + z^2}} 
\end{equation}

is a unit vector pointing radially outward. Following Griffiths, the \textbf{infinitesimal displacement vector} from $(x, y, z)$ to $(x+dx, y+dy, z+dz)$ is denoted as $d\vb{l}$

\begin{equation}
d\vb{l} = dx \, \vu{i} + dy \, \vu{j} + dz \, \vu{k} 
\end{equation}

Problems in electrodynamics frequently involve two points, typically a \textbf{source point}, $\vb{r}'$ and a \textbf{field point} $\vb{r}$ at which one needs to calculate the effect of the source. Following Griffiths we reserve the special symbol $\brcurs$ to denote the \textbf{separation vector} from the source point to the field point:

\begin{equation}
\begin{aligned}
\brcurs &\equiv \vb{r} - {\vb{r}}'\\
\rcurs &= \abs{\brcurs} = \abs{\vb{r} - {\vb{r}}'} \\
\hrcurs &= \frac{\brcurs}{\rcurs}
\end{aligned}
\end{equation}


\section{How Vectors Transform}
 




