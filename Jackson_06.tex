\chapter{Jackson -- Maxwell Equations, Conservation Laws}
\label{ch:Jackson_06} 

\section{Maxwell Equations}

\begin{equation}
\begin{aligned}
\div{\vb{B}} &= 0\\
\curl{\vb{E}} + \frac{1}{c} \pdv{\vb{B}}{t} &= 0\\
\div{\vb{E}} &= 4 \pi \rho\\
\curl{\vb{B}} &= \frac{4 \pi}{c} \vb{J} + \frac{1}{c} \pdv{\vb{E}}{t}
\end{aligned}
\label{eq:maxwell}
\end{equation}


\section{Potentials}

$\vb{B}$ has a vanishing divergence, therefore a vector field $\vb{A}$ exists such that

\begin{equation}
\vb{B} = \curl{\vb{A}}
\label{eq:Adef}
\end{equation}

Substituting \ref{eq:Adef} in the second of Maxwell equations one gets

\begin{equation*}
\curl{\vb{E}} + \frac{1}{c} \pdv{\curl{\vb{A}}}{t} = 0
\end{equation*}

whence 

\begin{equation}
\curl{ \left( \vb{E} + \frac{1}{c} \pdv{\vb{A}}{t} \right)} = 0
\end{equation}

The expression enclosed in parentheses in the above equation implies that a scalar field $\phi$ exists such that 

\begin{equation}
\vb{E} + \frac{1}{c} \pdv{\vb{A}}{t} = - \grad{\phi}
\label{eq:phidef}
\end{equation}


Using the last equation in \ref{eq:maxwell} and a known theorem of vector calculus

\begin{align*}
\curl{\vb{B}} &= \curl{(\curl{\vb{A}})} \\
&= \grad{(\div{\vb{A}})} - \laplacian{\vb{A}} \\
&= \frac{4 \pi}{c} \vb{J} + \frac{1}{c} \pdv{\vb{E}}{t}
\end{align*}

while from \ref{eq:phidef}

\begin{align*}
\frac{1}{c} \pdv{\vb{E}}{t} = - \frac{1}{c} \pdv{\grad{\phi}}{t} - \frac{1}{c^2} \pdv[2]{\vb{A}}{t}  
\end{align*}

therefore

\begin{align*}
\grad{(\div{\vb{A}})} - \laplacian{\vb{A}} &= \frac{4 \pi}{c} \vb{J} + \frac{1}{c} \pdv{\vb{E}}{t} \\
&= \frac{4 \pi}{c} \vb{J} - \frac{1}{c} \pdv{\grad{\phi}}{t} - \frac{1}{c^2} \pdv[2]{\vb{A}}{t}  
\end{align*}

With the goal of arriving at two independent equations for $\phi$ and $\vb{A}$ we put the equation for $\vb{A}$ in the following form 

\begin{equation}
\begin{aligned}
\laplacian{\vb{A}} - \frac{1}{c^2} \pdv[2]{\vb{A}}{t} &= - \frac{4 \pi}{c} \vb{J} \\
&+ \grad{\left( \div{\vb{A}} + \frac{1}{c} \pdv{\phi}{t} \right)}
\label{eq:uglyA}
\end{aligned}
\end{equation}

Turning to the scalar potential $\phi$ and combining the third equation in \ref{eq:maxwell} with \ref{eq:phidef} we obtain 

\begin{equation}
\begin{aligned}
\div{\vb{E}} &=   4 \pi \rho \\
&= - \laplacian{\phi} - \frac{1}{c} \pdv{\div{\vb{A}}}{t}
\label{eq:uglyPhi}
\end{aligned}
\end{equation}

\section{Gauge transformations}
Equations \ref{eq:uglyA} and \ref{eq:uglyPhi} are interdependent: one cannot be solved without the other having been solved. They stem from equations \ref{eq:Adef} and \ref{eq:phidef}, where potentials were introduced (respectively $\vb{A}$ and $\phi$) to yield fields $\vb{B}$ and $\vb{E}$ that by construction satisfy two of the four Maxwell equations \ref{eq:maxwell} (the homogeneous ones). 
However, the equations \ref{eq:Adef} and \ref{eq:phidef} do not uniquely determine the potentials. Indeed, if 
$\vb{A}$ and $\phi$ satisfy the equations, the same is true of the potentials $\vb{A}'$ and $\phi'$ obtained with the following transformation

\begin{equation}
\begin{aligned}
\vb{A}  &\longrightarrow \: \vb{A}' = \vb{A} + \grad{\lambda(\vb{r}, t)} \\   
\phi    &\longrightarrow \: \phi' = \phi - \frac{1}{c} \pdv{\lambda(\vb{r}, t)}{t}   
\label{eq:gaugexform}
\end{aligned}
\end{equation}

where $\lambda(\vb{r}, t)$ is any scalar function of position and time.

\subsection{Lorenz gauge}

Imposing the following constraint (\textit{Lorenz gauge}) on the potentials $\phi$ and $\vb{A}$

\begin{equation}
\div{\vb{A}} + \frac{1}{c} \pdv{\phi}{t} = 0
\label{eq:lorenzgauge} % Yes this is from Lorenz, not from the other (Lorentz)
\end{equation}

amounts to apply the transformation \ref{eq:gaugexform} with $\lambda(\vb{r}, t)$ satisfying the following equation: 

\begin{align*}
\laplacian{\lambda} - \frac{1}{c^2} \pdv[2]{\lambda}{t} = \div{\vb{A}} + \frac{1}{c} \pdv{\phi}{t}
\end{align*}


As a result, one obtains two independent \textit{wave equations} governing the dynamics of $\phi$ and $\vb{A}$, where the \textit{source fields} $\rho$ and $\vb{J}$ appear in the respective inhomogeneous terms: 


\begin{equation}
\laplacian{\phi} - \frac{1}{c^2} \pdv[2]{\phi}{t} = - 4 \pi \rho
\label{eq:PhiWave}
\end{equation}

\begin{equation}
\laplacian{\vb{A}} - \frac{1}{c^2} \pdv[2]{\vb{A}}{t} = - \frac{4 \pi}{c} \vb{J} 
\label{eq:AWave}
\end{equation}


\subsection{Coulomb gauge}

 

 




